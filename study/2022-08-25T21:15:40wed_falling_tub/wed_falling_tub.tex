\documentclass[12pt,b5paper]{ltjsarticle}

%\usepackage[margin=15truemm, top=5truemm, bottom=5truemm]{geometry}
\usepackage[margin=10truemm]{geometry}

\usepackage{amsmath,amssymb}
%\pagestyle{headings}
\pagestyle{empty}

%\usepackage{listings,url}
%\renewcommand{\theenumi}{(\arabic{enumi})}

%\usepackage{graphicx}

%\usepackage{tikz}
%\usetikzlibrary {arrows.meta}
%\usepackage{wrapfig}	% required for `\wrapfigure' (yatex added)
%\usepackage{bm}	% required for `\bm' (yatex added)

% ルビを振る
%\usepackage{luatexja-ruby}	% required for `\ruby'

%% 核Ker 像Im Hom を定義
%\newcommand{\Img}{\mathop{\mathrm{Im}}\nolimits}
%\newcommand{\Ker}{\mathop{\mathrm{Ker}}\nolimits}
%\newcommand{\Hom}{\mathop{\mathrm{Hom}}\nolimits}

%\DeclareMathOperator{\Rot}{rot}
%\DeclareMathOperator{\Div}{div}
%\DeclareMathOperator{\Grad}{grad}
%\DeclareMathOperator{\arcsinh}{arcsinh}
%\DeclareMathOperator{\arccosh}{arccosh}
%\DeclareMathOperator{\arctanh}{arctanh}



\begin{document}

\hrulefill

重さ2kgのたらいを5mの高さから落とすと落下の衝撃は400Nである

\hrulefill

重力加速度$g$は$10 \mathrm{m}/\mathrm{s}^2$として計算する

\dotfill

\fbox{たらいを落としたときの速さ}

高さ 5mの位置にあるたらいの位置エネルギーは$mgh=2\times 10 \times 5 =100$

たらいの速さを$v$として運動エネルギーを求めると
$\frac{1}{2}mv^2 = \frac{1}{2}2v^2=v^2$

位置エネルギーが全て運動エネルギーに変わったとすると
\begin{equation}
 mgh = \frac{1}{2}mv^2 \quad 100=v^2
\end{equation}

つまり、衝突直前の速さは$10\mathrm{m}/\mathrm{s}$


\fbox{たらいからの衝撃}

速さ$10\mathrm{m}/\mathrm{s}$のたらいが $0.05$s で速さ0になった場合の力を計算する

たらいが減速するときの加速度は
$(10-0)/0.05 = 200 \mathrm{m}/\mathrm{s}^2$

首に掛かる力は運動方程式から
$F=ma = 2 \times 200 =400 \mathrm{N}$

\dotfill

力が 400N になるように衝突の時間を 0.05s と設定

1kgの質量の物質が乗っている状態は10Nの力がかかっているので、
400Nだと40kgの物が乗っている状態



\end{document}

