\documentclass[11pt,a5paper]{ltjsarticle}

\usepackage[margin=15truemm, top=5truemm, bottom=5truemm]{geometry}
%\usepackage[margin=10truemm]{geometry}

\usepackage{amsmath,amssymb}
%\pagestyle{headings}
\pagestyle{empty}

%\usepackage{listings,url}
%\renewcommand{\theenumi}{(\arabic{enumi})}

%\usepackage{graphicx}

%\usepackage{tikz}
%\usetikzlibrary {arrows.meta}
%\usepackage{wrapfig}	% required for `\wrapfigure' (yatex added)
%\usepackage{bm}	% required for `\bm' (yatex added)

% ルビを振る
%\usepackage{luatexja-ruby}	% required for `\ruby'

%% 核Ker 像Im Hom を定義
%\newcommand{\Img}{\mathop{\mathrm{Im}}\nolimits}
%\newcommand{\Ker}{\mathop{\mathrm{Ker}}\nolimits}
%\newcommand{\Hom}{\mathop{\mathrm{Hom}}\nolimits}

%\DeclareMathOperator{\Rot}{rot}
%\DeclareMathOperator{\Div}{div}
%\DeclareMathOperator{\Grad}{grad}
%\DeclareMathOperator{\arcsinh}{arcsinh}
%\DeclareMathOperator{\arccosh}{arccosh}
%\DeclareMathOperator{\arctanh}{arctanh}



%\usepackage{listings,url}
%
%\lstset{
%%プログラム言語(複数の言語に対応,C,C++も可)
%%  language = Python,
%%  language = Lisp,
%  language = C,
%  %背景色と透過度
%  %backgroundcolor={\color[gray]{.90}},
%  %枠外に行った時の自動改行
%  breaklines = true,
%  %自動改行後のインデント量(デフォルトでは20[pt])
%  breakindent = 10pt,
%  %標準の書体
%%  basicstyle = \ttfamily\scriptsize,
%  basicstyle = \ttfamily,
%  %コメントの書体
%%  commentstyle = {\itshape \color[cmyk]{1,0.4,1,0}},
%  %関数名等の色の設定
%  classoffset = 0,
%  %キーワード(int, ifなど)の書体
%%  keywordstyle = {\bfseries \color[cmyk]{0,1,0,0}},
%  %表示する文字の書体
%  %stringstyle = {\ttfamily \color[rgb]{0,0,1}},
%  %枠 "t"は上に線を記載, "T"は上に二重線を記載
%  %他オプション:leftline,topline,bottomline,lines,single,shadowbox
%  frame = TBrl,
%  %frameまでの間隔(行番号とプログラムの間)
%  framesep = 5pt,
%  %行番号の位置
%  numbers = left,
%  %行番号の間隔
%  stepnumber = 1,
%  %行番号の書体
%%  numberstyle = \tiny,
%  %タブの大きさ
%  tabsize = 4,
%  %キャプションの場所("tb"ならば上下両方に記載)
%  captionpos = t
%}



\begin{document}

\hrulefill
\textbf{日本地図クイズ}
\hrulefill
\begin{enumerate}
 \item
      4県 [岩手-秋田-山形-宮城]で隣接しているのはどっち?
      \begin{center}
       (\textbf{岩手-山形}) or (\textbf{秋田-宮城})
      \end{center}

 \item
      4県 [福島-新潟-群馬-栃木]で隣接しているのはどっち?
      \begin{center}
       (\textbf{福島-群馬}) or (\textbf{新潟-栃木})
      \end{center}

 \item
      4県 [栃木-群馬-埼玉-茨城]で隣接しているのはどっち?
      \begin{center}
       (\textbf{栃木-埼玉}) or (\textbf{群馬-茨城})
      \end{center}

 \item
      4県 [群馬-長野-山梨-埼玉]で隣接しているのはどっち?
      \begin{center}
       (\textbf{群馬-山梨}) or (\textbf{長野-埼玉})
      \end{center}

 \item
      4県 [富山-石川-福井-岐阜]で隣接しているのはどっち?
      \begin{center}
       (\textbf{富山-福井}) or (\textbf{石川-岐阜})
      \end{center}

 \item
      4県 [岐阜-滋賀-三重-愛知]で隣接しているのはどっち?
      \begin{center}
       (\textbf{岐阜-三重}) or (\textbf{滋賀-愛知})
      \end{center}

 \item
      1府3県 [滋賀-京都-奈良-三重]で隣接しているのはどっち?
      \begin{center}
       (\textbf{滋賀-奈良}) or (\textbf{京都-三重})
      \end{center}

 \item
      4県 [鳥取-島根-広島-岡山]で隣接しているのはどっち?
      \begin{center}
       (\textbf{鳥取-広島}) or (\textbf{島根-岡山})
      \end{center}

 \item
      4県 [香川-愛媛-高知-徳島]で隣接しているのはどっち?
      \begin{center}
       (\textbf{香川-高知}) or (\textbf{愛媛-徳島})
      \end{center}

 \item
      4県 [大分-福岡-熊本-宮崎]で隣接しているのはどっち?
      \begin{center}
       (\textbf{大分-熊本}) or (\textbf{福岡-宮崎})
      \end{center}

\end{enumerate}

\hrulefill
\textbf{以上、10問}
\hrulefill


\end{document}
