\documentclass[10pt,b5paper]{ltjsarticle}

\usepackage[margin=15truemm]{geometry}
\pagestyle{empty}

\usepackage{amssymb}
\usepackage{amsmath}	% required for `\align' (yatex added)

%\usepackage{okumacro}	% required for `\ruby' (yatex added)
%\usepackage{pxrubrica}
\usepackage{luatexja-ruby}


\begin{document}


筆のすさび

\hrulefill

\ruby{麾下}{きか}の士(旗本の武士)\ruby{何某}{なにがし}の町奉行になられし時、
\ruby{堀田}{ほった}\ruby{筑前守}{ちくぜんのかみ}殿の
「必らず相手にならぬやうにあれかし(あれよ)」と申されしに、
何某\ruby{其}{その}時は\ruby{合点}{がてん}ゆかざりしが、
\ruby{訟}{うったへ}をきくにいたりてはじめて心付きしといはれしとぞ。
訟をきくは公の事ながら、
\ruby{悪}{にく}しとおもひ、
むつかし(不快だ)とおもへば、
必ず其人を我が相手とおもふやうになるものなり、
\ruby{我詞}{わがことば}するどくなれば、
其人言を尽すことあたはず、
必ずかたききになりて、
取さばき平かならず、
相手になるなといはれしは金言なりと、
子孫にもいひ置かれしとなり。

%麾下(きか)の士(旗本の武士)何某(なにがし)の町奉行になられし時、
%堀田筑前守(ちくぜんのかみ)殿の「必らず相手にならぬやうにあれかし」と申されしに、
%何某其(その)時は合点ゆかざりしが、
%訟(うったへ)をきくにいたりてはじめて心付きしといはれしとぞ。
%訟をきくは公の事ながら、悪(にく)しとおもひ、むつかしとおもへば、
%必ず其人を我が相手とおもふやうになるものなり、我詞(わがことば)するどくなれば、
%其人言を尽すことあたはず、必ずかたききになりて、取さばき平かならず、
%相手になるなといはれしは金言なりと、子孫にもいひ置かれしとなり。

%\hrulefill
\dotfill

とある旗本の武士が町奉行になった時、
堀田筑前守に
「決して相手にならないように」
と言われ、
その町奉行は合点がいかなかったが、
訴えを聞くようになると分かるようになった。

訴えを聞くことは公のことであるが、
気に入らない、不快であると思えば
必ず自分と対立する相手と思うようになるもので
自分の言葉は鋭くなり、
相手は言葉を尽くすことが出来ず、
一方的な聞き方になり、
取りさばきが公平ではなくなる。
相手になるなとは金言であると
子孫に言い伝えた。


\hrulefill


\begin{enumerate}\renewcommand{\theenumi}{(\arabic{enumi})}
 \item
      「相手にするなと言われたがよくわからなかった。
      訴えを聞くようになって理解できた。」

      ということで訴えを話してくる相手のことを指している。

      訴えを聞くことは公の事とあるので
%      相手とは仕事上で訴えを話してくる相手ということ。
%
      適当な選択肢としては
      \underline{仕事で協力する相手}

 \item きく、いはれし、いはれし、いひ置かれし

       \begin{enumerate}
        \item \ruby{訟}{うったへ}をきくにいたりて

        \item はじめて心付きしといはれし

        \item 相手になるなといはれしは金言なり

        \item 子孫にもいひ置かれしとなり。
       \end{enumerate}

       (a)、(b)、(d)は町奉行、
       (c)は筑前守

 \item 言を尽くすことあたはず

       言・・・言葉
       \qquad
       尽くす・・・出し切る
       \qquad
       あたはず・・・できない

       言いたいことが全て言うことが出来ないという意味であり、
       選択肢としては\\
       \underline{言いたいことをすべて言うことができない}

 \item 筑前守の言葉を何某がどのようにうけとったか

       親身になって訴えを聞いていると
       不快な思いなどから片側に肩入れしていまい
       公平ではなくなるので、
       筑前守の「相手にするな」とはいい言葉だとおもった。

       ということが書かれているので、
       \underline{常に公平に判断して裁けるように心がける}

 \item 文章の内容

       人の訴えを聞くことが仕事の町奉行になった旗本が、
       筑前守に「相手にしないように」と言われ何を言っているんだろうと思った。
       しかし、訴えを聞いていくうちに、
       不快だと感じた人に強くあたったりして、訴えをすべて聞けない場合もある。
       これでは公平さを保てないので
       筑前守の「相手にするな」とは良い言葉であると思い、
       子孫にも伝えた。

       適当な選択肢は
       \underline{筑前守の言葉を優れたものだと考えるようになった}
\end{enumerate}


\end{document}
