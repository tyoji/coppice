\documentclass[12pt,b5paper]{ltjsarticle}

%\usepackage[margin=15truemm, top=5truemm, bottom=5truemm]{geometry}
%\usepackage[margin=10truemm,left=15truemm]{geometry}
\usepackage[margin=10truemm]{geometry}

\usepackage{amsmath,amssymb}
%\pagestyle{headings}
\pagestyle{empty}

%\usepackage{listings,url}
%\renewcommand{\theenumi}{(\arabic{enumi})}

\usepackage{graphicx}

%\usepackage{tikz}
%\usetikzlibrary {arrows.meta}
%\usepackage{wrapfig}	% required for `\wrapfigure' (yatex added)
%\usepackage{bm}	% required for `\bm' (yatex added)

% ルビを振る
%\usepackage{luatexja-ruby}	% required for `\ruby'

%% 核Ker 像Im Hom を定義
%\newcommand{\Img}{\mathop{\mathrm{Im}}\nolimits}
%\newcommand{\Ker}{\mathop{\mathrm{Ker}}\nolimits}
%\newcommand{\Hom}{\mathop{\mathrm{Hom}}\nolimits}

%\DeclareMathOperator{\Rot}{rot}
%\DeclareMathOperator{\Div}{div}
%\DeclareMathOperator{\Grad}{grad}
%\DeclareMathOperator{\arcsinh}{arcsinh}
%\DeclareMathOperator{\arccosh}{arccosh}
%\DeclareMathOperator{\arctanh}{arctanh}



\usepackage{listings,url}

\lstset{
%プログラム言語(複数の言語に対応,C,C++も可)
  language = Python,
%  language = Lisp,
%  language = C,
  %背景色と透過度
  %backgroundcolor={\color[gray]{.90}},
  %枠外に行った時の自動改行
  breaklines = true,
  %自動改行後のインデント量(デフォルトでは20[pt])
  breakindent = 10pt,
  %標準の書体
%  basicstyle = \ttfamily\scriptsize,
  basicstyle = \ttfamily,
  %コメントの書体
%  commentstyle = {\itshape \color[cmyk]{1,0.4,1,0}},
  %関数名等の色の設定
  classoffset = 0,
  %キーワード(int, ifなど)の書体
%  keywordstyle = {\bfseries \color[cmyk]{0,1,0,0}},
  %表示する文字の書体
  %stringstyle = {\ttfamily \color[rgb]{0,0,1}},
  %枠 "t"は上に線を記載, "T"は上に二重線を記載
  %他オプション:leftline,topline,bottomline,lines,single,shadowbox
  frame = TBrl,
  %frameまでの間隔(行番号とプログラムの間)
  framesep = 5pt,
  %行番号の位置
  numbers = left,
  %行番号の間隔
  stepnumber = 1,
  %行番号の書体
%  numberstyle = \tiny,
  %タブの大きさ
  tabsize = 4,
  %キャプションの場所("tb"ならば上下両方に記載)
  captionpos = t
}



\begin{document}

\hrulefill

$f(x)=x^3-2x+2$と定義する。
Newton法で方程式$f(x)=0$の実数解の近似値を計算することを考える。
Newton法とは、与えられた初期値$x_0$に対して、
次の漸化式により反復計算を行う手法である。
\begin{equation}
 x_{n+1}=x_{n}-\frac{f(x_{n})}{f^{\prime}(x_{n})}
  \quad (n\geq 0)
  \label{form}
\end{equation}

方程式$f(x)=0$はただ一つの実数解
\begin{equation}
 a=\sqrt[3]{-1+\sqrt{\frac{19}{27}}}+\sqrt[3]{-1-\sqrt{\frac{19}{27}}}
\end{equation}
を持つことは既知として、以下の問いに答えよ。

\begin{enumerate}
 \item
      漸化式(\ref{form})を用いて、
      $x_1,x_2,\dots$を反復的に計算する
      関数\texttt{newton(x0)}を,
      Julia言語で次の要件を満たすように作成せよ。
      \begin{enumerate}
       \item 引数\texttt{x0}を初期値とすること
       \item $f^{\prime}(x_n)=0$のとき、反復計算を終了する
       \item 各$n\geq 0$に対して、\textbf{[$x_n$の値]}
             と\textbf{[誤差$\lvert x_n-a\rvert$]}
             を表示する
       \item $x_{n+1}$を計算した時点で、
             $\lvert x_{n+1}-x_n\rvert<10^{-6}$が満たされているならば、
             反復計算を終了する
      \end{enumerate}

      各要件がプログラムコードのどこの部分に該当するのかを明記し、
      プログラムに関する説明を必ず記述すること。

      \dotfill

      \begin{lstlisting}
function newton(x)
    # 引数 x を初期値とする ## 1 ##
    i=0
    while true
        println(i, "回目")
        # 計算の都度 近似解とその誤差を表示 ## 3 ##
        println("+-- x の値")
        println(x)
        println("+-- aとの誤差")
        println(abs(x-a))
        # 分母の微分係数が 0 になれば終了 ## 2 ##
        if f2(x) == 0
            println("f'(x)=0 により終了")
            return 0
        end
        # 漸化式の計算
        y = x - f1(big(x))/f2(big(x))
        # 近似解の差が小さい場合終了 ## 4 ##
        if abs(y-x) < big(10)^(-6)
            println("近似解の差が小さいので終了")
            return 0
        end
        # 30回を超えたら中止
        if i > 30
            println("30回を超えたので中止")
            return 1
        end
        x=y # 近似解の項を次にセットする
        i += 1
    end
end
      \end{lstlisting}


      \hrulefill

 \item
      \texttt{newton(-16)} および \texttt{newton(-2)}の実行結果を示し、
      誤差や収束の様子について考察を述べよ。

      \dotfill

      方程式の実数解は
      $a=-1.769292\cdots$
      であるので、
      初期値はこの値に近いと早くアルゴリズムが終わる。
      \texttt{newton(-16)}は$x_9$まで計算し、
      \texttt{newton(-2)}は$x_3$までの計算で近似解が得られた。


      \hrulefill


 \item
      初期値\texttt{x0}を 0 に近い値に選び、
      \texttt{newton(x0)} の実行結果を示せ。
      さらに、
      実行結果に関する考察を述べよ。

      \dotfill

      漸化式
      $x_{n+1}=x_{n}-\frac{f(x_{n})}{f^{\prime}(x_{n})}$
      は$x_n=1$のとき、$x_{n+1}=0$となり、
      $x_n=0$のとき、$x_{n+1}+1$となる。

      この為、数列$\{x_i\}_{i\in\mathbb{N}}$
      はどこかで$0$か$1$が現れると、
      それ以降$0,1$が交代に出てくるだけになる。
      近似解に近づかない数列となるので
      これを避けるように初期値を選択しないといけない。


      \hrulefill

\end{enumerate}




\hrulefill

\end{document}
