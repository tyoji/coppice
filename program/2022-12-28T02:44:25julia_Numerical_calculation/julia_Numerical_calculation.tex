\documentclass[12pt,b5paper]{ltjsarticle}

%\usepackage[margin=15truemm, top=5truemm, bottom=5truemm]{geometry}
%\usepackage[margin=10truemm,left=15truemm]{geometry}
\usepackage[margin=10truemm]{geometry}

\usepackage{amsmath,amssymb}
%\pagestyle{headings}
\pagestyle{empty}

%\usepackage{listings,url}
%\renewcommand{\theenumi}{(\arabic{enumi})}

%\usepackage{graphicx}

%\usepackage{tikz}
%\usetikzlibrary {arrows.meta}
%\usepackage{wrapfig}	% required for `\wrapfigure' (yatex added)
%\usepackage{bm}	% required for `\bm' (yatex added)

% ルビを振る
%\usepackage{luatexja-ruby}	% required for `\ruby'

%% 核Ker 像Im Hom を定義
%\newcommand{\Img}{\mathop{\mathrm{Im}}\nolimits}
%\newcommand{\Ker}{\mathop{\mathrm{Ker}}\nolimits}
%\newcommand{\Hom}{\mathop{\mathrm{Hom}}\nolimits}

%\DeclareMathOperator{\Rot}{rot}
%\DeclareMathOperator{\Div}{div}
%\DeclareMathOperator{\Grad}{grad}
%\DeclareMathOperator{\arcsinh}{arcsinh}
%\DeclareMathOperator{\arccosh}{arccosh}
%\DeclareMathOperator{\arctanh}{arctanh}



\usepackage{listings,url}

\lstset{
%プログラム言語(複数の言語に対応,C,C++も可)
%  language = Python,
%  language = Lisp,
%  language = C,
  %背景色と透過度
  %backgroundcolor={\color[gray]{.90}},
  %枠外に行った時の自動改行
  breaklines = true,
  %自動改行後のインデント量(デフォルトでは20[pt])
  breakindent = 10pt,
  %標準の書体
%  basicstyle = \ttfamily\scriptsize,
  basicstyle = \ttfamily,
  %コメントの書体
%  commentstyle = {\itshape \color[cmyk]{1,0.4,1,0}},
  %関数名等の色の設定
  classoffset = 0,
  %キーワード(int, ifなど)の書体
%  keywordstyle = {\bfseries \color[cmyk]{0,1,0,0}},
  %表示する文字の書体
  %stringstyle = {\ttfamily \color[rgb]{0,0,1}},
  %枠 "t"は上に線を記載, "T"は上に二重線を記載
  %他オプション:leftline,topline,bottomline,lines,single,shadowbox
  frame = TBrl,
  %frameまでの間隔(行番号とプログラムの間)
  framesep = 5pt,
  %行番号の位置
  numbers = left,
  %行番号の間隔
  stepnumber = 1,
  %行番号の書体
%  numberstyle = \tiny,
  %タブの大きさ
  tabsize = 4,
  %キャプションの場所("tb"ならば上下両方に記載)
  captionpos = t
}



\begin{document}

\hrulefill

\textbf{Julia (プログラミング言語)}

$\displaystyle f(x)=\frac{1-\cos{(x)}}{x^2}$と定義する。
以下の問について答えよ。
ここでは浮動小数点数として
デフォルトの倍精度(Float64)を用いるとする。

\dotfill

\begin{enumerate}
 \item
      関数$f(x)$を表記通りにJulia言語で実装し、
      \texttt{f(1.0e-3)}、\texttt{f(1.0e-5)}
      の値を出力せよ。
      この問に関しては
      コードに関する説明は記述しなくともよい。

      \dotfill

      
      Julia ソースコード
      \begin{lstlisting}
# f(x) の定義
f(x)=(1-cos(x))/(x^2)

# 計算
println("-- f(1.0e-3) の値 --")
println(f(1.0e-3))
println("-- f(1.0e-5) の値 --")
println(f(1.0e-5))
      \end{lstlisting}

      実行結果
      \begin{lstlisting}
-- f(1.0e-3) の値 --
0.49999995832550326
-- f(1.0e-5) の値 --
0.5000000413701854
      \end{lstlisting}

      \hrulefill

 \item
      理論的には$f(x)\leq 0.5$が成り立っているはずだが、
      \texttt{f(1.0e-5)}の出力する値は
      この不等式を満たしていない。
      このようなことが起こる原因について簡単に述べよ。

      \dotfill

      丸め誤差や桁落ちが発生している。
      特に$\cos{(0)}=1$であるので、
      $1-\cos{(x)}$の計算に問題が起きる。
      
      式を分けてそれぞれ
      $\cos{(x)},1-\cos{(x)},x^2,f(x)$
      を計算する。
      \begin{lstlisting}
# f(x) の定義
f(x)=(1-cos(x))/(x^2)

# 計算
println("-- cos(1.0e-3) の値 --")
println(cos(1.0e-3))
println("-- 1-cos(1.0e-3) の値 --")
println(1-cos(1.0e-3))
println("-- (1.0e-3)^2 の値 --")
println((1.0e-3)^2)
println("-- f(1.0e-3) の値 --")
println(f(1.0e-3))
println("-- cos(1.0e-5) の値 --")
println(cos(1.0e-5))
println("-- 1-cos(1.0e-5) の値 --")
println(1-cos(1.0e-5))
println("-- (1.0e-5)^2 の値 --")
println((1.0e-5)^2)
println("-- f(1.0e-5) の値 --")
println(f(1.0e-5))
      \end{lstlisting}

      実行結果
      \begin{lstlisting}
-- cos(1.0e-3) の値 --
0.9999995000000417
-- 1-cos(1.0e-3) の値 --
4.999999583255033e-7
-- (1.0e-3)^2 の値 --
1.0e-6
-- f(1.0e-3) の値 --
0.49999995832550326
-- cos(1.0e-5) の値 --
0.99999999995
-- 1-cos(1.0e-5) の値 --
5.000000413701855e-11
-- (1.0e-5)^2 の値 --
1.0000000000000002e-10
-- f(1.0e-5) の値 --
0.5000000413701854
      \end{lstlisting}

      関数電卓を用いた余弦$\cos{(x)}$の値
      \begin{align}
       \cos{(\mathtt{1.0e-3})} = \cos{(0.001)}
       =& 0.99999999984769129011051202417815\\
       \cos{(\mathtt{1.0e-5})} = \cos{(0.00001)}
       =& 0.9999999999999847691290106646087
      \end{align}


      \hrulefill

 \item
      $f(x)$を適切に「等式変形」することによって、
      問2のような不都合を回避する関数$g(x)$を導入せよ。


      \dotfill

      近い値同士の引き算をなくすために
      $f(x)$の分子分母に$1+\cos{(x)}$をかける。

      \begin{equation}
       g(x)
        = \frac{1-\cos{(x)}}{x^2}\cdot\frac{1+\cos{(x)}}{1+\cos{(x)}}
        = \frac{\sin^2{(x)}}{x^2(1+\cos{(x)})}
      \end{equation}

      \hrulefill

 \item
      問3で導入した$g(x)$をJulia言語で実装し、
      \texttt{g(1.0e-3)}、\texttt{g(1.0e-5)}
      の値を出力せよ。
      結果に関する説明も記述すること。

      \dotfill



      \begin{lstlisting}
# g(x) の定義
g(x)=(sin(x))^2/(x^2*(1+cos(x)))
# 計算
println("-- g(1.0e-3) の値 --")
println(g(1.0e-3))
println("-- g(1.0e-5) の値 --")
println(g(1.0e-5))
      \end{lstlisting}

      実行結果
      \begin{lstlisting}
-- g(1.0e-3) の値 --
0.4999999583333347
-- g(1.0e-5) の値 --
0.4999999999958333
      \end{lstlisting}


      \hrulefill


\end{enumerate}

\hrulefill

\end{document}
