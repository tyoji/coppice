\documentclass[12pt,b5paper]{ltjsarticle}

%\usepackage[margin=15truemm, top=5truemm, bottom=5truemm]{geometry}
%\usepackage[margin=10truemm,left=15truemm]{geometry}
\usepackage[margin=10truemm]{geometry}

\usepackage{amsmath,amssymb}
%\pagestyle{headings}
\pagestyle{empty}

%\usepackage{listings,url}

% 記号変更
%\renewcommand{\theenumi}{(\arabic{enumi})}
\renewcommand{\labelenumi}{(\arabic{enumi})}


%\usepackage{graphicx}

\usepackage{tikz} % draw graphics : TikZ ist kein Zeichenprogramm
\usetikzlibrary{arrows.meta}


\usepackage{circuitikz} % 電気回路図用ライブラリ 要Tikzパッケージ



%\usepackage{wrapfig}
%\usepackage{bm} % ベクトルの矢印

%\usepackage{luatexja-ruby} % ルビを振る

%% 核Ker 像Im Hom を定義
\newcommand{\Ker}{\mathop{\mathrm{Ker}}\nolimits}
\newcommand{\Img}{\mathop{\mathrm{Img}}\nolimits}
%\newcommand{\Hom}{\mathop{\mathrm{Hom}}\nolimits}

%\DeclareMathOperator{\Rot}{rot}
%\DeclareMathOperator{\Div}{div}
%\DeclareMathOperator{\Grad}{grad}
%\DeclareMathOperator{\arcsinh}{arcsinh}
%\DeclareMathOperator{\arccosh}{arccosh}
%\DeclareMathOperator{\arctanh}{arctanh}

\usepackage{url} % URLの記述

%\usepackage{listings}
%
%\lstset{
%%プログラム言語(複数の言語に対応,C,C++も可)
%  language = Python,
%%  language = Lisp,
%%  language = C,
%  %背景色と透過度
%  %backgroundcolor={\color[gray]{.90}},
%  %枠外に行った時の自動改行
%  breaklines = true,
%  %自動改行後のインデント量(デフォルトでは20[pt])
%  breakindent = 10pt,
%  %標準の書体
%%  basicstyle = \ttfamily\scriptsize,
%  basicstyle = \ttfamily,
%  %コメントの書体
%%  commentstyle = {\itshape \color[cmyk]{1,0.4,1,0}},
%  %関数名等の色の設定
%  classoffset = 0,
%  %キーワード(int, ifなど)の書体
%%  keywordstyle = {\bfseries \color[cmyk]{0,1,0,0}},
%  %表示する文字の書体
%  %stringstyle = {\ttfamily \color[rgb]{0,0,1}},
%  %枠 "t"は上に線を記載, "T"は上に二重線を記載
%  %他オプション:leftline,topline,bottomline,lines,single,shadowbox
%  frame = TBrl,
%  %frameまでの間隔(行番号とプログラムの間)
%  framesep = 5pt,
%  %行番号の位置
%  numbers = left,
%  %行番号の間隔
%  stepnumber = 1,
%  %行番号の書体
%%  numberstyle = \tiny,
%  %タブの大きさ
%  tabsize = 4,
%  %キャプションの場所("tb"ならば上下両方に記載)
%  captionpos = t
%}

%\usepackage{cancel}
%\usepackage{bussproofs}
%\usepackage{proof}

\begin{document}

%%\begin{figure}[htbp]
%  \begin{center}
%    \begin{circuitikz}[american currents]
%     \draw (0,0) to[sV=$E$] (0,2) to[short] (2,2) to[european resistor=$R$] (2,0) to[short] (0,0);
%      \draw (2,2)
%      to[short] (4,2)
%      to[L=$L$] (4,0)
%      to[short] (2,0);
%      \draw (4,2)
%      to[short] (6,2)
%      to[C=$C$] (6,0)
%      to[short] (4,0);
%    \end{circuitikz}
%%    \caption{RLC並列回路}
%  \end{center}
%%\end{figure}


\hrulefill

\begin{minipage}[c]{180pt}
 \begin{center}
  \begin{circuitikz}[scale=1.0]
   \draw(0,4) to[battery1] (0,0); % 左側
   \draw(0,0) to[short,-*] (2,0) to[short] (4,0); % 下部
   \draw(0,4) to[nos=$S_{1}$,-*] (2,4) to[short] (4,4); % 上部
   \draw(4,4) to[C=$C_{1}$,-*] (4,2) to[C=$C_{2}$] (4,0); % 右側
   \draw(2,0) to[european resistor=$R_{2}$,-*] (2,2) to[european resistor=$R_{1}$] (2,4); % 中央縦
   \draw(2,2) to[nos=$S_{2}$] (4,2); % 横中央

   \node[left] at (-0.5,2) {$E$};
   \node[above] at (2,4) {$a$};
   \node[below] at (2,0) {$b$};
   \node[left] at (2,2) {$c$};
   \node[right] at (4,2) {$d$};
  \end{circuitikz}
%  \caption{電気回路図}
%  \label{fig:electric_circuit}
 \end{center}
\end{minipage}
%%%%%%%%%%%%%%%%%%%%%%%%%%%%%%%%%%%
\begin{minipage}[c]{250pt}
コンデンサ$C_{1},\;C_{2}$の電気容量を$C,\;2C$、
抵抗$R_{1},\;R_{2}$の抵抗値を$R,\;2R$とし、
起電力$E$、スイッチ$S_{1},\;S_{2}$を接続し、%(図\ref{fig:electric_circuit})
次の操作を行った。
\begin{enumerate}
 \item $S_{1}$のみ閉じる
 \item コンデンサを流れる電流が0になる
 \item $S_{2}$も閉じる
 \item 回路を流れる電流は一定となる
\end{enumerate}

電位は点$b$を基準とする。
次の値を$C,\;R,\;E$を用いて求めよ。
\end{minipage}

\dotfill
\begin{enumerate}
 \item スイッチ$S_{2}$を閉じる直前の、点$d$の電位$V_{d}$
 \item スイッチ$S_{2}$を閉じる直前の、抵抗$R_{1}$を流れる電流の大きさ$I$
 \item スイッチ$S_{2}$を閉じて十分に時間経過したときの、点$d$の電位$V_{d}^{\prime}$
 \item スイッチ$S_{2}$を閉じてから、
       十分に時間経過するまでにスイッチ$S_{2}$を通って
       点$c$から点$d$に向かって流れた電気量$Q$
\end{enumerate}

その後、スイッチ$S_{2}$を開き、続いてスイッチ$S_{1}$を開いた。
十分に時間が経過すると回路を流れる電流は0になった。

\begin{enumerate}\setcounter{enumi}{4}
 \item スイッチ$S_{1}$を開いてから、十分に時間が経過するまでに、
       点$a$から抵抗$R_{1},\;R_{2}$を通り点$b$に向かって流れた電気量$Q^{\prime}$
 \item スイッチ$S_{1}$を開いてから、十分に時間が経過するまでに
       抵抗$R_{2}$で発生するジュール熱$P$
\end{enumerate}


\hrulefill


\begin{enumerate}
 \item
      コンデンサ$C_{1},\;C_{2}$の電荷は
      $Q_{1}=C_{1}V_{1},Q_{2}=C_{2}V_{2}$であり、
      直列のコンデンサの電荷は同じになるため、
      $C_{1}V_{1}=C_{2}V_{2}$。

      コンデンサ$C_{2}$の電位を$V_{d}$とすると
      コンデンサ$C_{1}$の電位は$E-V_{d}$であるので、
      次の式を得る。
      \begin{equation}
        C (E-V_{d})=2CV_{d}
      \end{equation}
      これを解くと次が得られる。
      \begin{equation}
       V_{d}=\frac{1}{3}E
      \end{equation}

      \dotfill
 \item
      抵抗$R_{1},R_{2}$の抵抗値はそれぞれ$R,2R$であり、
      これが直列に繋がれているので2つの抵抗の合計は$3R$である。
      電流$I$は電圧$E$を抵抗$3R$で割ることで求まるため次のようになる。

      \begin{equation}
       I=\frac{E}{3R}
      \end{equation}

      \dotfill
 \item

      スイッチ$S_{2}$を閉じ十分に電気を流すと、
      コンデンサには十分に電気が溜まっている状態となる。

      この場合、
      $R_{1},C_{1}$の並列接続と$R_{2},C_{2}$の並列接続が直列につながっている。
      コンデンサには電気が流れない状況であれば
      それぞれの電圧は$R_{1},R_{2}$に流れる電流から求まる。
      \begin{equation}
       V=IR = \frac{E}{3R} \times 2R
      \end{equation}
      上記計算により次のように電圧が求まる。

      \begin{equation}
       V_{d}^{\prime}=\frac{2}{3}E
      \end{equation}

      \dotfill
 \item
      スイッチ$S_{2}$を閉じる前に$C_{1},C_{2}$には電気が蓄えられている。
      電気容量が$C_{1}$のほうが小さいため、
      $S_{2}$を閉じたあとは
      その差分$2C\times E - C\times E$の電気が流れることになる。

      \begin{equation}
       Q=CE
      \end{equation}

      \dotfill
 \item

      コンデンサ$C_{1},C_{2}$の直列接続による静電容量$C$は次の式で求められる。
      \begin{equation}
       C=\frac{1}{\frac{1}{C_{1}}+\frac{1}{C_{2}}}=\frac{C_{1}C_{2}}{C_{1}+C_{2}}
      \end{equation}

      スイッチ$S_{2}$を閉じ、
      十分に電気を蓄えたあとにスイッチ$S_{2}$を開くと
      直列接続している$C_{1},C_{2}$の合成静電容量分の電気が蓄電されている状態となる。
      \begin{equation}
       \frac{C\times 2C}{C+2C}=\frac{2}{3}C
      \end{equation}

      スイッチ$S_{1}$を開くと、
      コンデンサに蓄えた電気が流れる事となるので、
      電力$E$と静電容量をかけた電気が流れることとなる。

      \begin{equation}
       Q^{\prime}=\frac{2}{3}CE
      \end{equation}

      \dotfill
 \item

      \begin{minipage}[t]{280pt}
       スイッチ$S_{1},S_{2}$を開いた状態だと、
       コンデンサ$C_{1},C_{2}$と抵抗$R_{1},R_{2}$を直列に繋いた閉回路
       となる。
       
       2つのコンデンサを合成したものを$C_{1,2}$とすると前問より、
       合成した電気容量$Q^{\prime}=\frac{2}{3}CE$を
       $C_{1,2}$が蓄えている状態であるといえる。
       このとき、直前まで起電力$E$により電気が流れていた為、
       $C_{1,2}$の電位差は$E$となっている。

       放電により$C_{1,2}$の電気が抵抗$R_{1},R_{2}$に流れる場合の
       $R_{2}$によって発生するジュール熱を求める。
      \end{minipage}
      \quad
      %%%%%%%%%%%%%%%%%%%%%%%%%%%%%%%%%%%
      \begin{minipage}[t]{100pt}
       \begin{center}
        \begin{circuitikz}[scale=1.0]
%         \draw(0,4) to[battery1] (0,0); % 左側
%         \draw(0,0) to[short,-*] (2,0) to[short] (4,0); % 下部
         \draw (2,0) to[short,*-] (4,0); % 下部
%         \draw(0,4) to[nos=$S_{1}$,-*] (2,4) to[short] (4,4); % 上部
         \draw (2,4) to[short,*-] (4,4); % 上部
%         \draw(4,4) to[C=$C_{1}$,-*] (4,2) to[C=$C_{2}$] (4,0); % 右側
         \draw(4,4) to[C=$C_{{1},{2}}$] (4,0); % 右側
         \draw(2,0) to[european resistor=$R_{2}$,-*] (2,2) to[european resistor=$R_{1}$] (2,4); % 中央縦
%         \draw(2,2) to[nos=$S_{2}$] (4,2); % 上部

%         \node[left] at (-0.5,2) {$E$};
         \node[above] at (2,4) {$a$};
         \node[below] at (2,0) {$b$};
         \node[left] at (2,2) {$c$};
%         \node[right] at (4,2) {$d$};
        \end{circuitikz}
       \end{center}
      \end{minipage}
      %%%%%%%%%%%%%%%%%%%%%%%%%%%%%%%%%%%

      \vspace{10pt}

      回路の静電エネルギーは
      コンデンサの電圧$V$と電気量$Q$を用いて
      $\frac{1}{2}QV$と表される。
      回路にはコンデンサが繋がれているだけであるから、
      2つの抵抗から発生するジュール熱の合計も
      $\frac{1}{2}QV$と等しくなる。

      今、コンデンサ$C_{1,2}$に蓄えられた電気量は
      $\frac{2}{3}CE$であり、
      直前まで繋がれていた起電力$E$がコンデンサの電位差となる。
      コンデンサの静電エネルギーは次のように求められる。
      \begin{equation}
       \frac{1}{2} \times \frac{2}{3}CE \times E
        = \frac{1}{3}CE^{2}
        \label{eq:electrostatic_energy}
      \end{equation}


      2つの抵抗値は$R,2R$であり、
      これにより電圧はそれぞれ$\frac{1}{3},\frac{2}{3}$がかかる。

      $R_{2}$に掛かる電圧が全体の$\frac{2}{3}$であるから
      式\eqref{eq:electrostatic_energy}
      の電位差$E$を$\frac{2}{3}E$に置き換えると
      $R_{2}$で発生するジュール熱が求まる。

      \begin{equation}
       P
        = \frac{1}{2} \times \frac{2}{3}CE \times \frac{2}{3}E
        = \frac{2}{9}CE^{2}
      \end{equation}
\end{enumerate}


\hrulefill



\end{document}
