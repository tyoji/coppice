\documentclass[12pt,b5paper]{ltjsarticle}

%\usepackage[margin=15truemm, top=5truemm, bottom=5truemm]{geometry}
%\usepackage[margin=10truemm,left=15truemm]{geometry}
\usepackage[margin=10truemm]{geometry}

\usepackage{amsmath,amssymb}
%\pagestyle{headings}
\pagestyle{empty}

%\usepackage{listings,url}
%\renewcommand{\theenumi}{(\arabic{enumi})}

%\usepackage{graphicx}

%\usepackage{tikz}
%\usetikzlibrary {arrows.meta}
%\usepackage{wrapfig}	% required for `\wrapfigure' (yatex added)
%\usepackage{bm}	% required for `\bm' (yatex added)

% ルビを振る
%\usepackage{luatexja-ruby}	% required for `\ruby'

%% 核Ker 像Im Hom を定義
%\newcommand{\Img}{\mathop{\mathrm{Im}}\nolimits}
%\newcommand{\Ker}{\mathop{\mathrm{Ker}}\nolimits}
%\newcommand{\Hom}{\mathop{\mathrm{Hom}}\nolimits}

%\DeclareMathOperator{\Rot}{rot}
%\DeclareMathOperator{\Div}{div}
%\DeclareMathOperator{\Grad}{grad}
%\DeclareMathOperator{\arcsinh}{arcsinh}
%\DeclareMathOperator{\arccosh}{arccosh}
%\DeclareMathOperator{\arctanh}{arctanh}



%\usepackage{listings,url}
%
%\lstset{
%%プログラム言語(複数の言語に対応,C,C++も可)
%  language = Python,
%%  language = Lisp,
%%  language = C,
%  %背景色と透過度
%  %backgroundcolor={\color[gray]{.90}},
%  %枠外に行った時の自動改行
%  breaklines = true,
%  %自動改行後のインデント量(デフォルトでは20[pt])
%  breakindent = 10pt,
%  %標準の書体
%%  basicstyle = \ttfamily\scriptsize,
%  basicstyle = \ttfamily,
%  %コメントの書体
%%  commentstyle = {\itshape \color[cmyk]{1,0.4,1,0}},
%  %関数名等の色の設定
%  classoffset = 0,
%  %キーワード(int, ifなど)の書体
%%  keywordstyle = {\bfseries \color[cmyk]{0,1,0,0}},
%  %表示する文字の書体
%  %stringstyle = {\ttfamily \color[rgb]{0,0,1}},
%  %枠 "t"は上に線を記載, "T"は上に二重線を記載
%  %他オプション:leftline,topline,bottomline,lines,single,shadowbox
%  frame = TBrl,
%  %frameまでの間隔(行番号とプログラムの間)
%  framesep = 5pt,
%  %行番号の位置
%  numbers = left,
%  %行番号の間隔
%  stepnumber = 1,
%  %行番号の書体
%%  numberstyle = \tiny,
%  %タブの大きさ
%  tabsize = 4,
%  %キャプションの場所("tb"ならば上下両方に記載)
%  captionpos = t
%}



\begin{document}

\hrulefill

命題$K$ : 王の状態が眠っている

命題$Q$ : 女王の状態が眠っている

$K,Q$はそれぞれ 0なら起きている、1なら寝ている


論理式$X$ : 王が信じていること

正しければ1、間違っていれば0


\dotfill

$X=(K=0)$は
「王が(王が起きている)と信じている」
のことであり、王が起きていても寝ていても真である。
$X=(K=0)=1$

$X=(K=1)$は
「王が(王が寝ている)と信じている」
のことであり、王が起きていても寝ていても偽である。
$X=(K=1)=0$


\dotfill

王は王も女王も眠っていると信じている。
\begin{equation}
 X=(K=1 \land Q=1)
\end{equation}

この式が真であれば$X=(K=1 \land Q=1)=1$である。
つまり、$(K=1)=1$かつ$(Q=1)=1$であり、
二人は寝ている事となる。


偽であれば$X=(K=1 \land Q=1)=0$である。
つまり、否定された式$\lnot(K=1 \land Q=1)$が真となる。
\begin{equation}
 \lnot(K=1 \land Q=1)
  = \lnot(K=1) \lor \lnot(Q=1)
  = (K=0) \lor (Q=0)
\end{equation}
これは、「王が起きているまたは女王が起きている」
ことを意味する。


王が起きている$K=0$なら
$X=1$であるので$(K=1)=1$かつ$(Q=1)=1$である。
つまり、$K=0$ならば$K=1$となり矛盾する。


王が寝ている$K=1$なら
$X=0$であるので$(K=0)=1$または$(Q=0)=1$である。

$K=1$ならば$(K=0)=1$または$(Q=0)=1$であるので、
$(K=1)=1$かつ$(Q=0)=1$の時において矛盾がない。

つまり、女王は起きている$Q=0$ことを意味する。

\hrulefill


\end{document}
