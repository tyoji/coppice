\documentclass[12pt,b5paper]{ltjsarticle}

%\usepackage[margin=15truemm, top=5truemm, bottom=5truemm]{geometry}
\usepackage[margin=10truemm]{geometry}

\usepackage{amsmath,amssymb}
%\pagestyle{headings}
\pagestyle{empty}

%\usepackage{listings,url}
%\renewcommand{\theenumi}{(\arabic{enumi})}

%\usepackage{graphicx}

%\usepackage{tikz}
%\usetikzlibrary {arrows.meta}
%\usepackage{wrapfig}	% required for `\wrapfigure' (yatex added)
%\usepackage{bm}	% required for `\bm' (yatex added)

% ルビを振る
\usepackage{luatexja-ruby}	% required for `\ruby'

%% 核Ker 像Im Hom を定義
%\newcommand{\Img}{\mathop{\mathrm{Im}}\nolimits}
%\newcommand{\Ker}{\mathop{\mathrm{Ker}}\nolimits}
%\newcommand{\Hom}{\mathop{\mathrm{Hom}}\nolimits}

%\DeclareMathOperator{\Rot}{rot}
%\DeclareMathOperator{\Div}{div}
%\DeclareMathOperator{\Grad}{grad}
%\DeclareMathOperator{\arcsinh}{arcsinh}
%\DeclareMathOperator{\arccosh}{arccosh}
%\DeclareMathOperator{\arctanh}{arctanh}



%\usepackage{listings,url}
%
%\lstset{
%%プログラム言語(複数の言語に対応,C,C++も可)
%  language = Python,
%%  language = Lisp,
%%  language = C,
%  %背景色と透過度
%  %backgroundcolor={\color[gray]{.90}},
%  %枠外に行った時の自動改行
%  breaklines = true,
%  %自動改行後のインデント量(デフォルトでは20[pt])
%  breakindent = 10pt,
%  %標準の書体
%%  basicstyle = \ttfamily\scriptsize,
%  basicstyle = \ttfamily,
%  %コメントの書体
%%  commentstyle = {\itshape \color[cmyk]{1,0.4,1,0}},
%  %関数名等の色の設定
%  classoffset = 0,
%  %キーワード(int, ifなど)の書体
%%  keywordstyle = {\bfseries \color[cmyk]{0,1,0,0}},
%  %表示する文字の書体
%  %stringstyle = {\ttfamily \color[rgb]{0,0,1}},
%  %枠 "t"は上に線を記載, "T"は上に二重線を記載
%  %他オプション:leftline,topline,bottomline,lines,single,shadowbox
%  frame = TBrl,
%  %frameまでの間隔(行番号とプログラムの間)
%  framesep = 5pt,
%  %行番号の位置
%  numbers = left,
%  %行番号の間隔
%  stepnumber = 1,
%  %行番号の書体
%%  numberstyle = \tiny,
%  %タブの大きさ
%  tabsize = 4,
%  %キャプションの場所("tb"ならば上下両方に記載)
%  captionpos = t
%}



\begin{document}

\hrulefill
\textbf{問題}
\hrulefill


\begin{enumerate}
 \item
      \begin{enumerate}
       \item
            距離空間と位相空間の定義、性質についてまとめ、
            その空間の違いについて述べよ。
            両者において開集合の定義を必ず含むこと。

            \dotfill

            \textbf{距離空間}

            空間$X$上に距離関数$d$が定義された
            空間を距離空間といい、
            $(X,d)$と書く。
            距離関数$d$とは、
            関数$d: X\times X \to \mathbb{R}$
            が次の条件を満たす時をいう。
            \begin{gather}
             d(x,y) \geq 0, \quad d(x,y)=0 \Leftrightarrow x=y\\
             d(x,y) = d(y,x), \quad d(x,z) \leq d(x,y)+d(y,z)
            \end{gather}


            \textbf{位相空間}

            空間$X$に次を満たす集合族$\mathcal{O}$が定義出来る時、
            その組$(X,\mathcal{O})$を位相空間という。
            \begin{gather}
             \emptyset, X \in\mathcal{O}\\
             O_{1},O_{2}\in\mathcal{O} \Rightarrow O_{1}\cap O_{2}\in\mathcal{O}\\
             O_{\lambda}\in\mathcal{O} \Rightarrow \bigcup_{\lambda\in\Lambda}O_{\lambda}\in\mathcal{O}
            \end{gather}

            開集合の集合族$\mathcal{O}$を位相といい、
            どのような空間にも位相を定めることが出来る。
            例えば、
            $\mathcal{O}=\{\emptyset,X\}$の場合は密着位相、
            $\mathcal{O}$が$X$の全ての部分集合の場合は離散位相
            というように何かの位相を導入することは出来る。

            距離空間であれば、距離関数を利用し
            $\epsilon$-近傍$B_{d}(x,\epsilon)$が定義でき、
            次を満たす部分集合$U\subset X$を開集合と定義できる。
            \begin{equation}
             {}^{\forall}x\in U, {}^{\exists}\epsilon>0
              \text{ s.t. } B_{d}(x,\epsilon) \subset U
            \end{equation}
            距離関数から誘導された位相が必ず定義出来るので
            距離空間は位相空間として扱うことが出来る。

            逆に位相空間は距離関数が定義できない場合もある為、
            距離空間となるわけではない。
            \begin{equation}
             \text{距離空間 } \Rightarrow \text{ 位相空間}
            \end{equation}


            \hrulefill

       \item
            距離化可能ではない空間とはどのような空間か?
            例を挙げ、
            それがなぜ距離化できないのか
            証明せよ。

            \dotfill

            \textbf{距離化可能でない空間の例}

            ハウフドルフ空間でない位相(
            \ruby{Zariski}{ザリスキ}位相、
            密着位相 等)

            \textbf{距離化できない理由}

            空間$X=\{a,b\}$に対し
            密着位相$\mathcal{O}=\{\emptyset,X\}$
            を導入する。

            $a$を含む開集合は全て$b$も含むことになるから
            これを誘導する距離関数は
            $a$の近傍に$b$も含めないといけない。
            つまり$d(a,b)=0$を満たさないといけない。
            しかし、距離関数は$d(x,y)=0 \Leftrightarrow x=y$
            を満たす必要があるので
            $a\ne b$に対し$d(a,b)=0$となる関数$d$は距離関数ではない。

            よって、
            $X=\{a,b\}$に密着位相を誘導する距離関数は存在しない。

            \hrulefill

      \end{enumerate}


 \item
      位相空間$(X,\mathcal{O})$において
      $A\subset X$の内部$A^{\circ}$、
      閉包$\overline{A}$を
      それぞれ
      $A$に包まれる最大の開集合、
      $A$を包む最小の閉集合
      として定義する。
      この2つの定義は、
      $(X,\mathcal{O})$が
      距離空間$(X,d)$からくる
      距離位相空間$(X,\mathcal{O}_{d})$と同値な場合、
      授業内に行った$\epsilon$-近傍を用いた定義と
      同じものであることを確かめよ。

      \dotfill

      距離空間$(X,d)$の$\epsilon$-近傍$B_{d}(x,\epsilon)$を
      次のように定義する。
      \begin{equation}
       B_{d}(x,\epsilon)=\{y \mid d(x,y)<\epsilon\}
      \end{equation}
      この近傍を用いて、$(X,d)$の開集合$U$を次のように定義する。
      \begin{equation}
       {}^{\forall}x\in U, {}^{\exists}\epsilon>0
        \text{ s.t. }
        B_{d}(x,\epsilon) \subset U
        \label{def_openset}
      \end{equation}

      この定義(\ref{def_openset})は
      内部の定義と同じである。
      \begin{equation}
       A^{\circ}=\{
        a\in A \mid a\text{の開球で}A\text{に含まれるものが存在}
        \}
        \label{def_int}
      \end{equation}

%      $A^{\circ}$に対して$\overline{A^{c}}$が
%      $A^{\circ}\cap\overline{A^{c}}=\emptyset$、
%      $A^{\circ}\cup\overline{A^{c}}=X$
%      であれば開集合の補集合が閉集合となり
%      $(X,\mathcal{O}_d)$の閉集合と一致する。

      また、補集合$A^{c}$の閉包$\overline{A^{c}}$は次のような集合である。
      \begin{equation}
       \overline{A^{c}}=\{
        a\in X \mid a\text{の任意の開球と}A^{c}\text{との共通部分が空でない}\}
      \end{equation}

      $(A^{\circ})^{c}$は(\ref{def_int})より
      $x\in X$の開球$B(x)$で$B(x)\subset A$となるものが
      存在しない点を集めたものである。
      これは$x\in X$の任意の開球$B(x)$で
      $B(x)\cap A^{c}\ne\emptyset$
      となることを意味する。
      つまり、$(A^{\circ})^{c}=\overline{A^{c}}$より
      $(A^{\circ})^{c}$は閉集合である。

      $(X,\mathcal{O})$と$(X,\mathcal{O}_d)$の開集合が一致し、
      それぞれの開集合の補集合が閉集合になる為、
      $(X,\mathcal{O})$は$\epsilon$-近傍を用いた定義と同じものである。


      \hrulefill

 \item
      距離空間$(X,d)$とその部分集合$A\subset X$において、
      $\overline{A}=\{ x\in X \mid d(x,A)=0\}$
      であることを示せ。

      \dotfill

      $x\in X$の開球を$B(x) \subset X$と表すと
      閉包$\overline{A}$の定義は次のようになる。
      \begin{equation}
       \overline{A}=\{
        x\in X \mid {}^{\forall}B(x)\subset X
        \text{ に対して } B(x)\cap A \ne\emptyset
        \}
        \label{def_clo}
      \end{equation}

      また、点と集合の距離$d(x,A)$は次のように定義される。
      \begin{equation}
       d(x,A)= \min \{ d(x,a) \mid a\in A \}
      \end{equation}

      次を示す。
      \begin{equation}
       \overline{A} \subset \{ x\in X \mid d(x,A)=0\}
      \end{equation}


      $\bar{a}\in \overline{A}$とする。
      定義(\ref{def_clo})より
      $\bar{a}$の任意の開球$B(\bar{a})$において
      $B(\bar{a})\cap A\ne\emptyset$である。
      距離空間$(X,d)$での開球$B(\bar{a})$は次のような集合である。
      \begin{equation}
       B(\bar{a}) = \{ x\in X \mid d(\bar{a},x)<\varepsilon \}
      \end{equation}

      $a^{\prime}\in B(\bar{a})\cap A$を取ってくると、
      $0\leq d(\bar{a},a^{\prime})<\varepsilon$である。
      任意の開球について$B(\bar{a})\cap A\ne\emptyset$であるので、
      $\varepsilon$はどれだけ$0$に近づけてもよい。
      $\varepsilon \rightarrow 0$とし、
      $\varepsilon$を取り直すごとに$a^{\prime}$も取り直すと
      $0\leq d(\bar{a},a^{\prime})<\varepsilon$から
      $d(\bar{a},a^{\prime})=0$が得られる。
      $a^{\prime}\in A$より
      $d(\bar{a},A)=0$となり、
      $\overline{A} \subset \{ x\in X \mid d(x,A)=0\}$
      であることがわかる。





      次に逆を示す。
      \begin{equation}
       \overline{A} \supset \{ x\in X \mid d(x,A)=0\}
      \end{equation}

      $\alpha \in \{ x\in X \mid d(x,A)=0\}$とする。
      $d(\alpha,A)=0$は${}^{\exists}a\in A$について
      $d(\alpha,a)=0$である。
      $d$は距離関数であるので、
      $d(\alpha,a)=0 \Leftrightarrow \alpha=a$
      である。
      $\alpha$の近傍を$B_{\varepsilon}(\alpha)$、
      $a$の近傍を$B_{\delta}(a)$とすれば、
      それぞれ任意の$\varepsilon,\delta$について
      $B_{\varepsilon}(\alpha)\cap B_{\delta}(a)\ne\emptyset$
      である。
      $B_{\delta}(a)\subset A$となる$\delta$を選部ことが出来るので、
      $B_{\varepsilon}(\alpha)\cap A \ne\emptyset$となる。

      よって、
      $\alpha\in \overline{A}$となり、
      $\overline{A} \supset \{ x\in X \mid d(x,A)=0\}$
      であることが示せる。


      以上により
      $\overline{A} = \{ x\in X \mid d(x,A)=0\}$
      である。


      \hrulefill

\end{enumerate}


\end{document}
