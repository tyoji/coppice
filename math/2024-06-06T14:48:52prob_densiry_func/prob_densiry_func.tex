\documentclass[12pt,b5paper]{ltjsarticle}

%\usepackage[margin=15truemm, top=5truemm, bottom=5truemm]{geometry}
%\usepackage[margin=10truemm,left=15truemm]{geometry}
\usepackage[margin=10truemm]{geometry}

\usepackage{amsmath,amssymb}
%\pagestyle{headings}
\pagestyle{empty}

%\usepackage{listings,url}
\renewcommand{\theenumi}{(\arabic{enumi})}

%\usepackage{graphicx}

%\usepackage{tikz}
%\usetikzlibrary {arrows.meta}
%\usepackage{wrapfig}
%\usepackage{bm}

% ルビを振る
%\usepackage{luatexja-ruby}	% required for `\ruby'

%% 核Ker 像Im Hom を定義
%\newcommand{\Img}{\mathop{\mathrm{Im}}\nolimits}
%\newcommand{\Ker}{\mathop{\mathrm{Ker}}\nolimits}
%\newcommand{\Hom}{\mathop{\mathrm{Hom}}\nolimits}

%\DeclareMathOperator{\Rot}{rot}
%\DeclareMathOperator{\Div}{div}
%\DeclareMathOperator{\Grad}{grad}
%\DeclareMathOperator{\arcsinh}{arcsinh}
%\DeclareMathOperator{\arccosh}{arccosh}
%\DeclareMathOperator{\arctanh}{arctanh}



%\usepackage{listings,url}
%
%\lstset{
%%プログラム言語(複数の言語に対応,C,C++も可)
%  language = Python,
%%  language = Lisp,
%%  language = C,
%  %背景色と透過度
%  %backgroundcolor={\color[gray]{.90}},
%  %枠外に行った時の自動改行
%  breaklines = true,
%  %自動改行後のインデント量(デフォルトでは20[pt])
%  breakindent = 10pt,
%  %標準の書体
%%  basicstyle = \ttfamily\scriptsize,
%  basicstyle = \ttfamily,
%  %コメントの書体
%%  commentstyle = {\itshape \color[cmyk]{1,0.4,1,0}},
%  %関数名等の色の設定
%  classoffset = 0,
%  %キーワード(int, ifなど)の書体
%%  keywordstyle = {\bfseries \color[cmyk]{0,1,0,0}},
%  %表示する文字の書体
%  %stringstyle = {\ttfamily \color[rgb]{0,0,1}},
%  %枠 "t"は上に線を記載, "T"は上に二重線を記載
%  %他オプション:leftline,topline,bottomline,lines,single,shadowbox
%  frame = TBrl,
%  %frameまでの間隔(行番号とプログラムの間)
%  framesep = 5pt,
%  %行番号の位置
%  numbers = left,
%  %行番号の間隔
%  stepnumber = 1,
%  %行番号の書体
%%  numberstyle = \tiny,
%  %タブの大きさ
%  tabsize = 4,
%  %キャプションの場所("tb"ならば上下両方に記載)
%  captionpos = t
%}



\begin{document}

$X$は連続分布に従う確率変数で、
その密度関数$\rho(x)$で与えられるとする。
\begin{equation}
 \rho(x) = C\frac{1}{(1+\lvert x-1 \rvert)^{5}}
  ,\qquad (Cは正定数)
\end{equation}

\begin{enumerate}
 \item
      定数$C$を計算せよ。

 \item
      $X$の期待値と分散を計算せよ。

 \item
      $\lvert X \rvert \leq 1$となる確率を計算せよ。
\end{enumerate}

\hrulefill

\begin{enumerate}
 \item
      定数$C$を計算せよ。

      \dotfill

      確率密度関数は実数全体で積分すると1になる。
      \begin{align}
       \int_{-\infty}^{\infty} \rho(x) \mathrm{d}x
        &= \int_{-\infty}^{1} C\frac{1}{(1+\lvert x-1 \rvert)^{5}} \mathrm{d}x
        + \int_{1}^{\infty} C\frac{1}{(1+\lvert x-1 \rvert)^{5}} \mathrm{d}x\\
        &= \int_{-\infty}^{1} C\frac{1}{(2-x)^{5}} \mathrm{d}x
        + \int_{1}^{\infty} C\frac{1}{x^{5}} \mathrm{d}x\\
       \int_{-\infty}^{1} \frac{1}{(2-x)^{5}} \mathrm{d}x
        &= \left[ \frac{1}{4}(2-x)^{-4} \right]_{x=-\infty}^{x=1} = \frac{1}{4}\\
       \int_{1}^{\infty} \frac{1}{x^{5}} \mathrm{d}x
        &= \left[ \frac{1}{-4}x^{-4} \right]_{x=1}^{x=\infty} = \frac{1}{4}\\
       \int_{-\infty}^{\infty} \rho(x) \mathrm{d}x
        &= \frac{C}{4} + \frac{C}{4} = \frac{C}{2}
      \end{align}

      よって、$C=2$である。


 \item
      $X$の期待値と分散を計算せよ。

      \dotfill

      \textbf{期待値}

      期待値$E[X]$は$x\rho(x)$を積分することで得られる。
      \begin{equation}
       E[X] = \int_{-\infty}^{\infty} x\rho(x) \mathrm{d}x
       = \int_{-\infty}^{1} \frac{2x}{(2-x)^{5}} \mathrm{d}x
        + \int_{1}^{\infty} \frac{2x}{x^{5}} \mathrm{d}x
      \end{equation}

      後半部分は次のようになる。
      \begin{equation}
       \int_{1}^{\infty} \frac{2x}{x^{5}} \mathrm{d}x
        = \left[ \frac{2}{-3}x^{-3} \right]_{x=1}^{x=\infty}
        = \frac{2}{3}
      \end{equation}

      前半部分は$t=2-x$と置き、置換積分を行う。
      \begin{align}
       \int_{-\infty}^{1} \frac{2x}{(2-x)^{5}} \mathrm{d}x
        &= \int_{\infty}^{1} \frac{2(2-t)}{t^{5}} (-\mathrm{d}t)
        = \int_{1}^{\infty} \left( 4t^{-5} -2t^{-4} \right) \mathrm{d}t\\
        &= \left[ \frac{4}{-4}t^{-4} -\frac{2}{-3}t^{-3}  \right]_{t=1}^{t=\infty}
        = \frac{1}{3}
      \end{align}

      よって、次のように期待値$E[X]$は1となる。
      \begin{equation}
       E[X] = \int_{-\infty}^{\infty} x\rho(x) \mathrm{d}x = 1
      \end{equation}

      \textbf{分散}

      分散$V[X]$は$E[(X-E[X])^{2}]$で定義される。
      つまり、
      $V[X]=E[X^{2}]-\left(E[X]\right)^{2}$である。
      先ほど$E[X]=1$であることがわかったので、
      $E[X^{2}]$を計算する。

      \begin{equation}
       E[X^{2}]
        = \int_{-\infty}^{\infty} x^{2}\rho(x) \mathrm{d}x
        = \int_{-\infty}^{1} \frac{2x^{2}}{(2-x)^{5}} \mathrm{d}x
        + \int_{1}^{\infty} \frac{2x^{2}}{x^{5}} \mathrm{d}x
      \end{equation}

      後半部分を計算する。
      \begin{equation}
       \int_{1}^{\infty} \frac{2x^{2}}{x^{5}} \mathrm{d}x
        = \int_{1}^{\infty} 2x^{-3} \mathrm{d}x
        = \left[\frac{2}{-2}x^{-2}\right]_{x=1}^{x=\infty}
        = 1
      \end{equation}

      前半部分は$t=2-x$として、置換積分を行う。
      \begin{align}
       \int_{-\infty}^{1} \frac{2x^{2}}{(2-x)^{5}} \mathrm{d}x
        &= \int_{\infty}^{1} \frac{2(2-t)^{2}}{t^{5}} (-\mathrm{d}t)
        = \int_{1}^{\infty} \left( 8t^{-5} -8t^{-4}+2t^{-3} \right) \mathrm{d}t\\
        &= \left[ \frac{8}{-4}t^{-4} -\frac{8}{-3}t^{-3} + \frac{2}{-2}t^{-2} \right]_{t=1}^{t=\infty}
        = \frac{1}{3}
      \end{align}

      これにより$E[X^{2}]=4/3$であるので、
      分散$V[X]$は次のように求まる。
      \begin{equation}
       V[X] = E[X^{2}]-(E[X])^{2} = \frac{4}{3} -1^{2} = \frac{1}{3}
      \end{equation}


 \item
      $\lvert X \rvert \leq 1$となる確率を計算せよ。

      \dotfill

      $\lvert X \rvert \leq 1$となる確率は密度関数$\rho(x)$を
      $\lvert x \rvert \leq1$の範囲で積分することで求まる。
      $t=2-x$として置換積分を用いて計算する。
      \begin{align}
       \int_{-1}^{1} \rho(x) \mathrm{d}x
        &= \int_{-1}^{1} \frac{2}{(2-x)^{5}} \mathrm{d}x
        = \int_{3}^{1} \frac{2}{t^{5}} (-\mathrm{d}t)\\
        &= \int_{1}^{3} 2t^{-5} \mathrm{d}t
        = \left[ \frac{2}{-4}t^{-4} \right]_{t=1}^{t=3}
        = \frac{40}{81}%\\
%        &= \left( \frac{2}{-4}3^{-4} \right) - \left( \frac{2}{-4}1^{-4} \right)
      \end{align}

      よって、確率は
      $P(\lvert X \rvert \leq 1) = \frac{40}{81}$
      である。
      
\end{enumerate}



\hrulefill

\end{document}
