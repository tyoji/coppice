\documentclass[12pt,b5paper]{ltjsarticle}

%\usepackage[margin=15truemm, top=5truemm, bottom=5truemm]{geometry}
%\usepackage[margin=10truemm,left=15truemm]{geometry}
\usepackage[margin=10truemm]{geometry}

\usepackage{amsmath,amssymb}
%\pagestyle{headings}
\pagestyle{empty}

%\usepackage{listings,url}
%\renewcommand{\theenumi}{(\arabic{enumi})}

%\usepackage{graphicx}

%\usepackage{tikz}
%\usetikzlibrary {arrows.meta}
%\usepackage{wrapfig}
%\usepackage{bm}

% ルビを振る
\usepackage{luatexja-ruby}	% required for `\ruby'

%% 核Ker 像Im Hom を定義
%\newcommand{\Img}{\mathop{\mathrm{Im}}\nolimits}
%\newcommand{\Ker}{\mathop{\mathrm{Ker}}\nolimits}
%\newcommand{\Hom}{\mathop{\mathrm{Hom}}\nolimits}

%\DeclareMathOperator{\Rot}{rot}
%\DeclareMathOperator{\Div}{div}
%\DeclareMathOperator{\Grad}{grad}
%\DeclareMathOperator{\arcsinh}{arcsinh}
%\DeclareMathOperator{\arccosh}{arccosh}
%\DeclareMathOperator{\arctanh}{arctanh}

\usepackage{url}

%\usepackage{listings}
%
%\lstset{
%%プログラム言語(複数の言語に対応,C,C++も可)
%  language = Python,
%%  language = Lisp,
%%  language = C,
%  %背景色と透過度
%  %backgroundcolor={\color[gray]{.90}},
%  %枠外に行った時の自動改行
%  breaklines = true,
%  %自動改行後のインデント量(デフォルトでは20[pt])
%  breakindent = 10pt,
%  %標準の書体
%%  basicstyle = \ttfamily\scriptsize,
%  basicstyle = \ttfamily,
%  %コメントの書体
%%  commentstyle = {\itshape \color[cmyk]{1,0.4,1,0}},
%  %関数名等の色の設定
%  classoffset = 0,
%  %キーワード(int, ifなど)の書体
%%  keywordstyle = {\bfseries \color[cmyk]{0,1,0,0}},
%  %表示する文字の書体
%  %stringstyle = {\ttfamily \color[rgb]{0,0,1}},
%  %枠 "t"は上に線を記載, "T"は上に二重線を記載
%  %他オプション:leftline,topline,bottomline,lines,single,shadowbox
%  frame = TBrl,
%  %frameまでの間隔(行番号とプログラムの間)
%  framesep = 5pt,
%  %行番号の位置
%  numbers = left,
%  %行番号の間隔
%  stepnumber = 1,
%  %行番号の書体
%%  numberstyle = \tiny,
%  %タブの大きさ
%  tabsize = 4,
%  %キャプションの場所("tb"ならば上下両方に記載)
%  captionpos = t
%}

%\usepackage{cancel}
%\usepackage{bussproofs}
%\usepackage{proof}

\begin{document}

\hrulefill

\textbf{冪零元}

環$R$の元$x\in R$が冪零元であるとは、
ある自然数$n$が存在し、
$x^{n}=0$となることをいう。

\hrulefill

\begin{enumerate}
 \item
      $(A, +, \;\cdot\;)$を
      可換環とする。
      $\mathcal{N} (A) = \{ a \in A \mid a が冪零元である \}$
      (冪零根基と呼ぶ)と定義する。
      \begin{enumerate}
       \item
            $a, b, k, \ell \in \mathbb{Z}_{\geq 1}$とする。
            $n = a^{k}b^{\ell}$であるとき、
            $ab + n\mathbb{Z} \in \mathcal{N}(\mathbb{Z}/n\mathbb{Z})$
            であることを示せ。

            \dotfill

            $k$と$\ell$の最小公倍数を$m$とする。
            つまり、$m=km_{k}=\ell m_{\ell}$となる
            自然数$m_{k},m_{\ell}$が存在する。
            \begin{equation}
             (ab)^{m} = (a^{k})^{m_{k}}(b^{\ell})^{m_{\ell}}
              = a^{k}b^{\ell} \cdot (a^{k})^{m_{k}-1}(b^{\ell})^{m_{\ell}-1}
              = n (a^{k})^{m_{k}-1}(b^{\ell})^{m_{\ell}-1}
            \end{equation}
            $(ab)^{m}$は$n$の倍数であるので、
            $\left(ab + n \mathbb{Z}\right)^{m}$も$n$の倍数となる。

            $\left(ab + n \mathbb{Z}\right)^{m} \in n\mathbb{Z}$
            であるから
            $\mathbb{Z}/n\mathbb{Z}$上で
            $\left(ab + n \mathbb{Z}\right)^{m}=0$である。

            よって、
            $ab + n\mathbb{Z} \in \mathcal{N}(\mathbb{Z}/n\mathbb{Z})$
            である。

            \hrulefill

       \item
            $\mathcal{N}(\mathbb{Z}/72\mathbb{Z})$を求めよ。

            \dotfill

            $\mathbb{Z}/72\mathbb{Z}$は次のような環である。
            
            \begin{equation}
             \mathbb{Z}/72\mathbb{Z}
              =\{\bar{0},\bar{1},\bar{2},\dots,\bar{71}\}
            \end{equation}

            この環では$72$の倍数は$0$となる。

            $72 = 2^{3}\cdot 3^{2}$であるので、
            $1,\dots,71$の内、$2$と$3$の因数のみで出来た数は
            何度かかけると$72$の倍数になる。

            \begin{align}
             2^{1} \times 3^{1} =& 6 &
             2^{2} \times 3^{1} =& 12 &
             2^{3} \times 3^{1} =& 24 &
             2^{4} \times 3^{1} =& 48 \\
             2^{1} \times 3^{2} =& 18 &
             2^{2} \times 3^{2} =& 36 \\
             2^{1} \times 3^{3} =& 54
            \end{align}

            よって、
            $\mathcal{N}(\mathbb{Z}/72\mathbb{Z})$は
            上記7つと$\bar{0}$を加えた8つの元を持つことがわかる。

            \begin{equation}
             \mathcal{N}(\mathbb{Z}/72\mathbb{Z})
              =\{
              \bar{0},
              \bar{6},\bar{12},
              \bar{18},\bar{24},
              \bar{36},\bar{48},
              \bar{54}
              \}
            \end{equation}


            \hrulefill

      \end{enumerate}
\end{enumerate}

\hrulefill

\end{document}
