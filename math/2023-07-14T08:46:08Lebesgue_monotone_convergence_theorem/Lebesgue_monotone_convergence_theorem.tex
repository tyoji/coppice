\documentclass[12pt,b5paper]{ltjsarticle}

%\usepackage[margin=15truemm, top=5truemm, bottom=5truemm]{geometry}
%\usepackage[margin=10truemm,left=15truemm]{geometry}
\usepackage[margin=10truemm]{geometry}

\usepackage{amsmath,amssymb}
%\pagestyle{headings}
\pagestyle{empty}

%\usepackage{listings,url}
%\renewcommand{\theenumi}{(\arabic{enumi})}

%\usepackage{graphicx}

%\usepackage{tikz}
%\usetikzlibrary {arrows.meta}
%\usepackage{wrapfig}
%\usepackage{bm}

% ルビを振る
\usepackage{luatexja-ruby}	% required for `\ruby'

%% 核Ker 像Im Hom を定義
%\newcommand{\Img}{\mathop{\mathrm{Im}}\nolimits}
%\newcommand{\Ker}{\mathop{\mathrm{Ker}}\nolimits}
%\newcommand{\Hom}{\mathop{\mathrm{Hom}}\nolimits}

%\DeclareMathOperator{\Rot}{rot}
%\DeclareMathOperator{\Div}{div}
%\DeclareMathOperator{\Grad}{grad}
%\DeclareMathOperator{\arcsinh}{arcsinh}
%\DeclareMathOperator{\arccosh}{arccosh}
%\DeclareMathOperator{\arctanh}{arctanh}

\usepackage{url}

%\usepackage{listings}
%
%\lstset{
%%プログラム言語(複数の言語に対応,C,C++も可)
%  language = Python,
%%  language = Lisp,
%%  language = C,
%  %背景色と透過度
%  %backgroundcolor={\color[gray]{.90}},
%  %枠外に行った時の自動改行
%  breaklines = true,
%  %自動改行後のインデント量(デフォルトでは20[pt])
%  breakindent = 10pt,
%  %標準の書体
%%  basicstyle = \ttfamily\scriptsize,
%  basicstyle = \ttfamily,
%  %コメントの書体
%%  commentstyle = {\itshape \color[cmyk]{1,0.4,1,0}},
%  %関数名等の色の設定
%  classoffset = 0,
%  %キーワード(int, ifなど)の書体
%%  keywordstyle = {\bfseries \color[cmyk]{0,1,0,0}},
%  %表示する文字の書体
%  %stringstyle = {\ttfamily \color[rgb]{0,0,1}},
%  %枠 "t"は上に線を記載, "T"は上に二重線を記載
%  %他オプション:leftline,topline,bottomline,lines,single,shadowbox
%  frame = TBrl,
%  %frameまでの間隔(行番号とプログラムの間)
%  framesep = 5pt,
%  %行番号の位置
%  numbers = left,
%  %行番号の間隔
%  stepnumber = 1,
%  %行番号の書体
%%  numberstyle = \tiny,
%  %タブの大きさ
%  tabsize = 4,
%  %キャプションの場所("tb"ならば上下両方に記載)
%  captionpos = t
%}



\begin{document}

\hrulefill

\begin{enumerate}
 \item
      $(X,\mathcal{M})$を可測空間とし、
      $A\in\mathcal{M}$とする。

      \begin{enumerate}
       \item
            $f:X \to [0,\infty]$は$\mathcal{M}$-可測であるとし、
            $f$を$A$に制限して得られる関数を
            $f|_{A}:A \to [0,\infty]$とかく。
            このとき、
            $f|_{A}$は
            $\mathcal{M}|_{A}$-可測であることを示せ。

            \dotfill

            ${}^{\forall}U\in \mathcal{B}([0,\infty])$に対して、
            $f^{-1}(U)\in\mathcal{M}$であるとする。
            \begin{equation}
             \mathcal{M}|_{A} = \{ M\cap A \mid M\in\mathcal{M} \}
            \end{equation}

            $f|_{A}^{-1}(U) = f^{-1}(U)\cap A \in \mathcal{M}|_{A}$
            であるので、
            $f|_{A}$は
            $\mathcal{M}|_{A}$-可測である。

            \hrulefill

       \item
            $f:X \to [0,\infty)$は$\mathcal{M}$-可測な単関数であるとする。
            この時、次の式が成り立つことを示せ。
            \begin{equation}
             \int_{A} f|_{A}\;d\mu|_{A} = \int_{X} f\mathbf{1}_{A} \;d\mu
            \end{equation}

            但し、
            $\mu|_{A}$は
            $\mu$を可測空間$(A,\mathcal{M}|_{A})$に制限したものである。

            \dotfill

            
            $\{\alpha_{j}\}_{j=1}^{k} \subset \mathbb{R}$
            と
            互いに素な$\{S_{j}\}_{j=1}^{k} \subset \mathcal{M}$
            により、
            $f=\sum_{j=1}^{k}\alpha_{j}\mathbf{1}_{S_{k}}$
            と表せる。
            これにより、
            $f\mathbf{1}_{A}$は次のように表せる。
            \begin{equation}
             f\mathbf{1}_{A}
              =\sum_{j=1}^{k}\alpha_{j}\mathbf{1}_{S_{k}}\mathbf{1}_{A}
              =\sum_{j=1}^{k}\alpha_{j}\mathbf{1}_{S_{k} \cap A}
            \end{equation}

            よって、右辺の積分は次のようになる。
            \begin{equation}
             \int_{X} f\mathbf{1}_{A} \;d\mu
              = \sum_{j=1}^{k}\alpha_{j} \mu(S_{k} \cap A)
            \end{equation}

            $f$を$A$に制限した関数は
            $f|_{A}=\sum_{j=1}^{k}\alpha_{j}\mathbf{1}_{S_{k} \cap A}$
            となる。
            また、
            $\mathbf{1}_{S_{k}}$ は $A$ 以外で $0$ となるので、
            積分範囲を制限しても積分結果は変わらない。
            その為、上記積分は次のようにかける。
            \begin{equation}
             \int_{X} f\mathbf{1}_{A} \;d\mu
              = \sum_{j=1}^{k}\alpha_{j} \mu(S_{k} \cap A)
              = \int_{A} f|_{A} \; d\mu|_{A}
            \end{equation}

            \hrulefill

       \item
            $f:X \to [0,\infty]$は
            $\mathcal{M}$-可測であるとする。
            単調収束定理を用いて
            次の式が成り立つことを示せ。
            \begin{equation}
             \int_{A} f|_{A} \; d\mu|_{A} = \int_{X} f\mathbf{1}_{A} \; d\mu
            \end{equation}

            \dotfill

            \textbf{単調収束定理}

            $f_{n}:X\to [0,\infty]$が$\mathcal{M}$-可測であり、
            $f_{n}(x)\leq f_{n+1}(x)$とする。
            \begin{equation}
             \int_{X} \lim_{n\to\infty}f_{n} d\mu
              = \lim_{n\to\infty} \int_{X} f_{n} d\mu
            \end{equation}

            \dotfill

            $f=\lim_{n\to\infty}f_{n}$となる
            $\mathcal{M}$-可測関数列
            $\{ f_{n} \}_{n=1}^{\infty} (f_{i}:X\to[0,\infty))$
            が存在し、
            $f_{n} \leq f_{n+1}$を満たす。

            上の問いの結果より、
            次の式が成り立つ。
            \begin{equation}
             \int_{A} f_{n}|_{A} \; d\mu|_{A}
              = \int_{X} f_{n}\mathbf{1}_{A} \; d\mu
            \end{equation}

            単調収束定理より次の式が得られる。
            \begin{gather}
              \int_{A} \lim_{n\to\infty} f_{n}|_{A} \; d\mu|_{A}
             = \lim_{n\to\infty} \int_{A} f_{n}|_{A} \; d\mu|_{A}
              \\
              \int_{X} \lim_{n\to\infty} f_{n}\mathbf{1}_{A} \; d\mu
             =  \lim_{n\to\infty} \int_{X} f_{n}\mathbf{1}_{A} \; d\mu
            \end{gather}

            $f|_{A} = \lim_{n\to\infty} f_{n}|_{A},\;
            f\mathbf{1}_{A} = \lim_{n\to\infty} f_{n}\mathbf{1}_{A}$
            であるので、
            次が成り立つ。
            \begin{equation}
             \int_{A} f|_{A} \; d\mu|_{A}
              = \int_{X} f\mathbf{1}_{A} \; d\mu
            \end{equation}

            \hrulefill

      \end{enumerate}


 \item
      $f:\mathbb{R} \to [0,\infty)$を連続関数とし、
      $n\in \mathbb{N}$に対して、
      $f_{n}:\mathbb{R} \to [0,\infty)$を
      $f_{n}(x)=f(x)\mathbf{1}_{[-n,n]}(x)$で定義する。
      $\{ f_{n} \}_{n=1}^{\infty}$に対して
      単調収束定理を用いて
      次の式が成り立つことを示せ。
      \begin{equation}
       \int_{\mathbb{R}}f d\mu = \int_{-\infty}^{\infty}f(x)dx
      \end{equation}

      ここで、
      $\mu$は$1$次元のルベーグ測度であり、
      右辺は広義リーマン積分である。

      \dotfill


      $n\in\mathbb{N}$に対して、
      $\mathbb{R}$の部分集合を$I_{n}=[-n,n]$とおく。

      $f$は連続関数であるので、可測関数である。
      よって、次の式が成り立つ。
      \begin{equation}
       \int_{I_{n}} f|_{I_{n}} d\mu|_{I_{n}}
        = \int_{\mathbb{R}} f \mathbf{1}_{I_{n}} d \mu
      \end{equation}

      相対位相により
      $f|_{I_{n}} : I_{n}\to[0,\infty)$
      は連続写像である。
      これにより次の式が成り立つ。
      \begin{equation}
       \int_{I_{n}} f|_{I_{n}} d\mu|_{I_{n}}
        %= \int_{-n}^{n} f|_{I_{n}} dx
        = \int_{-n}^{n} f(x) dx
      \end{equation}

      この2つの式と
      $f_{n}$の定義から次の式が成り立つ。
      \begin{equation}
       \int_{\mathbb{R}} f_{n} d \mu
       = \int_{-n}^{n} f(x) dx
      \end{equation}

      両辺の極限を考える。
      \begin{equation}
       \lim_{n\to\infty}
        \int_{\mathbb{R}} f_{n} d \mu
        =
        \lim_{n\to\infty}
        \int_{-n}^{n} f(x) dx
      \end{equation}

      広義積分の定義より右辺の式は次のように表せる。
      \begin{equation}
        \lim_{n\to\infty} \int_{-n}^{n}f(x)dx
        = \int_{-\infty}^{\infty}f(x)dx
      \end{equation}


      $f_{n}=f\mathbf{1}_{I_{n}}$であり、
      $f$が可測関数であるから$f_{n}$も可測である。
      また、$f_{n}\leq f_{n+1}$である。
      この為、
      単調収束定理により
      次の式が得られる。
      \begin{equation}
       \lim_{n\to\infty} \int_{\mathbb{R}}f_{n} d\mu
        = \int_{\mathbb{R}} \lim_{n\to\infty}f_{n} d\mu
      \end{equation}


      $f=\lim_{n\to\infty}f_{n}$であるから
      右辺を置き換えると最終的に次の式が得られる。
      \begin{equation}
       \int_{\mathbb{R}} f d\mu
        = \int_{-\infty}^{\infty}f(x)dx
      \end{equation}


      \hrulefill

\end{enumerate}

\hrulefill

\end{document}
