\documentclass[12pt,b5paper]{ltjsarticle}

%\usepackage[margin=15truemm, top=5truemm, bottom=5truemm]{geometry}
\usepackage[margin=10truemm]{geometry}

\usepackage{amsmath,amssymb}
%\pagestyle{headings}
\pagestyle{empty}

%\usepackage{listings,url}
%\renewcommand{\theenumi}{(\arabic{enumi})}

%\usepackage{graphicx}

%\usepackage{tikz}
%\usetikzlibrary {arrows.meta}
%\usepackage{wrapfig}	% required for `\wrapfigure' (yatex added)
%\usepackage{bm}	% required for `\bm' (yatex added)

% ルビを振る
%\usepackage{luatexja-ruby}	% required for `\ruby'

%% 核Ker 像Im Hom を定義
%\newcommand{\Img}{\mathop{\mathrm{Im}}\nolimits}
%\newcommand{\Ker}{\mathop{\mathrm{Ker}}\nolimits}
%\newcommand{\Hom}{\mathop{\mathrm{Hom}}\nolimits}

%\DeclareMathOperator{\Rot}{rot}
%\DeclareMathOperator{\Div}{div}
%\DeclareMathOperator{\Grad}{grad}
%\DeclareMathOperator{\arcsinh}{arcsinh}
%\DeclareMathOperator{\arccosh}{arccosh}
%\DeclareMathOperator{\arctanh}{arctanh}



\begin{document}


\textbf{写像と関数}

写像と関数の厳密な区別は無いが概ね以下のように使い分けられている。

\dotfill

写像は2つの集合の間にある対応づけた関係のことである。

関数は対応先が値と呼べる集合(実数や複素数等)である写像のことである。


関数は写像の一部である。


\begin{equation}
 f:\{1,2,3,4,5\}\rightarrow \{1,2,3\}
  \qquad
 g:\{1,2,3,4,5\}\rightarrow \mathbb{R}
\end{equation}

$f,g$は写像であるが、$g$は関数で$f$は関数ではない。


\hrulefill

記号$f^{-1}$は逆写像や逆関数を表すときに使うが、
$f$に対して逆写像が存在する場合にしかつかわない。


一般的に写像$f$に対して$f^{-1}$の記号は集合を表す記号として使う。

\begin{equation}
 f:\{1,2,3,4,5\}\rightarrow \{1,2,3\}
\end{equation}

上記写像$f$に対して$f^{-1}(1)$は
$f(x)=1$を満たす$x$の集合である。
\begin{equation}
f^{-1}(1) = \{x \mid f(x)=1\}
\end{equation}

もし、
それぞれの集合$f^{-1}(1),f^{-1}(2),f^{-1}(3)$
の要素の個数が$1$であるなら$f^{-1}$は逆写像を表す。


記号$f^{-1}$は集合を表す場合と写像を表す場合を兼ねているので
状況によって判断する必要がある。







\end{document}

