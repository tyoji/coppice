\documentclass[12pt,b5paper]{ltjsarticle}

%\usepackage[margin=15truemm, top=5truemm, bottom=5truemm]{geometry}
%\usepackage[margin=10truemm,left=15truemm]{geometry}
\usepackage[margin=10truemm]{geometry}

\usepackage{amsmath,amssymb}
%\pagestyle{headings}
\pagestyle{empty}

%\usepackage{listings,url}
%\renewcommand{\theenumi}{(\arabic{enumi})}

%\usepackage{graphicx}

%\usepackage{tikz}
%\usetikzlibrary {arrows.meta}
%\usepackage{wrapfig}	% required for `\wrapfigure' (yatex added)
%\usepackage{bm}	% required for `\bm' (yatex added)

% ルビを振る
%\usepackage{luatexja-ruby}	% required for `\ruby'

%% 核Ker 像Im Hom を定義
%\newcommand{\Img}{\mathop{\mathrm{Im}}\nolimits}
%\newcommand{\Ker}{\mathop{\mathrm{Ker}}\nolimits}
%\newcommand{\Hom}{\mathop{\mathrm{Hom}}\nolimits}

%\DeclareMathOperator{\Rot}{rot}
%\DeclareMathOperator{\Div}{div}
%\DeclareMathOperator{\Grad}{grad}
%\DeclareMathOperator{\arcsinh}{arcsinh}
%\DeclareMathOperator{\arccosh}{arccosh}
%\DeclareMathOperator{\arctanh}{arctanh}



%\usepackage{listings,url}
%
%\lstset{
%%プログラム言語(複数の言語に対応,C,C++も可)
%  language = Python,
%%  language = Lisp,
%%  language = C,
%  %背景色と透過度
%  %backgroundcolor={\color[gray]{.90}},
%  %枠外に行った時の自動改行
%  breaklines = true,
%  %自動改行後のインデント量(デフォルトでは20[pt])
%  breakindent = 10pt,
%  %標準の書体
%%  basicstyle = \ttfamily\scriptsize,
%  basicstyle = \ttfamily,
%  %コメントの書体
%%  commentstyle = {\itshape \color[cmyk]{1,0.4,1,0}},
%  %関数名等の色の設定
%  classoffset = 0,
%  %キーワード(int, ifなど)の書体
%%  keywordstyle = {\bfseries \color[cmyk]{0,1,0,0}},
%  %表示する文字の書体
%  %stringstyle = {\ttfamily \color[rgb]{0,0,1}},
%  %枠 "t"は上に線を記載, "T"は上に二重線を記載
%  %他オプション:leftline,topline,bottomline,lines,single,shadowbox
%  frame = TBrl,
%  %frameまでの間隔(行番号とプログラムの間)
%  framesep = 5pt,
%  %行番号の位置
%  numbers = left,
%  %行番号の間隔
%  stepnumber = 1,
%  %行番号の書体
%%  numberstyle = \tiny,
%  %タブの大きさ
%  tabsize = 4,
%  %キャプションの場所("tb"ならば上下両方に記載)
%  captionpos = t
%}



\begin{document}

\hrulefill

$\sigma\ne 0$とし、
$c\ne 0$を定数、
$\varepsilon =\{ \varepsilon_{t} ; t\in\mathbb{Z} \}$
をホワイトノイズ$WN(0,\sigma^2)$とする。
また、
\begin{equation}
 \phi_1 =\frac{1}{3},\quad
 \phi_2 =-\frac{1}{2},\quad
 \phi_3 =\frac{1}{6}
\end{equation}
として、
\begin{equation}
 \Phi (z) = 1-\phi_1 z -\phi_2 z^2 - \phi_3 z^3
  \quad (z\in\mathbb{C})
\end{equation}
とおく。
\begin{equation}
 X_t=c+\phi_1X_{t-1} + \phi_2X_{t-2} + \phi_3X_{t-3} + \varepsilon_t
  \quad (t\in\mathbb{Z})
  \label{Xt}
\end{equation}
によって表される時系列$X=\{X_t;t\in\mathbb{Z}\}$について、
以下の問に答えよ。

\hrulefill

\begin{enumerate}
 \item
      $\Phi(z)=0$の解$z\in\mathbb{C}$を求めよ。

      \dotfill

      \begin{equation}
       \Phi (z) = 1-\frac{1}{3}z +\frac{1}{2} z^2 - \frac{1}{6} z^3=0
      \end{equation}

      分母を払って計算すると次のように因数分解できる。
      \begin{align}
       6-2z +3 z^2 - z^3=0\\
       (3-z)(2+z^2)=0
      \end{align}

      よって、
      $z=3,\ \pm\sqrt{2}i$
      が解となる。

      \hrulefill

 \item
      $\{X_t-c;t\in\mathbb{Z}\}$がAR(3)モデル
      となることを示せ。

      \dotfill

%      AR(p)モデルとは、
%      次の多項式$\Phi(z)$の根の絶対値が全て1より大きいことをいう。
%      \begin{equation}
%       \Phi(z)=1-\phi_1z-\phi_2z^2-\cdots-\phi_pz^p
%      \end{equation}

      \textbf{自己回帰モデル}
      
      確率過程$\{ X_t ; t\in\mathbb{Z} \}$が
      次の式で表される時、
      $p$次の自己回帰モデル(Autoregressive Model, AR(p))
      という。
      \begin{equation}
       X_t+\sum_{i=1}^{p}a_{i}X_{t-i}=\varepsilon_t
      \end{equation}

      \dotfill

      式(\ref{Xt})より
      \begin{equation}
       X_{t}-c = \sum_{i=1}^{3}\phi_{i}X_{t-i}+\varepsilon_{t}
      \end{equation}
      である。

      この式を
      $\sum_{i=1}^{3}\phi_{i}=0$
      を利用して変形すると次のようになる。
      \begin{align}
       X_{t}-c =& \sum_{i=1}^{3}\phi_{i}X_{t-i}+\varepsilon_{t}\\
       =& \sum_{i=1}^{3}\phi_{i}(X_{t-i}-c+c)+\varepsilon_{t}\\
       =& \sum_{i=1}^{3}\phi_{i}(X_{t-i}-c)+c\sum_{i=1}^{3}\phi_{i}+\varepsilon_{t}\\
       =& \sum_{i=1}^{3}\phi_{i}(X_{t-i}-c)+\varepsilon_{t}
      \end{align}
      
      \begin{equation}
       (X_{t}-c) +\sum_{i=1}^{3}(-\phi_{i})(X_{t-i}-c)=\varepsilon_{t}
      \end{equation}

      これにより時系列$\{X_{t}-c\}$は3次の自己回帰モデルである。

      \hrulefill

 \item
      時系列$X$は因果的であることを示せ。

      \dotfill

      \textbf{因果性(定常性)}

      自己回帰モデルの時系列において、
      次の特性方程式$\Phi(z)$が単位円内($\lvert z\rvert \leq 1$)で
      零点を持たない時、因果性をもつという。
      \begin{equation}
       \Phi (z) = 1-\phi_1 z -\phi_2 z^2 -\cdots- \phi_p z^p
        \quad (z\in\mathbb{C})
      \end{equation}

      \dotfill

      時系列$\{ X_{t}-c ; t\in\mathbb{Z} \}$は
      次の式で表される。
      \begin{equation}
       X_{t}-c = \sum_{i=1}^{3}\phi_{i}(X_{t-i}-c)+\varepsilon_{t}
      \end{equation}

      この特性方程式$\Phi(z)$は次の式となる。
      \begin{equation}
       \Phi (z) = 1-\phi_1 z -\phi_2 z^2 - \phi_3 z^3
        \quad (z\in\mathbb{C})
      \end{equation}

      $\Phi(z)$の零点($\Phi(z)=0$を満たす$z\in\mathbb{C}$)は
      $z=3,\ \pm\sqrt{2}i$であり、
      これらは
      $\left\lvert 3 \right\rvert >1,\
      \left\lvert \pm\sqrt{2}i \right\rvert >1$
      となるので、
      時系列$\{ X_{t}-c ; t\in\mathbb{Z} \}$は因果的である。

      $\{ X_{t}-c ; t\in\mathbb{Z} \}$に
      定数$c$を加えた
      時系列$\{ X_{t} ; t\in\mathbb{Z} \}$
      も因果的である。
      

      \hrulefill

\end{enumerate}


\end{document}
