\documentclass[12pt,b5paper]{ltjsarticle}

%\usepackage[margin=15truemm, top=5truemm, bottom=5truemm]{geometry}
\usepackage[margin=15truemm]{geometry}

\usepackage{amsmath,amssymb}
%\pagestyle{headings}
\pagestyle{empty}

%\usepackage{listings,url}
%\renewcommand{\theenumi}{(\arabic{enumi})}

\usepackage{graphicx}

\usepackage{tikz}
\usetikzlibrary {arrows.meta}
\usepackage{wrapfig}	% required for `\wrapfigure' (yatex added)
\usepackage{bm}	% required for `\bm' (yatex added)

% ルビを振る
%\usepackage{luatexja-ruby}	% required for `\ruby'

%% 核Ker 像Im Hom を定義
%\newcommand{\Img}{\mathop{\mathrm{Im}}\nolimits}
%\newcommand{\Ker}{\mathop{\mathrm{Ker}}\nolimits}
%\newcommand{\Hom}{\mathop{\mathrm{Hom}}\nolimits}
%\newcommand{\Rot}{\mathop{\mathrm{rot}}\nolimits}
%\newcommand{\Div}{\mathop{\mathrm{div}}\nolimits}

\begin{document}

\hrulefill

\textbf{微分方程式}

\begin{itemize}
 \item 
       \begin{equation}
        y^\prime = f(x)y
       \end{equation}

       両辺を$y$で割り、
       左辺が対数の微分の形になっているので
       これを積分することで右辺も不定積分の形にできる。
       \begin{equation}
        \frac{y^\prime}{y} = f(x)
         \qquad
        \frac{\mathrm{d}}{\mathrm{d}x} \log y = f(x)
         \qquad
        \log y = \int f(x) \mathrm{d}x
       \end{equation}

 \item
      \begin{equation}
       y^\prime = f(x)y + g
      \end{equation}

      後ろに余計なものがある場合、
      最初に特殊解$\bar{y}$を求める。
      特殊解$\bar{y}$とは
      $\bar{y}^\prime = f(x)\bar{y} +g$
      を満たす式のことである。

      特殊解の求め方は$g$によって異なる。
      $g$が多項式であれば
      $y$も同じ次数の多項式と仮定して考える。
      $g$が定数なら$y$も定数だとして考える。

      特殊解が見つかれば
      \begin{equation}
       y^\prime = f(x)y + g
        \qquad
        \bar{y}^\prime = f(x)\bar{y} +g
      \end{equation}
      この2つの式の差を取ることで、
      $g$が消えて次の式になる。
      \begin{equation}
       y^\prime - \bar{y}^\prime = f(x)(y - \bar{y})
      \end{equation}

      $Y=y - \bar{y}$とおくと
      $Y^\prime = f(x)Y$である。
      この形が、
      $y^\prime = f(x)y$
      と同じなので、同様の方法で$Y$が求まり
      ここから解は$y=Y+\bar{y}$となる。

\end{itemize}



\hrulefill


%$A=\begin{pmatrix} 3 & -5 \\ 5 & -7 \end{pmatrix}$
%
%固有値$\lambda$と固有ベクトル$\bm{x}_\lambda$は
%\begin{equation}
% \lambda = -2, \qquad \bm{x}_\lambda =
%  \begin{pmatrix} 1 \\ 1 \end{pmatrix}
%\end{equation}
%

\begin{equation}
 \begin{pmatrix} x^\prime(t) \\ y^\prime(t) \end{pmatrix}
 = A
  \begin{pmatrix} x(t) \\ y(t) \end{pmatrix}
  +
  \begin{pmatrix} 2 \\ 2 \end{pmatrix}
\end{equation}

後ろに定数のベクトルがついているので、
$x(t),y(t)$を定数だと仮定し、
特殊解を求める。

$x^\prime(t),\ y^\prime(t)$は定数を微分している為、
ともに0となる。
\begin{equation}
 \begin{pmatrix} 0 \\ 0 \end{pmatrix}
 = A
  \begin{pmatrix} x \\ y \end{pmatrix}
  +
  \begin{pmatrix} 2 \\ 2 \end{pmatrix}
\end{equation}
この連立方程式の解$(x,y)=(1,1)$を
特殊解とし、2つの式の差を求める。

\begin{equation}
 \begin{pmatrix} x^\prime(t) \\ y^\prime(t) \end{pmatrix}
 = A
  \begin{pmatrix} x(t)-1 \\ y(t)-1 \end{pmatrix}
\end{equation}

これを$X(t)=x(t)-1,\ Y(t)=y(t)-1$と置いて次の式になる。
\begin{equation}
 \begin{pmatrix} X^\prime(t) \\ X^\prime(t) \end{pmatrix}
 = A
  \begin{pmatrix} X(t) \\ Y(t) \end{pmatrix}
\end{equation}





\end{document}
