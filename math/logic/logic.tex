\documentclass[12pt,b5paper]{ltjsarticle}

%\usepackage[margin=15truemm, top=5truemm, bottom=5truemm]{geometry}
%\usepackage[margin=10truemm,left=15truemm]{geometry}
\usepackage[margin=10truemm]{geometry}

\usepackage{amsmath,amssymb}
%\pagestyle{headings}
\pagestyle{empty}

%\usepackage{listings,url}
\renewcommand{\theenumi}{(\arabic{enumi})}

%\usepackage{graphicx}

%\usepackage{tikz}
%\usetikzlibrary {arrows.meta}
%\usepackage{wrapfig}
%\usepackage{bm}

% ルビを振る
%\usepackage{luatexja-ruby}	% required for `\ruby'

%% 核Ker 像Im Hom を定義
%\newcommand{\Img}{\mathop{\mathrm{Im}}\nolimits}
%\newcommand{\Ker}{\mathop{\mathrm{Ker}}\nolimits}
%\newcommand{\Hom}{\mathop{\mathrm{Hom}}\nolimits}

%\DeclareMathOperator{\Rot}{rot}
%\DeclareMathOperator{\Div}{div}
%\DeclareMathOperator{\Grad}{grad}
%\DeclareMathOperator{\arcsinh}{arcsinh}
%\DeclareMathOperator{\arccosh}{arccosh}
%\DeclareMathOperator{\arctanh}{arctanh}

\usepackage{url}

%\usepackage{listings}
%
%\lstset{
%%プログラム言語(複数の言語に対応,C,C++も可)
%  language = Python,
%%  language = Lisp,
%%  language = C,
%  %背景色と透過度
%  %backgroundcolor={\color[gray]{.90}},
%  %枠外に行った時の自動改行
%  breaklines = true,
%  %自動改行後のインデント量(デフォルトでは20[pt])
%  breakindent = 10pt,
%  %標準の書体
%%  basicstyle = \ttfamily\scriptsize,
%  basicstyle = \ttfamily,
%  %コメントの書体
%%  commentstyle = {\itshape \color[cmyk]{1,0.4,1,0}},
%  %関数名等の色の設定
%  classoffset = 0,
%  %キーワード(int, ifなど)の書体
%%  keywordstyle = {\bfseries \color[cmyk]{0,1,0,0}},
%  %表示する文字の書体
%  %stringstyle = {\ttfamily \color[rgb]{0,0,1}},
%  %枠 "t"は上に線を記載, "T"は上に二重線を記載
%  %他オプション:leftline,topline,bottomline,lines,single,shadowbox
%  frame = TBrl,
%  %frameまでの間隔(行番号とプログラムの間)
%  framesep = 5pt,
%  %行番号の位置
%  numbers = left,
%  %行番号の間隔
%  stepnumber = 1,
%  %行番号の書体
%%  numberstyle = \tiny,
%  %タブの大きさ
%  tabsize = 4,
%  %キャプションの場所("tb"ならば上下両方に記載)
%  captionpos = t
%}

%\usepackage{cancel}
%\usepackage{bussproofs}
%\usepackage{proof}

\begin{document}

\hrulefill

\begin{description}
 \item [問題 1]
       以下の問いに答えよ。
       \begin{enumerate}
        \item \label{q1.1}
             $(a,b)=(a^{\prime},b^{\prime})
             \rightarrow a=a^{\prime} \land b=b^{\prime}$を示せ。

              \dotfill

              順序対の定義
              \begin{equation}
               (a,b) = \{\{a,a\},\{a,b\}\},\quad
                (a^{\prime},b^{\prime})
                =\{\{a^{\prime},a^{\prime}\},\{a^{\prime},b^{\prime}\}\}
              \end{equation}

              外延性公理
              \begin{equation}
               (a,b)=(a^{\prime},b^{\prime})
                \leftrightarrow
                \forall x (x\in (a,b) \leftrightarrow x\in (a^{\prime},b^{\prime}))
              \end{equation}

              $x\in (a,b) \leftrightarrow x\in (a^{\prime},b^{\prime})$より
              $\{a,a\}\in (a^{\prime},b^{\prime})=\{\{a^{\prime},a^{\prime}\},\{a^{\prime},b^{\prime}\}\}$である。

              これより
              $\{a,a\}=\{a^{\prime},a^{\prime}\}$
              または
              $\{a,a\}=\{a^{\prime},b^{\prime}\}$
              である。

              \dotfill

              \textbf{$\{a,a\}=\{a^{\prime},a^{\prime}\}$の場合}

              対集合は次で定義される。
              \begin{equation}
               \{a,a\}\underset{\mathrm{def}}{=}\{x:x=a \lor x=a \}
                ,\quad
                \{a^{\prime},a^{\prime}\}\underset{\mathrm{def}}{=} \{x:x=a^{\prime}\lor x=a^{\prime}\}
              \end{equation}

              これにより
              $\{a,a\}=\{a^{\prime},a^{\prime}\}$
              であるなら
              $a=a^{\prime}$
              であることがわかる。

              $\{a,a\}=\{a^{\prime},a^{\prime}\}$より
              $\{a,b\}=\{a^{\prime},b^{\prime}\}$であり、
              $a=a^{\prime}$であるので、
              $\{a,b\}=\{a,b^{\prime}\}$である。

              同様に考えると$b=b^{\prime}$となる。

              つまり、次が得られる。
              \begin{equation}
               a=a^{\prime}\land b=b^{\prime}
              \end{equation}

              \dotfill

              \textbf{$\{a,a\}=\{a^{\prime},b^{\prime}\}$の場合}

              外延性公理より
              \begin{equation}
               \forall x ( x\in \{a,a\} \leftrightarrow x \in \{ a^{\prime}, b^{\prime}\} )
              \end{equation}
              であるので、
              $a=a^{\prime} \land a=b^{\prime}$
              であることがわかる。

              $\{a,b\}=\{a^{\prime},a^{\prime}\}$であるから、
              同様に考えて
              $a=a^{\prime} \land b=a^{\prime}$
              である。

              つまり、次が得られる。
              \begin{equation}
               a=a^{\prime}
                \land a=b^{\prime}
                \land b=a^{\prime}
              \end{equation}

              ここから次が得られる。
              \begin{equation}
               b=b^{\prime}
              \end{equation}

              \dotfill

              これらそれぞれの場合において
              \begin{equation}
               a=a^{\prime} \land b=b^{\prime}
              \end{equation}
              である。


              \hrulefill

        \item
             $((a,b)) \underset{\mathrm{def}}{=} \{\{a\},\{\emptyset,b\}\}$
             について
             \ref{q1.1}
             と同様の事実は成り立つか?

             \dotfill


             $((a,b)) = ((a^{\prime},b^{\prime}))$とする。

             外延性公理より次が成り立つ。
             \begin{equation}
              \forall x (x\in \{\{a\},\{\emptyset,b\}\}
               \leftrightarrow x\in \{\{a^{\prime}\},\{\emptyset,b^{\prime}\}\})
             \end{equation}

             \dotfill

             \textbf{$\{a\}=\{a^{\prime}\}$ の場合}

             $\{a\}=\{a^{\prime}\}$より
             $a=a^{\prime}$が得られる。

             また、
             $\{\emptyset,b\} = \{\emptyset,b^{\prime}\}$
             より、
             $b = b^{\prime}$または
             $b=\emptyset \land b^{\prime}=\emptyset$
             となる。

             \textbf{$\{a\}=\{\emptyset,b^{\prime}\}$ の場合}

             $\{a\}=\{\emptyset,b^{\prime}\}$より
             $a=\emptyset \land a=b^{\prime}$である。

             同様に
             $\{\emptyset,b\} = \{a^{\prime}\}$より
             $a^{\prime}=\emptyset \land b=a^{\prime}$である。


             $a=\emptyset \land b^{\prime}=\emptyset \land
             a^{\prime}=\emptyset \land b=\emptyset$である。

             \dotfill

             これらより
             \ref{q1.1}と同様に
             $a=a^{\prime} \land b=b^{\prime}$
             が得られる。


             \end{enumerate}

             \hrulefill

 \item[問題2]
            集合$A\ne \emptyset$について
            以下の問いに答えよ。
            \begin{enumerate}
             \item \label{q2.1}
                  $(\mathcal{P}(A);\Delta,\cap,\emptyset,A)$は
                  指標2の可換環であることを示せ。

                   \dotfill

                   \hrulefill

             \item
                  環に関する性質Pを一つ選んで
                  \ref{q2.1}の環が性質Pをみたすような$A$を特徴付けよ.

                  \dotfill

                  \hrulefill

            \end{enumerate}


 \item[問題3]
            $(\mathbb{Q},<) \simeq (\mathbb{Q} \lceil 0,<)$
            は成り立つか?

            \dotfill

            \hrulefill



 \item[問題4]
            以下を示せ。
            \begin{enumerate}
             \item
                  整列集合の自己同型写像は恒等写像に限る。

                  \dotfill

                  \hrulefill

             \item
                  全順序集合$(\mathbb{Z},<)$の
                  自己同型写像は無数に存在する。

                  \dotfill

                  \hrulefill

            \end{enumerate}

 \item[問題5]
            全順序集合$(\mathbb{Z},<),(\mathbb{Q},<),(\mathbb{R},<)$
            について次の補題と同様の主張は成り立つか?

            \begin{quotation}
             \textbf{補題}

             $(A; <)$を整列集合とする。
             $A$の真部分集合$X$が下に閉$ (x < y \in X \to x \in X)$ならば、
             $x \in A$ が存在して$X = A\lceil x$である。
             
            \end{quotation}

            \dotfill

            \hrulefill

\end{description}

\hrulefill

\end{document}
