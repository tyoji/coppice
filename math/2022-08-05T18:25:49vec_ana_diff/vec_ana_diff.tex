\documentclass[12pt,b5paper]{ltjsarticle}

%\usepackage[margin=15truemm, top=5truemm, bottom=5truemm]{geometry}
\usepackage[margin=10truemm]{geometry}

\usepackage{amsmath,amssymb}
%\pagestyle{headings}
\pagestyle{empty}

%\usepackage{listings,url}
%\renewcommand{\theenumi}{(\arabic{enumi})}

\usepackage{graphicx}

\usepackage{tikz}
\usetikzlibrary {arrows.meta}
\usepackage{wrapfig}	% required for `\wrapfigure' (yatex added)
\usepackage{bm}	% required for `\bm' (yatex added)

% ルビを振る
%\usepackage{luatexja-ruby}	% required for `\ruby'

%% 核Ker 像Im Hom を定義
%\newcommand{\Img}{\mathop{\mathrm{Im}}\nolimits}
%\newcommand{\Ker}{\mathop{\mathrm{Ker}}\nolimits}
%\newcommand{\Hom}{\mathop{\mathrm{Hom}}\nolimits}

\DeclareMathOperator{\Rot}{rot}
\DeclareMathOperator{\Div}{div}
\DeclareMathOperator{\Grad}{grad}
%\DeclareMathOperator{\arcsinh}{arcsinh}



\begin{document}

\hrulefill

\textbf{ナブラ $\nabla$}
\begin{equation}
 \nabla = \left( \frac{\partial}{\partial x}, \frac{\partial}{\partial y}, \frac{\partial}{\partial z} \right)
\end{equation}


ベクトル場$\bm{f}$
\begin{equation}
 \bm{f}: \mathbb{R}^3 \rightarrow \mathbb{R}^3,
  \quad
  (x,y,z) \mapsto ( f_1(x,y,z),f_2(x,y,z),f_3(x,y,z) )
\end{equation}

\textbf{$D\bm{f}$}
\begin{equation}
 D\bm{f}=\left( \frac{\partial \bm{f}}{\partial x}, \frac{\partial \bm{f}}{\partial y}, \frac{\partial \bm{f}}{\partial z} \right)
  =
  \begin{pmatrix}
   \frac{\partial f_1}{\partial x} & \frac{\partial f_1}{\partial y} & \frac{\partial f_1}{\partial z}\\
   \frac{\partial f_2}{\partial x} & \frac{\partial f_2}{\partial y} & \frac{\partial f_2}{\partial z}\\
   \frac{\partial f_3}{\partial x} & \frac{\partial f_3}{\partial y} & \frac{\partial f_3}{\partial z}
  \end{pmatrix}
\end{equation}

\textbf{回転 $\Rot$}
\begin{equation}
 \Rot \bm{f} = \nabla\times\bm{f}
  = \left(
     \frac{\partial f_3}{\partial y} - \frac{\partial f_2}{\partial z},\
     \frac{\partial f_1}{\partial z} - \frac{\partial f_3}{\partial x},\
     \frac{\partial f_2}{\partial x} - \frac{\partial f_1}{\partial y}
    \right)
\end{equation}


\textbf{発散 $\Div$}
\begin{equation}
 \Div \bm{f} = \langle \nabla, \bm{f}\rangle
  = \frac{\partial f_1}{\partial x}
    + \frac{\partial f_2}{\partial y}
    + \frac{\partial f_3}{\partial z}
\end{equation}



\textbf{内積、外積}
\begin{align}
 \langle \bm{a}, \bm{a} \rangle =& \lvert\bm{a}\rvert^2 &
 \bm{a}\times\bm{a} =& 0\\
 \langle \bm{a}, \bm{b} \rangle =& \langle \bm{b}, \bm{a} \rangle &
  \quad
 \bm{a}\times\bm{b} =& - \bm{b}\times\bm{a}\\
 \langle k\bm{a}, \bm{b} \rangle =& k \langle \bm{a}, \bm{b} \rangle &
  k\bm{a} \times \bm{b} =& k (\bm{a} \times \bm{b})
 \\
 \langle \bm{a}, \bm{b}+\bm{c} \rangle =& \langle \bm{a}, \bm{b} \rangle + \langle \bm{a}, \bm{c} \rangle &
 \bm{a}\times (\bm{b}+\bm{c}) =& \bm{a}\times \bm{b} + \bm{a}\times \bm{c}
\end{align}

\textbf{三重積}
\begin{align}
 \langle \bm{a}, \bm{b}\times\bm{c} \rangle
  =& \langle \bm{b}, \bm{c}\times\bm{a} \rangle
  = \langle \bm{c}, \bm{a}\times\bm{b} \rangle\\
 \bm{a}\times(\bm{b}\times\bm{c})
 =& \langle \bm{a}, \bm{c} \rangle\bm{b}
 - \langle \bm{a}, \bm{b} \rangle\bm{c}
\end{align}

\textbf{ヤコビの恒等式}
\begin{equation}
 \bm{a}\times(\bm{b}\times\bm{c})
  + \bm{b}\times(\bm{c}\times\bm{a})
  + \bm{c}\times(\bm{a}\times\bm{b})
  =0
\end{equation}

\hrulefill

\textbf{定義}

$S\subset\mathbb{R}^3$を$C^2$-級曲面片の像とし、
$f:S\to\mathbb{R}$を$C^1$-級関数、
$\bm{n}$を$S$上の外向き単位法ベクトル場とする。

この時、$p\in S$に対して、
\begin{equation}
 \lim_{h\to 0}\frac{f(p+h\bm{n}(p))-f(p)}{h}
  \qquad
  \left(\quad
   = \langle \nabla f(p), \bm{n}(p) \rangle
  \quad\right)
\end{equation}
を$f$の点$p$での法方向微分といい、
これを
\begin{equation}
 \frac{\partial f}{\partial \bm{n}}(p)
\end{equation}
と書く。

\hrulefill

$D\subset\mathbb{R}^3$
\begin{enumerate}
 \item \label{defg}
      $g:D\to\mathbb{R}$を$C^1$-級関数、
      $\bm{f}:D\to\mathbb{R}^3$を$C^1$-級ベクトル場とする。

      この時、
      \begin{equation}
       \int_{D} g \Div \bm{f} \mathrm{d}x \mathrm{d}y \mathrm{d}z
        =-\int_{D} \langle \nabla g, \bm{f}\rangle \mathrm{d}x \mathrm{d}y \mathrm{d}z
        + \int_{\partial D} g \langle\bm{f},\bm{n}\rangle \mathrm{d}A
      \end{equation}
      となることを示せ。


 \item
      \ref{defg}の状況で$\partial D$上で$g$が$0$であるなら、
      \begin{equation}
       \int_{D} g \Div \bm{f} \mathrm{d}x \mathrm{d}y \mathrm{d}z
        =-\int_{D} \langle \nabla g, \bm{f}\rangle \mathrm{d}x \mathrm{d}y \mathrm{d}z
      \end{equation}
      となることを示せ。



 \item
      \textbf{Greenの定理}
      
      $f:D\to\mathbb{R}$と
      $g:D\to\mathbb{R}$を
      共に$C^2$-級関数とする。

      この時、
      \begin{equation}
       \int_{D}(f\Delta g - g\Delta f) \mathrm{d}x \mathrm{d}y \mathrm{d}z
        = \int_{\partial D}
        \left(
         f\frac{\partial g}{\partial \bm{n}} - g\frac{\partial f}{\partial \bm{n}}
        \right)\mathrm{d}A
      \end{equation}
      となることを示せ。

\end{enumerate}

\hrulefill

\newpage

$e\in\mathbb{R}, \varepsilon_0 >0$は定数とする。
$p=(p_1,p_2,p_3)\in\mathbb{R}^3$とし、
変数を$\bm{x}=(x_1,x_2,x_3)$とする。
$\mathbb{E}: \mathbb{R}^3 \backslash \{p\} \to \mathbb{R}^3$を
\begin{equation}
 \mathbb{E}(\bm{x}) = \frac{e}{4\pi\varepsilon_0\lvert\bm{x}-p\rvert^3}
  \begin{pmatrix}
   x_1-p_1 \\ x_2-p_2 \\ x_3-p_3
  \end{pmatrix}
\end{equation}
とする。

(物理的には点$p$に電荷$e$を置いたときに出来る電場が$\mathbb{E}$)

\hrulefill

\begin{enumerate}
 \item
      $\Div\mathbb{E}=0$
      を示せ。

      \dotfill

      $\Div\mathbb{E}$を展開し、定数部分をまとめる。
      \begin{align}
       \Div\mathbb{E}
        =& \frac{\partial}{\partial x_1}\frac{e(x_1-p_1)}{4\pi\varepsilon_0\lvert\bm{x}-p\rvert^3}
        + \frac{\partial}{\partial x_2}\frac{e(x_2-p_2)}{4\pi\varepsilon_0\lvert\bm{x}-p\rvert^3}
        + \frac{\partial}{\partial x_3}\frac{e(x_3-p_3)}{4\pi\varepsilon_0\lvert\bm{x}-p\rvert^3}\\
       =& \frac{e}{4\pi\varepsilon_0}
       \left(
       \frac{\partial}{\partial x_1}\frac{x_1-p_1}{\lvert\bm{x}-p\rvert^3}
       + \frac{\partial}{\partial x_2}\frac{x_2-p_2}{\lvert\bm{x}-p\rvert^3}
       + \frac{\partial}{\partial x_3}\frac{x_3-p_3}{\lvert\bm{x}-p\rvert^3}
       \right)
       \label{exp_div}
      \end{align}
      $\mathbb{E}$の発散は上のように3つの偏微分からなる。

      この内の一つを取り出し、変数を$x_i$として計算する。
      \begin{equation}
        \frac{\partial}{\partial x_i}\frac{x_i-p_i}{\lvert\bm{x}-p\rvert^3}
         =
         \frac{1}{\lvert\bm{x}-p\rvert^6}
         \left(
          \lvert\bm{x}-p\rvert^3 \frac{\partial}{\partial x_i}(x_i-p_i)
          -
          (x_i-p_i)\frac{\partial}{\partial x_i}\lvert\bm{x}-p\rvert^3
         \right)
         \label{par_diff}
      \end{equation}
      $\frac{\partial}{\partial x_i}(x_i-p_i)=1$であるので、
      $\frac{\partial}{\partial x_i}\lvert\bm{x}-p\rvert^3$を計算する。
      \begin{align}
       \frac{\partial}{\partial x_i}\lvert\bm{x}-p\rvert^3
       =& \frac{\partial}{\partial x_i}( (x_1-p_1)^2+(x_2-p_2)^2+(x_2-p_2)^2 )^{\frac{3}{2}}\\
       =& \frac{3}{2}( (x_1-p_1)^2+(x_2-p_2)^2+(x_2-p_2)^2 )^{\frac{1}{2}}\times 2(x_i-p_i)\\
       =& 3\lvert \bm{x}-p\rvert (x_i-p_i)
      \end{align}

      これを式(\ref{par_diff})に代入する。
      \begin{equation}
        \frac{\partial}{\partial x_i}\frac{x_i-p_i}{\lvert\bm{x}-p\rvert^3}
         =
         \frac{1}{\lvert\bm{x}-p\rvert^6}
         \left(
          \lvert\bm{x}-p\rvert^3
          -
          3\lvert \bm{x}-p\rvert (x_i-p_i)^2
         \right)
      \end{equation}

      式(\ref{exp_div})の3つの偏微分の和を考える。
      \begin{align}
       & \sum_{i=1}^{3}\frac{\partial}{\partial x_i}\frac{x_i-p_i}{\lvert\bm{x}-p\rvert^3}\\
        =&
         \frac{1}{\lvert\bm{x}-p\rvert^6}
         \left(
          3\lvert\bm{x}-p\rvert^3
          -
          3\lvert \bm{x}-p\rvert ((x_1-p_1)^2+(x_2-p_2)^2+(x_3-p_3)^2)
         \right)\\
        =&
         \frac{1}{\lvert\bm{x}-p\rvert^6}
         \left(
          3\lvert\bm{x}-p\rvert^3
          -
          3\lvert \bm{x}-p\rvert \lvert \bm{x}-p \rvert^2
         \right)
       =0
      \end{align}

      これにより$\Div\mathbb{E}=0$がわかる。

      \hrulefill

 \item
      領域$D\subset\mathbb{R}^3$はGaussの発散定理の条件を満たすとする。
      ($D=\partial D \cup D^{\circ}$)
      \begin{enumerate}
       \item
            $p\not\in D$の時、次の式を示せ。
            \begin{equation}
             \int_{\partial D}
              \langle \mathbb{E},\bm{n} \rangle \mathrm{d}A
              =0
            \end{equation}

            \dotfill

            Gaussの発散定理より次のような変形ができる。
            \begin{equation}
             \int_{\partial D} \langle \mathbb{E},\bm{n} \rangle \mathrm{d}A
             = \int_{D} \Div \mathbb{E} \mathrm{d}D
            \end{equation}

            $p\not\in D$であるので、
            $\mathbb{E}$は$D$全体で定義されている。
            この為、先程の問より$\Div\mathbb{E}=0$である。

            よって、次の式が得られる。
            \begin{equation}
             \int_{\partial D}
              \langle \mathbb{E},\bm{n} \rangle \mathrm{d}A
              =0
            \end{equation}

            \hrulefill

       \item
            $p\in D^{\circ}$の時、次の式を示せ。
            \begin{equation}
             \int_{\partial D}
              \langle \mathbb{E},\bm{n} \rangle \mathrm{d}A
              =\frac{e}{\varepsilon_0}
            \end{equation}

            \dotfill

            $p\in D^{\circ}$であるので、
            $p$を中心とした半径$r$の球$S_r$を考える。
            $S_r \subset D^{\circ}$となるように十分小さな半径$r$とする。

            $D^{\prime}=D\backslash S_r$ とおくと、
            $p\not\in D^{\prime}$であるので、次の積分は0となる。
            \begin{equation}
             \int_{D^{\prime}} \Div \mathbb{E} \mathrm{d}D^{\prime}
              =0
            \end{equation}

            発散定理より次のように変形される。
            \begin{equation}
             \int_{D^{\prime}} \Div \mathbb{E} \mathrm{d}D^{\prime}
              =
             \int_{\partial D} \langle \mathbb{E},\bm{n} \rangle \mathrm{d}A
             -
             \int_{\partial S_r} \langle \mathbb{E},\bm{n} \rangle \mathrm{d}S_r
            \end{equation}

            左辺が0であるので、右辺を移項し次のようになる。
            \begin{equation}
             \int_{\partial D} \langle \mathbb{E},\bm{n} \rangle \mathrm{d}A
             =
             \int_{\partial S_r} \langle \mathbb{E},\bm{n} \rangle \mathrm{d}S_r
            \end{equation}

            右辺を計算することで左辺を求める。
            $\bm{n}$は法単位ベクトルであり、
%            $D^{\prime}$の外向きにとるので、
%            $S_r$の中心$p$へ向いている。
%            その為、$\bm{n}=-\frac{\bm{x}-p}{\lvert \bm{x}-p \rvert}$である。
            $\bm{n}$は$S_r$に垂直である。
            ベクトル$\bm{x}-p$も$S_r$と垂直であるので
            $\langle \bm{x}-p,\bm{n} \rangle = \lvert \bm{x}-p \rvert$である。
            \begin{align}
             \int_{\partial S_r} \langle \mathbb{E},\bm{n} \rangle \mathrm{d}S_r
%              =& \int_{\partial S_r} \left\langle
%              \frac{e}{4\pi\varepsilon_0\lvert\bm{x}-p\rvert^3} \begin{pmatrix} x_1-p_1 \\ x_2-p_2 \\ x_3-p_3 \end{pmatrix},
%              -\frac{\bm{x}-p}{\lvert \bm{x}-p \rvert}
%             \bm{n}
%             \right\rangle \mathrm{d}S_r\\
              =& \int_{\partial S_r}
             \frac{e}{4\pi\varepsilon_0\lvert\bm{x}-p\rvert^3}
             \left\langle
               \begin{pmatrix} x_1-p_1 \\ x_2-p_2 \\ x_3-p_3 \end{pmatrix}
              ,\bm{n} \right\rangle \mathrm{d}S_r\\
              =& \int_{\partial S_r} \frac{e}{4\pi\varepsilon_0\lvert\bm{x}-p\rvert^2} \mathrm{d}S_r
            \end{align}
            $S_r$は半径$r$の球であるため、$\lvert \bm{x}-p\rvert =r$である。
            \begin{equation}
             \int_{\partial S_r} \frac{e}{4\pi\varepsilon_0\lvert\bm{x}-p\rvert^2} \mathrm{d}S_r
              =\frac{e}{4\pi\varepsilon_0 r^2} \int_{\partial S_r} \mathrm{d}S_r
              =\frac{e}{4\pi\varepsilon_0 r^2} 4\pi r^2
              =\frac{e}{\varepsilon_0}
            \end{equation}

            半径$r$の球の表面積であるので
            $\int_{\partial S_r} \mathrm{d}S_r = 4\pi r^2$
            より、上記の結果が得られる。
      \end{enumerate}
\end{enumerate}

\hrulefill

\end{document}
