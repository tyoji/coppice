\documentclass[12pt,b5paper]{ltjsarticle}

%\usepackage[margin=15truemm, top=5truemm, bottom=5truemm]{geometry}
%\usepackage[margin=10truemm,left=15truemm]{geometry}
\usepackage[margin=10truemm]{geometry}

\usepackage{amsmath,amssymb}
%\pagestyle{headings}
\pagestyle{empty}

%\usepackage{listings,url}
%\renewcommand{\theenumi}{(\arabic{enumi})}

%\usepackage{graphicx}

%\usepackage{tikz}
%\usetikzlibrary {arrows.meta}
%\usepackage{wrapfig}
%\usepackage{bm}

% ルビを振る
\usepackage{luatexja-ruby}	% required for `\ruby'

%% 核Ker 像Im Hom を定義
%\newcommand{\Img}{\mathop{\mathrm{Im}}\nolimits}
%\newcommand{\Ker}{\mathop{\mathrm{Ker}}\nolimits}
%\newcommand{\Hom}{\mathop{\mathrm{Hom}}\nolimits}

%\DeclareMathOperator{\Rot}{rot}
%\DeclareMathOperator{\Div}{div}
%\DeclareMathOperator{\Grad}{grad}
%\DeclareMathOperator{\arcsinh}{arcsinh}
%\DeclareMathOperator{\arccosh}{arccosh}
%\DeclareMathOperator{\arctanh}{arctanh}



%\usepackage{listings,url}
%
%\lstset{
%%プログラム言語(複数の言語に対応,C,C++も可)
%  language = Python,
%%  language = Lisp,
%%  language = C,
%  %背景色と透過度
%  %backgroundcolor={\color[gray]{.90}},
%  %枠外に行った時の自動改行
%  breaklines = true,
%  %自動改行後のインデント量(デフォルトでは20[pt])
%  breakindent = 10pt,
%  %標準の書体
%%  basicstyle = \ttfamily\scriptsize,
%  basicstyle = \ttfamily,
%  %コメントの書体
%%  commentstyle = {\itshape \color[cmyk]{1,0.4,1,0}},
%  %関数名等の色の設定
%  classoffset = 0,
%  %キーワード(int, ifなど)の書体
%%  keywordstyle = {\bfseries \color[cmyk]{0,1,0,0}},
%  %表示する文字の書体
%  %stringstyle = {\ttfamily \color[rgb]{0,0,1}},
%  %枠 "t"は上に線を記載, "T"は上に二重線を記載
%  %他オプション:leftline,topline,bottomline,lines,single,shadowbox
%  frame = TBrl,
%  %frameまでの間隔(行番号とプログラムの間)
%  framesep = 5pt,
%  %行番号の位置
%  numbers = left,
%  %行番号の間隔
%  stepnumber = 1,
%  %行番号の書体
%%  numberstyle = \tiny,
%  %タブの大きさ
%  tabsize = 4,
%  %キャプションの場所("tb"ならば上下両方に記載)
%  captionpos = t
%}



\begin{document}

\hrulefill

\textbf{Vinogradov の記号}

\begin{equation}
 f:X\to\mathbb{C},\quad
  g:X\to\mathbb{R}_{\geq 0}
\end{equation}

部分集合$S\subset X$とする。

\begin{center}
 $x\in S$において$f(x) \ll g(x)$
 $\overset{\mathrm{def}}{\iff}$
 ${}^{\exists}C\geq 0 \ \text{s.t.} \
 {}^{\forall}s\in S, \ \lvert f(x) \rvert \leq Cg(x)$
\end{center}

定数$C$のことを
\textbf{implicit constant}
または
\textbf{implied constant}
という。


\textbf{Landau の記号}

\begin{equation}
 g:X\to\mathbb{R}_{\geq0}
\end{equation}

$S\subset X$に対して
$O(g(x))$
$\overset{\mathrm{def}}{\iff}$
範囲$x\in S$において
$f(x)\ll g(x)$
と評価されるような
項$f(x)$の省略



\textbf{Landau の記号}

\begin{equation}
 f(x)=O(g(x)) \ (x\to a)
  \overset{\mathrm{def}}{\iff}
  {}^{\exists}C\geq 0
  \ s.t. \
 \lim_{x\to a}
 \left\lvert \frac{f(x)}{g(x)} \right\rvert \leq C
\end{equation}

$O(g(x)) \ (x\to a)$とは、
$x\to a$において
同じぐらいの速さで収束する関数全体の集合である。

$O(g(x)) \ (x\to \infty)$であれば、
$\deg{g(x)}$と等しい次数の多項式等の集合であり、
$O(g(x)) \ (x\to 0)$であれば、
次数の低い項が同じ次数の多項式等の集合である。

正しい表記は$f(x) \in O(g(x)) \ (x\to a)$である。

\begin{equation}
 f(x)=o(g(x)) \ (x\to a)
  \overset{\mathrm{def}}{\iff}
 \lim_{x\to a}\frac{f(x)}{g(x)} = 0
\end{equation}

$o(g(x)) \ (x\to a)$とは、
$x\to a$において$g(x)$より速く$0$に収束する関数全体の集合である。

つまり、上の表記は正しくは$f(x)\in o(g(x)) \ (x\to a)$となる。



\hrulefill

\begin{enumerate}
 \item
      \begin{enumerate}
       \item
            関数$f:\mathbb{N}\to\mathbb{C}$
            と
            $F:\mathbb{N}\to\mathbb{R}_{\geq 0}$
            に対して、
            $f(n) \ll F(n) \quad (n\in\mathbb{N})$
            が成り立つとき、
            実数$x\geq 1$に対して、
            次が成り立つことを示せ。
            \begin{equation}
             \sum_{n\leq x} f(n) \ll \sum_{n\leq x} F(n)
            \end{equation}

            \dotfill

            $\sum_{n\leq x} f(n) = \sum_{n=1}^{[x]} f(n)$であり、
            $\sum_{n\leq x} F(n) = \sum_{n=1}^{[x]} F(n)$である。
            つまり、有限和である。


            $f(n) \ll F(n) \quad (n\in\mathbb{N})$が成り立つので、
            各自然数$k$に対して、次を満たす$C\geq0$が存在する。
            \begin{equation}
             \lvert f(k) \rvert \leq C F(k)
            \end{equation}

            よって、$1$から$[x]$までの和が次の不等式を満たす。
            \begin{equation}
              \sum_{n=1}^{[x]} \left\lvert f(n) \right\rvert
               \leq C\sum_{n=1}^{[x]} F(n)
            \end{equation}

            左辺は三角不等式から次の関係がある。
            \begin{equation}
              \left\lvert \sum_{n=1}^{[x]} f(n) \right\rvert
              \leq
              \sum_{n=1}^{[x]} \left\lvert f(n) \right\rvert
            \end{equation}

            よって、
            \begin{equation}
             \left\lvert \sum_{n=1}^{[x]} f(n) \right\rvert
               \leq C\sum_{n=1}^{[x]} F(n)
            \end{equation}
            であるので、
            \begin{equation}
             \sum_{n\leq x} f(n) \ll \sum_{n\leq x} F(n)
            \end{equation}
            である。





            \hrulefill

       \item
            関数$f_{i}:\mathbb{N}\to\mathbb{C}$
            と
            $F_{i}:\mathbb{N}\to\mathbb{R}_{\geq 0}$
            $(i=1,\dots,K)$
            に対して、
            条件
            $\lvert f_{k}(n) \rvert \leq 1, f_{k}(n) \ll F_{k}(n)$
            $(k\in\{1,\dots,K\}, \ n\in\mathbb{N})$
            (但し、ここで implicit constant は絶対定数)
            が成立すれば、
            次が成り立つことを示せ。
            \begin{equation}
             \prod_{k=1}^{K} (1 + f_{k}(n))
              = 1 + O_{K}\left( \sum_{k=1}^{K}F_{k}(n) \right)
            \end{equation}

            \dotfill

            \begin{equation}
             \prod_{k=1}^{K} (1 + f_{k}(n))
              = 1 + \sum_{k=1}^{K} f_{k}(n)
              + \sum_{i,j(i\ne j)} f_{i}(n) f_{j}(n) + \cdots
              + \prod_{k=1}^{K} f_{k}(n)
            \end{equation}



            $\lvert f_{k}(n) \rvert \leq 1$より
            $f_{k}(n)$を複数かけた方がより$0$に近い値となる。
            \begin{equation}
             0\leq \cdots \leq
             \lvert f_{k}(n)f_{i}(n) \rvert
              \leq
             \lvert f_{k}(n) \rvert
              \leq 1
            \end{equation}


            $f_{k}(n) \ll F_{k}(n)$より、
            $k=1,\dots,K$に対して$C_{k}>0$が存在する。
            \begin{equation}
             \lvert f_{k}(n) \rvert \leq C_{k} F_k(n)
            \end{equation}

            $C_{M} = \max\{C_{1},\dots,C_{K}\}$とおけば、
            \begin{equation}
             \left\lvert \sum_{k=1}^{K}f_{k}(n) \right\rvert
              \leq
              \sum_{k=1}^{K} \lvert f_{k}(n) \rvert
              \leq \sum_{k=1}^{K}C_{k} F_k(n)
              \leq \sum_{k=1}^{K}C_{M} F_k(n)
              = C_{M} \sum_{k=1}^{K} F_k(n)
            \end{equation}
            より、
            $\sum_{k=1}^{K}f_{k}(n) \ll \sum_{k=1}^{K} F_k(n)$
            であることがわかる。

            \textbf{---要確認---}

            後ろの項が小さいので次が成り立つ。
            \begin{equation}
               \sum_{k=1}^{K} f_{k}(n)
              + \sum_{i,j(i\ne j)} f_{i}(n) f_{j}(n) + \cdots
              + \prod_{k=1}^{K} f_{k}(n)
              \ll
              \sum_{k=1}^{K} F_k(n)
            \end{equation}

            \textbf{---要確認---}


            \begin{equation}
             \prod_{k=1}^{K} (1 + f_{k}(n))
              = 1 + O_{K}\left( \sum_{k=1}^{K}F_{k}(n) \right)
            \end{equation}


            \hrulefill
      \end{enumerate}

 \item
      集合$X$上の関数$f,g:X\to\mathbb{R}_{\geq 0}$に対して
      関係$\asymp$を次のように定義する。
      \begin{equation}
       F(x) \asymp G(x) \quad (x\in X)
        \overset{\mathrm{def}}{\iff}
        F(x) \ll G(x) \ \text{かつ} \ G(x) \ll F(x) \quad (x\in X)
      \end{equation}
      \begin{enumerate}
       \item
            集合$X$上の関数$f,g:X\to\mathbb{R}_{\geq 0}$に対して
            次が成り立つことを示せ。
            \begin{equation}
             f(x)+g(x) \asymp \max(f(x),g(x)) \quad (x\in X)
            \end{equation}

            \dotfill

            $f(x)\geq 0,\ g(x)\geq 0$である。

            \begin{equation}
             \lvert f(x)+g(x) \rvert
              \leq \lvert f(x) \rvert + \lvert g(x) \rvert
              \leq 2 \max(f(x),g(x)) \quad (x\in X)
            \end{equation}

            よって、
            $f(x)+g(x) \ll \max(f(x),g(x))$である。

            \begin{equation}
             \lvert \max(f(x),g(x)) \rvert
              \leq f(x)+g(x)
              \quad (x\in X)
            \end{equation}

            よって、
            $\max(f(x),g(x)) \ll f(x)+g(x)$である。

            以上から次が成り立つ。
            \begin{equation}
             f(x)+g(x) \asymp \max(f(x),g(x)) \quad (x\in X)
            \end{equation}


            \hrulefill

       \item
            集合$X$上の関数$f,g:X\to\mathbb{R}_{\geq 0}$に対して
            次が成り立つことを示せ。
            \begin{equation}
             (f(x)+g(x))^{\frac{1}{2}}
              \asymp
              f(x)^{\frac{1}{2}}+g(x)^{\frac{1}{2}} \quad (x\in X)
            \end{equation}

            \dotfill

            $f(x)\geq 0,\ g(x)\geq 0$である。
            これより
            $f(x)^{\frac{1}{2}}g(x)^{\frac{1}{2}} \geq 0$である。
            \begin{align}
             & \left( (f(x)+g(x))^{\frac{1}{2}} \right)^2
              = f(x)+g(x)\\
              \leq &
              f(x) + 2 f(x)^{\frac{1}{2}}g(x)^{\frac{1}{2}} + g(x)
              = \left( f(x)^{\frac{1}{2}}+g(x)^{\frac{1}{2}} \right)^{2}
            \end{align}

            $f(x)^{\frac{1}{2}}+g(x)^{\frac{1}{2}} \geq 0$であるので
            $2$乗を外すと次の式が得られる。
            \begin{equation}
             (f(x)+g(x))^{\frac{1}{2}}
              \leq
              f(x)^{\frac{1}{2}}+g(x)^{\frac{1}{2}}
            \end{equation}
            よって、
            $(f(x)+g(x))^{\frac{1}{2}} \ll f(x)^{\frac{1}{2}}+g(x)^{\frac{1}{2}}$
            である。


            相加相乗平均の関係より次の式が得られる。
            \begin{equation}
             ( f(x)g(x) )^{\frac{1}{2}}
              \leq \frac{1}{2} (f(x)+g(x))
            \end{equation}
            これを用いて次の不等式が成り立つ。
            \begin{align}
             & \left( f(x)^{\frac{1}{2}}+g(x)^{\frac{1}{2}} \right)^{2}
              =
              f(x) + 2 f(x)^{\frac{1}{2}}g(x)^{\frac{1}{2}} + g(x)\\
              \leq &
              f(x) + f(x)+g(x) + g(x)
              = 2(f(x)+g(x))
             = \left( 2^{\frac{1}{2}}(f(x)+g(x))^{\frac{1}{2}} \right)^{2}
            \end{align}
            この$2$乗を外すことで
            $f(x)^{\frac{1}{2}}+g(x)^{\frac{1}{2}} \ll (f(x)+g(x))^{\frac{1}{2}}$
            である。

            これらより
            次の式が成り立つことがわかる。
            \begin{equation}
             (f(x)+g(x))^{\frac{1}{2}}
              \asymp
              f(x)^{\frac{1}{2}}+g(x)^{\frac{1}{2}} \quad (x\in X)
            \end{equation}



            \hrulefill

      \end{enumerate}


 \item
      実数$x\in\mathbb{R}$に対して、
      $\exp(ix) = 1+O(\lvert x \rvert)$
      が成立することを示せ。

      \dotfill

      $\exp(ix)$のテイラー展開
      \begin{equation}
       \exp(ix)
        = \sum_{k=0}^{\infty} \frac{(ix)^{k}}{k!}
        = 1 + ix + \frac{(ix)^{2}}{2!} + \frac{(ix)^{3}}{3!} + \cdots
      \end{equation}

      \begin{equation}
       \exp(ix) -1
        = \sum_{k=1}^{\infty} \frac{(ix)^{k}}{k!}
      \end{equation}

      \hrulefill


 \item
      関数$\Phi:[1,+\infty)\to\mathbb{C}$と
      $F:[1,+\infty)\to\mathbb{R}_{\geq 0}$に対して
      次の式が成立するとする。
      \begin{equation}
       \Phi(x) = 1+ O(F(x)) \quad (x\geq 1)
      \end{equation}
      
      このとき、次を示せ。
      \begin{enumerate}
       \item
            もし、$\displaystyle \lim_{x\to\infty}F(x)=0$だったなら、
            ある$x_{0}=x_{0}(\Phi)$が存在して
            次が成立する。
            \begin{equation}
             \frac{1}{\Phi(x)}= 1+ O(F(x)) \quad (x\geq x_{0})
            \end{equation}

            \dotfill

            \hrulefill

       \item
            もし、
            $\frac{1}{\Phi(x)} \ll 1$ $(x\geq 1)$だったなら
            次が成立する。
            \begin{equation}
             \frac{1}{\Phi(x)}= 1+ O(F(x)) \quad (x\geq 1)
            \end{equation}
            但し、ここで implicit constant は
            $\frac{1}{\Phi(x)} \ll 1$ $(x\geq 1)$
            の implicit constant に依存する。

            \dotfill

            \hrulefill


      \end{enumerate}


 \item
      実数$x\geq 1$に対して、
      次が成立することを示せ。
      \begin{equation}
       \sum_{n\leq x} \sum_{d \mid n}(-1)^{d}
        = (-\log{2})\cdot x + O(x^{\frac{1}{2}})
      \end{equation}
      (Hint: hyperbola method を用いる)


      \dotfill

      \hrulefill


 \item
      数論的関数
      $\chi_{4} : \mathbb{Z}\to\mathbb{R}
      ,\ 
      r:\mathbb{N}\to\mathbb{R}$
      を
      次のように定める。
      \begin{equation}
       \chi_{4}(n) =
        \begin{cases}
         +1 & ( n\equiv 1 \pmod{4})\\
         0 & ( n\equiv 0 \pmod{2})\\
         -1 & ( n\equiv 3 \pmod{4})
        \end{cases}
        ,\quad
        r(n)=4\sum_{d\mid n} \chi_{4}(d)
      \end{equation}

      このとき、
      $x\geq 1$に対して、次が成り立つことを示せ。
      \begin{equation}
       \sum_{n\leq x} r(n) = \pi x + O(x^{\frac{1}{2}})
      \end{equation}

      (Hint: hyperbola method を用いる)

      (補足 :
      実は、$n\in\mathbb{N}$に対して、
      $r(n) = \# \{ (u,v) \in\mathbb{Z}^2 \mid u^2+v^2=n \}$
      となることが知られている。
      格子点の数え上げと上記の結果を比較してみると良い
      )



      \dotfill

      \hrulefill




\end{enumerate}

\hrulefill

\newpage


\hrulefill

\textbf{\ruby{Bernoulli}{ベルヌーイ} 多項式}

有理数係数多項式の列
$\{ B_{k}(X) \}_{k=0}^{\infty}$
を初期値$B_{0}(X)=1$および
$k\geq 1$に対して成立する次の漸化式で定める。
\begin{equation}
 \frac{1}{k}\frac{d}{dX}B_{k}(X) = B_{k-1}(X)
  \quad かつ \quad
  \int_{0}^{1} B_{k}(u)du=0
\end{equation}

この多項式たち$B_{k}(X)$のことを
\ruby{Bernoulli}{ベルヌーイ} 多項式
という。

また、
多項式$B_{k}(X)$の定数項$B_{k}=B_{k}(0)$のことを
\ruby{Bernoulli}{ベルヌーイ} 数
と呼ぶ。


Bernoulli 多項式$B_{n}(X)$は
Bernoulli 数
$B_{k}$
を用いて
$B_{n}(X) = \sum_{k=0}^{n}\begin{pmatrix}n \\ k\end{pmatrix}B_{k}X^{n-k}$
と定義することもできる。
Bernoulli 数 は次のようにも定義できる。

\begin{equation}
 \sum_{j=0}^{k} \begin{pmatrix}k+1 \\ j\end{pmatrix} B_{j}
 = k+1
  \quad (k=0,1,2,\dots)
\end{equation}


Bernoulli 多項式の例
\begin{align}
 B_{0}(X)=& 1,
   & B_{1}(X)=& X-\frac{1}{2}\\
 B_{2}(X)=& X^{2}-X+\frac{1}{6},
   & B_{3}(X)=& X^{3}-\frac{3}{2}X^{2}+\frac{1}{2}X\\
 B_{4}(X)=& X^{4}-2X^{3}+X^{2}-\frac{1}{30},
 & B_{5}(X)=& X^{5}-\frac{5}{2}X^{4}+\frac{5}{3}X^{3}-\frac{1}{6}X
\end{align}

\hrulefill

\begin{enumerate}
 \item
      \begin{enumerate}
       \item

            自然数$k\geq 0$に対して、次が成立することを示せ。
            \begin{equation}
             B_{k}(X) = (-1)^{k}B_{k}(1-X)
            \end{equation}

\dotfill

            $Y=1-X$とすると、
            $\frac{dY}{dX} = -1$となる。
            \begin{align}
             \frac{1}{k} \frac{d}{dX} (-1)^{k}B_{k}(1-X)
              =& \frac{(-1)^{k-1}}{k} \frac{d}{dX} \frac{dX}{dY} B_{k}(Y)\\
              =& (-1)^{k-1} \frac{1}{k} \frac{d}{dY}B_{k}(Y)\\
              =& (-1)^{k-1} B_{k-1}(Y)\\
              =& (-1)^{k-1} B_{k-1}(1-X)
            \end{align}

            $v=1-u$とすれば$\frac{d}{du}v = -1$である。
            \begin{align}
             \int_{0}^{1} (-1)^{k}B_{k}(1-u) du
              =& (-1)^{k-1} \int_{0}^{1} B_{k}(1-u)\frac{dv}{du} du\\
              =& (-1)^{k-1} \int_{0}^{1} B_{k}(v)dv
              =0
            \end{align}


            以上により、
            $(-1)^{k}B_{k}(1-X)$
            はBernoulli 多項式である。


\hrulefill

       \item
            奇数$k\geq 3$に対して、$B_{k}=0$を示せ。

\dotfill


            $n\in\mathbb{N}$とする。

            上の結果より次の式が得られる。
            \begin{equation}
             B_{2n-1}(X) = -B_{2n-1}(1-X)
              ,\quad
              B_{2n}(X) = B_{2n}(1-X)
            \end{equation}

            これより、
            $B_{2n-1}(X) + B_{2n-1}(1-X) =0$
            である。

            $B_{2n-1}(X)=\sum_{i=0}^{2n-1}a_{i}X^{i}$とする。
            この時、$B_{2n-1}(X)$の定数項は$a_{0}$で、
            $B_{2n-1}(1-X)$は$\sum_{i=0}^{2n-1}a_{i}$である。
            



\hrulefill

      \end{enumerate}



 \item
      \begin{enumerate}
       \item
            自然数$k\geq 1$に対して次が成立することを示せ。
            \begin{equation}
             B_{k}(X) = \sum_{j=0}^{k}X^{j}B_{k-j}
            \end{equation}

\dotfill

\hrulefill

       \item
            最初の5つのBernoulli 多項式の
            $B_{i}(X) \ (i=0,\dots,4)$を求めよ。

\dotfill

\hrulefill

      \end{enumerate}

 \item
      実数$x\geq 1$と自然数$K\geq 1$に対して、
      \begin{equation}
       \sum_{n\leq x}\frac{1}{n}
        = \log{x} + \gamma
        + \sum_{k=1}^{K}(-1)^{k}\frac{B_{k}(\{x\})}{k}x^{-k}
        + (-1)^{K}\int_{x}^{\infty} B_{K}(\{u\})u^{-(K+1)}du
      \end{equation}
      したがって
      \begin{equation}
       \sum_{n\leq x}\frac{1}{n}
        = \log{x} + \gamma
        + \sum_{k=1}^{K}(-1)^{k}\frac{B_{k}(\{x\})}{k}x^{-k}
        + O(x^{-(K+1)})
      \end{equation}
      が成立することを示せ。

\dotfill

\hrulefill

 \item
      $a,b\in\mathbb{R},\ a\leq b$、
      $K\in\mathbb{N}$、
      $C^{K}$-級関数$f: [a,b]\to\mathbb{C}$
      とする。
      この時、次の\textbf{Euler-Maclaurin 和公式}が成り立つことを示せ。
      \begin{equation}
        \begin{split}
         \sum_{a<n\leq b}f(n)
        =
         \int_{a}^{b}f(u)du
         + & \sum_{k=1}^{K}(-1)^{k}
           \left[ \frac{B_{k}(\{u\})}{k!}f^{(k-1)}(u) \right]_{a}^{b}\\
        - & \frac{(-1)^{K}}{K!}
          \int_{a}^{b} B_{K}(\{u\})f^{(K)}(u)du
        \end{split}
      \end{equation}

\dotfill

\hrulefill

 \item
      ある定数$c_{0}$が存在して、
      自然数$N$に対して、
      次の式が成立することを示せ。
      \begin{equation}
       N! = \left(\frac{N}{e}\right)^{N} \sqrt{N}e^{c_{0}+O(\frac{1}{N})}
      \end{equation}
      これは Gamma関数の Stirling の公式の特殊な場合になっている。

      (HINT: Euler-Maclaurin 和公式を利用する。)

\dotfill

\hrulefill

 \item
      次の積分$I_{n}$について考える。
      \begin{equation}
       I_{n} = \int_{0}^{\pi} (\sin{x})^{n}dx
      \end{equation}

      \begin{enumerate}
       \item
            $I_{0} = \pi$および$I_{1} = 2$を示せ。

\dotfill

\hrulefill

       \item
            整数$n\geq 0$に対して次の式を示せ。
            \begin{equation}
             I_{n+2} = \frac{n+1}{n+2}\cdot I_{n}
            \end{equation}

\dotfill

\hrulefill

       \item
            次の極限を計算せよ。
            \begin{equation}
             \lim_{n\to\infty} \frac{I_{n}}{I_{n+1}}
            \end{equation}

\dotfill

\hrulefill

       \item
            上記3つの結果から
            次の
            \textbf{Wallis の公式}
            を示せ。
            \begin{equation}
             \frac{2}{\pi}
              = \prod_{n=1}^{\infty}\left( 1-\frac{1}{4n^2} \right)
            \end{equation}

\dotfill

\hrulefill


       \item
            $e-{c_{0}} = \sqrt{2\pi}$を示せ。

\dotfill

\hrulefill
            
      \end{enumerate}
\end{enumerate}


\hrulefill

\end{document}
