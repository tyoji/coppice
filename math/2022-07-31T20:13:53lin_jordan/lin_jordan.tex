\documentclass[12pt,b5paper]{ltjsarticle}

%\usepackage[margin=15truemm, top=5truemm, bottom=5truemm]{geometry}
\usepackage[margin=15truemm]{geometry}

\usepackage{amsmath,amssymb}
%\pagestyle{headings}
\pagestyle{empty}




%\usepackage{listings,url}
%\renewcommand{\theenumi}{(\arabic{enumi})}

\usepackage{graphicx}

\usepackage{tikz}
\usetikzlibrary {arrows.meta}
\usepackage{wrapfig}	% required for `\wrapfigure' (yatex added)
\usepackage{bm}	% required for `\bm' (yatex added)

% ルビを振る
%\usepackage{luatexja-ruby}	% required for `\ruby'

%% 核Ker 像Im Hom を定義
%\newcommand{\Img}{\mathop{\mathrm{Im}}\nolimits}
%\newcommand{\Ker}{\mathop{\mathrm{Ker}}\nolimits}
%\newcommand{\Hom}{\mathop{\mathrm{Hom}}\nolimits}

%\DeclareMathOperator{\Rot}{rot}
%\DeclareMathOperator{\Div}{div}
%\DeclareMathOperator{\Grad}{grad}
%\DeclareMathOperator{\arcsinh}{arcsinh}

\begin{document}
%
%\textbf{問題14.1}
%
%$V=\{ f\in\mathbb{C}[X] \mid \deg{f}\leq n \}$
%とおく。
%このとき、
%$\mathbb{C}$-線形写像$\varphi$のJordan標準形を求めよ。
%(Jordan細胞$J_n(\lambda)$と直和の記号で答えよ。)
%\begin{equation}
% \varphi : V\to V \qquad f(X) \mapsto f(X+1)
%\end{equation}
%
%\dotfill
%
%$f(X)=\sum_{k=0}^{n}a_kX^k$とする。
%\begin{align}
% \varphi(f) =& f(X+1) = \sum_{k=0}^{n}a_k(X+1)^k\\
% =& \sum_{k=0}^{n}a_k \sum_{i=0}^{k} \frac{k!}{i!(k-i)!}X^i\\
%\end{align}
%





\hrulefill

\textbf{問題14.2}

行列$A$に対して$A^n$を求めよ。
\begin{equation}
 A=
  \begin{pmatrix}
   4 & 0 & -1 \\
   5 & 1 & -2 \\
   1 & 1 & 1 \\
  \end{pmatrix}
\end{equation}

\dotfill

$A$のJordan標準形と正則行列$P$は次のようになる。
\begin{equation}
 P=
  \begin{pmatrix}
   3 & 2 & 1 \\
   3 & 5 & 0 \\
   6 & 1 & 0
  \end{pmatrix}
  ,\qquad
  P^{-1}=
  \begin{pmatrix}
   0 & \frac{-1}{27} & \frac{5}{27} \\
   0 & \frac{2}{9} & \frac{-1}{9} \\
   1 & \frac{-1}{3} & \frac{-1}{3}
  \end{pmatrix}
  ,\qquad
  A=P
  \begin{pmatrix}
   2 & 1 & 0 \\
   0 & 2 & 1 \\
   0 & 0 & 2
  \end{pmatrix}
  P^{-1}
\end{equation}

Jordan標準形の$n$乗を考える。
\begin{equation}
 D=
  \begin{pmatrix}
   2 & 1 & 0 \\
   0 & 2 & 1 \\
   0 & 0 & 2
  \end{pmatrix}
  ,\qquad
 D^n=
  \begin{pmatrix}
   a_n & b_n & c_n \\
   0 & a_n & b_n \\
   0 & 0 & a_n
  \end{pmatrix}
\end{equation}
行列$D^n$の$0$でない成分を$a_n,b_n,c_n$とおき、
この数列の一般項を求める。

$a_n=2^n$であるので、$D^{n+1}$は次のようになる。
\begin{equation}
 D^{n+1}=
  \begin{pmatrix}
   2^n & b_n & c_n \\
   0 & 2^n & b_n \\
   0 & 0 & 2^n
  \end{pmatrix}
  \begin{pmatrix}
   2 & 1 & 0 \\
   0 & 2 & 1 \\
   0 & 0 & 2
  \end{pmatrix}
  =
  \begin{pmatrix}
   2^n & 2^n+2b_n & b_n+2c_n \\
   0 & 2^n & 2^n+2b_n \\
   0 & 0 & 2^n
  \end{pmatrix}
\end{equation}

初項$b_1=1$である漸化式$b_{n+1}=2^n+2b_n$を解く。
\begin{align}
 b_{n+1} =& 2^n+2b_n\\
 \frac{b_{n+1}}{2^{n+1}} =& \frac{2^n}{2^{n+1}}+\frac{2b_n}{2^{n+1}}
\end{align}
これにより数列$\frac{b_n}{2^n}$は初項$\frac{1}{2}$、公差$\frac{1}{2}$の等差数列である。
よって、$b_n$の一般項は
$b_n=2^{n-1}n$となる。

これを用いて
初項$c_1=0$である漸化式$c_{n+1}=2^{n-1}n+2c_n$を解く。
\begin{align}
 c_{n+1} =& 2^{n-1}n+2c_n\\
 \frac{c_{n+1}}{2^{n+1}} =& \frac{2^{n-1}n}{2^{n+1}}+\frac{2c_n}{2^{n+1}}\\
 \frac{c_{n+1}}{2^{n+1}} =& \frac{n}{4}+\frac{c_n}{2^{n}}
\end{align}
これにより階差数列$\frac{c_{n+1}}{2^{n+1}}-\frac{c_{n}}{2^{n}}$は
初項$\frac{1}{4}$、差$\frac{n}{4}$である。
よって、$c_n$の一般項は次のように求まる。
\begin{align}
 \frac{c_{n}}{2^{n}}
 =& \frac{c_1}{2^1} + \sum_{k=1}^{n-1}\frac{n}{4}
 = \frac{n(n-1)}{8}\\
 c_n =& 2^{n-3}n(n-1)
\end{align}

よって、
$A^n$は次のように計算できる。
\begin{equation}
 A^n = P
  \begin{pmatrix}
   2 & 1 & 0 \\
   0 & 2 & 1 \\
   0 & 0 & 2
  \end{pmatrix}^n
  P^{-1}
  =
  P
  \begin{pmatrix}
   2^n & 2^{n-1}n & 2^{n-3}n(n-1) \\
   0 & 2^n & 2^{n-1}n \\
   0 & 0 & 2^n
  \end{pmatrix}
  P^{-1}
\end{equation}


\hrulefill

\textbf{問題 14.3}

$A$を$n$次複素正方行列とする。
$A$の固有値を$\lambda_1,\dots,\lambda_n$(重複あり)とし、
$\rho(A)$を次のように定める。
\begin{equation}
 \rho(A) = \max \{\lvert \lambda_1 \rvert ,\dots, \lvert \lambda_n \rvert \}
\end{equation}

$\rho(A)<1$のとき、$\lim_{k\rightarrow \infty} A^k =0$であることを示せ。

\dotfill


$A$が対角か可能である場合、
$A^k$は次のように表せる。
\begin{equation}
 A^k=P
  \begin{pmatrix}
   \lambda_1^k & \cdots & 0\\
   \vdots & \ddots & \vdots\\
   0 & \cdots & \lambda_n^k
  \end{pmatrix}
  P^{-1}
\end{equation}
$\rho(A)<1$より
対角行列の$n$乗は$k$が大きくなると$0$に収束する。
よって、
$\lim_{k\rightarrow \infty} A^k =0$
となる。

$A$が対角か可能でない場合、
Jordan標準形を考える。
\begin{equation}
 A^k=P
  \begin{pmatrix}
   J_{m_1}(\lambda_1)^k & \cdots & 0\\
   \vdots & \ddots & \vdots\\
   0 & \cdots & J_{m_n}(\lambda_n)^k
  \end{pmatrix}
  P^{-1}
  \label{jordanform}
\end{equation}
$A^k$はJordan細胞の$k$乗という形で表現できる。

固有値$0$のJordan細胞$J_m(0)$は冪零行列であるので、
固有値は$0$でない場合を考える。

$J_m(\lambda)^k$の各成分を計算する。
\begin{equation}
 J_m(\lambda)^k
  =
  \begin{pmatrix}
   \lambda & 1 & 0 & \cdots & 0\\
   0 & \ddots & \ddots & \ddots & \vdots\\
   \vdots & \ddots & \ddots & \ddots & 0\\
   \vdots &  & \ddots & \ddots & 1\\
   0 & \cdots & \cdots & 0 & \lambda
  \end{pmatrix}^k
  =
  \begin{pmatrix}
   a_{(1)_k} & a_{(2)_k} & a_{(3)_k} & \cdots & a_{(m)_k}\\
   0 & \ddots & \ddots & \ddots & \vdots\\
   \vdots & \ddots & \ddots & \ddots & a_{(3)_k}\\
   \vdots &  & \ddots & \ddots & a_{(2)_k}\\
   0 & \cdots & \cdots & 0 & a_{(1)_k}
  \end{pmatrix}
\end{equation}
左下は常に0であるので、右上の成分を数列$a_{(i)_k} \ (i=1,\dots,m)$で表している。
これを利用し$k+1$乗を計算する。
\begin{align}
 J_m(\lambda)^{k+1}
  =&
  \begin{pmatrix}
   a_{(1)_k} & a_{(2)_k} & a_{(3)_k} & \cdots & a_{(m)_k}\\
   0 & \ddots & \ddots & \ddots & \vdots\\
   \vdots & \ddots & \ddots & \ddots & a_{(3)_k}\\
   \vdots &  & \ddots & \ddots & a_{(2)_k}\\
   0 & \cdots & \cdots & 0 & a_{(1)_k}
  \end{pmatrix}
  \begin{pmatrix}
   \lambda & 1 & 0 & \cdots & 0\\
   0 & \ddots & \ddots & \ddots & \vdots\\
   \vdots & \ddots & \ddots & \ddots & 0\\
   \vdots &  & \ddots & \ddots & 1\\
   0 & \cdots & \cdots & 0 & \lambda
  \end{pmatrix}\\
  =&
  \begin{pmatrix}
   \lambda a_{(1)_k} & a_{(1)_k} + \lambda a_{(2)_k} & a_{(2)_k} + \lambda a_{(3)_k} & \cdots & a_{(m-1)_k} + \lambda a_{(m)_k}\\
   0 & \ddots & \ddots & \ddots & \vdots\\
   \vdots & \ddots & \ddots & \ddots & a_{(2)_k} + \lambda a_{(3)_k}\\
   \vdots &  & \ddots & \ddots & a_{(1)_k} + \lambda a_{(2)_k}\\
   0 & \cdots & \cdots & 0 & \lambda a_{(1)_k}
  \end{pmatrix}
\end{align}

ここから次の漸化式を解くことで$J_m(\lambda)^k$の成分が求まる。
\begin{itemize}
 \item $a_{(1)_{k+1}} = \lambda a_{(1)_k}$、\quad 初項$a_{(1)_1}=\lambda$
 \item $a_{(2)_{k+1}} = a_{(1)_k} + \lambda a_{(2)_k}$、 \quad 初項$a_{(2)_1}=1$
 \item $a_{(3)_{k+1}} = a_{(2)_k} + \lambda a_{(3)_k}$、 \quad 初項$a_{(3)_1}=0$
 \item $a_{(4)_{k+1}} = a_{(3)_k} + \lambda a_{(4)_k}$ 、\quad 初項$a_{(4)_1}=0$

       $\vdots$
 \item $a_{(m)_{k+1}} = a_{(m-1)_k} + \lambda a_{(m)_k}$ 、\quad 初項$a_{(m)_1}=0$
\end{itemize}

数列の一般項は次のようになる。
\begin{align}
 a_{(1)_{k}} =& \lambda^k &
 a_{(2)_{k}} =& k\lambda^{k-1} &
 a_{(3)_{k}} =& \frac{1}{2}k(k-1)\lambda^{k-2}
\end{align}
これらは$\lvert \lambda \rvert <1$の範囲で、
$k$を大きくすると0に収束する。

$A$が$n$次正方行列であるので、
数列$a_{(i)_k}$は最大で$n$種類($a_{(1)_{k}},a_{(2)_{k}},\dots,a_{(n)_{k}}$)であるので、
$a_{(n)_{k}}$の一般項は$k^n$の式と$\lambda^k$の式の積で表される。
(ランダウの記号で書けば$O(k^n)$と$O(\lambda^k)$)

よって、
$J_m(\lambda)^k$の各成分は$k$を大きくすると0に収束する。
つまり次の式が成り立つ。
\begin{equation}
 \lim_{k\rightarrow \infty}J_m(\lambda)^k =0
\end{equation}

よって、
式(\ref{jordanform})の
Jordan細胞は零行列に収束する。

つまり、次が成り立つ。
\begin{equation}
 \rho(A)<1 \Rightarrow \lim_{k\rightarrow \infty} A^k =0
\end{equation}

\hrulefill

\end{document}
