\documentclass[12pt,b5paper]{ltjsarticle}

%\usepackage[margin=15truemm, top=5truemm, bottom=5truemm]{geometry}
\usepackage[margin=10truemm]{geometry}

\usepackage{amsmath,amssymb}
%\pagestyle{headings}
\pagestyle{empty}

%\usepackage{listings,url}
%\renewcommand{\theenumi}{(\arabic{enumi})}

%\usepackage{graphicx}

%\usepackage{tikz}
%\usetikzlibrary {arrows.meta}
%\usepackage{wrapfig}	% required for `\wrapfigure' (yatex added)
%\usepackage{bm}	% required for `\bm' (yatex added)

% ルビを振る
%\usepackage{luatexja-ruby}	% required for `\ruby'

%% 核Ker 像Im Hom を定義
%\newcommand{\Img}{\mathop{\mathrm{Im}}\nolimits}
%\newcommand{\Ker}{\mathop{\mathrm{Ker}}\nolimits}
%\newcommand{\Hom}{\mathop{\mathrm{Hom}}\nolimits}

%\DeclareMathOperator{\Rot}{rot}
%\DeclareMathOperator{\Div}{div}
%\DeclareMathOperator{\Grad}{grad}
%\DeclareMathOperator{\arcsinh}{arcsinh}
%\DeclareMathOperator{\arccosh}{arccosh}
%\DeclareMathOperator{\arctanh}{arctanh}



%\usepackage{listings,url}
%
%\lstset{
%%プログラム言語(複数の言語に対応,C,C++も可)
%%  language = Python,
%  language = Lisp,
%  %背景色と透過度
%  %backgroundcolor={\color[gray]{.90}},
%  %枠外に行った時の自動改行
%  breaklines = true,
%  %自動改行後のインデント量(デフォルトでは20[pt])
%  breakindent = 10pt,
%  %標準の書体
%%  basicstyle = \ttfamily\scriptsize,
%  basicstyle = \ttfamily,
%  %コメントの書体
%%  commentstyle = {\itshape \color[cmyk]{1,0.4,1,0}},
%  %関数名等の色の設定
%  classoffset = 0,
%  %キーワード(int, ifなど)の書体
%%  keywordstyle = {\bfseries \color[cmyk]{0,1,0,0}},
%  %表示する文字の書体
%  %stringstyle = {\ttfamily \color[rgb]{0,0,1}},
%  %枠 "t"は上に線を記載, "T"は上に二重線を記載
%  %他オプション:leftline,topline,bottomline,lines,single,shadowbox
%  frame = TBrl,
%  %frameまでの間隔(行番号とプログラムの間)
%  framesep = 5pt,
%  %行番号の位置
%  numbers = left,
%  %行番号の間隔
%  stepnumber = 1,
%  %行番号の書体
%%  numberstyle = \tiny,
%  %タブの大きさ
%  tabsize = 4,
%  %キャプションの場所("tb"ならば上下両方に記載)
%  captionpos = t
%}



\begin{document}



\hrulefill
\textbf{二面体群}
\hrulefill

6次二面体群$D_6$は次のように定義される。
\begin{equation}
 D_{6} = \langle \alpha , \beta \mid \alpha^{6}=\beta^{2}=e ,\ \alpha\beta=\beta\alpha^{-1} \rangle
\end{equation}
$e$は単位元、$\alpha$は回転、$\beta$は鏡映を意味している。

具体的な$D_{6}$の元は次の通り。
\begin{equation}
 D_{6}=
  \{ e,\ \alpha,\ \alpha^2,\ \alpha^3,\ \alpha^4,\ \alpha^5,\
  \beta,\ \alpha\beta,\ \alpha^2\beta,\
  \alpha^3\beta,\ \alpha^4\beta,\ \alpha^5\beta \}
\end{equation}


\hrulefill
\textbf{問題}
\hrulefill

6次二面体群$D_6$の
部分群を全て求めよ。

\dotfill



$D_{6}$
の元の位数
\begin{align}
 \textbf{[位数 1]} &\
 e
 &
 \textbf{[位数 2]} &\
 \alpha^3,\beta,\alpha\beta,\alpha^2\beta,
 \alpha^3\beta,\alpha^4\beta,\alpha^5\beta
 \\
 \textbf{[位数 3]} &\
 \alpha^2,\alpha^4
 &
% \textbf{[位数 4]} &\
% -
% &
 \textbf{[位数 6]} &\
 \alpha,\alpha^5
\end{align}

元の位数から巡回群は次の10個になる。
\begin{align}
 \textbf{[位数 1]} &\
 \{ e \}
 \\
 \textbf{[位数 2]} &\
 \{ e, \alpha^3 \}, \{ e, \beta \}, \{ e, \alpha\beta \},
 \{ e, \alpha^2\beta \}, \{ e, \alpha^3\beta \},
 \{ e, \alpha^4\beta\}, \{ e, \alpha^5\beta \}
 \\
 \textbf{[位数 3]} &\
  \{ e, \alpha^2, \alpha^4 \}
 \\
 \textbf{[位数 6]} &\
 \{ e, \alpha, \alpha^2, \alpha^3, \alpha^4, \alpha^5 \}
\end{align}


部分群$\langle \alpha^k, \alpha^l \rangle$は
$\alpha^{k+l} \in \langle \alpha^k,\alpha^l \rangle$である。
これにより$\langle \alpha^{k+l} \rangle \subset \langle \alpha^k,\alpha^l \rangle$である。
$\beta$を含まない部分群のうち最大のものは位数6の$\langle \alpha \rangle$である。
この為、$\alpha$や$\alpha^5$を含む部分群$\langle \alpha^k, \alpha^l \rangle$は$\langle \alpha \rangle$と一致する。
$\alpha$や$\alpha^5$を含まないのは
$\langle \alpha^2,\alpha^4 \rangle$のみだが、
$\langle \alpha^2,\alpha^4 \rangle = \langle \alpha^2 \rangle$
である。
これらにより部分群$\langle \alpha^k, \alpha^l \rangle$は
巡回群のどれかと一致する。
これは生成元が3つ以上の場合($\langle \alpha^l,\alpha^m,\alpha^n \rangle$)でも同様である。



次に$\alpha^k$と$\beta$を含む元から生成される部分群を考える。

$\langle \alpha^k,\beta \rangle$の部分群は
$\langle \alpha^k \rangle$に$\beta$を付け加えた部分群になる。
$\langle \alpha^k \rangle$は3種類あったので、
つぎの3つとなる。
\begin{align}
 \langle \alpha,\beta \rangle =& D_{6}\\
 \langle \alpha^2,\beta \rangle =& \{e, \alpha^2, \alpha^4, \beta, \alpha^2\beta, \alpha^4\beta \}\\
 \langle \alpha^3,\beta \rangle =& \{e, \alpha^3, \beta, \alpha^3\beta \}
% \langle \alpha^4,\beta \rangle =& \langle \alpha^2,\beta \rangle\\
% \langle \alpha^5,\beta \rangle =& \langle \alpha,\beta \rangle\\
\end{align}


$\langle \alpha^k,\alpha^l\beta \rangle$の形の部分群は
元$\beta$を含めば上の3つのどれかと一致する。
この為、上と一致しない部分群は次の3つになる。
\begin{align}
 \langle \alpha^2,\alpha\beta \rangle =& \{e, \alpha^2, \alpha^4, \alpha\beta, \alpha^3\beta, \alpha^5\beta \}\\
 \langle \alpha^3,\alpha\beta \rangle =& \{e, \alpha^3, \alpha\beta, \alpha^4\beta \}\\
 \langle \alpha^3,\alpha^2\beta \rangle =& \{e, \alpha^3, \alpha^2\beta, \alpha^5\beta \}
\end{align}


$\langle \alpha^k\beta, \alpha^l\beta \rangle$の形の部分群は
必ず$\alpha^{k-l}$を含む。
\begin{equation}
  \alpha^k\beta \alpha^l\beta
  = \beta\alpha^{-k} \alpha^l\beta
  = \beta \alpha^{l-k}\beta
  = \alpha^{k-l}
  \label{alpha}
\end{equation}
この為、上記6つの部分群のどれかと一致する。


3つ以上の元から生成される部分群について考える。

$\langle \alpha^k,\alpha^l,\alpha^m \rangle$
であれば巡回群となる。
$\beta$を含む元を生成元としてもつ場合($\langle \alpha^k\beta,\alpha^l\beta,\alpha^m\beta \rangle$等)、
式(\ref{alpha})より$\alpha^k$の形の元が含まれる。
逆元が含まれるので、
$\alpha^k$は$\alpha,\alpha^2\alpha^3$のどれかである。
$\beta$が生成される場合、
$\langle \alpha,\beta \rangle, \langle \alpha^2,\beta \rangle, \langle \alpha^3,\beta \rangle$のどれかと一致する。
$\beta$が生成されない場合、
$\langle \alpha^2,\alpha\beta \rangle, \langle \alpha^3,\alpha\beta \rangle, \langle \alpha^3,\alpha^2\beta \rangle$のどれかと一致する。



以上により$D_{6}$の部分群は次の16個である。
\begin{align}
 \textbf{[位数 1]} &\
 \{ e \}
 \\
 \textbf{[位数 2]} &\
 \{ e, \alpha^3 \}, \{ e, \beta \}, \{ e, \alpha\beta \},
 \{ e, \alpha^2\beta \}, \{ e, \alpha^3\beta \},
 \{ e, \alpha^4\beta\}, \{ e, \alpha^5\beta \}
 \\
 \textbf{[位数 3]} &\
  \{ e, \alpha^2, \alpha^4 \}
 \\
 \textbf{[位数 4]} &\
 \{e, \alpha^3, \beta, \alpha^3\beta \},
 \{e, \alpha^3, \alpha\beta, \alpha^4\beta \},
 \{e, \alpha^3, \alpha^2\beta, \alpha^5\beta \}
 \\
 \textbf{[位数 6]} &\
 \{ e, \alpha, \alpha^2, \alpha^3, \alpha^4, \alpha^5 \}\\
&\ \{e, \alpha^2, \alpha^4, \beta, \alpha^2\beta, \alpha^4\beta \},
 \{e, \alpha^2, \alpha^4, \alpha\beta, \alpha^3\beta, \alpha^5\beta \}
 \\
 \textbf{[位数 12]} &\
 D_{6}
\end{align}

\hrulefill

\end{document}
