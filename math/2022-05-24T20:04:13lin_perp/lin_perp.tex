\documentclass[12pt,b5paper]{ltjsarticle}

%\usepackage[margin=15truemm, top=5truemm, bottom=5truemm]{geometry}
\usepackage[margin=15truemm]{geometry}

\usepackage{amsmath,amssymb}
%\pagestyle{headings}
\pagestyle{empty}

%\usepackage{listings,url}
%\renewcommand{\theenumi}{(\arabic{enumi})}

\usepackage{graphicx}

\usepackage{tikz}
\usetikzlibrary {arrows.meta}
\usepackage{wrapfig}	% required for `\wrapfigure' (yatex added)
\usepackage{bm}	% required for `\bm' (yatex added)
\usepackage{luatexja-ruby}	% required for `\ruby'
%% 像Im を定義
%\newcommand{\Img}{\mathop{\mathrm{Im}}\nolimits}

\begin{document}

\begin{enumerate}
 \item
      $A$を$n$次実対称行列とする。
      $A$の固有値が全て正の場合、$A$の全ての対角成分は正となることを示せ。
 \item
      $K$を体とし、$n>0$を整数とする。
      $V\subset K^n$を部分空間、
      $K^n \times K^n \rightarrow K$を $(x,y)={}^{t}xy$(標準内積)で定義する。
      この時、
      \begin{equation}
       V^{\perp}=\{
        x\in K^n \mid {}^{\forall} v\in V ,\ (x,v)=0
        \}
      \end{equation}

      \begin{enumerate}
       \item
            $n=2$かつ$\dim_{K}V = 1$の時、
            $V=V^{\perp}$となる例をあげよ。
       \item
            $\dim_{K}V + \dim_{K}V^{\perp} =n$を示せ。
      \end{enumerate}
\end{enumerate}

\hrulefill


対称行列
$\Leftrightarrow$
ある直交行列$P$にて対角化可能

$P$が直交行列
$\stackrel{\mathrm{def}}{\Leftrightarrow}$
$P^{-1} = {}^{t}P$


$A$は対称行列とする。

\textbf{定義}\qquad
$A$が正定値
$\stackrel{\mathrm{def}}{\Leftrightarrow}$
任意のベクトル$\bm{x}$に対し、${}^{t}\bm{x}A\bm{x} >0$

$A$が正定値$\Leftrightarrow$$A$の固有値は全て正

$A$が正定値$\Rightarrow$$A$の対角成分は全て正

\dotfill

\textbf{直交補空間}

$n$次元ベクトル空間$V$の部分空間$U$について
次の$U^{\perp}$を直交補空間という。
\begin{equation}
 U^{\perp} =\{\bm{x}\in V \mid {}^{\forall}\bm{u}\in U ,\ \bm{x}\cdot\bm{u}=0 \}
\end{equation}

性質

\begin{enumerate}
 \item $V=U\oplus U^{\perp}$
 \item $n =\dim  U + \dim U^{\perp}$
 \item $U \cap U^{\perp} =\{\bm{0}\}$
\end{enumerate}

\hrulefill

%固有値(eigenvalue)

\begin{enumerate}
 \item
      $A$は対称行列なので、
      直交行列$P$を用いて対角化可能である。
      \begin{equation}
       {}^{t}PAP= \Lambda,
        \qquad
        \Lambda =
        \begin{pmatrix}
         \lambda_1 & 0 & \cdots & 0\\
         0 & \lambda_2 & \ddots & \vdots\\
         \vdots & \ddots & \ddots & 0\\
         0 & \cdots & 0 & \lambda_n
        \end{pmatrix}
       \quad (\lambda_i \text{は固有値})
      \end{equation}

      ${}^{\forall}\bm{x}$に対し、
      $\bm{y}=P^{-1}\bm{x}$とする。
      つまり、$\bm{x}=P\bm{y}$となる。

      この時、${}^{t}\bm{x}A\bm{x}$を計算する。
      \begin{equation}
       {}^{t}\bm{x}A\bm{x}
        = {}^{t}(P\bm{y})AP\bm{y}
        = {}^{t}\bm{y}{}^{t}PAP\bm{y}
        = {}^{t}\bm{y}\Lambda\bm{y}
        = \sum_{i=1}^{n}\lambda_i \lvert y_i \rvert^2
        > 0
        \label{pos}
      \end{equation}

      $y_i$はベクトル$\bm{y}$の$i$成分である。
      固有値$\lambda_i$は全て正である為、
      ${}^{t}\bm{x}A\bm{x}>0$である。

      $\bm{x}$を第$i$成分のみ$1$でそれ以外が$0$のベクトルとする。
      この時、${}^{t}\bm{x}A\bm{x}=a_{ii}$となるので、
      式(\ref{pos})より$a_{ii}>0$である。

 \item
      \begin{enumerate}
       \item
            $V \cap V^{\perp} = \{\bm{0}\}$より、
            $\dim_{K}V = 1$ かつ $V=V^{\perp}$となるものは存在しない。

            もし、$V=V^{\perp}$が同型という意味であるとする。

            同型であれば、
            $V \simeq V^{\perp}$ とか
            $V \cong V^{\perp}$とか
            $V \approx V^{\perp}$で表す。
            代数だと$V \cong V^{\perp}$と書くことが多い。

            \begin{align}
             V= \{ (k,0) \in K^2 \mid  k\in K\}\\
             V^{\perp} = \{ (0,k) \in K^2 \mid  k\in K\}
            \end{align}
            とすれば
            $\dim_{K} V = 1$であり、$V\cong V^{\perp} \cong K$である。

       \item
            $\dim_{K}V + \dim_{K}V^{\perp} =n$

            $r=\dim_{K}V$とする。
            
            $V$の正規直交基底を$v_1,\dots,v_r$とする。
            この基底に$n-r$個の基底を追加し
            $K^n$の基底とする。
            直交化法により$K^n$の正規直交基底を
            $v_1,\dots,v_r,v_{r+1},\dots,v_n$とする。

            追加して基底$v_{r+1},\dots,v_n$を基底として出来る
            部分空間を$V^{\prime}=\langle v_{r+1},\dots,v_n \rangle$
            とすると、
            $V^{\prime}$の任意の元は$V$の元と直交する為、
            $V^{\prime} \subset V^{\perp}$となる。

            逆に$V^{\prime} \supset V^{\perp}$を示す。

            ${}^{\forall}\bm{v}\in V^{\perp}$とする。
            $\bm{v}\in K^{n}$より$\bm{v}=\sum_{i=1}^{n}k_iv_i$と基底を用いて表せる。
            $\bm{v}\in V^{\perp}$より$v_{i}\in V \ (i=1,\dots,r)$との内積は$0$である。
            \begin{equation}
             0=v_{i}\cdot\bm{v}=v_{i}\cdot\sum_{i=1}^{n}k_iv_i = k_i
            \end{equation}
            $k_i=0\ (i=1,\dots,r)$である為、
            $\bm{v}\in V^{\prime}$となり、$V^{\prime} \supset V^{\perp}$である。

            これにより$\dim_{K}V^{\perp}=n-r$となることが分かる。
      \end{enumerate}
\end{enumerate}












\end{document}
