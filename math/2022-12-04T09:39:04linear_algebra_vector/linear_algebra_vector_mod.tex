\documentclass[12pt,b5paper]{ltjsarticle}

%\usepackage[margin=15truemm, top=5truemm, bottom=5truemm]{geometry}
%\usepackage[margin=10truemm,left=15truemm]{geometry}
\usepackage[margin=10truemm]{geometry}

\usepackage{amsmath,amssymb}
%\pagestyle{headings}
\pagestyle{empty}

%\usepackage{listings,url}
%\renewcommand{\theenumi}{(\arabic{enumi})}

%\usepackage{graphicx}

%\usepackage{tikz}
%\usetikzlibrary {arrows.meta}
%\usepackage{wrapfig}	% required for `\wrapfigure' (yatex added)
\usepackage{bm}	% required for `\bm' (yatex added)

% ルビを振る
\usepackage{luatexja-ruby}	% required for `\ruby'

%% 核Ker 像Im Hom を定義
%\newcommand{\Img}{\mathop{\mathrm{Im}}\nolimits}
%\newcommand{\Ker}{\mathop{\mathrm{Ker}}\nolimits}
%\newcommand{\Hom}{\mathop{\mathrm{Hom}}\nolimits}

%\DeclareMathOperator{\Rot}{rot}
%\DeclareMathOperator{\Div}{div}
%\DeclareMathOperator{\Grad}{grad}
%\DeclareMathOperator{\arcsinh}{arcsinh}
%\DeclareMathOperator{\arccosh}{arccosh}
%\DeclareMathOperator{\arctanh}{arctanh}



%\usepackage{listings,url}
%
%\lstset{
%%プログラム言語(複数の言語に対応,C,C++も可)
%  language = Python,
%%  language = Lisp,
%%  language = C,
%  %背景色と透過度
%  %backgroundcolor={\color[gray]{.90}},
%  %枠外に行った時の自動改行
%  breaklines = true,
%  %自動改行後のインデント量(デフォルトでは20[pt])
%  breakindent = 10pt,
%  %標準の書体
%%  basicstyle = \ttfamily\scriptsize,
%  basicstyle = \ttfamily,
%  %コメントの書体
%%  commentstyle = {\itshape \color[cmyk]{1,0.4,1,0}},
%  %関数名等の色の設定
%  classoffset = 0,
%  %キーワード(int, ifなど)の書体
%%  keywordstyle = {\bfseries \color[cmyk]{0,1,0,0}},
%  %表示する文字の書体
%  %stringstyle = {\ttfamily \color[rgb]{0,0,1}},
%  %枠 "t"は上に線を記載, "T"は上に二重線を記載
%  %他オプション:leftline,topline,bottomline,lines,single,shadowbox
%  frame = TBrl,
%  %frameまでの間隔(行番号とプログラムの間)
%  framesep = 5pt,
%  %行番号の位置
%  numbers = left,
%  %行番号の間隔
%  stepnumber = 1,
%  %行番号の書体
%%  numberstyle = \tiny,
%  %タブの大きさ
%  tabsize = 4,
%  %キャプションの場所("tb"ならば上下両方に記載)
%  captionpos = t
%}



\begin{document}

\hrulefill
\textbf{問題}
\hrulefill

次の行列$A$に対し、
$\mathrm{Im}L_A$
と
$\mathrm{Ker}L_A$
の基底を求めよ。

\dotfill


\begin{enumerate}
 \item
      $A=\begin{pmatrix} 0 & 1 \\ 1 & 2 \\ 2 & 3 \end{pmatrix}$

      \dotfill

      $L_A$とは次のような写像である。
      \begin{equation}
       L_A : K^2 \to K^3 ,\quad \bm{x} \to A\bm{x}
      \end{equation}

      $\bm{a}_i,\bm{x}$を次のように置く。
      \begin{equation}
       \bm{a}_1=\begin{pmatrix} 0 \\ 1 \\ 2 \end{pmatrix}
       ,\quad
       \bm{a}_2=\begin{pmatrix} 1 \\ 2 \\ 3 \end{pmatrix}
       ,\quad
       \bm{x}=\begin{pmatrix} x_1 \\ x_2 \end{pmatrix}
      \end{equation}

      $x_i\in K$に対して$\mathrm{Im} L_A$
      を考える。
      \begin{equation}
       L_A(\bm{x})=A\bm{x}
        =\begin{pmatrix}
          \bm{a}_1&\bm{a}_2
         \end{pmatrix}\bm{x}
         = x_1\bm{a}_1 + x_2\bm{a}_2
         \label{form}
      \end{equation}

      $\bm{a}_1,\bm{a}_2$は線形独立であるので
      これが$\mathrm{Im}L_A$の基底となる。
      \begin{equation}
       \mathrm{Im}L_A
        = \langle \bm{a}_1,\bm{a}_2 \rangle
        =\left\langle
          \begin{pmatrix} 0 \\ 1 \\ 2 \end{pmatrix}
          ,
          \begin{pmatrix} 1 \\ 2 \\ 3 \end{pmatrix}
         \right\rangle
      \end{equation}

      $\mathrm{Ker}L_A$は
      $L_A(\bm{x})=0$を満たす
      $\bm{x}\in K^2$全体の集合である。

      式(\ref{form})より
      $x_1\bm{a}_1 + x_2\bm{a}_2=\bm{0}$
      の解空間の基底を求める。
      左辺を計算すると次のようになる。
      \begin{equation}
       x_1\bm{a}_1 + x_2\bm{a}_2
          = \begin{pmatrix} x_2 \\ x_1+2x_2 \\ 2x_1+ 3x_2 \end{pmatrix}
      \end{equation}

      これは成分ごとに0となるときは$x_1=0,x_2=0$となる。

      よって、$\mathrm{Ker}L_A$は次のように生成される。
      \begin{equation}
       \mathrm{Ker}L_A
        = \left\langle \bm{0} \right\rangle
        = \{\bm{0}\}
      \end{equation}

      \hrulefill

 \item
      $A=\begin{pmatrix} 1 & 2 & 3 \\ 1 & 2 & 3 \\ 1 & 2 & 3 \end{pmatrix}$

      \dotfill

      $A$による線形写像$L_A$は次のような写像である。
      \begin{equation}
       L_A : K^3 \to K^3 ,\quad \bm{x} \to A\bm{x}
      \end{equation}

      行列$A$を列ベクトル$\bm{a}_i$に分け、$\bm{x}$を次のように置く。
      \begin{equation}
        \bm{a}_1= \begin{pmatrix} 1 \\ 1 \\ 1 \end{pmatrix}
        ,\quad
        \bm{a}_2= \begin{pmatrix} 2 \\ 2 \\ 2 \end{pmatrix}
        ,\quad
        \bm{a}_3= \begin{pmatrix} 3 \\ 3 \\ 3 \end{pmatrix}
        ,\quad
        \bm{x}=\begin{pmatrix} x_1 \\ x_2 \\ x_3 \end{pmatrix}
      \end{equation}

      $\bm{a}_1,\bm{a}_2,\bm{a}_3$は一次従属であり、
      $\bm{a}_2=2\bm{a}_1,\ \bm{a}_3=3\bm{a}_1$となるので
      $A\bm{x}$は次のように計算できる。
      \begin{equation}
       A\bm{x}=
        \begin{pmatrix} \bm{a}_1 & \bm{a}_2 & \bm{a}_3 \end{pmatrix}
        \bm{x}
         =x_1\bm{a}_1 + x_2\bm{a}_2 + x_3 \bm{a}_3
         =(x_1+2x_2+3x_3)\bm{a}_1
         \label{form2}
      \end{equation}

      これにより
      $\mathrm{Im}L_A$は次のようになる。
      \begin{equation}
       \mathrm{Im}L_A
        =\left\langle \bm{a}_1 \right\rangle
        =\left\langle
          \begin{pmatrix} 1 \\ 1 \\ 1 \end{pmatrix}
         \right\rangle
      \end{equation}

      $\mathrm{Ker}L_A$は
      式(\ref{form2})が
      $A\bm{x}=\bm{0}$になるような$\bm{x}$の
      解空間を求めればよい。
      式(\ref{form2})より、$x_1+2x_2+3x_3=0$となるときの
      $\bm{x}$について調べる。
      \begin{equation}
       \bm{x}=
        \begin{pmatrix} x_1 \\ x_2 \\ x_3 \end{pmatrix}
        = \begin{pmatrix} -2x_2-3x_3 \\ x_2 \\ x_3 \end{pmatrix}
        = x_2 \begin{pmatrix} -2 \\ 1 \\ 0 \end{pmatrix}
        + x_3 \begin{pmatrix} -3 \\ 0 \\ 1 \end{pmatrix}
      \end{equation}

      よって、$\mathrm{Ker}L_A$は次のように生成される。
      \begin{equation}
       \mathrm{Ker}L_A
        = \left\langle
           \begin{pmatrix} -2 \\ 1 \\ 0 \end{pmatrix},
           \begin{pmatrix} -3 \\ 0 \\ 1 \end{pmatrix}
          \right\rangle
      \end{equation}



\hrulefill

\end{enumerate}



\end{document}
