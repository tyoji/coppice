\documentclass[12pt,b5paper]{ltjsarticle}

%\usepackage[margin=15truemm, top=5truemm, bottom=5truemm]{geometry}
\usepackage[margin=15truemm]{geometry}

\usepackage{amsmath,amssymb}
%\pagestyle{headings}
\pagestyle{empty}

%\usepackage{listings,url}
\renewcommand{\theenumi}{(\arabic{enumi})}

\usepackage{graphicx}

\usepackage{tikz}
\usetikzlibrary {arrows.meta}
\usepackage{wrapfig}	% required for `\wrapfigure' (yatex added)
\begin{document}



\textbf{数列の極限}

数列 $\{a_k\}_{k=0,1,2,\dots ,n,\dots}$の極限値$\displaystyle \lim_{n\rightarrow\infty}a_n$
が$\alpha$に収束することを
\[
 \lim_{n\rightarrow\infty}a_n = \alpha
\]
と書く。

\dotfill

$\displaystyle \lim_{n\rightarrow\infty}a_n = \alpha$
の定義を次のように定める。
\begin{equation}
 {}^\forall \varepsilon > 0 ,
 {}^\exists N_{\varepsilon}\in \mathbb{N}
 \ s.t. \ {}^\forall k > N_\varepsilon ,
 \lvert a_k - \alpha \rvert< \varepsilon\label{220910_17Apr22}
\end{equation}

\dotfill

\[
 {}^\forall \varepsilon \in \{ \varepsilon \in \mathbb{R} \mid \varepsilon > 0 \},
 %{}^\forall \varepsilon > 0 ,
 {}^\exists N_{\varepsilon}\in \mathbb{N}
 \ s.t. \ {}^\forall k \in \{ k \in \mathbb{N} \mid k > N_\varepsilon \},
 \lvert a_k - \alpha \rvert< \varepsilon
\]

\dotfill

\textbf{解説}

\begin{equation}
 {}^\forall \varepsilon > 0 ,
 {}^\exists N_{\varepsilon}\in \mathbb{N}
 \ s.t. \ {}^\forall k > N_\varepsilon ,
 \lvert a_k - \alpha \rvert< \varepsilon
\end{equation}

上の式は s.t. (such that) で分けられる。

前半部分は次のような意味になる。

\begin{equation}
 {}^\forall \varepsilon > 0 ,
  {}^\exists N_{\varepsilon}\in \mathbb{N}
\end{equation}
「正の実数 $\varepsilon$を好きに取ってくると
条件を満たすある自然数$N_{\varepsilon}$が必ず存在する。」


この条件が後半部分で示されている

\begin{equation}
 {}^\forall k > N_\varepsilon ,
 \lvert a_k - \alpha \rvert< \varepsilon
\end{equation}
「$N_{\varepsilon}$よりも大きな自然数$k$を任意に取ってきた時、
$a_k$と$\alpha$との差が$\varepsilon$より小さくなる。」



つまり、(\ref{220910_17Apr22}) の式は次のようなことを意味します。

どんなに0に近い実数$\varepsilon$を取ってきたとしても
数列$\{a_k\}$のずっと先の
あるところ以降は
全て$\alpha$に近い値になっている








\end{document}
