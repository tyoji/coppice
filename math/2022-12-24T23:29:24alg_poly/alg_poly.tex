\documentclass[12pt,b5paper]{ltjsarticle}

%\usepackage[margin=15truemm, top=5truemm, bottom=5truemm]{geometry}
%\usepackage[margin=10truemm,left=15truemm]{geometry}
\usepackage[margin=10truemm]{geometry}

\usepackage{amsmath,amssymb}
%\pagestyle{headings}
\pagestyle{empty}

%\usepackage{listings,url}
%\renewcommand{\theenumi}{(\arabic{enumi})}

%\usepackage{graphicx}

%\usepackage{tikz}
%\usetikzlibrary {arrows.meta}
%\usepackage{wrapfig}	% required for `\wrapfigure' (yatex added)
%\usepackage{bm}	% required for `\bm' (yatex added)

% ルビを振る
%\usepackage{luatexja-ruby}	% required for `\ruby'

%% 核Ker 像Im Hom を定義
%\newcommand{\Img}{\mathop{\mathrm{Im}}\nolimits}
%\newcommand{\Ker}{\mathop{\mathrm{Ker}}\nolimits}
%\newcommand{\Hom}{\mathop{\mathrm{Hom}}\nolimits}

%\DeclareMathOperator{\Rot}{rot}
%\DeclareMathOperator{\Div}{div}
%\DeclareMathOperator{\Grad}{grad}
%\DeclareMathOperator{\arcsinh}{arcsinh}
%\DeclareMathOperator{\arccosh}{arccosh}
%\DeclareMathOperator{\arctanh}{arctanh}



%\usepackage{listings,url}
%
%\lstset{
%%プログラム言語(複数の言語に対応,C,C++も可)
%  language = Python,
%%  language = Lisp,
%%  language = C,
%  %背景色と透過度
%  %backgroundcolor={\color[gray]{.90}},
%  %枠外に行った時の自動改行
%  breaklines = true,
%  %自動改行後のインデント量(デフォルトでは20[pt])
%  breakindent = 10pt,
%  %標準の書体
%%  basicstyle = \ttfamily\scriptsize,
%  basicstyle = \ttfamily,
%  %コメントの書体
%%  commentstyle = {\itshape \color[cmyk]{1,0.4,1,0}},
%  %関数名等の色の設定
%  classoffset = 0,
%  %キーワード(int, ifなど)の書体
%%  keywordstyle = {\bfseries \color[cmyk]{0,1,0,0}},
%  %表示する文字の書体
%  %stringstyle = {\ttfamily \color[rgb]{0,0,1}},
%  %枠 "t"は上に線を記載, "T"は上に二重線を記載
%  %他オプション:leftline,topline,bottomline,lines,single,shadowbox
%  frame = TBrl,
%  %frameまでの間隔(行番号とプログラムの間)
%  framesep = 5pt,
%  %行番号の位置
%  numbers = left,
%  %行番号の間隔
%  stepnumber = 1,
%  %行番号の書体
%%  numberstyle = \tiny,
%  %タブの大きさ
%  tabsize = 4,
%  %キャプションの場所("tb"ならば上下両方に記載)
%  captionpos = t
%}



\begin{document}

\hrulefill

\textbf{多項式環}

可換環$R$と文字$x$
を用いた形式的な有限和
\begin{equation}
 a_0+a_1x+a_2x^2+\cdots+a_nx^n
  \qquad (a_i\in R)
\end{equation}
を$R$上の多項式という。

$R$上の多項式全体の集合を
多項式環$R[x]$と書く。

\hrulefill

\textbf{既約}

多項式環$R[x]$において、
$f\in R[x]$が可約であるとは、
$f=gh$となる単元ではない$g,h\in R[x]$が存在することをいう。

(単元とは逆元をもつ元のこと。)

$f$が可約でない時、既約であるという。

\hrulefill

$f\in\mathbb{R}[x]$を次のように定める。
$f$が$\mathbb{R}[x]$において既約であるか否かを判定せよ。
\begin{enumerate}
 \item $f=x+\sqrt{2}$

       \dotfill

       ある$g,h\in\mathbb{R}[x]$により
       $f=gh$とする。

       $\deg{f}=1$より$\deg{gh}=1$である。
       $\deg{g}\geq 0, \ \deg{h}\geq 0$であるので、
       $\deg{g}= 1, \ \deg{h}= 0$または
       $\deg{g}= 0, \ \deg{h}= 1$である。

       $\deg{h}=0$であれば、$h\in\mathbb{R}$である。
       $\mathbb{R}$の元は全て逆元を持つので
       $f=gh$となる場合、一方は単元となる。
       よって、$f$は既約となる。

       \hrulefill

 \item $f=2x+4$

       \dotfill

       上記の問いと同様に
       $\deg{f}=1$より既約である。

       $f=2(x+4)$

       \hrulefill

 \item $f=x^2+4x+5$

       \dotfill

       $f$が可約であるとする。
       可約であれば
       $\deg{g}\geq1,\ \deg{h}\geq1$となる
       $g, h\in\mathbb{R}[x]$が存在し、
       $f=gh$となる。

       $\deg{f}=2$より、$\deg{g}=\deg{h}=1$である。
       %$f$の$x^2$の係数が1であるので、
       1次多項式は$ax+b\ (a,b\in\mathbb{R})$
       の形をしているので、
       $f(x)=(ax+b)(cx+d) \ (a,b,c,d\in\mathbb{R})$
       となる。
       この時、$x=-\frac{b}{a},-\frac{d}{c}$を$f$に代入すると
       $f(x)=0$となる。

       しかし、$f=x^2+4x+5=(x+2)^2+1>0$であるので、
       $f(x)=0$となる実数は存在しない。

       よって、$f$は既約多項式である。

       \hrulefill

 \item $f=x^2-6x+8$

       \dotfill

       $f=(x - 2)(x - 4)$であり、
       $\deg{(x - 2)}=\deg{(x - 4)}=1$である為、
       $f$は既約な多項式の積で表せる。

       よって、$f$は可約である。

       \hrulefill

 \item $f=x^3+12x^2-x+1$

       \dotfill


       $f=x^2(x+12)-x+1$より次のように計算ができる。
       \begin{align}
       f(-12)&=(-12)^2(-12+12)-(-12)+1=13>0\\
       f(-13)&=(-13)^2(-13+12)-(-13)+1=-(-13)^2+14<0
       \end{align}
       これにより$-12$と$-13$の間に実数$a$が存在し、
       $f(a)=0$となる。

       代数学の基本定理により$\mathbb{C}$において
       $f$は複素係数の1次多項式の積に分解できる。
       この複素係数の1次式の一つは実数$a$を用いて$(x-a)$である。


       $\deg{f}=3$であるので、
       $x-a,g\in\mathbb{R}[x]$で
       $\deg{(x-a)}=1$、
       $\deg{g}=2$となる多項式により
       $f=(x-a)g$と分解される。

       よって、$f$は可約である。

       \hrulefill

 \item $f=x^4+12$

       \dotfill

%       $f=(x^2+2\sqrt{3}i)(x^2-2\sqrt{3}i)$
       $\deg{f}=4$であるので、
       $f=gh$と分解できるなら
       $\deg{g}=1,\deg{h}=3$か
       $\deg{g}=\deg{h}=2$である。

       任意の実数に対して
       $f=x^4+12>0$であるので、
       $f(a)=0$を満たす$a\in\mathbb{R}$は存在しない。
       つまり、$\deg{g}=1,\deg{h}=3$となる多項式の積に分けられない。

       %$X=x^2$とおくと$f=X^2+12$である。
       $\deg{g}=\deg{h}=2$の場合を考える。

       $a_0,a_1,b_0,b_1\in\mathbb{R}$とし、
       $g=x^2+a_1x+a_0,\ h=x^2+b_1x+b_0$とする。
       \begin{align}
        gh =& (x^2+a_1x+a_0)(x^2+b_1x+b_0)\\
         =&
         x^{4}
         + {\left(a_{1} + b_{1}\right)} x^{3}
         + {\left(a_{1} b_{1} + a_{0} + b_{0}\right)} x^{2}
         + {\left(a_{1} b_{0} + a_{0} b_{1}\right)} x
         + a_{0} b_{0}
       \end{align}

       $x^4+12=(x^2+a_1x+a_0)(x^2+b_1x+b_0)$
       と分解できるとしたら次の式を満たす必要がある。
       \begin{equation}
        \begin{cases}
         a_{1} + b_{1}=0\\
         a_{1} b_{1} + a_{0} + b_{0}=0\\
         a_{1} b_{0} + a_{0} b_{1}=0\\
         a_{0} b_{0}=12
        \end{cases}
       \end{equation}

       $a_{0} b_{0}=12$より
       $a_{0}\ne 0,\ b_{0}\ne0$に注意し
       $b_0,b_1$を消すと
       \begin{equation}
        \begin{cases}
         a_{1} (-a_{1}) + a_{0} + \frac{12}{a_{0}}=0\\
         a_{1} \frac{12}{a_{0}} + a_{0} (-a_{1})=0
        \end{cases}
        \ \Rightarrow \
        \begin{cases}
         -a_{0} a_{1}^2 + a_{0}^2 + 12 =0\\
         a_{1} (12 - a_{0}^2 )=0
        \end{cases}
       \end{equation}

       $a_{1}=0$であれば、
       $a_{0}^2+12=0$であるので$a_{0}\not\in\mathbb{R}$となる。

       $12-a_{0}^2=0$であれば$a_{0}=\pm 2\sqrt{3}$であるので、
       \begin{equation}
        -a_{0} a_{1}^2 + a_{0}^2 + 12 =0
        \ \Rightarrow \
         a_{1}^2 = \frac{ a_{0}^2 + 12 }{a_{0}}
         = \frac{ 24 }{\pm 2\sqrt{3}}
         = \pm 4\sqrt{3}
       \end{equation}
       となるが、
       $a_{0}=- 2\sqrt{3}$であれば
       $a_{1}^2=-4\sqrt{3}$であるから
       $a_{1}\not\in\mathbb{R}$である。

       つまり、$a_{0}= 2\sqrt{3},\ a_{1}=\pm2\sqrt[4]{3}$である。

       $a_{0} b_{0}=12$より$b_{0}= 2\sqrt{3}$、
       $a_{1} + b_{1}=0$より$b_{1}=\mp 2\sqrt[4]{3}$が得られる。

       これにより$x^{4}+12$は
       $\mathbb{R}$係数多項式
       ${\left(x^{2} + 2 \sqrt[4]{3} x + 2 \sqrt{3}\right)},
       {\left(x^{2} - 2 \sqrt[4]{3} x + 2 \sqrt{3}\right)}$
       に分解できる。
       \begin{equation}
        x^{4} + 12
         =
         {\left(x^{2} + 2 \sqrt[4]{3} x + 2 \sqrt{3}\right)}
         {\left(x^{2} - 2 \sqrt[4]{3} x + 2 \sqrt{3}\right)}
       \end{equation}

       よって、$x^{4}+12$は
       $\mathbb{R}[x]$上可約である。


       \hrulefill

\end{enumerate}

\hrulefill

\end{document}
