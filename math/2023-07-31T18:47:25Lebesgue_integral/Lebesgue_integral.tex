\documentclass[12pt,b5paper]{ltjsarticle}

%\usepackage[margin=15truemm, top=5truemm, bottom=5truemm]{geometry}
%\usepackage[margin=10truemm,left=15truemm]{geometry}
\usepackage[margin=10truemm]{geometry}

\usepackage{amsmath,amssymb}
%\pagestyle{headings}
\pagestyle{empty}

%\usepackage{listings,url}
%\renewcommand{\theenumi}{(\arabic{enumi})}

%\usepackage{graphicx}

%\usepackage{tikz}
%\usetikzlibrary {arrows.meta}
%\usepackage{wrapfig}
%\usepackage{bm}

% ルビを振る
\usepackage{luatexja-ruby}	% required for `\ruby'

%% 核Ker 像Im Hom を定義
%\newcommand{\Img}{\mathop{\mathrm{Im}}\nolimits}
%\newcommand{\Ker}{\mathop{\mathrm{Ker}}\nolimits}
%\newcommand{\Hom}{\mathop{\mathrm{Hom}}\nolimits}

%\DeclareMathOperator{\Rot}{rot}
%\DeclareMathOperator{\Div}{div}
%\DeclareMathOperator{\Grad}{grad}
%\DeclareMathOperator{\arcsinh}{arcsinh}
%\DeclareMathOperator{\arccosh}{arccosh}
%\DeclareMathOperator{\arctanh}{arctanh}

\usepackage{url}

%\usepackage{listings}
%
%\lstset{
%%プログラム言語(複数の言語に対応,C,C++も可)
%  language = Python,
%%  language = Lisp,
%%  language = C,
%  %背景色と透過度
%  %backgroundcolor={\color[gray]{.90}},
%  %枠外に行った時の自動改行
%  breaklines = true,
%  %自動改行後のインデント量(デフォルトでは20[pt])
%  breakindent = 10pt,
%  %標準の書体
%%  basicstyle = \ttfamily\scriptsize,
%  basicstyle = \ttfamily,
%  %コメントの書体
%%  commentstyle = {\itshape \color[cmyk]{1,0.4,1,0}},
%  %関数名等の色の設定
%  classoffset = 0,
%  %キーワード(int, ifなど)の書体
%%  keywordstyle = {\bfseries \color[cmyk]{0,1,0,0}},
%  %表示する文字の書体
%  %stringstyle = {\ttfamily \color[rgb]{0,0,1}},
%  %枠 "t"は上に線を記載, "T"は上に二重線を記載
%  %他オプション:leftline,topline,bottomline,lines,single,shadowbox
%  frame = TBrl,
%  %frameまでの間隔(行番号とプログラムの間)
%  framesep = 5pt,
%  %行番号の位置
%  numbers = left,
%  %行番号の間隔
%  stepnumber = 1,
%  %行番号の書体
%%  numberstyle = \tiny,
%  %タブの大きさ
%  tabsize = 4,
%  %キャプションの場所("tb"ならば上下両方に記載)
%  captionpos = t
%}

%\usepackage{cancel}
%\usepackage{bussproofs}
%\usepackage{proof}

\begin{document}

\hrulefill

\begin{enumerate}

 \item
      次の極限値を求めよ。
      \begin{equation}
       \lim_{n\to \infty}
        \int_{1}^{\infty} \frac{1}{1+x^{n}}dx
      \end{equation}

      \dotfill

      $n\in\mathbb{N}$とし、
      連続関数$f_{n}:[1,\infty)\to\mathbb{R}$を次で定義する。
      \begin{equation}
       f_{n}(x) = \frac{1}{1+x^{n}}
      \end{equation}

      $x\in [1,\infty)$に対し、
      次のような不等式が成り立つ。
      \begin{equation}
       \frac{1}{1+x}
        > \frac{1}{1+x^{2}}
        > \frac{1}{1+x^{3}}
        > \cdots
      \end{equation}

      そこで、
      $n>1$の自然数と$x\in [1,\infty)$に対し、
      次の不等式が成り立つ。
      \begin{equation}
       \lvert f_{n}(x) \rvert
        = \left\lvert \frac{1}{1+x^{n}} \right\rvert
        \leq (1+x^{2})^{-1}
      \end{equation}

      $\int_{1}^{\infty}(1+x^2)^{-1}dx$は
      $x=\tan\theta$として置換積分をすると
      次のように計算できる。
      \begin{align}
       \int_{1}^{\infty} (1+x^{2})^{-1} dx
       = \int_{\frac{\pi}{4}}^{\frac{\pi}{2}} (1+\tan^{2}\theta)^{-1} \frac{1}{\cos^{2}\theta}d\theta
       = \frac{\pi}{2} - \frac{\pi}{4} = \frac{\pi}{4}
      \end{align}

      $\int_{1}^{\infty}(1+x^{2})^{-1}dx < \infty$であるので、
      $\mu$を$[1,\infty)$上のルベーグ速度とすると、
      $n\geq 2$において
      次の式が成り立つ。
      \begin{equation}
       \int_{1}^{\infty} f_{n}(x) dx
        = \int_{[1,\infty)} f_{n} d\mu
      \end{equation}

      $n\in\mathbb{N}$に対し、
      $f_{n}$は連続関数であるので、
      $\mathcal{B}([1,\infty))$-可測関数である。
      また、
      $x\in(1,\infty)$に対し、
      $\lim_{n\to\infty} f_{n}(x) =0$
      であり、
      $x=1$のとき、
      $f_{n}(x)=1/2$である。

      関数$g:[1,\infty) \to [0,\infty)$を
      $g(x)=(1+x^{2})^{-1}$で定義する。
      $g$は連続関数であり、
      $\int_{1}^{\infty}\lvert g(x) \rvert dx = \frac{\pi}{4} < \infty$
      である。
      したがって、
      $\int_{[1,\infty)} g d\mu < \infty$
      である。

      $n\geq 2$と$x\in [1,\infty)$に対し、
      $\lvert f_{n}(x) \rvert \leq g(x)$が成り立つので、
      ルベーグの収束定理から次が得られる。
      \begin{equation}
       \lim_{n\to \infty}
        \int_{1}^{\infty} \frac{1}{1+x^{n}}dx
        =
        \lim_{n\to \infty}
        \int_{[1,\infty)} f_{n} d\mu
        =
        \int_{[1,\infty)} \lim_{n\to \infty} f_{n} d\mu
        =0
      \end{equation}

      \hrulefill

 \item
      $a\in (0,\infty)$と
      $a=0$のそれぞれの場合で
      次の極限値を求めよ。
      \begin{equation}
       \lim_{n\to \infty}
        \int_{a}^{\infty} \frac{n^{2} x e^{-n^{2}x^{2}}}{1+x^{2}}dx
      \end{equation}

      \dotfill

      $n\in\mathbb{N}$に対して、連続関数$f_{n}(x)$を次のように定義する。
      \begin{equation}
       f_{n}(x) = \frac{n^{2} x e^{-n^{2}x^{2}}}{1+x^{2}}
      \end{equation}

      $x\in (0,\infty),\; n\in\mathbb{N}$において
      $n^{2}x^{2} < e^{n^{2}x^{2}}$である。
      これにより$n^{2}x^{2}e^{-n^{2}x^{2}} < 1$であるので、
      次のような不等式が成り立つ。

      \begin{equation}
       \lvert f_{n}(x) \rvert
        = \left\lvert \frac{n^{2} x e^{-n^{2}x^{2}}}{1+x^{2}} \right\rvert
        = \frac{\lvert n^{2} x^{2} e^{-n^{2}x^{2}}\rvert}{\lvert x(1+x^{2})\rvert}
        < \frac{1}{x(1+x^{2})}
        < \frac{1}{x^{3}}
      \end{equation}


      $a>0$において
      $\int_{a}^{\infty} x^{-3} dx = \frac{1}{2a^2} < \infty$
      である。
      したがって、$\mu$を$[a,\infty)$上のルベーグ測度とすると
      次の式が得られる。
      \begin{equation}
       \int_{a}^{\infty} f_{n}(x) dx = \int_{[a,\infty)} f_{n} d\mu
      \end{equation}


      $f_{n}$は連続関数だから
      $\mathcal{B}([a,\infty))$-可測関数である。
      また、
      ${}^{\forall}x\in [a,\infty)$に対して
      \begin{equation}
       \lim_{n\to \infty} \frac{n^2x^2}{e^{n^2x^2}}
        =0
      \end{equation}
      であるので、
      \begin{equation}
       \lim_{n\to\infty}f_{n}(x) =0
      \end{equation}


      関数$g:[a,\infty)\to[0,\infty)$を
      $g(x)=x^{-3}$で定義する。
      $g$は連続関数で
      $\int_{a}^{\infty}\lvert g(x) \rvert dx <\infty$
      である。
      したがって、
      $\int_{[a,\infty)} g  d\mu < \infty$
      である。

      また、
      ${}^{\forall} n\in\mathbb{N}$と
      ${}^{\forall}x\in[a,\infty)$に対し
      $\lvert f_{n}(x) \rvert \leq g(x)$
      が成り立つ。

      これにより、収束定理から次の式が成り立つ。

      \begin{equation}
       \lim_{n\to\infty} \int_{a}^{\infty}f_{n}(x)dx
        = \lim_{n\to\infty} \int_{[a,\infty)}f_{n} d\mu
        = \int_{[a,\infty)} \lim_{n\to\infty} f_{n} d\mu
        =0
      \end{equation}


      $a=0$の場合を考える。
      ルベーグ測度$\mu$において
      1点集合$\{0\}$は測度$0 \;(\mu(\{0\}) = 0)$
      である。
      その為、$a=0$を付け加えても積分値は変わらない。
      \begin{equation}
       \lim_{n\to\infty}
        \int_{a}^{\infty} \frac{n^{2} x e^{-n^{2}x^{2}}}{1+x^{2}}dx
        =0
      \end{equation}
      


      \hrulefill

\end{enumerate}

\hrulefill

\end{document}
