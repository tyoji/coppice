\documentclass[12pt,b5paper]{ltjsarticle}

%\usepackage[margin=15truemm, top=5truemm, bottom=5truemm]{geometry}
%\usepackage[margin=10truemm,left=15truemm]{geometry}
\usepackage[margin=10truemm]{geometry}

\usepackage{amsmath,amssymb}
%\pagestyle{headings}
\pagestyle{empty}

%\usepackage{listings,url}
%\renewcommand{\theenumi}{(\arabic{enumi})}

%\usepackage{graphicx}

%\usepackage{tikz}
%\usetikzlibrary {arrows.meta}
%\usepackage{wrapfig}
\usepackage{bm}

% ルビを振る
%\usepackage{luatexja-ruby}	% required for `\ruby'

%% 核Ker 像Im Hom を定義
%\newcommand{\Img}{\mathop{\mathrm{Im}}\nolimits}
%\newcommand{\Ker}{\mathop{\mathrm{Ker}}\nolimits}
%\newcommand{\Hom}{\mathop{\mathrm{Hom}}\nolimits}

%\DeclareMathOperator{\Rot}{rot}
%\DeclareMathOperator{\Div}{div}
%\DeclareMathOperator{\Grad}{grad}
%\DeclareMathOperator{\arcsinh}{arcsinh}
%\DeclareMathOperator{\arccosh}{arccosh}
%\DeclareMathOperator{\arctanh}{arctanh}

\usepackage{url}

%\usepackage{listings}
%
%\lstset{
%%プログラム言語(複数の言語に対応,C,C++も可)
%  language = Python,
%%  language = Lisp,
%%  language = C,
%  %背景色と透過度
%  %backgroundcolor={\color[gray]{.90}},
%  %枠外に行った時の自動改行
%  breaklines = true,
%  %自動改行後のインデント量(デフォルトでは20[pt])
%  breakindent = 10pt,
%  %標準の書体
%%  basicstyle = \ttfamily\scriptsize,
%  basicstyle = \ttfamily,
%  %コメントの書体
%%  commentstyle = {\itshape \color[cmyk]{1,0.4,1,0}},
%  %関数名等の色の設定
%  classoffset = 0,
%  %キーワード(int, ifなど)の書体
%%  keywordstyle = {\bfseries \color[cmyk]{0,1,0,0}},
%  %表示する文字の書体
%  %stringstyle = {\ttfamily \color[rgb]{0,0,1}},
%  %枠 "t"は上に線を記載, "T"は上に二重線を記載
%  %他オプション:leftline,topline,bottomline,lines,single,shadowbox
%  frame = TBrl,
%  %frameまでの間隔(行番号とプログラムの間)
%  framesep = 5pt,
%  %行番号の位置
%  numbers = left,
%  %行番号の間隔
%  stepnumber = 1,
%  %行番号の書体
%%  numberstyle = \tiny,
%  %タブの大きさ
%  tabsize = 4,
%  %キャプションの場所("tb"ならば上下両方に記載)
%  captionpos = t
%}

%\usepackage{cancel}
%\usepackage{bussproofs}
%\usepackage{proof}



\begin{document}

\hrulefill

$A$を4次正則(可逆)行列とする。
以下の行列の行列式の値を
$\det{A}$
を用いて表せ。

\begin{enumerate}
 \item ${}^{t}\!A$
       \quad \dotfill \quad $\det{{}^{t}\!A} = \det{A}$
 \item $-A$
       \quad \dotfill \quad $\det{(-A)} = (-1)^{4}\det{A}=\det{A}$
 \item $2A$
       \quad \dotfill \quad $\det{(2A)} = 2^{4}\det{A}=16\det{A}$
 \item $A^{-1}$
       \quad \dotfill \quad $\det{(A^{-1})} = \frac{1}{\det{A}}$
 \item $A$の1行目と3行目を入れ替えた行列$B$
       \quad \dotfill \quad $\det{B} = -\det{A}$
 \item $A$の2列目に3列目の成分の$-1$倍を足した行列$C$
       \quad \dotfill \quad $\det{C} = \det{A}$
 \item $A$の3列目の成分を$5$倍した行列$D$
       \quad \dotfill \quad $\det{D} = 5\det{A}$
\end{enumerate}


\dotfill

$i,j \in \{1,2,3,4\}$とし、
行列$A=(a_{ij})$とする。


\begin{enumerate}
 \item
      正則行列$A$は基本行列($P(i,j),Q(i,j;k),R(i;c)$)の積で表すことができる。
      これらの行列式は転置行列と同じになる。
      \begin{equation}
       \det{P(i,j)} = \det{{}^{t}\!P(i,j)}
        ,\;
        \det{Q(i,j;k)} = \det{{}^{t}\!Q(i,j;k)}
        ,\;
        \det{R(i;c)} = \det{{}^{t}\!R(i;c)}
      \end{equation}

      ${}^{t}\!A$は
      基本行列の転置行列(${}^{t}\!P(i,j),{}^{t}\!Q(i,j;k),{}^{t}\!R(i;c)$)
      の積で表すことができるが、
      行列式は一致するので
      $\det{{}^{t}\!A}=\det{A}$
      である。


 \item
      $-A$は$A$の全ての成分に$-1$をかけた行列である。
      つまり、基本行列$R(i;c)$を用いて次のようになる。
      \begin{equation}
       -A = R(1;-1)R(2;-1)R(3;-1)R(4;-1)A
      \end{equation}

      $\det{R(i;-1)}=-1$であるので、
      \begin{align}
       \det{(-A)}
       &=\det{R(1;-1)}\det{R(2;-1)}\det{R(3;-1)}\det{R(4;-1)}\det{A}\\
       &= (-1)^{4} \det{A} = \det{A}
      \end{align}


 \item
      $2A$は$A$の全ての成分に$2$をかけた行列である。
      \begin{equation}
       2A = R(1;2)R(2;2)R(3;2)R(4;2)A
      \end{equation}

      $\det{R(i;2)}=2$であるので、
      \begin{align}
       \det{(2A)}
       &=\det{R(1;2)}\det{R(2;2)}\det{R(3;2)}\det{R(4;2)}\det{A}\\
       &= 2^{4} \det{A} = 16 \det{A}
      \end{align}


 \item
      $A$が正則行列であれば、
      $A A^{-1} = E$である。

      \begin{equation}
       \det{(A A^{-1})}=\det{A}\det{A^{-1}}
        ,\quad
        \det{E}=1
      \end{equation}

      これにより
      $\det{A}\det{A^{-1}}=1$であるから、
      $\det{A^{-1}}=\frac{1}{\det{A}}$である。



 \item
      $B$は$A$の1行目と3行目を入れ替えた行列であるので、
      基本行列$P(1,3)$を用いて
      次のように表せる。
      \begin{equation}
       B=P(1,3)A
      \end{equation}

      $\det{P(1,3)}=-1$であるので、
      $\det{B}=\det{P(1,3)} \det{A} = -\det{A}$
      である。


 \item
      $C$は$A$の2列目に3列目の成分の$-1$倍を足した行列であるので、
      基本行列$Q(2,3;-1)$を用いて次のように表せる。
      \begin{equation}
       C=A Q(2,3;-1)
      \end{equation}

      $\det{Q(2,3;-1)}=1$であるので、
      $\det{C}=\det{A} \det{Q(2,3;-1)} = \det{A}$
      である。



 \item
      $D$は$A$の3列目の成分を$5$倍した行列であるので、
      基本行列$R(3;5)$を用いて次のように表せる。
      \begin{equation}
       C=A R(3;5)
      \end{equation}

      $\det{R(3;5)}=5$であるので、
      $\det{C}=\det{A} \det{R(3;5)} = 5\det{A}$
      である。



\end{enumerate}

\dotfill

\textbf{基本行列}

$i\ne j$とする。
単位行列の$i$行と$j$行を入れ替えた行列を$P(i,j)$、
単位行列の$(i,j)$成分を$k$に変えた行列を$Q(i,j;k)$とする。
\begin{equation}
 P(i,j)=
  \begin{pmatrix}
   1 & & & & & & \\
    & \ddots & & & & & \\
    & & 0 & \cdots & 1 & & \\
    & & \vdots & \ddots & \vdots & & \\
    & & 1 & \cdots & 0 & & \\
   & & & & & \ddots & \\
   & & & & & & 1 \\
  \end{pmatrix}
  ,\;
 Q(i,j;k)=
  \begin{pmatrix}
   1 & & & & & & \\
    & \ddots & & & & & \\
    & & 1 & \cdots & k & & \\
    & & & \ddots & \vdots & & \\
    & & & & 1 & & \\
   & & & & & \ddots & \\
   & & & & & & 1 \\
  \end{pmatrix}
\end{equation}

また、
単位行列の$i$行目に$c$をかけたものを
$R(i;c)$とする。
\begin{equation}
 R(i;c)=
  \begin{pmatrix}
   1 & & & & & & \\
    & \ddots & & & & & \\
   & & & c & & & \\
   & & & & & \ddots & \\
   & & & & & & 1 \\
  \end{pmatrix}
\end{equation}

この3つの行列を基本行列という。
行列に対して左から基本行列をかけることで
行についての基本変形を行える。
右からかけることで列についての基本変形となる。

全ての正則行列はこの3つの行列の積で表すことができる。


\dotfill

\textbf{行列式の定義}

成分が一つだけの行列$(c)$の行列式は
$\det{(c)}=c$である。

n次行列$A=(a_{ij})$における$(i,j)$成分の\textbf{余因子}$A_{ij}$とは、
$A$の$(i,j)$成分$a_{ij}$を含む行と列を除いてできる
$n-1$次行列式に$(-1)^{i+j}$をかけたものである。
例えば、
$3$次正方行列$A$の余因子$A_{12}$は次のようになる。
\begin{equation}
 A=
  \begin{pmatrix}
   \alpha & \beta & \gamma \\ \delta & \epsilon & \zeta \\ \eta & \theta & \iota
   \end{pmatrix}
   ,\quad
   A_{12}
   =(-1)^{1+2} \det{
  \begin{pmatrix}
   \delta & \zeta \\ \eta & \iota
   \end{pmatrix}
}
\end{equation}


$n$次正方行列$A=(a_{ij})$の行列式は
余因子を用いて次のように定義する。
\begin{equation}
 \det{A} =
  \sum_{j=1}^{n} a_{ij}A_{ij}
\end{equation}
例えば、
$3$次正方行列$A=(a_{ij})$の行列式$\det{A}$は
次のようになる。
\begin{equation}
 \det{A} = a_{11}A_{11} + a_{12}A_{12} + a_{13}A_{13}
\end{equation}



\textbf{行列式の性質}

\begin{enumerate}
 \item $1$つの行を$k$倍した行列式は、もとの行列式の$k$倍となる
       \begin{equation}
        \det
        \begin{pmatrix}
         k\alpha & k\beta \\ \gamma & \delta
        \end{pmatrix}
        =k
        \det
        \begin{pmatrix}
         \alpha & \beta \\ \gamma & \delta
        \end{pmatrix}
       \end{equation}

 \item $2$つの行を入れ替えてできる行列式は、もとの行列式の$-1$倍となる
       \begin{equation}
        \det
        \begin{pmatrix}
         \alpha & \beta \\ \gamma & \delta
        \end{pmatrix}
        =
        -\det
        \begin{pmatrix}
         \gamma & \delta \\
         \alpha & \beta
        \end{pmatrix}
       \end{equation}

 \item 行列式の第$i$行が$\bm{a}_{i}+\bm{b}_{i}$であれば、
       第$i$行を$\bm{a}_{i},\bm{b}_{i}$にした行列式の和である
       \begin{equation}
        \det
        \begin{pmatrix}
         \alpha + \epsilon & \beta + \zeta \\ \gamma & \delta
        \end{pmatrix}
        =
        \det
        \begin{pmatrix}
         \alpha & \beta \\
         \gamma & \delta
        \end{pmatrix}
       +
        \det
        \begin{pmatrix}
         \epsilon & \zeta \\
         \gamma & \delta
        \end{pmatrix}
       \end{equation}

 \item $1$つの行の$k$倍を他の行に加えてできる行列式は
       もとの行列式と等しい
       \begin{equation}
        \det
        \begin{pmatrix}
         \alpha + k\gamma & \beta + k\delta \\ \gamma & \delta
        \end{pmatrix}
        =
        \det
        \begin{pmatrix}
         \alpha & \beta \\
         \gamma & \delta
        \end{pmatrix}
       \end{equation}

 \item 1つの行の成分が全て$0$であるなら、行列式は0である
       \begin{equation}
        \det
        \begin{pmatrix}
         \alpha & \beta \\ 0 & 0
        \end{pmatrix}
        =
        0
       \end{equation}

\end{enumerate}

上記性質は列に対して行っても成立する

\dotfill

\textbf{記号}

\begin{description}
 \item[$M(n,\mathbb{K})$]
            $\mathbb{K}$を成分とする$n$次正方行列の全体
 \item[$GL(n,\mathbb{K})$]
            $\mathbb{K}$を成分とする$n$次正則正方行列の全体
            (\textbf{一般線型群})
\end{description}
\begin{equation}
 GL(n,\mathbb{K}) \subset  M(n,\mathbb{K})
\end{equation}


\hrulefill

\end{document}
