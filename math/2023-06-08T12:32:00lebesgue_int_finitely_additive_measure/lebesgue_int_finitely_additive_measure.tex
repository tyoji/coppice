\documentclass[12pt,b5paper]{ltjsarticle}

%\usepackage[margin=15truemm, top=5truemm, bottom=5truemm]{geometry}
%\usepackage[margin=10truemm,left=15truemm]{geometry}
\usepackage[margin=10truemm]{geometry}

\usepackage{amsmath,amssymb}
%\pagestyle{headings}
\pagestyle{empty}

%\usepackage{listings,url}
%\renewcommand{\theenumi}{(\arabic{enumi})}

%\usepackage{graphicx}

%\usepackage{tikz}
%\usetikzlibrary {arrows.meta}
%\usepackage{wrapfig}
%\usepackage{bm}

% ルビを振る
\usepackage{luatexja-ruby}	% required for `\ruby'

%% 核Ker 像Im Hom を定義
%\newcommand{\Img}{\mathop{\mathrm{Im}}\nolimits}
%\newcommand{\Ker}{\mathop{\mathrm{Ker}}\nolimits}
%\newcommand{\Hom}{\mathop{\mathrm{Hom}}\nolimits}

%\DeclareMathOperator{\Rot}{rot}
%\DeclareMathOperator{\Div}{div}
%\DeclareMathOperator{\Grad}{grad}
%\DeclareMathOperator{\arcsinh}{arcsinh}
%\DeclareMathOperator{\arccosh}{arccosh}
%\DeclareMathOperator{\arctanh}{arctanh}



%\usepackage{listings,url}
%
%\lstset{
%%プログラム言語(複数の言語に対応,C,C++も可)
%  language = Python,
%%  language = Lisp,
%%  language = C,
%  %背景色と透過度
%  %backgroundcolor={\color[gray]{.90}},
%  %枠外に行った時の自動改行
%  breaklines = true,
%  %自動改行後のインデント量(デフォルトでは20[pt])
%  breakindent = 10pt,
%  %標準の書体
%%  basicstyle = \ttfamily\scriptsize,
%  basicstyle = \ttfamily,
%  %コメントの書体
%%  commentstyle = {\itshape \color[cmyk]{1,0.4,1,0}},
%  %関数名等の色の設定
%  classoffset = 0,
%  %キーワード(int, ifなど)の書体
%%  keywordstyle = {\bfseries \color[cmyk]{0,1,0,0}},
%  %表示する文字の書体
%  %stringstyle = {\ttfamily \color[rgb]{0,0,1}},
%  %枠 "t"は上に線を記載, "T"は上に二重線を記載
%  %他オプション:leftline,topline,bottomline,lines,single,shadowbox
%  frame = TBrl,
%  %frameまでの間隔(行番号とプログラムの間)
%  framesep = 5pt,
%  %行番号の位置
%  numbers = left,
%  %行番号の間隔
%  stepnumber = 1,
%  %行番号の書体
%%  numberstyle = \tiny,
%  %タブの大きさ
%  tabsize = 4,
%  %キャプションの場所("tb"ならば上下両方に記載)
%  captionpos = t
%}



\begin{document}

\hrulefill

\textbf{有限加法族}

集合$X$の部分集合族$\mathcal{F}$が
\textbf{有限加法族}である
とは次を満たすときをいう。
\begin{enumerate}
 \item $\emptyset \in \mathcal{F}$
 \item $A \in \mathcal{F} \Rightarrow X\backslash A \in\mathcal{F}$
 \item $A,B\in\mathcal{F}
       \Rightarrow A\cup B \in\mathcal{F}$
\end{enumerate}


\textbf{有限加法的測度}

集合$X$上の有限加法族$\mathcal{F}$について、
$m:\mathcal{F}\to [0,\infty]$が
$(X,\mathcal{F})$上の
\textbf{有限加法的測度}であるとは、
次の2つの条件を満たすときをいう。
\begin{enumerate}
 \item $m(\emptyset) =0$
 \item $A,B\in\mathcal{F}$が互いに素である時、
       $m(A\cup B) = m(A) + m(B)$
\end{enumerate}


\textbf{外測度}

$X$を集合とする。
$\Gamma : 2^{X}\to [0,\infty]$が
$X$上の\textbf{外測度}であるとは、
次の3つの条件を満たすときをいう。
\begin{enumerate}
 \item
      $\Gamma (\emptyset) = 0$
 \item
      $A,B \subset X$が
      $A\subset B$を満たす時、
      $\Gamma(A)\leq \Gamma(B)$
 \item
      $X$の任意の部分集合列$\{A_{n}\}_{n=1}^{\infty}$
      に対し、
      $\Gamma(\bigcup_{n=1}^{\infty}A_{n}) \leq \sum_{n=1}^{\infty}\Gamma(A_{n})$
\end{enumerate}


\textbf{$\Gamma$-可測}

$X$を集合とする。
$\Gamma : 2^{X}\to [0,\infty]$を
$X$上の外測度とする。

集合$E\subset X$が\textbf{$\Gamma$-可測}
(または \ruby{Carath\'eodory}{カラテオドリ}の意味で可測)
とは、
任意の$A\subset X$に対し次を満たすときをいう。
\begin{equation}
 \Gamma(A\cap E) + \Gamma(A\cap (X\backslash E)) = \Gamma(A)
\end{equation}

また、$\Gamma$-可測集合全体を$\mathcal{M}_{\Gamma}$と表す。



\textbf{命題}($X$上の外測度)

$X$を集合、
$\mathcal{F}$を$X$上の有限加法族、
$\mu$を$(X,\mathcal{F})$上の有限加法的測度
とする。
$\mu^{*}:2^{X}\to [0,\infty]$を次で定義する。
\begin{equation}
 \mu^{*}(A)
  = \inf\left\{
         \sum_{j=1}^{\infty}\mu(E_{j})
         \ \middle| \
         A \subset \bigcup_{j=1}^{\infty}E_{j}
         であり、
         E_{j}\in\mathcal{F}
         、j\in\mathbb{N}
        \right\}
\end{equation}

このとき、
$\mu^{*}$は$X$上の外側度である。



\hrulefill

\begin{enumerate}
 \item
      $X$を集合とし、
      $\mathcal{M}$を$X$上の有限加法族とする。
      また、
      $m$を$(X,\mathcal{M})$上の
      有限加法的測度とする。
      \begin{enumerate}
       \item
            $A,B\subset \mathcal{M}$が
            $A\subset B$を満たすならば
            $m(A)\leq m(B)$、
            更に
            $m(A)<\infty$ならば
            $m(B\backslash A) = m(B)- m(A)$
            が成り立つことを示せ。

            \dotfill

            $A\subset B$より次の式が成り立つ。
            \begin{gather}
             A \cup (B\backslash A) = B\\
             m(A)+m(B\backslash A) = m(B)
            \end{gather}

            これより、$m(A) \leq m(B)$である。
            また、
            $m(A)<\infty$であれば$m(A)$を移項し
            $m(B\backslash A) = m(B)- m(A)$
            となる。

            \hrulefill

       \item
            $N\in\mathbb{N}$とし、
            $\{A_{n}\}_{n=1}^{N}\subset \mathcal{M}$とする。
            このとき、
            $m\left( \bigcup_{n=1}^{N}A_{n} \right) \leq \sum_{n=1}^{N}m(A_{n})$
            が成り立つことを示せ。

            \dotfill

            $A_{1},A_{2}\subset \mathcal{M}$について次の式が成り立つ。
            \begin{equation}
             A_{1}=(A_{1}\cap A_{2})\cup (A_{1}\backslash A_{2}),\
             A_{2}=(A_{2}\cap A_{1})\cup (A_{2}\backslash A_{1})
            \end{equation}
            これにより次が得られる。
            \begin{equation}
              m(A_{1})=m(A_{1}\cap A_{2}) + m(A_{1}\backslash A_{2}),\
              m(A_{2})=m(A_{2}\cap A_{1}) + m(A_{2}\backslash A_{1})
            \end{equation}

            また、$A_{1}\cup A_{2}$は次のように分けられる。
            \begin{equation}
             A_{1}\cup A_{2}
              =
              (A_{1}\cap A_{2})
              \cup (A_{1}\backslash A_{2})
              \cup (A_{2}\backslash A_{1})
            \end{equation}

            $m(A_{1}\cup A_{2})$と
            $m(A_{1}),\ m(A_{2})$の関係が次のようになる。
            \begin{align}
             m(A_{1}\cup A_{2})
             =& \
              m(A_{1}\cap A_{2}) + m(A_{1}\backslash A_{2}) + m(A_{2}\backslash A_{1})\\
             \leq &\
              2 m(A_{1}\cap A_{2}) + m(A_{1}\backslash A_{2}) + m(A_{2}\backslash A_{1})\\
             =& \ m(A_{1}) + m(A_{2})
            \end{align}

            $A_{1}\cup A_{2}$と$A_{3}$について同様に行うと
            次が得られる。
            \begin{equation}
             m(A_{1}\cup A_{2} \cup A_{3}) \leq m(A_{1}) + m(A_{2}) + m(A_{3})
            \end{equation}

            これを繰り返すと次の式が得られる。
            \begin{equation}
             m\left( \bigcup_{n=1}^{N}A_{n} \right) \leq \sum_{n=1}^{N}m(A_{n})
            \end{equation}

            \hrulefill

      \end{enumerate}

 \item
      関数$m:2^{\mathbb{N}}\to[0,\infty]$を次で定義する。
      \begin{equation}
       m(A)=
       \begin{cases}
        \infty & A\subset\mathbb{N}が無限集合\\
        0 & A\subset\mathbb{N}が有限集合
       \end{cases}
      \end{equation}

      ($m$は
      $(\mathbb{N},\ 2^{\mathbb{N}})$上の有限加法的測度である。)

      $m$に\textbf{命題}($X$上の外測度)
      を適用し、
      $\mathbb{N}$上の外測度$m^{*}$を得る。
      $m^{*}$-可測な集合の全体$\mathcal{M}_{m^{*}}$はどのようなものか。
      理由をつけて答えよ。

      \dotfill

      $m^{*}$-可測な集合$E\subset \mathbb{N}$は次の式を満たす。
      \begin{equation}
       m^{*}(A\cap E) + m^{*}(A\cap (\mathbb{N}\backslash E)) = m^{*}(A)
        ,\quad {}^{\forall}A\subset \mathbb{N}
      \end{equation}

      $m^{*}(S)$は$S$を被覆する集合列$\{A_{k}\}_{k=1}^{\infty}$を用いて
      $\sum m(A_{k})$の下限で定義している。
      $m(A_{k})$は$0$か$\infty$のどちらかの値のみをとる。
      つまり、$S$が有限集合のみで被覆できれば$m^{*}(S)=0$、
      そうでなければ$m^{*}(S)=\infty$である。

      $k\in\mathbb{N}$に対して、
      要素一つだけの集合$A_{k}=\{k\}$とする。
      これにより$\mathbb{N}=\bigcup_{k} A_{k}$である。
      この為、$m^{*}(\mathbb{N})=0$となる。
      任意の部分集合$S$は$A_{k}$で被覆できる為、
      $m^{*}(S)=0$である。

      よって、全て部分集合は$m^{*}$-可測であり、
      $\mathcal{M}_{m^{*}} = 2^{\mathbb{N}}$
      である。
      

      \hrulefill

\end{enumerate}

\hrulefill

\end{document}
