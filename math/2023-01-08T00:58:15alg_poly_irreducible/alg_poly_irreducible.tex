\documentclass[12pt,b5paper]{ltjsarticle}

%\usepackage[margin=15truemm, top=5truemm, bottom=5truemm]{geometry}
%\usepackage[margin=10truemm,left=15truemm]{geometry}
\usepackage[margin=10truemm]{geometry}

\usepackage{amsmath,amssymb}
%\pagestyle{headings}
\pagestyle{empty}

%\usepackage{listings,url}
%\renewcommand{\theenumi}{(\arabic{enumi})}

\usepackage{graphicx}

%\usepackage{tikz}
%\usetikzlibrary {arrows.meta}
%\usepackage{wrapfig}	% required for `\wrapfigure' (yatex added)
%\usepackage{bm}	% required for `\bm' (yatex added)

% ルビを振る
%\usepackage{luatexja-ruby}	% required for `\ruby'

%% 核Ker 像Im Hom を定義
%\newcommand{\Img}{\mathop{\mathrm{Im}}\nolimits}
%\newcommand{\Ker}{\mathop{\mathrm{Ker}}\nolimits}
%\newcommand{\Hom}{\mathop{\mathrm{Hom}}\nolimits}

%\DeclareMathOperator{\Rot}{rot}
%\DeclareMathOperator{\Div}{div}
%\DeclareMathOperator{\Grad}{grad}
%\DeclareMathOperator{\arcsinh}{arcsinh}
%\DeclareMathOperator{\arccosh}{arccosh}
%\DeclareMathOperator{\arctanh}{arctanh}



%\usepackage{listings,url}
%
%\lstset{
%%プログラム言語(複数の言語に対応,C,C++も可)
%  language = Python,
%%  language = Lisp,
%%  language = C,
%  %背景色と透過度
%  %backgroundcolor={\color[gray]{.90}},
%  %枠外に行った時の自動改行
%  breaklines = true,
%  %自動改行後のインデント量(デフォルトでは20[pt])
%  breakindent = 10pt,
%  %標準の書体
%%  basicstyle = \ttfamily\scriptsize,
%  basicstyle = \ttfamily,
%  %コメントの書体
%%  commentstyle = {\itshape \color[cmyk]{1,0.4,1,0}},
%  %関数名等の色の設定
%  classoffset = 0,
%  %キーワード(int, ifなど)の書体
%%  keywordstyle = {\bfseries \color[cmyk]{0,1,0,0}},
%  %表示する文字の書体
%  %stringstyle = {\ttfamily \color[rgb]{0,0,1}},
%  %枠 "t"は上に線を記載, "T"は上に二重線を記載
%  %他オプション:leftline,topline,bottomline,lines,single,shadowbox
%  frame = TBrl,
%  %frameまでの間隔(行番号とプログラムの間)
%  framesep = 5pt,
%  %行番号の位置
%  numbers = left,
%  %行番号の間隔
%  stepnumber = 1,
%  %行番号の書体
%%  numberstyle = \tiny,
%  %タブの大きさ
%  tabsize = 4,
%  %キャプションの場所("tb"ならば上下両方に記載)
%  captionpos = t
%}



\begin{document}

\hrulefill

\textbf{既約元}

環$R$において、
単元でない$a\in R\ \{0\}$が既約であるとは
$a=bc$となる$b,c\in R$が存在すれば
$b$または$c$が単元となるときをいう。

$a=bc$となる$b,c\in R$が
ともに単元でないのであれば
可約という。

\hrulefill

$f\in\mathbb{Z}[x]$を
次のように定める。
$f$が$\mathbb{Z}[x]$において既約か否かを判定せよ。
\begin{enumerate}
 \item $2x-4$
 \item $-3x+1$
 \item $x^2 + 2x + 10$
 \item $x^2 + 3x + 6$
 \item $x^2 + 6x + 9$
 \item $x^2 + 9$
 \item $x^3 + 8$
 \item $x^4 + 12$
 \item $x^4 + 64$
\end{enumerate}

\dotfill

\begin{enumerate}
 \item
      $2x-4=2(x-2)$であり、
      $2\in\mathbb{Z}[x]$も
      $x-2\in\mathbb{Z}[x]$も
      単元ではないので
      可約である。

 \item
      $-3x+1$は次数1の多項式である。
      もし可約であれば、
      $f=ag \ (a,g\in\mathbb{Z}[x])$
      で$\deg{a}=0,\deg{g}=1$
      と分けられる。

      $-3$と$1$の最大公約数は$1$であるので、
      $a=1$である。
      $1$は単元であるので、
      $-3x+1$は既約である。

 \item
      $x^2 + 2x + 10$
      が可約であるとする。
      つまり、$f=gh$となる単元でない$g,h\in\mathbb{Z}[x]$が存在するとする。

      $(\deg{g},\deg{h})=(0,2),(2,0)$の場合と
      $(\deg{g},\deg{h})=(1,1)$の場合を考える。

      $f$の係数は$1,2,10$であるので、最大公約数は1であり単元となるので、
      $(\deg{g},\deg{h})=(0,2),(2,0)$となる分解はできない。

      $(\deg{g},\deg{h})=(1,1)$の場合を考える。

      アイゼンシュタインの定理より、
      $2,10$を割り切る素数$2$は
      $x^2$の係数を割り切れず、
      $2^2$も定数$10$を割り切れない。
      よって、$\mathbb{Q}[x]$において$f$は既約であり、
      $\mathbb{Z}[x] (\subset \mathbb{Q}[x])$においても
      $f$は既約である。

 \item
      $x^2 + 3x + 6$

      係数$1,3,6$の最大公約数は1である。
      よって、$f=ag \ (a\in\mathbb{Z},g\in\mathbb{Z}[x])$
      という分解はできない。

      $3,6$を割り切る素数$3$は
      $x^2$の係数を割り切れず、
      $3^2$も定数$6$を割り切れない。
      よって、
      アイゼンシュタインの定理より、
      $\mathbb{Q}[x]$において$f$は既約であり、
      $\mathbb{Z}[x] (\subset \mathbb{Q}[x])$においても
      $f$は既約である。

 \item
      $x^2 + 6x + 9 = (x+3)^2$
      より可約。

 \item
      $x^2 + 9$

      係数$1,9$を割り切る素数はないので、
      $f=ag \ (a\in\mathbb{Z},g\in\mathbb{Z}[x])$
      という分解はできない。

      $x^2 + 9=(x+a)(x+b) \ (a,b\in\mathbb{Z})$と割り切れるとする。

      $(x+a)(x+b)=x^2+(a+b)x+ab$であるので、
      $a,b\in\mathbb{Z}$は
      $a+b=0, \ ab=9$
      を満たす。
      $a+b=0$より$b=-a$であるので、
      $ab=9$より$-a^2=9$となるが、
      これを満たす整数$a$は存在しない。

      よって、
      $x^2 + 9=(x+a)(x+b) \ (a,b\in\mathbb{Z})$と
      分けられない。

      これにより
      既約であることがわかる。

 \item
      $x^3 + 8 = (x+2)(x^2-2x+4)$
      より可約。


 \item
      $f(x)=x^4 + 12$
      とする。

      $x^4$の係数が$1$であるので
      整数と多項式の積に分けられない。

      そこで、$f(x+3)=(x+3)^4 +12$について既約かどうかを考える。

      \begin{equation}
       f(x+3) = x^4 + 12x^3 + 54x^2 + 108x + 93
      \end{equation}

      最高次数の項以外の係数は
      $12,54,108,93$でありこれらの最大公約数は$3$である。
      $x^4$の係数は$3$で割り切れず
      $93$は$3^2$で割り切れない。
      このため、アイゼンシュタインの既約判定法により
      $f(x+3)$は既約である。
      よって、平行移動する前の多項式$f(x)$も既約である。

 \item
      $x^4 + 64 = (x^2+4x+8)(x^2-4x+8)$
      より可約。

\end{enumerate}

\hrulefill

\end{document}
