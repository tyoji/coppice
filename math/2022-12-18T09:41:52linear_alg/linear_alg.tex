\documentclass[12pt,b5paper]{ltjsarticle}

%\usepackage[margin=15truemm, top=5truemm, bottom=5truemm]{geometry}
%\usepackage[margin=10truemm,left=15truemm]{geometry}
\usepackage[margin=10truemm]{geometry}

\usepackage{amsmath,amssymb}
%\pagestyle{headings}
\pagestyle{empty}

%\usepackage{listings,url}
%\renewcommand{\theenumi}{(\arabic{enumi})}

%\usepackage{graphicx}

%\usepackage{tikz}
%\usetikzlibrary {arrows.meta}
%\usepackage{wrapfig}	% required for `\wrapfigure' (yatex added)
\usepackage{bm}	% required for `\bm' (yatex added)

% ルビを振る
%\usepackage{luatexja-ruby}	% required for `\ruby'

%% 核Ker 像Im Hom を定義
%\newcommand{\Img}{\mathop{\mathrm{Im}}\nolimits}
%\newcommand{\Ker}{\mathop{\mathrm{Ker}}\nolimits}
%\newcommand{\Hom}{\mathop{\mathrm{Hom}}\nolimits}

%\DeclareMathOperator{\Rot}{rot}
%\DeclareMathOperator{\Div}{div}
%\DeclareMathOperator{\Grad}{grad}
%\DeclareMathOperator{\arcsinh}{arcsinh}
%\DeclareMathOperator{\arccosh}{arccosh}
%\DeclareMathOperator{\arctanh}{arctanh}



%\usepackage{listings,url}
%
%\lstset{
%%プログラム言語(複数の言語に対応,C,C++も可)
%  language = Python,
%%  language = Lisp,
%%  language = C,
%  %背景色と透過度
%  %backgroundcolor={\color[gray]{.90}},
%  %枠外に行った時の自動改行
%  breaklines = true,
%  %自動改行後のインデント量(デフォルトでは20[pt])
%  breakindent = 10pt,
%  %標準の書体
%%  basicstyle = \ttfamily\scriptsize,
%  basicstyle = \ttfamily,
%  %コメントの書体
%%  commentstyle = {\itshape \color[cmyk]{1,0.4,1,0}},
%  %関数名等の色の設定
%  classoffset = 0,
%  %キーワード(int, ifなど)の書体
%%  keywordstyle = {\bfseries \color[cmyk]{0,1,0,0}},
%  %表示する文字の書体
%  %stringstyle = {\ttfamily \color[rgb]{0,0,1}},
%  %枠 "t"は上に線を記載, "T"は上に二重線を記載
%  %他オプション:leftline,topline,bottomline,lines,single,shadowbox
%  frame = TBrl,
%  %frameまでの間隔(行番号とプログラムの間)
%  framesep = 5pt,
%  %行番号の位置
%  numbers = left,
%  %行番号の間隔
%  stepnumber = 1,
%  %行番号の書体
%%  numberstyle = \tiny,
%  %タブの大きさ
%  tabsize = 4,
%  %キャプションの場所("tb"ならば上下両方に記載)
%  captionpos = t
%}



\begin{document}

$U=\langle \bm{a}_1,\bm{a}_2,\bm{a}_3 \rangle$
ただし、
$\bm{a}_1=\begin{pmatrix}1\\2\\3\end{pmatrix},
\bm{a}_2=\begin{pmatrix}1\\5\\1\end{pmatrix},
\bm{a}_3=\begin{pmatrix}1\\-1\\a\end{pmatrix}$

\dotfill

$\langle \bm{a}_1,\bm{a}_2,\bm{a}_3 \rangle$
の次元と基底の数は同じであるので、
基底を求めればよい。

基底は一次独立でないといけないため、
$\bm{a}_1,\bm{a}_2,\bm{a}_3$が
一次従属かどうかを調べる必要がある。

\dotfill

$\bm{a}_1,\bm{a}_2,\bm{a}_3$のうち
$\bm{a}_1,\bm{a}_2$は
$\bm{a}_2=k\bm{a}_1$となる
スカラー$k$が存在しない為
一次独立である。

$2\bm{a}_1-\bm{a}_2 =\begin{pmatrix}1\\-1\\5\end{pmatrix}$
であるので、
$a=5$のとき、$\bm{a}_3=2\bm{a}_1-\bm{a}_2$となり一次従属である。
$a\ne5$のとき、3つのベクトルは一次独立である。

よって、以下の二種類に分かれる。
\begin{enumerate}
 \item $a=5$のとき、$U=\langle \bm{a}_1,\bm{a}_2\rangle$であり、2次元空間
 \item $a\ne5$のとき、$U=\langle \bm{a}_1,\bm{a}_2,\bm{a}_3 \rangle$であり、3次元空間
\end{enumerate}


\hrulefill


$U=\langle 1,t,t^2\rangle=\{a_01+a_1t+a_2t^2 \mid a_i\in\mathbb{R}\}$
とおく。
\begin{enumerate}
 \item
      $W=\{f\in U \mid f(-2)=0 \}$
      は$U$の部分空間であることを示せ。

      \dotfill

      $W$が部分空間であるとは
      次を満たすときをいう。
      \begin{enumerate}
       \item $0\in W$
       \item $f_1,f_2\in W$に対して$f_1+f_2\in W$
       \item $f\in W,\ k\in\mathbb{R}$に対して$kf\in W$
      \end{enumerate}

      \dotfill

      $f(t)=a_01+a_1t+a_2t^2$とする。
      $(a_0,a_1,a_2)=(0,0,0)$であるとき、
      $f(t)=0\cdot1+0t+0t^2=0$であるので
      $f(-2)=0$である。つまり、$0\in W$である。

      $f_1,f_2\in W$とする。このとき、$f_1(-2)=f_2(-2)=0$である。
      $g(t)=f_1(t)+f_2(t)$とすれば、
      $g(-2)=f_1(-2)+f_2(-2)=0+0=0$であるので、
      $f_1+f_2\in W$である。

      $f\in W$、$k\in\mathbb{R}$とする。
      このとき、$f(-2)=0$である。
      よって、$kf(-2)=k\cdot 0=0$であるので、
      $kf\in W$である。

      以上により$W$は$U$の部分空間である。


      \hrulefill

 \item\label{second}
      $W$の基底と次元を求めよ。

      \dotfill

      任意の$f\in W$について
      $f(-2)=0$であるので、
      $f(t)=\alpha(t+2)(t+\beta)\ (\alpha,\beta\in\mathbb{R})$
      と因数分解できる。
      ここから$f(t)=\alpha t(t+2)+\alpha\beta(t+2)$
      であるので、
      $\alpha,\alpha\beta$を別の記号$b_1,b_2$におきかえると、
      $f(t)=b_1t(t+2)+b_2(t+2)$となる。

      $t(t+2$)$と$(t+2)$は次数が異なるので実数倍で等しくならない。
      つまり、$t(t+2$)$と$(t+2)$は一次独立である。

      よって、$W$は2次元空間であり、
      $W=\langle t(t+2),(t+2)\rangle$となる。

      \hrulefill

 \item
      $f(t)=2+3t+t^2$のとき、
      (\ref{second})の基底に関する
      $f$の列ベクトル表示を求めよ。

      \dotfill

      $f(t)=2+3t+t^2$を因数分解すると
      $f(t)=(1+t)(2+t)$である。
      これを次のように変形する。
      \begin{equation}
       f(t)=(1+t)(2+t)=(2+t)+t(2+t)
        =\begin{pmatrix}(2+t) & t(2+t)\end{pmatrix}
        \begin{pmatrix}1 \\ 1\end{pmatrix}
      \end{equation}

      よって、
      $W=\langle t(t+2),(t+2)\rangle$に対し、
      $f(t)=\begin{pmatrix}1 \\ 1\end{pmatrix}$となる。

      \hrulefill
\end{enumerate}



\end{document}
