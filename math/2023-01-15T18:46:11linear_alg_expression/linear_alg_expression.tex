\documentclass[12pt,b5paper]{ltjsarticle}

%\usepackage[margin=15truemm, top=5truemm, bottom=5truemm]{geometry}
%\usepackage[margin=10truemm,left=15truemm]{geometry}
\usepackage[margin=10truemm]{geometry}

\usepackage{amsmath,amssymb}
%\pagestyle{headings}
\pagestyle{empty}

%\usepackage{listings,url}
%\renewcommand{\theenumi}{(\arabic{enumi})}

\usepackage{graphicx}

%\usepackage{tikz}
%\usetikzlibrary {arrows.meta}
%\usepackage{wrapfig}
%\usepackage{bm}

% ルビを振る
\usepackage{luatexja-ruby}

%% 核Ker 像Im Hom を定義
%\newcommand{\Img}{\mathop{\mathrm{Im}}\nolimits}
%\newcommand{\Ker}{\mathop{\mathrm{Ker}}\nolimits}
%\newcommand{\Hom}{\mathop{\mathrm{Hom}}\nolimits}

%\DeclareMathOperator{\Rot}{rot}
%\DeclareMathOperator{\Div}{div}
%\DeclareMathOperator{\Grad}{grad}
%\DeclareMathOperator{\arcsinh}{arcsinh}
%\DeclareMathOperator{\arccosh}{arccosh}
%\DeclareMathOperator{\arctanh}{arctanh}



%\usepackage{listings,url}
%
%\lstset{
%%プログラム言語(複数の言語に対応,C,C++も可)
%  language = Python,
%%  language = Lisp,
%%  language = C,
%  %背景色と透過度
%  %backgroundcolor={\color[gray]{.90}},
%  %枠外に行った時の自動改行
%  breaklines = true,
%  %自動改行後のインデント量(デフォルトでは20[pt])
%  breakindent = 10pt,
%  %標準の書体
%%  basicstyle = \ttfamily\scriptsize,
%  basicstyle = \ttfamily,
%  %コメントの書体
%%  commentstyle = {\itshape \color[cmyk]{1,0.4,1,0}},
%  %関数名等の色の設定
%  classoffset = 0,
%  %キーワード(int, ifなど)の書体
%%  keywordstyle = {\bfseries \color[cmyk]{0,1,0,0}},
%  %表示する文字の書体
%  %stringstyle = {\ttfamily \color[rgb]{0,0,1}},
%  %枠 "t"は上に線を記載, "T"は上に二重線を記載
%  %他オプション:leftline,topline,bottomline,lines,single,shadowbox
%  frame = TBrl,
%  %frameまでの間隔(行番号とプログラムの間)
%  framesep = 5pt,
%  %行番号の位置
%  numbers = left,
%  %行番号の間隔
%  stepnumber = 1,
%  %行番号の書体
%%  numberstyle = \tiny,
%  %タブの大きさ
%  tabsize = 4,
%  %キャプションの場所("tb"ならば上下両方に記載)
%  captionpos = t
%}



\begin{document}

\hrulefill

2次式全体の集合$\mathbb{R}[x]_{2}$
\begin{equation}
 \mathbb{R}[x]_{2}=
  \{ a_0+a_1x+a_2x^2 \mid a_{i}\in\mathbb{R} \}
\end{equation}
これは$1,x,x^2$をベクトル空間の基底と考えて
3次元空間をみれる。





\hrulefill

線形変換 $T : \mathbb{R}[x]_{2}\to\mathbb{R}[x]_{2}$
に対して次の3つを求めよ。
\begin{enumerate}
 \item
      $g_{T}(t)$
 \item
      $T$の固有値$\lambda$
 \item
      $T$の各固有値$\lambda$について固有空間$W(\lambda ; T)$
\end{enumerate}

\dotfill

\begin{enumerate}
 \item
      $T(f(x))=f(1-x)$

      \dotfill

      $f(x)\in\mathbb{R}[x]_{2}$であるので
      $f(x)=a_0+a_1x+a_2x^2$とする。
      線形変換$T$で$f(x)$を移すと
      次のようになる。
      \begin{equation}
       T(f(x))
        =a_0+a_1(1-x)+a_2(1-x)^2
        =a_0+a_1+a_2 - (a_1 +2a_2)x + a_2x^2
      \end{equation}

      これは基底$1,x,x^2$がどのように対応するかを調べることで
      表現行列が得られる。
      \begin{equation}
       T(1)=1,\ T(x)=1-x,\ T(x^2)=(1-x)^2=1-2x+x^2
      \end{equation}
      基底の対応に合わせて次のように行列で表記する。
      \begin{equation}
       \begin{pmatrix}
        1\\x\\x^2
       \end{pmatrix}
       \mapsto
       \begin{pmatrix}
        1\\1-x\\1-2x+x^2
       \end{pmatrix}
       =
       \begin{pmatrix}
        1 & 0 & 0 \\
        1 & -1 & 0 \\
        1 & -2 & 1
       \end{pmatrix}
       \begin{pmatrix}
        1\\x\\x^2
       \end{pmatrix}
      \end{equation}


      この時現れる行列が表現行列である。
      \begin{equation}
       \begin{pmatrix}
        1 & 0 & 0 \\
        1 & -1 & 0 \\
        1 & -2 & 1
       \end{pmatrix}
      \end{equation}


      \hrulefill

 \item
      $T(f(x))=f(2x)+f^{\prime}(x)$

      \dotfill

      基底$1,x,x^2$について対応を調べる

      \begin{align}
       T(1) =& 1 + (1)^{\prime} =1\\
       T(x) =& 2x + (x)^{\prime} =2x+1\\
       T(x^2) =& 2x^2 + (x^2)^{\prime} =2x^2+2x
      \end{align}

      これを行列で表す。
      \begin{equation}
       \begin{pmatrix}
        1\\2x+1\\2x^2+2x
       \end{pmatrix}
       =
       \begin{pmatrix}
        1 & 0 & 0 \\
        1 & 2 & 0 \\
        0 & 2 & 2
       \end{pmatrix}
       \begin{pmatrix}
        1\\x\\x^2
       \end{pmatrix}
      \end{equation}

      つまり表現行列は次の通り。
      \begin{equation}
       \begin{pmatrix}
        1 & 0 & 0 \\
        1 & 2 & 0 \\
        0 & 2 & 2
       \end{pmatrix}
      \end{equation}

      \hrulefill

\end{enumerate}



\hrulefill

\end{document}
