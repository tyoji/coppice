\documentclass[12pt,b5paper]{ltjsarticle}

%\usepackage[margin=15truemm, top=5truemm, bottom=5truemm]{geometry}
\usepackage[margin=15truemm]{geometry}

\usepackage{amsmath,amssymb}
%\pagestyle{headings}
\pagestyle{empty}

%\usepackage{listings,url}
%\renewcommand{\theenumi}{(\arabic{enumi})}

\usepackage{graphicx}

\usepackage{tikz}
\usetikzlibrary {arrows.meta}
\usepackage{wrapfig}	% required for `\wrapfigure' (yatex added)
\usepackage{bm}	% required for `\bm' (yatex added)

% ルビを振る
\usepackage{luatexja-ruby}	% required for `\ruby'

%% 核Ker 像Im Hom を定義
%\newcommand{\Img}{\mathop{\mathrm{Im}}\nolimits}
%\newcommand{\Ker}{\mathop{\mathrm{Ker}}\nolimits}
%\newcommand{\Hom}{\mathop{\mathrm{Hom}}\nolimits}
\newcommand{\Rot}{\mathop{\mathrm{rot}}\nolimits}
\newcommand{\Div}{\mathop{\mathrm{div}}\nolimits}
\newcommand{\Grad}{\mathop{\mathrm{grad}}\nolimits}

\begin{document}

\hrulefill

\textbf{$\log$のTaylor展開}

$f(x)=\log(1+2x)$とする。
\begin{align}
 f(x) =& \log(1+2x) & f(0) =& 0\\
 f^{\prime}(x) =& \frac{2}{1+2x} & f^{\prime}(0) =& 2\\
 f^{\prime\prime}(x) =& \frac{-4}{(1+2x)^2} & f^{\prime}(0) =& -4\\
 f^{\prime\prime\prime}(x) =& \frac{16}{(1+2x)^3} & f^{\prime}(0) =& 16\\
 f^{(4)}(x) =& \frac{-96}{(1+2x)^4} & f^{\prime}(0) =& -96\\
 f^{(n)}(x) =& \frac{2^n(-1)^{n-1}(n-1)!}{(1+2x)^n} & f^{\prime}(0) =& 2^n(-1)^{n-1}(n-1)!
\end{align}

$x=0$におけるTaylor展開
\begin{align}
 f(x) =& \frac{f(0)}{0!}x^0 + \frac{f^{\prime}(0)}{1!}x^1 + \frac{f^{\prime\prime}(0)}{2!}x^2 + 
   \frac{f^{\prime\prime\prime}(0)}{3!}x^3 + \frac{f^{(4)}(0)}{4!}x^4 +
   \cdots + \frac{f^{(n)}(0)}{n!}x^n + \cdots\\
 \log(1+2x) =& \frac{0}{1}x^0 + \frac{2}{1}x + \frac{-4}{2}x^2 + \frac{16}{6}x^3 +
   \frac{-96}{24}x^4 + \cdots + \frac{2(-2)^{n-1}(n-1)!}{n!}x^n + \cdots \\
 =& 2x-2x^2+\frac{8}{3}x^3 - 4x^4 + \cdots + \frac{2^n(-1)^{n-1}}{n}x^n + \cdots
\end{align}


$x\in [0,1]$において、$x^n > x^{n+1}$である。

上記、Taylor展開により
\begin{equation}
 2x-2x^2+\frac{8}{81}x^3
  \leq
  2x-2x^2+\frac{8}{3}x^3 - 4x^4 + \cdots
  = \log(1+2x)
  \label{low}
\end{equation}
であり、
\begin{equation}
 \log(1+2x)
  = 2x-2x^2+\frac{8}{3}x^3 - 4x^4 + \cdots
  \leq
  2x-2x^2+\frac{8}{3}x^3
  \label{hi}
\end{equation}
となるので、
\begin{equation}
  2x-2x^2+\frac{8}{81}x^3
 \leq
 \log(1+2x)
 \leq
  2x-2x^2+\frac{8}{3}x^3
\end{equation}

\hrulefill

Lagrangeの剰余項
\begin{equation}
 R_n=f^{(n)}(c)\times \frac{(x-a)^n}{n!} ,\quad c\in(a,x)
\end{equation}

$x=0$におけるTaylor展開とその剰余項(Lagrange)
\begin{align}
 \log(1+2x)
  =& 2x-2x^2+ \frac{16}{(1+2c)^3}\frac{x^3}{3!}\label{3ji}\\
  =& 2x-2x^2+\frac{8}{3}x^3 + \frac{-96}{(1+2c)^4}\frac{x^4}{4!}\label{4ji}
\end{align}

\textbf{左の不等式}

(\ref{3ji})は第3項がLagrangeの剰余項である。
(\ref{low})を示すには次の不等式を確認すればよい。
\begin{equation}
 2x-2x^2+\frac{8}{81}x^3
  \leq 2x-2x^2+ \frac{16}{(1+2c)^3}\frac{x^3}{3!}
\end{equation}

そこで右辺$-$左辺が正になることを確認する。
\begin{align}
 & \left(2x-2x^2+ \frac{16}{(1+2c)^3}\frac{x^3}{3!}\right)
 - \left(2x-2x^2+\frac{8}{81}x^3\right)\\
 =& \frac{16}{(1+2c)^3}\frac{x^3}{3!} - \frac{8}{81}x^3\\
 =& \left( \frac{1}{(1+2c)^3} - \frac{1}{27} \right)\frac{8}{3}x^3
 > 0
\end{align}
$c\in(0,x)$より$c<1$であるので、$1+2c<3$である。
また、$x\in[0,1]$であるので、正であることがわかる。


\textbf{右の不等式}

(\ref{4ji})は第4項がLagrangeの剰余項である。
(\ref{hi})を示すには次の不等式を確認すればよい。
\begin{equation}
 2x-2x^2+\frac{8}{3}x^3 + \frac{-96}{(1+2c)^4}\frac{x^4}{4!}
  \leq 2x-2x^2+\frac{8}{3}x^3
\end{equation}

そこで右辺$-$左辺が正になることを確認する。
\begin{align}
 & \left( 2x-2x^2+\frac{8}{3}x^3 \right)
 - \left( 2x-2x^2+\frac{8}{3}x^3 + \frac{-96}{(1+2c)^4}\frac{x^4}{4!}\right)\\
 =& - \frac{-96}{(1+2c)^4}\frac{x^4}{4!}
 = \frac{4}{(1+2c)^4}x^4
 > 0
\end{align}
$c\in(0,x),\ x\in[0,1]$であるので正であることがわかる。




\end{document}
