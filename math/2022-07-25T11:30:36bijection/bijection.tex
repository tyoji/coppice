\documentclass[12pt,b5paper]{ltjsarticle}

%\usepackage[margin=15truemm, top=5truemm, bottom=5truemm]{geometry}
\usepackage[margin=15truemm]{geometry}

\usepackage{amsmath,amssymb}
%\pagestyle{headings}
\pagestyle{empty}

%\usepackage{listings,url}
%\renewcommand{\theenumi}{(\arabic{enumi})}

\usepackage{graphicx}

\usepackage{tikz}
\usetikzlibrary {arrows.meta}
\usepackage{wrapfig}	% required for `\wrapfigure' (yatex added)
\usepackage{bm}	% required for `\bm' (yatex added)

% ルビを振る
%\usepackage{luatexja-ruby}	% required for `\ruby'

%% 核Ker 像Im Hom を定義
%\newcommand{\Img}{\mathop{\mathrm{Im}}\nolimits}
%\newcommand{\Ker}{\mathop{\mathrm{Ker}}\nolimits}
%\newcommand{\Hom}{\mathop{\mathrm{Hom}}\nolimits}
\newcommand{\Rot}{\mathop{\mathrm{rot}}\nolimits}
\newcommand{\Div}{\mathop{\mathrm{div}}\nolimits}

\begin{document}

\hrulefill

$f:\mathbb{Z}\to\mathbb{Z}$が次で示す写像である時、
全単射であることを示せ。
\begin{equation}
 f(x)=
  \begin{cases}
   x & (x:\text{偶数})\\
   -x & (x:\text{奇数})
  \end{cases}
\end{equation}

\dotfill

$f$が全射であることと単射であることを示す。

${}^\forall z\in\mathbb{Z}$について
$z$は偶数か奇数のどちらかである。
\begin{itemize}
 \item
      $z$が偶数であるなら$f(z)=z$である
 \item
      $z$が奇数であるなら$f(-z)=z$である
\end{itemize}
この為、$f$は全射である。

${}^{\forall}a,b\in\mathbb{Z},\ a\ne b$とする。
\begin{itemize}
 \item
      $a,b$の組合せは (偶数、偶数)の場合

      $f(a)=a, \ f(b)=b$より、$f(a)\ne f(b)$となる。
      つまり、$a\ne b \Rightarrow f(a)\ne f(b)$
 \item
      $a,b$の組合せは (奇数、奇数)の場合

      $f(a)=-a, \ f(b)=-b$より、$f(a)\ne f(b)$となる。
      つまり、$a\ne b \Rightarrow f(a)\ne f(b)$
 \item
      $a,b$の組合せは (偶数、奇数)の場合

      $f(a)=a, \ f(b)=-b$である。
      $a$は偶数より$f(a)$も偶数で、
      $b$は奇数より$f(b)$も奇数である。
      よって、偶数奇数が異なる為$f(a)\ne f(b)$となる。
      つまり、$a\ne b \Rightarrow f(a)\ne f(b)$
 \item
      $a,b$の組合せは (奇数、偶数)の場合

      $a,b$の組合せは (偶数、奇数)の場合と同様
\end{itemize}
これらから$f$は単射である。

よって、$f$は全単射である。

\hrulefill

写像$g:A\to B$について

\textbf{全射}

$B$の任意の要素について$A$の要素が対応することである。
記号で書くと次の通り。
\begin{equation}
 {}^{\forall}b\in B %\Rightarrow
  {}^{\exists}a\in A s.t. b=g(a)
\end{equation}


\textbf{単射}

$A$の任意の異なる2つの要素に
対応する$B$の要素も異なることである。
記号で書くと次の通り。
\begin{equation}
 {}^{\forall}\alpha, \beta \in A, \alpha \ne \beta
  \Rightarrow g(\alpha)\ne g(\beta)
\end{equation}

これは待遇を取って次のように考えることも多い。
\begin{equation}
 g(\alpha)= g(\beta)
  \Rightarrow \alpha = \beta
\end{equation}
\end{document}
