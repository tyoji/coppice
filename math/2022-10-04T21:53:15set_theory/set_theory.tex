\documentclass[12pt,b5paper]{ltjsarticle}

%\usepackage[margin=15truemm, top=5truemm, bottom=5truemm]{geometry}
\usepackage[margin=10truemm]{geometry}

\usepackage{amsmath,amssymb}
%\pagestyle{headings}
\pagestyle{empty}

%\usepackage{listings,url}
%\renewcommand{\theenumi}{(\arabic{enumi})}

%\usepackage{graphicx}

%\usepackage{tikz}
%\usetikzlibrary {arrows.meta}
%\usepackage{wrapfig}	% required for `\wrapfigure' (yatex added)
%\usepackage{bm}	% required for `\bm' (yatex added)

% ルビを振る
%\usepackage{luatexja-ruby}	% required for `\ruby'

%% 核Ker 像Im Hom を定義
%\newcommand{\Img}{\mathop{\mathrm{Im}}\nolimits}
%\newcommand{\Ker}{\mathop{\mathrm{Ker}}\nolimits}
%\newcommand{\Hom}{\mathop{\mathrm{Hom}}\nolimits}

%\DeclareMathOperator{\Rot}{rot}
%\DeclareMathOperator{\Div}{div}
%\DeclareMathOperator{\Grad}{grad}
%\DeclareMathOperator{\arcsinh}{arcsinh}
%\DeclareMathOperator{\arccosh}{arccosh}
%\DeclareMathOperator{\arctanh}{arctanh}



\begin{document}


\hrulefill

$\{0,1\}^{\mathbb{N}}$は非可算集合であることを示せ。

\dotfill

$\{0,1\}^{\mathbb{N}}$は次のような写像の集合である。
\begin{equation}
 \{0,1\}^{\mathbb{N}} =
  \{
  f \mid f:\mathbb{N}\to \{0,1\}
  \}
\end{equation}

べき集合$2^{\mathbb{N}}$は自然数$\mathbb{N}$の部分集合全体の集合とする。
\begin{equation}
 2^{\mathbb{N}} =
  \{
  A \mid A\subset \mathbb{N}
  \}
\end{equation}

写像$F:\{0,1\}^{\mathbb{N}} \to 2^{\mathbb{N}}$を定義する。
$F$は$f\in\{0,1\}^\mathbb{N}$に対し、
$f(a)=1$となる自然数$a\in\mathbb{N}$を集めた集合を対応させる。
\begin{equation}
 F(f)=\{a\in \mathbb{N} \mid f(a)=1 \}
\end{equation}

これにより、$f\ne g \Rightarrow F(f)\ne F(g)$となるので単射、
${}^\forall A\in 2^{\mathbb{N}}$に対して
$a\in A$なら$f(a)=1$、$a\not\in A$なら$f(a)\ne1$となる写像$f$が
存在し
$F(f)=A$となるので全射である。

この写像$F$により2つの集合$\{0,1\}^{\mathbb{N}} ,\ 2^{\mathbb{N}}$は
濃度が同じである。
$2^{\mathbb{N}}$は非可算濃度なので
$\{0,1\}^{\mathbb{N}}$も非可算となる。


\dotfill

\textbf{$2^{\mathbb{N}}$が非可算濃度である}

$2^{\mathbb{N}}$は自然数のべき集合とする。
つまり、%自然数全体の集合
$\mathbb{N}$の部分集合をすべて集めた集合とする。
\begin{equation}
 2^{\mathbb{N}}
  =
  \{ A \mid A \subset \mathbb{N}\}
\end{equation}

任意の自然数$n\in\mathbb{N}$に対して
自然数一つだけを要素に持つ集合$\{n\}\subset\mathbb{N}$
を対応させる写像$f$を考える。
\begin{equation}
 f: \mathbb{N} \to 2^{\mathbb{N}}
  ,\qquad
  n \mapsto \{n\}
\end{equation}

このとき、$f$は単射である。
これにより
$2^{\mathbb{N}}$は可算濃度以上であることが分かる。


次に逆向きの写像$g$を考える。
\begin{equation}
 g: 2^{\mathbb{N}} \to \mathbb{N}
\end{equation}


$g$は単射でないことを背理法で示す。

$g$を単射であると仮定
し、
$\mathbb{N}$の部分集合$S$を次のように定義する。
\begin{equation}
 S = \{
  g(A)\in\mathbb{N} \mid A\in 2^{\mathbb{N}},\ g(A)\not\in A
  \}
\end{equation}

$S\subset \mathbb{N}$であるので、
すべての自然数$a\in\mathbb{N}$は
$a\in S$又は$a\not\in S$である。
そこで、$a=g(S)$として$a$が$S$に含まれるかどうかを考える。



\textbf{$a\not\in S$の場合}

$S\in 2^{\mathbb{N}}$ かつ $a=g(S)\not\in S$
であるので
$a$は$S$の定義を満たし$a\in S$となり
矛盾。


\textbf{$a\in S$の場合}

$S$の定義から
$X\in 2^{\mathbb{N}}$
であり
$g(X)\not\in X$
となる集合$X$が存在し、
$a=g(X)$である。
$g$は単射であるので$g(X)=g(S) \Rightarrow X=S$である。
つまり、$g(S)\not\in S$となり矛盾。


以上により写像$g$は単射ではないことが分かる。

つまり、$2^{\mathbb{N}}$は
可算濃度以上ではあるが、可算濃度ではないことから
非加算であることが分かる。




\dotfill
$\{0,1\}^{\mathbb{N}}$と$\mathbb{N}$を比較する場合
\dotfill

写像$f$を次のように定義する。
\begin{equation}
 f:
 \mathbb{N}
  \rightarrow
  \{0,1\}^{\mathbb{N}}
  \qquad
  n \mapsto h
  \quad
  h(x)=
  \begin{cases}
   1 & (x=n)\\
   0 & (\text{otherwise})
  \end{cases}
\end{equation}
自然数$n$に対して
$n$を1に写しそれ以外は0に写す写像$h\in\{0,1\}^{\mathbb{N}}$
を対応させると
$f$は単射になる。

逆に次のような写像$g$を考える。
\begin{equation}
 g:
  \{0,1\}^{\mathbb{N}}
  \rightarrow
   \mathbb{N}
\end{equation}
$g$は単射でないことを示す。

$g$を単射と仮定する。

$\mathbb{N}$の部分集合$S$を次のように定義する。
\begin{equation}
 S=\{g(h)\in\mathbb{N} \mid h\in\{0,1\}^{\mathbb{N}} ,\ h(g(h))=0\}
\end{equation}






\hrulefill

2つの区間$[0,1]$と$(0,1]$の間に
全単射を具体的に構成せよ。

\dotfill

2つの区間はほとんど同じなので次のような写像を作りたい。
\begin{equation}
 f_0 : [0,1] \to (0,1] \quad x \mapsto x
\end{equation}

しかし$f_0(0)$が定義されていない為、$f_0$は写像ではない。
そこで、$f_0(0)=1/2$となるように修正をした写像$f_1$を考える。

\begin{equation}
 f_1 : [0,1] \to (0,1] ,\qquad
 f_1(x) = 
  \begin{cases}
   \frac{1}{2} & (x = 0)\\
   x & (\text{otherwise})
  \end{cases}
\end{equation}
これにより$f_1$は写像となる。
任意の$(0,1]$に対応する点が$[0,1]$にある為
全射である。
ただし、$f_1(0)=f_1(1/2)=1/2$より単射ではない。
そこで$f_1(1/2)$を$1/2$ではない点に写す。


\begin{equation}
 f_2 : [0,1] \to (0,1] ,\qquad
 f_2(x) = 
  \begin{cases}
   \frac{1}{2} & (x = 0)\\
   \frac{3}{4} = \frac{1}{2}+\frac{1}{2^2} & (x = \frac{1}{2})\\
   x & (\text{otherwise})
  \end{cases}
\end{equation}

写像$f_2$は全射となる。
ある一点が重なるため単射ではない。
そこで、先程と同じように$f_2(3/4)$を他の点に写す。

\begin{equation}
 f_3 : [0,1] \to (0,1] ,\qquad
 f_3(x) = 
  \begin{cases}
   \frac{1}{2} & (x = 0)\\
   \frac{3}{4} = \frac{1}{2}+\frac{1}{2^2} & (x = \frac{1}{2})\\
   \frac{7}{8} = \frac{1}{2}+\frac{1}{2^2}+\frac{1}{2^3} & (x = \frac{3}{4})\\
   x & (\text{otherwise})
  \end{cases}
\end{equation}

これを繰り返すと
1までの距離が$\frac{1}{2^n}$である点は
1までの距離が$\frac{1}{2^{n+1}}$の点に写すことになる。


つまり、次の写像$f$が全単射となる。
\begin{equation}
 f : [0,1] \to (0,1] ,\qquad
 f(x) =
  \begin{cases}
   1-\frac{1}{2^{n+1}} & (x= 1-\frac{1}{2^n}, n=0,1,2,\dots)\\
   x & (\text{otherwise})
  \end{cases}
\end{equation}



\hrulefill


%\begin{equation}
% \mathbb{TEST} \quad
% \mathcal{TEST} \quad
%% \mathscr{TEST} \quad
% \mathfrak{TEST}
%\end{equation}


\end{document}

