\documentclass[12pt,b5paper]{ltjsarticle}

%\usepackage[margin=15truemm, top=5truemm, bottom=5truemm]{geometry}
\usepackage[margin=15truemm]{geometry}

\usepackage{amsmath,amssymb}
%\pagestyle{headings}
\pagestyle{empty}

%\usepackage{listings,url}
\renewcommand{\theenumi}{(\arabic{enumi})}

\usepackage{graphicx}

\usepackage{tikz}
\usetikzlibrary {arrows.meta}
\usepackage{wrapfig}	% required for `\wrapfigure' (yatex added)

%% 像Im を定義
%\newcommand{\Img}{\mathop{\mathrm{Im}}\nolimits}

\begin{document}

\textbf{問}

\begin{enumerate}
 \item $\displaystyle
       \lim_{x\rightarrow +0}\left( \frac{\mathrm{d}^2}{\mathrm{d}x^2}e^{-\frac{1}{x}}\right)$
       を計算せよ。
       \label{213949_25Apr22}
 \item $\alpha \in \mathbb{R}$、$f(x)=x^\alpha \; (x\ne 0)$とする。
       この時、
       $\displaystyle \int_1^2 f(x) \mathrm{d}x$
       を求めよ。
       \label{214119_25Apr22}
 \item $f(x,y,z) = x^2+y^3+z^4$とし、$g(t)=f(t-1,t-2,t-3)$とする。
       この時、
       $\displaystyle \frac{\mathrm{d}}{\mathrm{d}t}g(t)$の
       $t=3$での値を求めよ。
       \label{214131_25Apr22}
 \item $f: [0,\pi] \rightarrow \mathbb{R}$を
       \label{214140_25Apr22}
       \begin{equation*}
        f(x) =
         \begin{cases}
          \sin x & x\in \left(\frac{\pi}{3},\frac{2\pi}{3}\right)\\
          0 & \text{other}
         \end{cases}
       \end{equation*}
       とする。
       この時、
       $\displaystyle \int_0^{\pi} f(x) \mathrm{d}x$
       を求めよ。
\end{enumerate}

\hrulefill

\textbf{問\ref{213949_25Apr22}}
$
\lim_{x\rightarrow +0}\left( \frac{\mathrm{d}^2}{\mathrm{d}x^2}e^{-\frac{1}{x}}\right)
$

極限を求めるためにまず微分をする。
\begin{equation}
 \frac{\mathrm{d}}{\mathrm{d}x}e^{-\frac{1}{x}} = x^{-2}e^{-x^{-1}}
\end{equation}

\begin{align}
 \frac{\mathrm{d}^2}{\mathrm{d}x^2}e^{-\frac{1}{x}}
  =& \frac{\mathrm{d}}{\mathrm{d}x} x^{-2}e^{-x^{-1}}\\
  =& -2x^{-3}e^{-x^{-1}} + x^{-4}e^{-x^{-1}}\\
  =& x^{-4}e^{-x^{-1}} (  -2x + 1 )
\end{align}

後半部分の極限は
$  -2x + 1 \rightarrow 1 (x\rightarrow +0)$
となるので、前半部分について考える。

\begin{equation}
 \lim_{x\rightarrow +0}\frac{1}{x^4e^{\frac{1}{x}}} = 0
\end{equation}

\dotfill
$\frac{1}{x^4e^{\frac{1}{x}}}$
\dotfill


\begin{equation}
 x^{-4}e^{-x^{-1}} = \frac{1}{x^4e^{\frac{1}{x}}}
\end{equation}
この分母がどのような値に近づくかを調べる。
$f(x)= x^4e^{\frac{1}{x}}$とする。
$f^\prime (x)$を計算する。
\begin{align}
 f^\prime (x)
 =& 4x^3e^{\frac{1}{x}} - x^2e^{\frac{1}{x}}\\
 =& x^2e^{\frac{1}{x}} (4x - 1)
\end{align}
$f^\prime (x)=0$を満たすのは$x=0,1/4$であるが、
$x>0$の範囲では$x=1/4$だけになる。

$x>0$の範囲では$f(x)>0$であり、
$f(1)=1<f(2)$であるから$x=1/4$のときに極小となる。
実際に$0<x<1/4$において$f^\prime (x)<0$である。

よって
\begin{equation}
 \lim_{x\rightarrow +0} x^4e^{\frac{1}{x}} = +\infty
\end{equation}



\dotfill


\begin{equation}
 \lim_{x\rightarrow +0}\frac{1}{x^4e^{\frac{1}{x}}} = 0
\end{equation}
であるから

\begin{align}
 \lim_{x\rightarrow +0}\left( \frac{\mathrm{d}^2}{\mathrm{d}x^2}e^{-\frac{1}{x}}\right)
 &= \lim_{x\rightarrow +0} x^{-4}e^{-x^{-1}} (  -2x + 1 )\\
 &= 0
\end{align}

\hrulefill

\textbf{問\ref{214119_25Apr22}}

$f(x)=x^\alpha \; (x\ne 0)$の積分は$\alpha$によって変わる。

$\alpha=0$の場合

\begin{equation}
  \int_1^2 f(x) \mathrm{d}x = \int_1^2 \mathrm{d}x = 1
\end{equation}

$\alpha=-1$の場合

\begin{align}
 \int_1^2 f(x) \mathrm{d}x
 &= \int_1^2 x^{-1} \mathrm{d}x\\
 &= \left[\log x\right]_1^2 = \log 2
\end{align}

上記以外

\begin{align}
 \int_1^2 f(x) \mathrm{d}x
 &= \int_1^2 x^{\alpha} \mathrm{d}x\\
 &= \left[ \frac{1}{\alpha+1} x^{\alpha+1} \right]_1^2 = 
 \frac{1}{\alpha+1} (2^{\alpha+1} -1)
\end{align}


\hrulefill

\textbf{問\ref{214131_25Apr22}}
\begin{align}
 g(t) &= f(t-1,t-2,t-3)\\
 &= (t-1)^2+(t-2)^3+(t-3)^4
\end{align}

これを$t$で微分する。

\begin{equation}
 \frac{\mathrm{d}}{\mathrm{d}t}g(t) = 2(t-1) + 3(t-2)^2 + 4(t-3)^3
\end{equation}

これに$t=3$を代入し
$2(3-1) + 3(3-2)^2 + 4(3-3)^3 = 2\times 2 + 3 = 7$
となる。

\hrulefill

\textbf{問\ref{214140_25Apr22}}

積分を区間ごとに分け、
その区間における$f(x)$に置き換える。

\begin{align}
 \displaystyle \int_0^{\pi} f(x) \mathrm{d}x
  &= \int_{\frac{2\pi}{3}}^{\pi} f(x) \mathrm{d}x
  + \int_{\frac{\pi}{3}}^{\frac{2\pi}{3}} f(x) \mathrm{d}x
  + \int_0^{\frac{\pi}{3}} f(x) \mathrm{d}x\\
  &= \int_{\frac{2\pi}{3}}^{\pi} 0 \mathrm{d}x
  + \int_{\frac{\pi}{3}}^{\frac{2\pi}{3}} \sin x \mathrm{d}x
  + \int_0^{\frac{\pi}{3}} 0 \mathrm{d}x\\
  &= \int_{\frac{\pi}{3}}^{\frac{2\pi}{3}} \sin x \mathrm{d}x
 = \left[ -\cos x \right]_{\frac{\pi}{3}}^{\frac{2\pi}{3}}\\
  &= -\left( -\frac{1}{2}\right) - \left( -\frac{1}{2} \right)
 = 1
\end{align}







\end{document}
