\documentclass[12pt,b5paper]{ltjsarticle}

%\usepackage[margin=15truemm, top=5truemm, bottom=5truemm]{geometry}
%\usepackage[margin=10truemm,left=15truemm]{geometry}
\usepackage[margin=10truemm]{geometry}

\usepackage{amsmath,amssymb}
%\pagestyle{headings}
\pagestyle{empty}

%\usepackage{listings,url}
%\renewcommand{\theenumi}{(\arabic{enumi})}

\usepackage{graphicx}

%\usepackage{tikz}
%\usetikzlibrary {arrows.meta}
%\usepackage{wrapfig}	% required for `\wrapfigure' (yatex added)
%\usepackage{bm}	% required for `\bm' (yatex added)

% ルビを振る
%\usepackage{luatexja-ruby}	% required for `\ruby'

%% 核Ker 像Im Hom を定義
%\newcommand{\Img}{\mathop{\mathrm{Im}}\nolimits}
%\newcommand{\Ker}{\mathop{\mathrm{Ker}}\nolimits}
%\newcommand{\Hom}{\mathop{\mathrm{Hom}}\nolimits}

%\DeclareMathOperator{\Rot}{rot}
%\DeclareMathOperator{\Div}{div}
%\DeclareMathOperator{\Grad}{grad}
%\DeclareMathOperator{\arcsinh}{arcsinh}
%\DeclareMathOperator{\arccosh}{arccosh}
%\DeclareMathOperator{\arctanh}{arctanh}



%\usepackage{listings,url}
%
%\lstset{
%%プログラム言語(複数の言語に対応,C,C++も可)
%  language = Python,
%%  language = Lisp,
%%  language = C,
%  %背景色と透過度
%  %backgroundcolor={\color[gray]{.90}},
%  %枠外に行った時の自動改行
%  breaklines = true,
%  %自動改行後のインデント量(デフォルトでは20[pt])
%  breakindent = 10pt,
%  %標準の書体
%%  basicstyle = \ttfamily\scriptsize,
%  basicstyle = \ttfamily,
%  %コメントの書体
%%  commentstyle = {\itshape \color[cmyk]{1,0.4,1,0}},
%  %関数名等の色の設定
%  classoffset = 0,
%  %キーワード(int, ifなど)の書体
%%  keywordstyle = {\bfseries \color[cmyk]{0,1,0,0}},
%  %表示する文字の書体
%  %stringstyle = {\ttfamily \color[rgb]{0,0,1}},
%  %枠 "t"は上に線を記載, "T"は上に二重線を記載
%  %他オプション:leftline,topline,bottomline,lines,single,shadowbox
%  frame = TBrl,
%  %frameまでの間隔(行番号とプログラムの間)
%  framesep = 5pt,
%  %行番号の位置
%  numbers = left,
%  %行番号の間隔
%  stepnumber = 1,
%  %行番号の書体
%%  numberstyle = \tiny,
%  %タブの大きさ
%  tabsize = 4,
%  %キャプションの場所("tb"ならば上下両方に記載)
%  captionpos = t
%}



\begin{document}

\hrulefill

$a\in\mathbb{Z}$で、$a\ne 1, -3, -5$とする。
この時、
$x^3+ax+2$は既約$\mathbb{Z}$-多項式であることを示せ。

\dotfill

$x^3+ax+2$はモニック多項式であるので、
次数が0と3の式に分けることはない。

つまり、$x^3+ax+2=fg$となる$\deg{f}=1,\deg{g}=2$があるとする。
$f,g$は次のような式とする。
\begin{equation}
 f=x+\alpha_0
  ,\quad
 g=x^2+\beta_1 x + \beta_0
 \qquad
 \alpha_0,\beta_0,\beta_1 \in\mathbb{Z}
\end{equation}


$fg$を計算する。
\begin{align}
 fg=&(x+\alpha_0)(x^2+\beta_1 x + \beta_0)\\
  =& x^3 + (\alpha_0+\beta_1)x^2
 + (\alpha_0\beta_1 + \beta_0)x + \alpha_0\beta_0
\end{align}

$x^3+ax+2=fg$より
次の3つの式が得られる。
\begin{equation}
 \alpha_0+\beta_1=0
  ,\quad
 \alpha_0\beta_1 + \beta_0=a
  ,\quad
 \alpha_0\beta_0=2
\end{equation}

$\alpha_0\beta_0=2$を満たす整数の組は次の4つである。
\begin{equation}
 (\alpha_0,\ \beta_0)=
  (1,2),\ (2,1),\ (-1,-2),\ (-2,-1)
\end{equation}

$\alpha_0+\beta_1=0$より$\beta_1=-\alpha_0$となるので、
$\alpha_0\beta_1 + \beta_0=a$より
$-\alpha_0^2 + \beta_0=a$となる。
これに$(\alpha_0,\ \beta_0)$を代入し
$a$を求める。

\begin{itemize}
 \item
      $(\alpha_0,\beta_0)=(1,2)$の時、
      $-\alpha_0^2 + \beta_0=1$となる。

 \item
      $(\alpha_0,\beta_0)=(-1,-2)$の時、
      $-\alpha_0^2 + \beta_0=-3$となる。

 \item
      $(\alpha_0,\beta_0)=(2,1)$の時、
      $-\alpha_0^2 + \beta_0=-3$となる。

 \item
      $(\alpha_0,\beta_0)=(-2,-1)$の時、
      $-\alpha_0^2 + \beta_0=-5$となる。

\end{itemize}

よって、
$x^3+ax+2=fg$となるのであれば$a=1,-3,-5$となる。

つまり、
$a\ne1,-3,-5$のとき
$x^3+ax+2$は既約である。

\hrulefill

\end{document}
