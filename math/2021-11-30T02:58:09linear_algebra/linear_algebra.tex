\documentclass[10pt,b5paper]{ltjsarticle}

\usepackage[margin=15truemm]{geometry}

\usepackage{amsmath}
\usepackage{amssymb}
\pagestyle{empty}

\usepackage{bm}	% required for `\bm' (yatex added)
\begin{document}

\begin{enumerate}
 \item 
       \begin{enumerate}
        \item $c$:定数、$E$:3次単位行列、$\lvert cE \rvert$の値
              
              3次の正方行列Aの行列式は次のようにして計算できる。
              \begin{align}
               A =& \left(
                 \begin{matrix}a & b & c\\ d & e & f \\ g & h & i\end{matrix}
               \right)\\
               \lvert A \rvert =& \left|
               \begin{matrix}a & b & c\\ d & e & f \\ g & h & i\end{matrix} \right| 
                         = a\left|\begin{matrix}e & f \\h & i\end{matrix} \right|
                        - b\left|\begin{matrix}d & f \\g & i\end{matrix} \right|
                        + c\left|\begin{matrix}d & e \\g & h\end{matrix} \right|\\
               =& a(ei-fh)-b(di-fg)+c(dh-eg)
              \end{align}

              三角行列で行う場合、$0$となる場所が出てくるので、
              行列式は対角成分の積を求めるだけでよい。

                        $cE=\left(\begin{matrix}c&0&0\\0&c&0\\0&0&c\\\end{matrix}\right)$より
              $\lvert cE \rvert = c^3$となる。

        \item 3次正方行列$A$の余因子行列$\tilde{A}$に対し、
              $\lvert \tilde{A} \rvert =\lvert A \rvert^2$

              $A^{-1}=\frac{1}{\lvert A \rvert}\tilde{A}$を認めるなら
              次のようになる。
              \begin{align}
               A^{-1}=&\frac{1}{\lvert A \rvert}\tilde{A} &  両辺に \lvert A \rvert Aをかける\\
               \lvert A \rvert E =&\tilde{A}A & 両辺の行列式を求める\\
               \lvert \lvert A \rvert E\rvert =& \lvert\tilde{A}A\rvert\\
               \lvert A \rvert ^3 =& \lvert\tilde{A}\rvert \lvert A\rvert & \lvert A \rvert で割る \\
               \lvert A \rvert ^2 =& \lvert\tilde{A}\rvert
              \end{align}

              講義ではこれを逆にたどり
              $A^{-1}=\frac{1}{\lvert A \rvert}\tilde{A}$を示したと思います。
              $\lvert A \rvert ^2 = \lvert \tilde{A}\rvert$の
              証明をするのであれば講義の順を追う必要があります。

       \end{enumerate}
 \item 連立方程式$A\boldsymbol{x}=\bm{0}$の解を求めよ。ただし、$A$は正則。

       行列$A$は正則である為、逆行列$A^{-1}$が存在する。
       これを連立方程式の両辺にかけると$\bm{x}$が求まる。
       \begin{equation}
        \bm{x}=A^{-1}\bm{0} = \bm{0}
       \end{equation}

 \item
       $\bm{a_1}=\left(\begin{matrix}1\\0\end{matrix}\right)$、
       $\bm{a_2}=\left(\begin{matrix}1\\1\end{matrix}\right)$
      が線形独立であることを示せ。
      \begin{enumerate}
       \item
            線形結合$c_1\bm{a_1}+c_2\bm{a_2}=\bm{0}$を書き換えると次のようになる。
            \begin{gather}
             \left(\bm{a_1}\ \bm{a_2}\right)
                   \left(\begin{matrix}c_1\\c_2\end{matrix}\right)=\bm{0}\\
             \left(\begin{matrix}1&1\\0&1\end{matrix}\right)
                   \left(\begin{matrix}c_1\\c_2\end{matrix}\right)=\bm{0}
            \end{gather}
            これにより$A=\left(\begin{matrix}1&1\\0&1\end{matrix}\right)$

       \item
            行列$A$は正則である。

            行列式が$0$でない場合に正則であるので、
            $\lvert A \rvert=1\ne0$より$A$は正則。

       \item
            $\bm{a_1}, \bm{a_2}$の線形独立性。

            $A=(\bm{a_1}\ \bm{a_2})$とすると
            \begin{gather}
             c_1\bm{a_1}+c_2\bm{a_2}=\bm{0}\\
             (\bm{a_1}\ \bm{a_2})\left(\begin{matrix}c_1\\c_2\end{matrix}\right)=\bm{0}\\
             A\left(\begin{matrix}c_1\\c_2\end{matrix}\right)=\bm{0}\\
             A^{-1}A\left(\begin{matrix}c_1\\c_2\end{matrix}\right)=A^{-1}\bm{0}\\
             \left(\begin{matrix}c_1\\c_2\end{matrix}\right)=\bm{0}
            \end{gather}
            となる。

            $c_1\bm{a_1}+c_2\bm{a_2}=\bm{0}$を満たす時、$c_1=0, c_2=0$となるので
            $\bm{a_1}, \bm{a_2}$は線形独立である。

      \end{enumerate}

 \item
      $\bm{a}=\left(\begin{matrix}1\\0\\3\end{matrix}\right)$、
      $\bm{b}=\left(\begin{matrix}1\\-1\\4\end{matrix}\right)$の時、
      外積$\bm{a}\times \bm{b}$を求める。

      外積で求まるベクトルの第$i$成分は、
      元の2つのベクトルの$i$以外の成分より求まる。
      具体的には次のように行列式の形になる。
      \begin{equation}
       \bm{a}\times \bm{b} =
              \left(
               \begin{matrix}
                \left| \begin{matrix} 0 & -1 \\ 3 & 4 \end{matrix} \right| \\
                \left| \begin{matrix} 3 & 4 \\ 1 & 1 \end{matrix} \right|\\
                \left| \begin{matrix} 1 & 1 \\ 0 & -1 \end{matrix} \right|
               \end{matrix}
              \right)
              =
              \left(
               \begin{matrix}
                3\\-1\\-1
               \end{matrix}
              \right)
      \end{equation}

 \item
      線形変換$f:\mathbb{R}^2\rightarrow \mathbb{R}^2$は
      $\left(\begin{matrix}1\\2\end{matrix}\right)\mapsto
      \left(\begin{matrix}1\\8\end{matrix}\right)$、
      $\left(\begin{matrix}3\\1\end{matrix}\right)\mapsto
      \left(\begin{matrix}8\\14\end{matrix}\right)$
      とする。

      この時、$f\left(\begin{matrix}-1\\3\end{matrix}\right)$を求めよ。


      この線形写像を表す行列を
               $A=\left(\begin{matrix}a_{11}&a_{12}\\a_{21}&a_{22}\end{matrix}\right)$
      とおく。
      これにより
      $A\left(\begin{matrix}1\\2\end{matrix}\right)=
      \left(\begin{matrix}1\\8\end{matrix}\right)$、
      $A\left(\begin{matrix}3\\1\end{matrix}\right)=
      \left(\begin{matrix}8\\14\end{matrix}\right)$
      である。

      \underline{(補足)
      2つの式を一つにまとめ、左から逆行列をかけて求めてもよい。}

      成分ごとに計算をすると次のようになる。
      \begin{gather}
             \left(\begin{matrix}a_{11}&a_{12}\\a_{21}&a_{22}\end{matrix}\right)
             \left(\begin{matrix}1\\2\end{matrix}\right)=
             \left(\begin{matrix}1\\8\end{matrix}\right)
       \ \Rightarrow 
       \begin{cases}
        a_{11}+2a_{12}=1\\
        a_{21}+2a_{22}=8
       \end{cases}\\
             \left(\begin{matrix}a_{11}&a_{12}\\a_{21}&a_{22}\end{matrix}\right)
             \left(\begin{matrix}3\\1\end{matrix}\right)=
             \left(\begin{matrix}8\\14\end{matrix}\right)
       \ \Rightarrow 
       \begin{cases}
        3a_{11}+a_{12}=8\\
        3a_{21}+a_{22}=14
       \end{cases}
      \end{gather}

      4つの式の連立方程式を解くことで行列$A$の成分が求まる。
      \begin{equation}
       A =
              \left(\begin{matrix}3&-1\\4&2\end{matrix}\right)
      \end{equation}
      これを利用し計算を行う。
      \begin{equation}
       f\left(\begin{matrix}-1\\3\end{matrix}\right)
       =\left(\begin{matrix}3&-1\\4&2\end{matrix}\right)
       \left(\begin{matrix}-1\\3\end{matrix}\right)
       =\left(\begin{matrix}-6\\2\end{matrix}\right)
      \end{equation}

 \item
      ベクトル$\bm{a}=\left(\begin{matrix}1\\2\end{matrix}\right)$とする。

      直線$l:y=ax$に関して対称に移動する線形変換を表す行列が
      \begin{equation}
        \frac{1}{1+a^2}\left(\begin{matrix}1-a^2 & 2a\\2a&a^2-1\end{matrix}\right)
      \end{equation}
      であることを用いても良いものとする。

      \begin{enumerate}
       \item ベクトル$\bm{a}$を原点を中心に$\frac{\pi}{4}$回転させたベクトルを求めよ。

             原点を中心として$\theta$だけ回転させる線形変換を表す行列は
             \begin{equation}
               \left(\begin{matrix}\cos\theta & -\sin\theta\\\sin\theta&\cos\theta\end{matrix}\right)
             \end{equation}
             である。
             これを用いて$\frac{\pi}{4}$回転させたベクトルは
             \begin{equation}
               \left(\begin{matrix}
                     \cos\frac{\pi}{4} & -\sin\frac{\pi}{4}\\
                     \sin\frac{\pi}{4} & \cos\frac{\pi}{4}
                     \end{matrix}\right)
               \left(\begin{matrix}1\\2\end{matrix}\right)
               =
                \left(\begin{matrix}-\frac{1}{\sqrt{2}}\\\frac{3}{\sqrt{2}}\end{matrix}\right)
             \end{equation}
       \item ベクトル$\bm{a}$を$x$軸に関して対称に折り返したベクトルを求めよ。
             また、その移動を表す行列を求めよ。

             $x$軸について対称であるので$y$成分の符号を変えれば良い。
             \begin{equation}
              \bm{a}=\left(\begin{matrix}1\\2\end{matrix}\right) \mapsto
              \left(\begin{matrix}1\\-2\end{matrix}\right)
             \end{equation}
             この変換に対応する行列$A$は
             \begin{equation}
              A\left(\begin{matrix}\alpha\\\beta\end{matrix}\right) =
              \left(\begin{matrix}\alpha\\-\beta\end{matrix}\right)
             \end{equation}
             を満たす為、
             $A=\left(\begin{matrix}1&0\\0&-1\end{matrix}\right)$

       \item ベクトル$\bm{a}$を直線$y=~\frac{1}{2}x$に関して対称に移動したベクトルを求めよ。

             \begin{align}
              \frac{1}{1+\left(\frac{1}{2}\right)^2}
              \left(
                \begin{matrix}
                1-\left(\frac{1}{2}\right)^2 & 2\cdot\left(\frac{1}{2}\right)\\
                2\cdot\left(\frac{1}{2}\right)&\left(\frac{1}{2}\right)^2-1
                \end{matrix}
              \right)
              \left(\begin{matrix}1\\2\end{matrix}\right)
              = \frac{1}{5}\left(\begin{matrix}11\\-2\end{matrix}\right)
             \end{align}
      \end{enumerate}
\end{enumerate}

\end{document}
