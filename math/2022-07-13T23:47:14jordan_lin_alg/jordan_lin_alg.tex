\documentclass[12pt,b5paper]{ltjsarticle}

%\usepackage[margin=15truemm, top=5truemm, bottom=5truemm]{geometry}
\usepackage[margin=15truemm]{geometry}

\usepackage{amsmath,amssymb}
%\pagestyle{headings}
\pagestyle{empty}

%\usepackage{listings,url}
%\renewcommand{\theenumi}{(\arabic{enumi})}

\usepackage{graphicx}

\usepackage{tikz}
\usetikzlibrary {arrows.meta}
\usepackage{wrapfig}	% required for `\wrapfigure' (yatex added)
\usepackage{bm}	% required for `\bm' (yatex added)

% ルビを振る
\usepackage{luatexja-ruby}	% required for `\ruby'

%% 核Ker 像Im Hom を定義
%\newcommand{\Img}{\mathop{\mathrm{Im}}\nolimits}
\newcommand{\Ker}{\mathop{\mathrm{Ker}}\nolimits}
%\newcommand{\Hom}{\mathop{\mathrm{Hom}}\nolimits}
%\newcommand{\Rot}{\mathop{\mathrm{rot}}\nolimits}
%\newcommand{\Div}{\mathop{\mathrm{div}}\nolimits}

\begin{document}

\hrulefill

\textbf{次元に関する定義}

$f$を次のような線形写像(線形変換)とする。
\begin{equation}
 f: V \rightarrow V \quad (V:\text{vec sp})
\end{equation}

この時、$s_d(f)$を次のように定義する。
\begin{equation}
 s_d(f) = \dim \Ker f^d / \Ker f^{d-1}
\end{equation}

つまり、$s_d(f)$は整数値を取り、次の式により求まる。
\begin{equation}
 s_d(f) = \dim \Ker f^d - \dim \Ker f^{d-1}
\end{equation}

\hrulefill


\begin{enumerate}
 \item
      4次冪零行列$A$に対し、
      $(s_1(A),s_2(A),s_3(A),s_4(A))$
      の候補をすべて求め、
      各々に対応する\ruby{Jordan}{ジョルダン}標準形を
      求めよ。

\dotfill

      $(s_1(A),s_2(A),s_3(A),s_4(A))$の候補は
      次の5通り。
      \begin{equation}
%       (s_1(A),s_2(A),s_3(A),s_4(A))=
        (4,0,0,0),\ (3,1,0,0),\ (2,2,0,0),\ (2,1,1,0),\ (1,1,1,1)
      \end{equation}

      $(s_i(A))=(4,0,0,0)$の場合、
      $(m_1,m_2,m_3,m_4)=(1,1,1,1)$
      となるので、
      ジョルダン標準形は次のようになる。
      \begin{equation}
       J_1(0)\oplus J_1(0)\oplus J_1(0)\oplus J_1(0) = 0
      \end{equation}

      $(s_i(A))=(3,1,0,0)$の場合、
      $(m_1,m_2,m_3,m_4)=(2,1,1,0)$
      となるので、
      ジョルダン標準形は次のようになる。
      \begin{equation}
       J_1(0)\oplus J_1(0)\oplus J_2(0)
      \end{equation}

      $(s_i(A))=(2,2,0,0)$の場合、
      $(m_1,m_2,m_3,m_4)=(2,2,0,0)$
      となるので、
      ジョルダン標準形は次のようになる。
      \begin{equation}
       J_2(0)\oplus J_2(0)
      \end{equation}

      $(s_i(A))=(2,1,1,0)$の場合、
      $(m_1,m_2,m_3,m_4)=(3,1,0,0)$
      となるので、
      ジョルダン標準形は次のようになる。
      \begin{equation}
       J_1(0)\oplus J_3(0)
      \end{equation}

      $(s_i(A))=(1,1,1,1)$の場合、
      $(m_1,m_2,m_3,m_4)=(4,0,0,0)$
      となるので、
      ジョルダン標準形は次のようになる。
      \begin{equation}
       J_4(0)
      \end{equation}

\hrulefill

 \item
      次の行列の\ruby{Jordan}{ジョルダン}標準形を求めよ。
%      \begin{enumerate}
%       \item
            \begin{equation}
             A=
              \begin{pmatrix}
               5 & -3 & 2 \\ 10 & -9 & 6 \\ 15 & -6 & 4
              \end{pmatrix}
              , \quad
              B=
              \begin{pmatrix}
               0&0&0&0\\0&0&1&0\\0&0&0&0\\1&0&0&0
              \end{pmatrix}
              , \quad
              C=
              \begin{pmatrix}
              2 & 2 & 2 & 2 & -4\\
              7 & 1 & 1 & 1 & -5\\
              1 & 7 & 1 & 1 & -5\\
              1 & 1 & 7 & 1 & -5\\
              1 & 1 & 1 & 7 & -5
              \end{pmatrix}
            \end{equation}
%      \end{enumerate}

\dotfill

      \textbf{$A$のジョルダン標準形}
      
      $A$の固有値を求める。

      固有方程式$\det(A-\lambda E) =0$を解くと、
      $\det(A-\lambda E) = -\lambda(\lambda-5)(\lambda+5)=0$であるので、
      $\lambda = 0,5,-5$が得られる。
      それぞれの固有ベクトル$\bm{x}$は次のようになる。
      \begin{align}
       [\lambda=0]\ &
       \bm{x}=k
       \begin{pmatrix}0\\2\\3\end{pmatrix} &
       [\lambda=5]\ &
       \bm{x}=k
       \begin{pmatrix}1\\2\\3\end{pmatrix} &
       [\lambda=-5]\ &
       \bm{x}=k
       \begin{pmatrix}1\\4\\1\end{pmatrix}
      \end{align}

      3次正方行列で異なる固有値が3つあるので、
      ジョルダン標準形はこれを並べたものになる。
      \begin{equation}
       P=
        \begin{pmatrix}0&1&1\\2&2&4\\3&3&1\end{pmatrix}
        \qquad
         P^{-1}AP=
         \begin{pmatrix}0&0&0\\0&5&0\\0&0&-5\end{pmatrix}
      \end{equation}


      \textbf{$B$のジョルダン標準形}

      $B$の固有値を求める。

      固有方程式$\det(B-\lambda E) =0$を解くと、
      $\det(B-\lambda E) = \lambda^4=0$であるので、
      $\lambda=0$(4重解)となる。

      この時の固有空間は
      \begin{gather}
       (B-0 E)\bm{x}=0\\
       \begin{pmatrix}
        0&0&0&0\\0&0&1&0\\0&0&0&0\\1&0&0&0
       \end{pmatrix}
       \bm{x}=0\\
       \left\{
       k_1 \begin{pmatrix}0\\1\\0\\0\end{pmatrix}
       + k_2 \begin{pmatrix}0\\0\\0\\1\end{pmatrix}
       \mid k_1,k_2 \in K
       \right\}
      \end{gather}
      であり、固有ベクトルは
      $\begin{pmatrix}0\\1\\0\\0\end{pmatrix},\begin{pmatrix}0\\0\\0\\1\end{pmatrix}$
      である。

      この固有ベクトルに対し、次を満たすベクトル$\bm{\alpha},\ \bm{\beta}$を求める。
      \begin{align}
       \begin{pmatrix}
        0&0&0&0\\0&0&1&0\\0&0&0&0\\1&0&0&0
       \end{pmatrix}
       \bm{\alpha}&=\begin{pmatrix}0\\1\\0\\0\end{pmatrix}
       &
       \begin{pmatrix}
        0&0&0&0\\0&0&1&0\\0&0&0&0\\1&0&0&0
       \end{pmatrix}
       \bm{\beta}&=\begin{pmatrix}0\\0\\0\\1\end{pmatrix}\\
       \bm{\alpha}&=\begin{pmatrix}0\\0\\1\\0\end{pmatrix}
       &
       \bm{\beta}&=\begin{pmatrix}1\\0\\0\\0\end{pmatrix}
      \end{align}


      固有値とそれに対応するベクトル$\bm{\alpha},\bm{\beta}$を並べた行列$P$
      を用いて次のようにジョルダン標準形が求まる。
      \begin{equation}
       P=
        \begin{pmatrix}0&0&0&1\\1&0&0&0\\0&1&0&0\\0&0&1&0\end{pmatrix}
        \qquad
         P^{-1}BP=
         \begin{pmatrix}0&1&0&0\\0&0&0&0\\0&0&0&1\\0&0&0&0\end{pmatrix}
      \end{equation}


      \textbf{$C$のジョルダン標準形}

      $C^4\ne 0,\ C^5=0$である。
      $C^4\bm{p}\ne 0$となるベクトル$\bm{p}\ne0$に対し、
      $\bm{p},C\bm{p},C^2\bm{p},C^3\bm{p},C^4\bm{p}$は1次独立である。

      $C$の固有値を求める。

      固有方程式$\det(C-\lambda E) =0$を解くと、
      $\det(C-\lambda E) = -\lambda^5=0$であるので、
      $\lambda=0$(5重解)となる。

      この固有値に対する固有ベクトル$\bm{x}_\lambda$は
      $\bm{x}_\lambda = \begin{pmatrix}1\\1\\1\\1\\2\end{pmatrix}$
      である。

      $C\bm{x}_\lambda =0$より$\bm{x}_\lambda = C^4\bm{p}$と考え、
      $C^3\bm{p},C^2\bm{p},C\bm{p},\bm{p}$を求める。
      そこで、次を満たすベクトル
      $\bm{\alpha},\bm{\beta},\bm{\gamma},\bm{\delta}$
      を求める。
      \begin{align}
       \begin{pmatrix}
        2 & 2 & 2 & 2 & -4\\
        7 & 1 & 1 & 1 & -5\\
        1 & 7 & 1 & 1 & -5\\
        1 & 1 & 7 & 1 & -5\\
        1 & 1 & 1 & 7 & -5
       \end{pmatrix}
       \bm{\alpha}=&\begin{pmatrix}1\\1\\1\\1\\2\end{pmatrix}
       &
       \bm{\alpha}=&k_1\begin{pmatrix}1\\1\\1\\1\\2\end{pmatrix}
       +\frac{1}{12}\begin{pmatrix}1\\1\\1\\3\\0\end{pmatrix}\\
       %
       \begin{pmatrix}
        2 & 2 & 2 & 2 & -4\\
        7 & 1 & 1 & 1 & -5\\
        1 & 7 & 1 & 1 & -5\\
        1 & 1 & 7 & 1 & -5\\
        1 & 1 & 1 & 7 & -5
       \end{pmatrix}
       \bm{\beta}=&\frac{1}{12}\begin{pmatrix}1\\1\\1\\3\\0\end{pmatrix}
       &
       \bm{\beta}=&k_2\begin{pmatrix}1\\1\\1\\1\\2\end{pmatrix}
       +\frac{1}{144}\begin{pmatrix}1\\1\\5\\-1\\0\end{pmatrix}\\
       %
       \begin{pmatrix}
        2 & 2 & 2 & 2 & -4\\
        7 & 1 & 1 & 1 & -5\\
        1 & 7 & 1 & 1 & -5\\
        1 & 1 & 7 & 1 & -5\\
        1 & 1 & 1 & 7 & -5
       \end{pmatrix}
       \bm{\gamma}=&\frac{1}{144}\begin{pmatrix}1\\1\\5\\-1\\0\end{pmatrix}
       &
       \bm{\gamma}=&k_3\begin{pmatrix}1\\1\\1\\1\\2\end{pmatrix}
       +\frac{1}{1728}\begin{pmatrix}1\\9\\-3\\-1\\0\end{pmatrix}\\
       %
       \begin{pmatrix}
        2 & 2 & 2 & 2 & -4\\
        7 & 1 & 1 & 1 & -5\\
        1 & 7 & 1 & 1 & -5\\
        1 & 1 & 7 & 1 & -5\\
        1 & 1 & 1 & 7 & -5
       \end{pmatrix}
       \bm{\delta}=&\frac{1}{1728}\begin{pmatrix}1\\9\\-3\\-1\\0\end{pmatrix}
       &
       \bm{\delta}=&k_4\begin{pmatrix}1\\1\\1\\1\\2\end{pmatrix}
       +\frac{1}{20736}\begin{pmatrix}17\\-7\\-3\\-1\\0\end{pmatrix}
      \end{align}



      この為、$\bm{p}=\frac{1}{20736}\begin{pmatrix}17\\-7\\-3\\-1\\0\end{pmatrix}$
      と置くと$C^3\bm{p},C^2\bm{p},C\bm{p}$は次のようになる。
      \begin{equation}
       C^3\bm{p}=\frac{1}{12}\begin{pmatrix}1\\1\\1\\3\\0\end{pmatrix}
       ,\quad
       C^2\bm{p}=\frac{1}{144}\begin{pmatrix}1\\1\\5\\-1\\0\end{pmatrix}
       ,\quad
       C\bm{p}=\frac{1}{1728}\begin{pmatrix}1\\9\\-3\\-1\\0\end{pmatrix}
      \end{equation}


      これに$C^4\bm{p}=\bm{x}_\lambda$を加えた
      正則行列$P=(C^4\bm{p} \ C^3\bm{p} \ C^2\bm{p} \ C\bm{p} \ \bm{p})$
      により$C$ジョルダン標準形は次のようになる。
      \begin{equation}
       P^{-1}CP=
        \begin{pmatrix}0&1&0&0&0\\0&0&1&0&0\\0&0&0&1&0\\0&0&0&0&1\end{pmatrix}
      \end{equation}




\hrulefill

\end{enumerate}






\hrulefill


\end{document}
