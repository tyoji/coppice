\documentclass[12pt,b5paper]{ltjsarticle}

%\usepackage[margin=15truemm, top=5truemm, bottom=5truemm]{geometry}
\usepackage[margin=15truemm]{geometry}

\usepackage{amsmath,amssymb}
%\pagestyle{headings}
\pagestyle{empty}

%\usepackage{listings,url}
\renewcommand{\theenumi}{(\arabic{enumi})}

\usepackage{graphicx}

\usepackage{tikz}
\usetikzlibrary {arrows.meta}
\usepackage{wrapfig}	% required for `\wrapfigure' (yatex added)
\begin{document}



\textbf{上界 upper bound}

実数の部分集合$M$が空でないとする。
\begin{equation}
 M \subset \mathbb{R}, \ M \ne \emptyset
\end{equation}

ある実数$K$が存在し、
任意の要素$m \in M$に対し
$m \leq K$である時、
$K$を上界という。


\dotfill

\textbf{例}

閉集合$[1,5]$、開集合$(1,5)$
\begin{gather}
 [1,5] = \{ x \in \mathbb{R} \mid 1\leq x \leq 5\}\\
 (1,5) = \{ x \in \mathbb{R} \mid 1 < x < 5\}
\end{gather}
において、どちらの集合に対しても$5\in\mathbb{R}$は上界である。

$[1,5]$の全ての要素$x$に対し、$x\leq 5$であり、
同様に$(1,5)$の全ての要素$x$に対しても、$x\leq 5$である。

\dotfill

$[1,5]$の上界のうち、$5\in\mathbb{R}$は最も小さい数である。

これは$[1,5]$には$5$が含まれる為$5$より小さい数は上界にならない。

\dotfill

$(1,5)$の上界のうち、$5\in\mathbb{R}$は最も小さい数である。

もしも$5\in\mathbb{R}$より小さい上界があるとする。
この上界を$r\in\mathbb{R}$とすると、$r<5$である。
この時次のような式が成り立つ。
\begin{equation}
 r< \frac{r+5}{2} <5
\end{equation}
$\frac{r+5}{2} \in (1,5)$であるので $r$は上界ではないことがわかる。
この為、$(1,5)$の上界のうち、$5\in\mathbb{R}$は最も小さい数であるといえる。

\end{document}
