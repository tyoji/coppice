\documentclass[12pt,b5paper]{ltjsarticle}

%\usepackage[margin=15truemm, top=5truemm, bottom=5truemm]{geometry}
%\usepackage[margin=10truemm,left=15truemm]{geometry}
\usepackage[margin=10truemm]{geometry}

\usepackage{amsmath,amssymb}
%\pagestyle{headings}
\pagestyle{empty}

%\usepackage{listings,url}
%\renewcommand{\theenumi}{(\arabic{enumi})}

%\usepackage{graphicx}

%\usepackage{tikz}
%\usetikzlibrary {arrows.meta}
%\usepackage{wrapfig}
%\usepackage{bm}

% ルビを振る
%\usepackage{luatexja-ruby}	% required for `\ruby'

%% 核Ker 像Im Hom を定義
%\newcommand{\Img}{\mathop{\mathrm{Im}}\nolimits}
%\newcommand{\Ker}{\mathop{\mathrm{Ker}}\nolimits}
%\newcommand{\Hom}{\mathop{\mathrm{Hom}}\nolimits}

%\DeclareMathOperator{\Rot}{rot}
%\DeclareMathOperator{\Div}{div}
%\DeclareMathOperator{\Grad}{grad}
%\DeclareMathOperator{\arcsinh}{arcsinh}
%\DeclareMathOperator{\arccosh}{arccosh}
%\DeclareMathOperator{\arctanh}{arctanh}



%\usepackage{listings,url}
%
%\lstset{
%%プログラム言語(複数の言語に対応,C,C++も可)
%  language = Python,
%%  language = Lisp,
%%  language = C,
%  %背景色と透過度
%  %backgroundcolor={\color[gray]{.90}},
%  %枠外に行った時の自動改行
%  breaklines = true,
%  %自動改行後のインデント量(デフォルトでは20[pt])
%  breakindent = 10pt,
%  %標準の書体
%%  basicstyle = \ttfamily\scriptsize,
%  basicstyle = \ttfamily,
%  %コメントの書体
%%  commentstyle = {\itshape \color[cmyk]{1,0.4,1,0}},
%  %関数名等の色の設定
%  classoffset = 0,
%  %キーワード(int, ifなど)の書体
%%  keywordstyle = {\bfseries \color[cmyk]{0,1,0,0}},
%  %表示する文字の書体
%  %stringstyle = {\ttfamily \color[rgb]{0,0,1}},
%  %枠 "t"は上に線を記載, "T"は上に二重線を記載
%  %他オプション:leftline,topline,bottomline,lines,single,shadowbox
%  frame = TBrl,
%  %frameまでの間隔(行番号とプログラムの間)
%  framesep = 5pt,
%  %行番号の位置
%  numbers = left,
%  %行番号の間隔
%  stepnumber = 1,
%  %行番号の書体
%%  numberstyle = \tiny,
%  %タブの大きさ
%  tabsize = 4,
%  %キャプションの場所("tb"ならば上下両方に記載)
%  captionpos = t
%}


%% 表
\usepackage{diagbox}


\begin{document}

\begin{table}
 \caption{問の表}
 \label{table:prob}
 \centering
 \begin{tabular}{|c|c|c|c|c|c|c|}
  \hline
  \diagbox{$a$}{$b$} & 1 & 2 & 3 & 4 & 5 & 6 \\
  \hline
  1 & 1+1 & 1+2 & 1+3 & 1+4 & 1+5 & 1+6 \\
  \hline
  2 & 2+1 & 2+2 & 2+3 & 2+4 & 2+5 & 2+6 \\
  \hline
  3 & 3+1 & 3+2 & 3+3 & 3+4 & 3+5 & 3+6 \\
  \hline
  4 & 4+1 & 4+2 & 4+3 & 4+4 & 4+5 & 4+6 \\
  \hline
  5 & 5+1 & 5+2 & 5+3 & 5+4 & 5+5 & 5+6 \\
  \hline
  6 & 6+1 & 6+2 & 6+3 & 6+4 & 6+5 & 6+6 \\
  \hline
 \end{tabular}
\end{table}



上記の表\ref{table:prob}
の中にある加法の左側だけを抜き出すと表\ref{table:prob2}になる。
\begin{table}
 \caption{問の表の一部}
 \label{table:prob2}
 \centering
 \begin{tabular}{|c|c|c|c|c|c|c|}
  \hline
  \diagbox{$a$}{$b$} & 1 & 2 & 3 & 4 & 5 & 6 \\
  \hline
  1 & 1\phantom{+1} & 1\phantom{+2} & 1\phantom{+3} & 1\phantom{+4} & 1\phantom{+5} & 1\phantom{+6} \\
  \hline
  2 & 2\phantom{+1} & 2\phantom{+2} & 2\phantom{+3} & 2\phantom{+4} & 2\phantom{+5} & 2\phantom{+6} \\
  \hline
  3 & 3\phantom{+1} & 3\phantom{+2} & 3\phantom{+3} & 3\phantom{+4} & 3\phantom{+5} & 3\phantom{+6} \\
  \hline
  4 & 4\phantom{+1} & 4\phantom{+2} & 4\phantom{+3} & 4\phantom{+4} & 4\phantom{+5} & 4\phantom{+6} \\
  \hline
  5 & 5\phantom{+1} & 5\phantom{+2} & 5\phantom{+3} & 5\phantom{+4} & 5\phantom{+5} & 5\phantom{+6} \\
  \hline
  6 & 6\phantom{+1} & 6\phantom{+2} & 6\phantom{+3} & 6\phantom{+4} & 6\phantom{+5} & 6\phantom{+6} \\
  \hline
 \end{tabular}
\end{table}

この表\ref{table:prob2}には縦に1から6が並んでいるので、
表の中の数の和は
$(1+2+3+4+5+6)\times 6$
である。

同様に加法の右側だけを足し合わせると
$(1+2+3+4+5+6)\times 6$
である。

よって、
表\ref{table:prob}
の総和は
\[
 (1+2+3+4+5+6)\times 6 +
(1+2+3+4+5+6)\times 6
= (1+2+3+4+5+6)\times 6\times 2
\]
である。



\end{document}
