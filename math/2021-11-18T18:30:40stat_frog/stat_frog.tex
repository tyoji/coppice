\documentclass[12pt,b5paper]{ltjsarticle}

\usepackage{amssymb}
\pagestyle{empty}

\usepackage{amsmath}	% required for `\align' (yatex added)
\begin{document}

正規分布$N(m,\sigma)$において
\begin{align}
 平均 &\ m\\
 分散 &\ \sigma\\
 確率密度関数 &\
 \frac{1}{\sqrt{2\pi\sigma^2}}\exp\left(-\frac{(x-m)^2}{2\sigma^2}\right)
\end{align}

\hrulefill


母集団は正規分布$N(\mu,2)$である。
母平均$\mu$、母分散$2$

標本として50個取り出す。
取り出した個体の平均は$10$である。

信頼係数$0.95$における$\mu$の信頼区間を算出



\begin{enumerate}\renewcommand{\theenumi}{(\arabic{enumi})}
 \item
      取り出した50個の標本$X_i (i=1,\dots,50)$に対し、
      次のように
      標本平均$\overline{X_{50}}$を定め、
      分散$V(\overline{X_{50}})$を求める。
      \[
       \overline{X_{50}}=\frac{1}{50}\sum_{i=1}^{50}X_i
      \]

      $X_i (i=1,\dots,50)$が独立同分布(平均$\mu$、分散$2$)であれば
      正規分布でなくても$V(\overline{X_{50}}) = \frac{2}{50}$である。
      \begin{align}
       V(\overline{X_{50}}) & = V\left(\frac{1}{50}\sum_{i=1}^{50}X_i\right)\\
       & = \frac{1}{50^2}\sum_{i=1}^{50}V(X_i)\\
       & = \frac{1}{50^2}\sum_{i=1}^{50}2 & V(X_i)=2 \ (i=1,\dots,50)\\
       & = \frac{2}{50}
      \end{align}

      $\overline{X_{50}}$の平均$E(\overline{X_{50}})$は、
      $E(\overline{X_{50}}) = \mu$である。

      平均と分散を利用し確率変数$\overline{X_{50}}$を正規化すると
      $\frac{\overline{X_{50}}-\mu}{\sqrt{V(\overline{X_{50}})}}$であり、
      この正規化した確率変数の平均、分散、密度関数は
      \begin{align}
       平均 &\ 0\\
       分散 &\ 1\\
       確率密度関数 &\
%       \frac{1}{\sqrt{2\pi}}\exp\left(-\frac{x^2}{2}\right)
       \frac{1}{\sqrt{2\pi}}e^{-\frac{x^2}{2}}
      \end{align}

 \item
      \[
       \int_{-1.96}^{1.96}\frac{1}{\sqrt{2\pi}}e^{-\frac{x^2}{2}}dx = 0.95
      \]
      この式は正規分布(平均 $0$、分散 $1$)の空間で
      $-1.96$から$1.96$に含まれる部分が全体の$95\%$であることを意味している。

      標準正規分布$N(0,1)$において確率変数$S$についての確率は
      \[
       P(-1.96\leq S \leq 1.96)=0.95
      \]
      となる。
      これは$95\%$の確率で確率変数$S$が区間$[-1.96, 1.96]$に含まれることを意味している。

      $\frac{\overline{X_{50}}-\mu}{\sqrt{V(\overline{X_{50}})}}$は
      $N(0,1)$に従うので、
      これが区間$[-1.96, 1.96]$に含まれる確率が$95\%$である。

      そこで、式の変形を行う。
      \begin{gather}
       -1.96 \leq \frac{\overline{X_{50}}-\mu}{\sqrt{V(\overline{X_{50}})}} \leq 1.96\\
       \overline{X_{50}}-1.96\sqrt{V(\overline{X_{50}})}
           \leq \mu
           \leq \overline{X_{50}}+1.96\sqrt{V(\overline{X_{50}})}
      \end{gather}

      $V(\overline{X_{50}})=\frac{2}{50}$は先ほど求めたとおり$\frac{2}{50}$であり、
      標本平均$\overline{X_{50}}$の代わりにサンプルとして得た平均値$10$を当てはめる。

      \begin{gather}
       10-1.96\sqrt{\frac{2}{50}}
           \leq \mu
           \leq 10+1.96\sqrt{\frac{2}{50}}\\
       9.608 \leq \mu \leq 10.392
      \end{gather}

      求める区間は$9.608 \leq \mu \leq 10.392$。

\end{enumerate}


\end{document}
