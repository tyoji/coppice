\documentclass[12pt,b5paper]{ltjsarticle}

\usepackage[margin=15truemm]{geometry}
\pagestyle{empty}

\usepackage{amssymb}
\usepackage{amsmath}	% required for `\align' (yatex added)

\begin{document}

\begin{gather}
 xy+x-4y=7\label{200422_22Nov21}\\
 x(y+1)-4y=7\label{200553_22Nov21}\\
 x(y+1)-4y-4+4=7\label{200646_22Nov21}\\
 x(y+1)-4(y+1)+4=7\label{200836_22Nov21}\\
 (y+1)(x-4)+4=7\label{200916_22Nov21}
\end{gather}

(\ref{200422_22Nov21})
 取り敢えず、何かの変数でまとめます。

(\ref{200553_22Nov21})
 今は$x$の次数でまとめ、$x$の1次の項と
 $x$の無い項に分けました。

(\ref{200646_22Nov21})
 $x$はまとまりましたが、$y$が複数あります。
 $x(y+1)$があるので、何とか$y+1$を作るために
 $-4+4$を式に加えます。

 (\ref{200836_22Nov21})
 $-4y-4$が出来たので$-4$でまとめます。

 (\ref{200916_22Nov21})
 $x(y+1)-4(y+1)$が出来ました。
 この式は$y+1$を共通因数として持つので
 $(y+1)(x-4)$に変形します。

 最終的に次の式が変形できます。

 \begin{gather}
  xy+x-4y=7\\
  (y+1)(x-4)+4=7\\
  (x-4)(y+1)=3
 \end{gather}
\end{document}
