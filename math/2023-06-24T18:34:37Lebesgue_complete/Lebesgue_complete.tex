\documentclass[12pt,b5paper]{ltjsarticle}

%\usepackage[margin=15truemm, top=5truemm, bottom=5truemm]{geometry}
%\usepackage[margin=10truemm,left=15truemm]{geometry}
\usepackage[margin=10truemm]{geometry}

\usepackage{amsmath,amssymb}
%\pagestyle{headings}
\pagestyle{empty}

%\usepackage{listings,url}
%\renewcommand{\theenumi}{(\arabic{enumi})}

%\usepackage{graphicx}

%\usepackage{tikz}
%\usetikzlibrary {arrows.meta}
%\usepackage{wrapfig}
%\usepackage{bm}

% ルビを振る
\usepackage{luatexja-ruby}	% required for `\ruby'

%% 核Ker 像Im Hom を定義
%\newcommand{\Img}{\mathop{\mathrm{Im}}\nolimits}
%\newcommand{\Ker}{\mathop{\mathrm{Ker}}\nolimits}
%\newcommand{\Hom}{\mathop{\mathrm{Hom}}\nolimits}

%\DeclareMathOperator{\Rot}{rot}
%\DeclareMathOperator{\Div}{div}
%\DeclareMathOperator{\Grad}{grad}
%\DeclareMathOperator{\arcsinh}{arcsinh}
%\DeclareMathOperator{\arccosh}{arccosh}
%\DeclareMathOperator{\arctanh}{arctanh}



%\usepackage{listings,url}
%
%\lstset{
%%プログラム言語(複数の言語に対応,C,C++も可)
%  language = Python,
%%  language = Lisp,
%%  language = C,
%  %背景色と透過度
%  %backgroundcolor={\color[gray]{.90}},
%  %枠外に行った時の自動改行
%  breaklines = true,
%  %自動改行後のインデント量(デフォルトでは20[pt])
%  breakindent = 10pt,
%  %標準の書体
%%  basicstyle = \ttfamily\scriptsize,
%  basicstyle = \ttfamily,
%  %コメントの書体
%%  commentstyle = {\itshape \color[cmyk]{1,0.4,1,0}},
%  %関数名等の色の設定
%  classoffset = 0,
%  %キーワード(int, ifなど)の書体
%%  keywordstyle = {\bfseries \color[cmyk]{0,1,0,0}},
%  %表示する文字の書体
%  %stringstyle = {\ttfamily \color[rgb]{0,0,1}},
%  %枠 "t"は上に線を記載, "T"は上に二重線を記載
%  %他オプション:leftline,topline,bottomline,lines,single,shadowbox
%  frame = TBrl,
%  %frameまでの間隔(行番号とプログラムの間)
%  framesep = 5pt,
%  %行番号の位置
%  numbers = left,
%  %行番号の間隔
%  stepnumber = 1,
%  %行番号の書体
%%  numberstyle = \tiny,
%  %タブの大きさ
%  tabsize = 4,
%  %キャプションの場所("tb"ならば上下両方に記載)
%  captionpos = t
%}



\begin{document}

\hrulefill

%\textbf{有限加法族}
%
%集合$X$の部分集合族$\mathcal{F}$が
%\textbf{有限加法族}である
%とは次を満たすときをいう。
%\begin{enumerate}
% \item $\emptyset \in \mathcal{F}$
% \item $A \in \mathcal{F} \Rightarrow X\backslash A \in\mathcal{F}$
% \item $A,B\in\mathcal{F}
%       \Rightarrow A\cup B \in\mathcal{F}$
%\end{enumerate}
%
%
%\textbf{有限加法的測度}
%
%集合$X$上の有限加法族$\mathcal{F}$について、
%$m:\mathcal{F}\to [0,\infty]$が
%$(X,\mathcal{F})$上の
%\textbf{有限加法的測度}であるとは、
%次の2つの条件を満たすときをいう。
%\begin{enumerate}
% \item $m(\emptyset) =0$
% \item $A,B\in\mathcal{F}$が互いに素である時、
%       $m(A\cup B) = m(A) + m(B)$
%\end{enumerate}
%
%
%\textbf{外測度}
%
%$X$を集合とする。
%$\Gamma : 2^{X}\to [0,\infty]$が
%$X$上の\textbf{外測度}であるとは、
%次の3つの条件を満たすときをいう。
%\begin{enumerate}
% \item
%      $\Gamma (\emptyset) = 0$
% \item
%      $A,B \subset X$が
%      $A\subset B$を満たす時、
%      $\Gamma(A)\leq \Gamma(B)$
% \item
%      $X$の任意の部分集合列$\{A_{n}\}_{n=1}^{\infty}$
%      に対し、
%      $\Gamma(\bigcup_{n=1}^{\infty}A_{n}) \leq \sum_{n=1}^{\infty}\Gamma(A_{n})$
%\end{enumerate}
%
%
%\textbf{$\Gamma$-可測}
%
%$X$を集合とする。
%$\Gamma : 2^{X}\to [0,\infty]$を
%$X$上の外測度とする。
%
%集合$E\subset X$が\textbf{$\Gamma$-可測}
%(または \ruby{Carath\'eodory}{カラテオドリ}の意味で可測)
%とは、
%任意の$A\subset X$に対し次を満たすときをいう。
%\begin{equation}
% \Gamma(A\cap E) + \Gamma(A\cap (X\backslash E)) = \Gamma(A)
%\end{equation}
%
%また、$\Gamma$-可測集合全体を$\mathcal{M}_{\Gamma}$と表す。
%
%
%\textbf{命題}($X$上の外測度)
%
%$X$を集合、
%$\mathcal{F}$を$X$上の有限加法族、
%$\mu$を$(X,\mathcal{F})$上の有限加法的測度
%とする。
%$\mu^{*}:2^{X}\to [0,\infty]$を次で定義する。
%\begin{equation}
% \mu^{*}(A)
%  = \inf\left\{
%         \sum_{j=1}^{\infty}\mu(E_{j})
%         \ \middle| \
%         A \subset \bigcup_{j=1}^{\infty}E_{j}
%         であり、
%         E_{j}\in\mathcal{F}
%         、j\in\mathbb{N}
%        \right\}
%\end{equation}
%
%このとき、
%$\mu^{*}$は$X$上の外側度である。


\textbf{可測関数}

$(X,\Sigma_{X}), \: (Y,\Sigma_{Y})$を可測空間、
つまり、
$X,Y$は集合で、
$\Sigma_{X},\Sigma_{Y}$は$\sigma$-加法族
とする。

関数$f: X\to Y$について
${}^{\forall}E\in \Sigma_{Y}$に対して
$f^{-1}(E)\in\Sigma_{X}$が成り立つとき、
関数$f: X\to Y$が可測であるという。
この集合$Y$が$\overline{\mathbb{R}}=[-\infty,\infty]$
の時、$\Sigma_{Y}$はボレル集合族として定義する。



\textbf{$\mu$-零集合}

$(X, \mathcal{M}, \mu)$を測度空間とする。
$A\subset X$が$\mu$-零集合であるとは,
$A\subset N$
かつ
$\mu(N) = 0$
を満たす
$ N\in \mathcal{M}$
が存在することをいう。


\textbf{完備}

$(X, \mathcal{M}, \mu)$を測度空間とする。
全ての$\mu$-零集合が
$\mathcal{M}$に属する時、
$(X, \mathcal{M}, \mu)$
あるいは$\mu$のことを
完備という。


\textbf{完備化}

$(X, \mathcal{M}, \mu)$を測度空間とする。
$X$の部分集合族$\overline{\mathcal{M}}$を
次のように定義する。
\begin{equation}
 \overline{\mathcal{M}}
  =\{ A \subset X \mid B_{1},B_{2}\in\mathcal{M}
   \: が存在して、\:
  B_{1}\subset A \subset B_{2} \:かつ\: \mu (B_{2}\backslash B_{1}) =0 \}
\end{equation}
また、$A\in\overline{\mathcal{M}}$に対し、
$\overline{\mathcal{M}}$の定義中の$B_{1}$をとり、
$\overline{\mu}(A) = \mu(B_{1})$と定める。
この時、
$(X, \overline{\mathcal{M}}, \overline{\mu})$は完備測度空間となる。

この測度空間$(X, \overline{\mathcal{M}}, \overline{\mu})$
を
$(X, \mathcal{M}, \mu)$の
完備化という。




\hrulefill

$(X,\mathcal{M},\mu)$を測度空間とし、
その完備化を
$(X,\overline{\mathcal{M}},\overline{\mu})$で表す。
また、$f:X\to \overline{\mathbb{R}}$とする。

\begin{enumerate}
 \item
      $g: X\to \overline{\mathbb{R}}$
      は
      $\mathcal{M}$-可測であるとする。
      $\{ x\in X \mid f(x)\ne g(x)\}$
      が$\mu$-零集合で
      あるならば、
      $f$は$\overline{\mathcal{M}}$-可測であることを示せ。

      \dotfill


      $B \subset \overline{\mathbb{R}}$をボレル集合とする。
      $g$は$\mathcal{M}$-可測であるので、
      $g^{-1}(B)\in\mathcal{M}$である。

      $S=\{ x\in X \mid f(x)\ne g(x) \}$
      が$\mu$-零集合であるので、
      $N\in\mathcal{M}$が存在し、
      $S \subset N$かつ
      $\mu(N)=0$である。

      任意のボレル集合$B \subset \overline{\mathbb{R}}$に対し、
      $f^{-1}(B)\in\overline{\mathcal{M}}$を示せればよい。
      
      集合$f^{-1}(B)$
      は次の2つの集合に分けられる。
      \begin{equation}
       f^{-1}(B)
        = \{ x\in f^{-1}(B) \mid f(x)\ne g(x) \}
        \cup \{ x\in f^{-1}(B) \mid f(x) = g(x) \}
      \end{equation}

      1つ目の集合は
      次のような包含関係がある。
      \begin{equation}
       \emptyset \subset \{ x\in f^{-1}(B) \mid f(x)\ne g(x) \} \subset S \subset N
      \end{equation}
      この時、$\mu(N\backslash \emptyset) = \mu(N) =0$であるので、
      $\{ x\in f^{-1}(B) \mid f(x)\ne g(x) \} \in \overline{\mathcal{M}}$
      である。
      同様に$g^{-1}(B)$についても考えられる。
      \begin{equation}
       \emptyset
        \subset \{ x\in g^{-1}(B) \mid f(x)\ne g(x) \}
        \subset S \subset N
      \end{equation}
      つまり、
      $\{ x\in g^{-1}(B) \mid f(x)\ne g(x) \} \in \overline{\mathcal{M}}$
      である。

      2つ目の集合$\{ x\in f^{-1}(B) \mid f(x) = g(x) \}$
      は$g^{-1}(B)\in\mathcal{M}$の部分集合である。
      \begin{equation}
       \{ x\in f^{-1}(B) \mid f(x) = g(x) \}
        = \{ x\in g^{-1}(B) \mid f(x) = g(x) \}
        \subset g^{-1}(B)
      \end{equation}

      つまり、次のような式が成り立つ。
      \begin{equation}
       \{ x\in g^{-1}(B) \mid f(x)= g(x) \}
        = g^{-1}(B) \backslash \{ x\in g^{-1}(B) \mid f(x)\ne g(x) \}
      \end{equation}

      $g^{-1}(B)\in\mathcal{M} \subset \overline{\mathcal{M}}$であり、
      $\{ x\in g^{-1}(B) \mid f(x)\ne g(x) \}\in\overline{\mathcal{M}}$
      であるので、
      $\{ x\in g^{-1}(B) \mid f(x)= g(x) \}\in\overline{\mathcal{M}}$
      である。

      $\{ x\in f^{-1}(B) \mid f(x)\ne g(x) \} \in \overline{\mathcal{M}}$
      であり、
      $\{ x\in f^{-1}(B) \mid f(x)= g(x) \}\in\overline{\mathcal{M}}$
      であるので、
      $f^{-1}(B)\in\overline{\mathcal{M}}$であることがわかる。

      これにより、
      $f$は$\overline{\mathcal{M}}$-可測である。


      \hrulefill

 \item
      $\{ f_{n} \}_{n=1}^{\infty}$は
      $X$上の$\overline{\mathbb{R}}$-値関数の列とし、
      任意の$n\in\mathbb{N}$に対し、
      $f_{n}$は$\mathcal{M}$-可則であるとする。
      $\displaystyle \left\{ x\in X \:\middle|\: \lim_{n\to\infty}f_{n}(x) \ne f(x) \right\}$
      が$\mu$-零集合であるならば、
      $f$は
      $\overline{\mathcal{M}}$-可測になることを示せ。

      \dotfill

      $B\subset \overline{\mathbb{R}}$
      をボレル集合とする。

      ${}^{\forall}n\in\mathbb{N}$に対して、
      $f_{n}^{-1}(B)\in\mathcal{M}$である。

      $\displaystyle
      S= \left\{ x\in X \:\middle|\:
      \lim_{n\to\infty}f_{n}(x) \ne f(x) \right\}$とすると、
      $S$は$\mu$-零集合であるので、
      $S\subset N$が存在し、$\mu(N)=0$である。


      $f^{-1}(B)\in\overline{\mathcal{M}}$となることを示す。

      $f^{-1}(B)$は$S$の内外に分けられる。
      \begin{equation}
       f^{-1}(B) = (f^{-1}(B) \cap S) \cup (f^{-1}(B)\cap S^{c})
      \end{equation}

      $f^{-1}(B) \cap S$
      は次の包含関係がある。
      \begin{equation}
       \emptyset \subset f^{-1}(B) \cap S \subset S \subset N
      \end{equation}

      $\mu(N\backslash \emptyset) = \mu(N) =0$であるので、
      $f^{-1}(B) \cap S \in \overline{\mathcal{M}}$である。


%\dotfill

      $f^{-1}(B)\cap S^{c}$
      について考える。


      任意の$x\in S$について
      関数列$\{f_{n}\}_{n=1}^{\infty}$の極限が
      あり、
      次のように定義される。
      \begin{equation}
       \lim_{n\to\infty}f_{n}(x) = f(x)
        \overset{\mathrm{def}}{\iff}
        {}^{\forall}\varepsilon >0,\: {}^{\exists}N_{0}\in\mathbb{N}
        \ \ \text{s.t.}\ \
        N \geq N_{0} \Rightarrow
        \lvert f_{N}(x) - f(x) \rvert <\varepsilon
      \end{equation}

      つまり、$\varepsilon$に対して
      十分に大きい$N\in\mathbb{N}$をとってくれば
      $\lvert f_{N}(x) - f(x) \rvert <\varepsilon$を満たす。
%      そこで、$\varepsilon$と$N$を使い、
%      $f^{-1}(B)\cap S^{c}$を変形する。
      \begin{align}
       f^{-1}(B)\cap S^{c}
        =& \left\{
           x\in f^{-1}(B) \:\middle|\: \lim_{n\to\infty}f_{n}(x) = f(x)
        \right\}
%        =& \left\{
%           x\in f^{-1}(B) \:\middle|\:
%%        {}^{\forall}\varepsilon >0,\: {}^{\exists}N_{0}\in\mathbb{N}
%%        \:\text{s.t.}\:
%%        N \geq N_{0} \Rightarrow
%        \lvert f_{N}(x) - f(x) \rvert <\varepsilon
%        \right\}
      \end{align}

      $x\in f^{-1}(B)$に対して、
      $\varepsilon_{x} >0$を任意に定めると
      十分に大きな$N_{x}\in\mathbb{N}$
      により
      $\lvert f_{N_{x}}(x) - f(x) \rvert <\varepsilon_{x}$
      となる。

      そこで、
      区間$I_{x}\subset \overline{\mathbb{R}}$
      を次のように定義する。
      \begin{equation}
       I_{x} =
        \begin{cases}
         [f(x),\ f_{N_{x}}(x)] ,& (f_{N_{x}}(x) \geq f(x))\\
         [f_{N_{x}}(x),\ f(x)] ,& ( \text{otherwise})
        \end{cases}
      \end{equation}

      $\lvert I_{x} \rvert < \varepsilon_{x}$
      であるが、
      十分小さな$\varepsilon_{x}$を取ってくることにより
      $I_{x} \subset B$とする。

      これにより
      $f^{-1}(B)\cap S^{c} = \bigcup_{x} f_{N_{x}}^{-1}(I_{x})$
      である。

      $f_{N_{x}}^{-1}(I_{x}) \in \mathcal{M}$であるので、
      $f^{-1}(B)\cap S^{c} \in \mathcal{M} \subset \overline{\mathcal{M}}$
      である。
      
      $f^{-1}(B) \cap S \in \overline{\mathcal{M}}$であることと合わせると
      $f^{-1}(B) \in \overline{\mathcal{M}}$である事がわかる。
      よって、
      $f$は$\overline{\mathcal{M}}$可測である。
      

%      ここで、
%      $x\in B$の$\varepsilon$近傍$U_{x,\varepsilon}$
%      を使い
%      $f_{N}^{-1}(B)$を近傍の逆像で表す。
%      \begin{align}
%       B = \bigcup_{x\in B}U_{x,\varepsilon}
%       & \Longrightarrow
%       f_{N}^{-1}(B)
%         = \bigcup_{x\in B} f_{N}^{-1}(U_{x,\varepsilon})\\
%       & \Longrightarrow
%       f_{N}^{-1}(B)
%         = \bigcup_{x\in f^{-1}(B)} f_{N}^{-1}(U_{f(x),\varepsilon})
%      \end{align}
%
%      
%
%
%      全ての$S$の要素に対し$N_{0}$が存在するので、
%      その中で最大の自然数を$N\in\mathbb{N}$とおく。
%      これにより
%      $f(x)\in\overline{\mathbb{R}}$の
%      近傍$U_{(f(x),\varepsilon)} \subset \overline{\mathbb{R}}$
%      が存在し、
%      $f_{N}(x)\in U_{(f(x),\varepsilon)}$である。
%      つまり、$x\in f_{N}f^{-1}(U_{(f(x),\varepsilon)})$となる。
%
%      \begin{equation}
%       S \subset \bigcup_{x\in S} f_{N}^{-1} (U_{(f(x),\varepsilon)})
%      \end{equation}
%
%      \begin{equation}
%       f^{-1}(B) \subset \bigcup_{x\in f^{-1}(B)} f_{N}^{-1} (U_{(f(x),\varepsilon)})
%      \end{equation}

      \hrulefill
\end{enumerate}
\hrulefill

\end{document}
