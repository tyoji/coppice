\documentclass[12pt,b5paper]{ltjsarticle}

%\usepackage[margin=15truemm, top=5truemm, bottom=5truemm]{geometry}
\usepackage[margin=10truemm]{geometry}

\usepackage{amsmath,amssymb}
%\pagestyle{headings}
\pagestyle{empty}

%\usepackage{listings,url}
%\renewcommand{\theenumi}{(\arabic{enumi})}

%\usepackage{graphicx}

%\usepackage{tikz}
%\usetikzlibrary {arrows.meta}
%\usepackage{wrapfig}	% required for `\wrapfigure' (yatex added)
%\usepackage{bm}	% required for `\bm' (yatex added)

% ルビを振る
%\usepackage{luatexja-ruby}	% required for `\ruby'

%% 核Ker 像Im Hom を定義
%\newcommand{\Img}{\mathop{\mathrm{Im}}\nolimits}
%\newcommand{\Ker}{\mathop{\mathrm{Ker}}\nolimits}
%\newcommand{\Hom}{\mathop{\mathrm{Hom}}\nolimits}

%\DeclareMathOperator{\Rot}{rot}
%\DeclareMathOperator{\Div}{div}
%\DeclareMathOperator{\Grad}{grad}
%\DeclareMathOperator{\arcsinh}{arcsinh}
%\DeclareMathOperator{\arccosh}{arccosh}
%\DeclareMathOperator{\arctanh}{arctanh}



\begin{document}

\hrulefill
\textbf{Laplace 変換}
\hrulefill

実関数$f(t)$から
複素関数$F(s)$を得る。
$F(s)=\mathcal{L}[f(t)]$
\begin{equation}
 F(s) = \int_{0}^{\infty} f(t)e^{-st} \mathrm{d}t
\end{equation}


\dotfill
\textbf{性質}
\dotfill


複素数$s$ が $\mathrm{Re}(s)>0$ を満たす時
\begin{align}
 \mathcal{L}[t^n] =& \frac{n!}{s^{n+1}} &
 \mathcal{L}[\sin \omega t] =& \frac{\omega}{s^{2}+\omega^2} &
 \mathcal{L}[\cos \omega t] =& \frac{s}{s^{2}+\omega^2}
\end{align}

複素数$s$ が $\mathrm{Re}(s)>a$ を満たす時
\begin{equation}
 \mathcal{L}[e^{at}] = \frac{1}{s-a}
\end{equation}

\textbf{線形性}
$a.b$が定数
\begin{equation}
 \mathcal{L}[af(t)+bg(t)] = a\mathcal{L}[f(t)] + b\mathcal{L}[g(t)]
\end{equation}


定数$a$が$a>0$、$F(s)=\mathcal{L}[f(t)]$とする。
\begin{equation}
 \mathcal{L}[f(at)] = \frac{1}{a}F\left(\frac{s}{a}\right)
  \qquad
 \mathcal{L}[e^{at}f(t)] = F(s-a)
\end{equation}


\dotfill

\textbf{微分}

$f^{(i)}(x)$ は連続 $(i=0,\dots, n-1)$
\begin{equation}
 \mathcal{L}[ f^{(n)}(t) ]
  = s^n F(s) - \sum_{k=1}^{n} s^{n-k}f^{(k-1)}(0)
\end{equation}


\hrulefill
逆ラプラス変換
\hrulefill

ある条件下において
ラプラス変換した後の関数に対応する変換前の関数は一つだけになる。
この為、次のような対応表が作ることが出来る。

\begin{align}
% F(s) (= & \mathcal{L}[f(t)]) & \mathcal{L}^{-1}[F(s)] (= & f(t) ) \quad (t>0)\\
%
 F(s) =& \frac{1}{s^{n}}  & \mathcal{L}^{-1}[F(s)] = & \frac{t^{n-1}}{(n-1)!} \label{base_01}\\
 F(s) =& \frac{1}{s-a} & \mathcal{L}^{-1}[F(s)] = & e^{at} \label{base_02}\\
 F(s) =& \frac{\omega}{s^2+\omega^2} & \mathcal{L}^{-1}[F(s)] = & \sin\omega t \\
 F(s) =& \frac{s}{s^2+\omega^2} & \mathcal{L}^{-1}[F(s)] = & \cos\omega t \\
 F(s) =& \frac{1}{(s-a)^{n}}  & \mathcal{L}^{-1}[F(s)] = & \frac{t^{n-1}e^{at}}{(n-1)!} \\
 F(s) =& \frac{\omega}{(s-a)^2+\omega^2} & \mathcal{L}^{-1}[F(s)] = & e^{at}\sin\omega t \\
 F(s) =& \frac{s-a}{(s-a)^2+\omega^2} & \mathcal{L}^{-1}[F(s)] = & e^{at}\cos\omega t \label{base_07}
\end{align}




\textbf{逆変換の公式}
\begin{equation}
 f(t) = \lim_{T\rightarrow\infty} \frac{1}{2\pi i}
  \int_{\sigma -Ti}^{\sigma +Ti} e^{st}F(s)\mathrm{d}s
\end{equation}


\hrulefill
\textbf{問題}
\hrulefill

次の関数$F(s)$から逆ラプラス変換で$f(t)$を求めよ。
\begin{enumerate}
 \item
      \begin{equation}
       F(s) = \frac{1}{s^2+3s+2}
      \end{equation}

 \item
      \begin{equation}
       F(s) = \frac{s^2+9s-6}{s(s-1)(s+3)}
      \end{equation}

\end{enumerate}

\dotfill

\textbf{解法}

ラプラス変換は線形性がある為、
逆変換を行いたい式を
式(\ref{base_01})~(\ref{base_07})
の和と定数倍に分けることを考える。

うまく分けることが出来たのなら
それぞれの式を対応する逆変換の式に置き換えて
式を整理する。

\dotfill

\begin{enumerate}
 \item

      \begin{equation}
       F(s) = \frac{1}{s^2+3s+2}
        = \frac{1}{(s+1)(s+2)}
        = \frac{1}{s+1} - \frac{1}{s+2}
      \end{equation}

      式(\ref{base_02})より次の式が得られる。
      \begin{align}
       \mathcal{L}^{-1}\left[ \frac{1}{s+1} \right]
        =& \mathcal{L}^{-1}\left[ \frac{1}{s-(-1)} \right]
        = e^{-t}\\
       \mathcal{L}^{-1}\left[ \frac{1}{s+2} \right]
        =& \mathcal{L}^{-1}\left[ \frac{1}{s-(-2)} \right]
        = e^{-2t}\\
      \end{align}
      これにより次のようなラプラス変換があることが分かる。
      \begin{equation}
       \mathcal{L}\left[ e^{-t}-e^{-2t} \right] = \frac{1}{s+1} - \frac{1}{s+2}
      \end{equation}

      ラプラスの逆変換は次のようになる。
      \begin{equation}
       \mathcal{L}^{-1}\left[ \frac{1}{s^2+3s+2} \right]
        = e^{-t}-e^{-2t}
      \end{equation}

 \item

      \begin{align}
       F(s) =& \frac{s^2+9s-6}{s(s-1)(s+3)}
        = \frac{s^2+3s}{s(s-1)(s+3)} + \frac{6s-6}{s(s-1)(s+3)}\\
        =& \frac{1}{s-1} + \frac{6}{s(s+3)}
        = \frac{1}{s-1} + \frac{2}{s} - \frac{2}{s+3}
      \end{align}

      式(\ref{base_01})より次の式が得られる。
      \begin{equation}
       \mathcal{L}^{-1}\left[ \frac{1}{s} \right]
        = \frac{t^{1-1}}{(1-1)!}
        =1
      \end{equation}
      
      式(\ref{base_02})より次の式が得られる。
      \begin{align}
       \mathcal{L}^{-1}\left[ \frac{1}{s-1} \right]
       =& e^{t} \\
       \mathcal{L}^{-1}\left[ \frac{1}{s+3} \right]
        =& \mathcal{L}^{-1}\left[ \frac{1}{s-(-3)} \right]
        = e^{-3t}
      \end{align}

      これらにより次のようなラプラス変換があることが分かる。
      \begin{equation}
       \mathcal{L}\left[ e^{t} + 2 -2e^{-3t} \right]
        = \frac{1}{s-1} + \frac{2}{s} - \frac{2}{s+3}
      \end{equation}

      よって求めるべき逆変換は次のような式となる。
      \begin{equation}
       \mathcal{L}^{-1}\left[ \frac{s^2+9s-6}{s(s-1)(s+3)} \right]
        = e^{t} + 2 -2e^{-3t}
      \end{equation}

\end{enumerate}


\end{document}

