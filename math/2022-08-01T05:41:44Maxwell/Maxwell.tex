\documentclass[12pt,b5paper]{ltjsarticle}

%\usepackage[margin=15truemm, top=5truemm, bottom=5truemm]{geometry}
\usepackage[margin=15truemm]{geometry}

\usepackage{amsmath,amssymb}
%\pagestyle{headings}
\pagestyle{empty}




%\usepackage{listings,url}
%\renewcommand{\theenumi}{(\arabic{enumi})}

\usepackage{graphicx}

\usepackage{tikz}
\usetikzlibrary {arrows.meta}
\usepackage{wrapfig}	% required for `\wrapfigure' (yatex added)
\usepackage{bm}	% required for `\bm' (yatex added)

% ルビを振る
%\usepackage{luatexja-ruby}	% required for `\ruby'

%% 核Ker 像Im Hom を定義
%\newcommand{\Img}{\mathop{\mathrm{Im}}\nolimits}
%\newcommand{\Ker}{\mathop{\mathrm{Ker}}\nolimits}
%\newcommand{\Hom}{\mathop{\mathrm{Hom}}\nolimits}

\DeclareMathOperator{\Rot}{rot}
\DeclareMathOperator{\Div}{div}
\DeclareMathOperator{\Grad}{grad}
\DeclareMathOperator{\arcsinh}{arcsinh}

\begin{document}



\textbf{Maxwell方程式}

\hrulefill

$\bm{x}\in\mathbb{R}^3,t\in\mathbb{R}$を変数とし、
$\varepsilon_0\in\mathbb{R},\mu_0\in\mathbb{R}$を定数とする。

ベクトル場$\mathbb{E},\mathbb{B},\bm{j}$と関数$\rho$は$C^\infty$-級であり、
次で示すような写像である。
\begin{align}
 \mathbb{E} :&\ \mathbb{R}\times\mathbb{R}^3 \to \mathbb{R}^3, \quad (t,\bm{x})\mapsto \mathbb{E}(t,\bm{x})\\
  \mathbb{B} :&\ \mathbb{R}\times\mathbb{R}^3 \to \mathbb{R}^3, \quad (t,\bm{x})\mapsto \mathbb{B}(t,\bm{x})\\
  \bm{j} :&\ \mathbb{R}\times\mathbb{R}^3 \to \mathbb{R}^3, \quad (t,\bm{x})\mapsto \bm{j}(t,\bm{x})\\
  \mathbb{\rho} :&\ \mathbb{R}\times\mathbb{R}^3 \to \mathbb{R}, \quad (t,\bm{x})\mapsto \rho(t,\bm{x})
\end{align}

ナブラ$\nabla$は次で示すような演算子のベクトルである。
$t$については微分を行わない。
\begin{equation}
 \nabla = \begin{pmatrix} \frac{\partial}{\partial x} & \frac{\partial}{\partial y} & \frac{\partial}{\partial z} \end{pmatrix}
\end{equation}

発散$\Div$や回転$\Rot$は次のようなものである。
\begin{equation}
 \Div f = \nabla \cdot f
  ,\qquad
 \Rot f = \nabla \times f
\end{equation}

次の4つの式をまとめてMaxwell方程式という。
\begin{equation}
 \Div\mathbb{E} = \frac{1}{\varepsilon_0}\rho,
  \quad
 \Div\mathbb{B} = 0,
  \quad
 \Rot\mathbb{E} = -\frac{\partial \mathbb{B}}{\partial t},
  \quad
 \Rot\mathbb{B} = \mu_0\bm{j} + \mu_0\varepsilon_0\frac{\partial \mathbb{E}}{\partial t}
\end{equation}

$\mathbb{E}$は電場、
$\mathbb{B}$は磁場、
$\bm{j}$は電流密度、
$\rho$は電化密度、
$\varepsilon_0$は真空の誘電率、
$\mu_0$は真空の透磁率
である。



\hrulefill


Maxwell方程式を満たしているとする。




\begin{enumerate}
 \item
      $D\subset\mathbb{R}^3$において、
      $\rho$の積分値を$\varepsilon_0$で割ったものは
      $D$の表面$\partial D$上での電場$\mathbb{E}$の面積分に
      等しいことを示せ。

      つまり、次の積分が一致する。
      \begin{equation}
       \frac{1}{\varepsilon_0} \int_{D} \rho \mathrm{d}D
        = \int_{\partial D}\mathbb{E}\mathrm{d}S
      \end{equation}

      \dotfill

      \textbf{発散定理}
      \begin{equation}
       \int\!\!\!\int\!\!\!\int_V \Div\bm{f}\mathrm{d}V
        = \int\!\!\!\int_S \bm{f}\cdot\bm{n}\mathrm{d}S
      \end{equation}

      \dotfill

      Maxwell方程式より$\Div\mathbb{E}=\rho/\varepsilon_0$であるので、
      次のような変形が出来る。
      \begin{equation}
       \frac{1}{\varepsilon_0} \int_{D} \rho \mathrm{d}D
        = \int_{D} \frac{\rho}{\varepsilon_0} \mathrm{d}D
        = \int_{D} \Div\mathbb{E} \mathrm{d}D
      \end{equation}

      ここに発散定理を用いれば次の式となる。
      \begin{equation}
       \frac{1}{\varepsilon_0} \int_{D} \rho \mathrm{d}D
        = \int_{\partial D}\mathbb{E}\mathrm{d}S
      \end{equation}

      \hrulefill

 \item
      $D\subset\mathbb{R}^3$の表面$\partial D$上での
      磁場$\mathbb{B}$の面積分は
      常に0であることを示せ。

      つまり、次の式が成り立つことを示せ。
      \begin{equation}
       \int_{\partial D}\mathbb{B} =0
      \end{equation}

      \dotfill

      発散定理より次のような式が得られる。
      \begin{equation}
       \int_{\partial D}\mathbb{B}
        = \int_{D}\Div\mathbb{B} \mathrm{d}D
      \end{equation}

      Maxwell方程式より$\Div\mathbb{B}=0$であるので、
      上記積分は$0$である。


      \hrulefill

 \item
      曲面片の像$S\subset \mathbb{R}^3$上での
      磁場$\mathbb{B}$の
      面積分の時間微分は
      電場$\mathbb{E}$の$\partial S$に沿った
      線積分の$-1$倍に等しいことを示せ。

      つまり、次が成り立つことを示せ。
      \begin{equation}
       \frac{\mathrm{\partial}}{\partial t}\int_{S} \mathbb{B}\mathrm{d}S
        = -\int_{\partial S} \mathbb{E}\mathrm{d}s
      \end{equation}

      \dotfill

      ストークスの定理
      \begin{equation}
       \int_{S}\Rot \bm{f} \mathrm{d}S
        = \int_{\partial S} \bm{f} \mathrm{d}s
      \end{equation}

      \dotfill

      ストークスの定理より
      \begin{equation}
       \int_{\partial S} \mathbb{E}\mathrm{d}s
        = \int_{S} \Rot\mathbb{E}\mathrm{d}S
      \end{equation}
      Maxwell方程式より
      $\Rot\mathbb{E}=-\frac{\partial \mathbb{B}}{\partial t}$
      が成り立つので、
      \begin{equation}
       -\int_{\partial S} \mathbb{E}\mathrm{d}s
        = \int_{S} (-\Rot\mathbb{E})\mathrm{d}S
        = \int_{S} \frac{\partial \mathbb{B}}{\partial t}\mathrm{d}S
      \end{equation}

      積分は$t$について影響しないので、
      積分と微分を入れ替えることで、次の式が得られる。
      \begin{equation}
       -\int_{\partial S} \mathbb{E}\mathrm{d}s
        =\frac{\partial}{\partial t}
        \int_{S} \mathbb{B}\mathrm{d}S
      \end{equation}


      \hrulefill

 \item
      曲面片の像$S\subset \mathbb{R}^3$上で
      $\bm{j}$を面積分したものに
      $\mu_0$をかけた値と
      $S$上で$\mathbb{E}$を
      面積分したものの時間微分に
      $\mu_0$と$\varepsilon_0$をかけたものの和は
      $\partial S$に沿った$\mathbb{B}$の
      線積分の値と等しいことを示せ。

      つまり、次が成り立つことを示せ。
      \begin{equation}
       \mu_0\int_{S}\bm{j}\mathrm{d}S
        + \mu_0 \varepsilon_0 \frac{\partial}{\partial t}
        \int_{S}\mathbb{E}\mathrm{d}S
        = \int_{\partial S}\mathbb{B}\mathrm{d}s
      \end{equation}



      \dotfill

      Maxwell方程式より次が成り立っている。
      \begin{equation}
       \Rot\mathbb{B}
        = \mu_0\bm{j}
        + \mu_0\varepsilon_0\frac{\partial \mathbb{E}}{\partial t}
      \end{equation}

      両辺を$S$で積分すると次の式が得られる。
      \begin{equation}
      \int_{S} \Rot\mathbb{B} \mathrm{d}S
      = \mu_0 \int_{S} \bm{j} \mathrm{d}S
      + \mu_0\varepsilon_0
      \int_{S} \frac{\partial \mathbb{E}}{\partial t} \mathrm{d}S
      \end{equation}

      右辺の偏微分は$t$の積分ではないので入れ替えて
      \begin{equation}
       \int_{S} \frac{\partial \mathbb{E}}{\partial t} \mathrm{d}S
        = \frac{\partial}{\partial t} \int_{S} \mathbb{E} \mathrm{d}S
      \end{equation}

      左辺はストークスの定理から
      \begin{equation}
       \int_{S} \Rot\mathbb{B} \mathrm{d}S
        = \int_{\partial S} \mathbb{B} \mathrm{d}s
      \end{equation}

      よって次の式が成り立つ。
      \begin{equation}
       \int_{\partial S} \mathbb{B} \mathrm{d}s
        =
        \mu_0 \int_{S} \bm{j} \mathrm{d}S
        + \mu_0\varepsilon_0
        \frac{\partial}{\partial t} \int_{S} \mathbb{E} \mathrm{d}S
      \end{equation}


      \hrulefill


\end{enumerate}




\end{document}
