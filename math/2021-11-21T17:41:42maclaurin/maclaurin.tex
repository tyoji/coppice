\documentclass[12pt,b5paper]{ltjsarticle}

\usepackage[margin=15truemm]{geometry}
\pagestyle{empty}

\usepackage{amssymb}
\usepackage{amsmath}	% required for `\align' (yatex added)

\begin{document}


$f(x)=\sin x$の剰余項付きマクローリンの展開
\begin{align}
 f^{(n)}(x)&=\sin(x+\frac{\pi}{2}n)\\
 f^{(n)}(0)&=\sin(\frac{\pi}{2}n)\\
 f^{(n)}(\theta x)&=\sin(\theta x+\frac{\pi}{2}n)
\end{align}
$f^{(n)}(0)=\sin(\frac{\pi}{2}n)$は奇数番目は$0$で
偶数番目は$1$と$-1$が交互に現れる。

$n=2m$と置くと
$f^{(n)}(\theta x)=\sin(\theta x+m\pi)$
となる。
\begin{align}
 f(x) =& \frac{0}{0!}x^0 + \frac{1}{1!}x^1 + \frac{0}{2!}x^2
         + \frac{-1}{3!}x^3 + \frac{0}{4!}x^4 + \frac{1}{5!}x^5 + \frac{0}{6!}x^6 + \cdots \\
 =& \frac{1}{1!}x - \frac{1}{3!}x^3 + \frac{1}{5!}x^5 - \frac{1}{7!}x^7 + \cdots \nonumber\\
   & \quad + \frac{(-1)^{m-1}}{(2m-1)!}x^{2m-1} + \frac{\sin(\theta x+m\pi)}{(2m)!}x^{2m}\\
 =& \sum_{k=0}^{m-1}\frac{(-1)^k}{(2k+1)!}x^{2k+1} + \frac{\sin(\theta x+m\pi)}{(2m)!}x^{2m}
\end{align}

\hrulefill


補足

微分は多項式の形をしていると行いやすいので、
次のような形(級数)にしてみる。
\begin{equation}
 \sin x = a_0 + a_1x^1 + a_2x^2 + a_3x^3 + a_4x^4 + \cdots + a_kx^k + \cdots\label{124330_21Nov21}
\end{equation}

$a_0$は$x=0$を代入すると求まる。
$0=a_0$

$a_1$は1回微分をしてから$x=0$を代入すると求まる。
$1 = a_1$
\begin{equation}
 \cos x = a_1 + 2a_2x^1 + 3a_3x^2 + 4a_4x^3 + \cdots + ka_kx^{k-1} + \cdots
\end{equation}

$a_2$は2回微分をしてから$x=0$を代入すると求まる。
$0 = 2a_2$
\begin{equation}
 -\sin x = 2a_2 + 3\cdot 2a_3x^1 + 4\cdot 3a_4x^2 + \cdots + k(k-1)a_kx^{k-2} + \cdots
\end{equation}


$a_3$は3回微分をしてから$x=0$を代入すると求まる。
$-1 = 3!a_3$
\begin{equation}
 -\cos x = 3\cdot 2a_3 + 4\cdot 3\cdot 2a_4x^1 + \cdots + k(k-1)(k-2)a_kx^{k-3} + \cdots
\end{equation}

$C^{\infty}$級であれば無限に行える為、
求めた$a_k$を(\ref{124330_21Nov21})の式に代入し級数が得られる。
\begin{equation}
 \sin x = \sum_{k=0}^{\infty}\frac{(-1)^k}{(2k+1)!}x^{2k+1}
\end{equation}


\end{document}
