\documentclass[12pt,b5paper]{ltjsarticle}

%\usepackage[margin=15truemm, top=5truemm, bottom=5truemm]{geometry}
%\usepackage[margin=10truemm,left=15truemm]{geometry}
\usepackage[margin=10truemm]{geometry}

\usepackage{amsmath,amssymb}
%\pagestyle{headings}
\pagestyle{empty}

%\usepackage{listings,url}
%\renewcommand{\theenumi}{(\arabic{enumi})}

%\usepackage{graphicx}

%\usepackage{tikz}
%\usetikzlibrary {arrows.meta}
%\usepackage{wrapfig}
%\usepackage{bm}

% ルビを振る
%\usepackage{luatexja-ruby}	% required for `\ruby'

%% 核Ker 像Im Hom を定義
%\newcommand{\Img}{\mathop{\mathrm{Im}}\nolimits}
%\newcommand{\Ker}{\mathop{\mathrm{Ker}}\nolimits}
%\newcommand{\Hom}{\mathop{\mathrm{Hom}}\nolimits}

%\DeclareMathOperator{\Rot}{rot}
%\DeclareMathOperator{\Div}{div}
%\DeclareMathOperator{\Grad}{grad}
%\DeclareMathOperator{\arcsinh}{arcsinh}
%\DeclareMathOperator{\arccosh}{arccosh}
%\DeclareMathOperator{\arctanh}{arctanh}

%\usepackage{url}

%\usepackage{listings}
%
%\lstset{
%%プログラム言語(複数の言語に対応,C,C++も可)
%  language = Python,
%%  language = Lisp,
%%  language = C,
%  %背景色と透過度
%  %backgroundcolor={\color[gray]{.90}},
%  %枠外に行った時の自動改行
%  breaklines = true,
%  %自動改行後のインデント量(デフォルトでは20[pt])
%  breakindent = 10pt,
%  %標準の書体
%%  basicstyle = \ttfamily\scriptsize,
%  basicstyle = \ttfamily,
%  %コメントの書体
%%  commentstyle = {\itshape \color[cmyk]{1,0.4,1,0}},
%  %関数名等の色の設定
%  classoffset = 0,
%  %キーワード(int, ifなど)の書体
%%  keywordstyle = {\bfseries \color[cmyk]{0,1,0,0}},
%  %表示する文字の書体
%  %stringstyle = {\ttfamily \color[rgb]{0,0,1}},
%  %枠 "t"は上に線を記載, "T"は上に二重線を記載
%  %他オプション:leftline,topline,bottomline,lines,single,shadowbox
%  frame = TBrl,
%  %frameまでの間隔(行番号とプログラムの間)
%  framesep = 5pt,
%  %行番号の位置
%  numbers = left,
%  %行番号の間隔
%  stepnumber = 1,
%  %行番号の書体
%%  numberstyle = \tiny,
%  %タブの大きさ
%  tabsize = 4,
%  %キャプションの場所("tb"ならば上下両方に記載)
%  captionpos = t
%}

%\usepackage{cancel}
%\usepackage{bussproofs}
%\usepackage{proof}

\begin{document}

\hrulefill

\begin{enumerate}
 \item
      $f\in C^{m}_{\mathrm{per}} [-\pi,\pi]$
      ならば
      $\left( \widehat{f^{(m)}}(n) \right)_{n\in\mathbb{Z}}$
      が有界であることを示せ。
      \begin{quotation}
       \textbf{Hint}:
       有界閉区間上の連続関数に対しては
       最大値、最小値の存在定理が成り立つので、
       特に有界である。
      \end{quotation}

      \dotfill

      \begin{equation}
       C^{m}_{\mathrm{per}}[-\pi,\pi]
        =\left\{
          f:[-\pi,\pi]\to\mathbb{C} \ \middle|\
          \begin{split}
           & fはC^{m}級であり、\
          0\leq {}^{\forall}k \leq m \\
          & に対し
          f^{(k)}(\pi)=f^{(k)}(-\pi)
          \end{split}
         \right\}
      \end{equation}

%      関数$f$のフーリエ級数が
%       $\displaystyle f(x) = \sum_{n=-\infty}^{\infty} \widehat{f}(n)e^{inx}$
%      であるから、
%      $f^{(m)}$のフーリエ級数は
%       $\displaystyle f^{(m)}(x) = \sum_{n=-\infty}^{\infty} \widehat{f^{(m)}}(n)e^{inx}$
%      である。


      $f\in C^{m}_{\mathrm{per}}[-\pi,\pi]$であるので、
      導関数$f^{(m)}(x)$は連続関数である。

      閉区間$[-\pi,\pi]$上の連続関数$f^{(m)}(x)$は
      最大値最小値が存在する。

      \begin{equation}
       \widehat{f^{(m)}}(n)
        = \frac{1}{2\pi} \int_{-\pi}^{\pi} f^{(m)}(x)e^{-inx} \mathrm{d}x
      \end{equation}


      $\widehat{f^{(m)}}(n)$を次のように評価する。
      \begin{equation}
       \int_{-\pi}^{\pi} \left| f^{(m)}(x)e^{-inx} \right| \mathrm{d}x
        \leq
        \int_{-\pi}^{\pi} \left| f^{(m)}(x) \right| \left| e^{-inx} \right| \mathrm{d}x
        =
        \int_{-\pi}^{\pi} \left| f^{(m)}(x) \right| \mathrm{d}x
        < \infty
      \end{equation}

      よって、
      $\widehat{f^{(m)}}(n)$は値を持つため、
      $\left( \widehat{f^{(m)}}(n) \right)_{n\in\mathbb{Z}}$
      は有界である。

      \hrulefill

 \item
      Fourier 級数の対称な部分和を
      $\displaystyle \sum_{n=-N}^{N}\widehat{f}(n)e^{inx}
       = \int f(y) D_{N}(x-y)\mathrm{d}y$
      と書いたときの
      $D_{N}$ (Dirichlet 核)が
      $\displaystyle D_{N}(x) =
       \frac{\sin\left(\frac{2N+1}{2}x\right)}{2\pi\sin\frac{x}{2}}$
      であることを示せ。
      \begin{quotation}
       \textbf{Hint}:
       $\sin\frac{x}{2}$を移項して
       三角関数の積和公式を使う。
      \end{quotation}

      \dotfill


      フーリエ級数とフーリエ係数
      \begin{equation}
       f(x)
        = \sum_{n=-\infty}^{\infty} \widehat{f}(n)e^{inx},\quad
       \widehat{f}(n)
        = \frac{1}{2\pi} \int_{-\pi}^{\pi} f(x)e^{-inx} \mathrm{d}x
      \end{equation}

      $\widehat{f}(n)e^{inx}$は次のように変形できる。
      \begin{align}
       \widehat{f}(n)e^{inx}
       &= \frac{e^{inx}}{2\pi} \int_{-\pi}^{\pi} f(x)e^{-inx} \mathrm{d}x
       = \frac{e^{inx}}{2\pi} \int_{-\pi}^{\pi} f(y)e^{-iny} \mathrm{d}y\\
       &= \int_{-\pi}^{\pi} f(y)\frac{e^{in(x-y)}}{2\pi} \mathrm{d}y
      \end{align}

      $\displaystyle \sum_{n=-N}^{N}\widehat{f}(n)e^{inx}$
      を計算する。
      \begin{align}
       \sum_{n=-N}^{N}\widehat{f}(n)e^{inx}
       &= \sum_{n=-N}^{N} \int_{-\pi}^{\pi} f(y)\frac{e^{in(x-y)}}{2\pi} \mathrm{d}y
       = \int_{-\pi}^{\pi} f(y) \sum_{n=-N}^{N} \frac{e^{in(x-y)}}{2\pi} \mathrm{d}y
      \end{align}

      つまり、$D_{N}(x-y)$は
      \begin{equation}
       D_{N}(x-y)
        = \sum_{n=-N}^{N} \frac{e^{in(x-y)}}{2\pi}
        \label{eq:Dn}
      \end{equation}
      である。

      $\sum_{n=-N}^{N} e^{inX}$を考える。
      \begin{equation}
       \begin{cases}
        e^{inX}=\cos{nX}+i\sin{nX}\\
        e^{-inX}=\cos{nX}-i\sin{nX}
       \end{cases}
        \quad \Rightarrow \quad
        \cos{nX}= \frac{1}{2}\left( e^{inX}+e^{-inX} \right)
      \end{equation}
      より次の式が得られる。
      \begin{equation}
       \sum_{n=-N}^{N} e^{inX}
        = 1+ 2\sum_{n=1}^{N} \cos{nX}
        \label{eq:sumexp}
      \end{equation}

      ここで、三角関数の積和の公式から次が得られる。
      \begin{gather}
       \sin\left( \frac{2n+1}{2}x \right) - \sin\left( \frac{2n-1}{2}x \right)
        = 2\cos{nx}\sin{\frac{x}{2}}\\
       \cos{nx} = \frac{1}{2\sin{\frac{x}{2}}} \left( \sin\left( \frac{2n+1}{2}x \right) - \sin\left( \frac{2n-1}{2}x \right) \right)
      \end{gather}

      これを用いると
      \begin{align}
       \sum_{n=1}^{N}\cos{nx}
        &= \sum_{n=1}^{N} \frac{1}{2\sin{\frac{x}{2}}} \left( \sin\left( \frac{2n+1}{2}x \right) - \sin\left( \frac{2n-1}{2}x \right) \right)\\
        &= \frac{1}{2\sin{\frac{x}{2}}} \left( \sin\left( \frac{2N+1}{2}x \right) - \sin\left( \frac{x}{2} \right) \right)\\
       &= \frac{\sin\left( \frac{2N+1}{2}x \right)}{2\sin{\frac{x}{2}}} - \frac{1}{2}
      \end{align}
      を得る。
      式\eqref{eq:sumexp}にこれを代入する。
      \begin{equation}
       \sum_{n=-N}^{N} e^{inX}
        = \frac{\sin\left( \frac{2N+1}{2}X \right)}{\sin{\frac{X}{2}}}
      \end{equation}

      式\eqref{eq:Dn}より
      \begin{equation}
       D_{N}(x) = \frac{\sin\left( \frac{2N+1}{2}x \right)}{2\pi \sin{\frac{x}{2}}}
      \end{equation}
      である。



      \hrulefill

\end{enumerate}

\hrulefill

\end{document}
