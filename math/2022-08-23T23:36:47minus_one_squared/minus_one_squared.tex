\documentclass[12pt,b5paper]{ltjsarticle}

%\usepackage[margin=15truemm, top=5truemm, bottom=5truemm]{geometry}
\usepackage[margin=10truemm]{geometry}

\usepackage{amsmath,amssymb}
%\pagestyle{headings}
\pagestyle{empty}

%\usepackage{listings,url}
%\renewcommand{\theenumi}{(\arabic{enumi})}

%\usepackage{graphicx}

%\usepackage{tikz}
%\usetikzlibrary {arrows.meta}
%\usepackage{wrapfig}	% required for `\wrapfigure' (yatex added)
%\usepackage{bm}	% required for `\bm' (yatex added)

% ルビを振る
%\usepackage{luatexja-ruby}	% required for `\ruby'

%% 核Ker 像Im Hom を定義
%\newcommand{\Img}{\mathop{\mathrm{Im}}\nolimits}
%\newcommand{\Ker}{\mathop{\mathrm{Ker}}\nolimits}
%\newcommand{\Hom}{\mathop{\mathrm{Hom}}\nolimits}

%\DeclareMathOperator{\Rot}{rot}
%\DeclareMathOperator{\Div}{div}
%\DeclareMathOperator{\Grad}{grad}
%\DeclareMathOperator{\arcsinh}{arcsinh}
%\DeclareMathOperator{\arccosh}{arccosh}
%\DeclareMathOperator{\arctanh}{arctanh}



\begin{document}

\hrulefill
\textbf{$(-1)\times(-1)$}
\hrulefill

次の式
(\ref{zero})、
(\ref{inv})、
(\ref{one})、
(\ref{distri})
を用いて$(-1)\times (-1)=1$を示す。


\dotfill
加法(足し算)
\dotfill

零元(加法の単位元)$0$の性質
\begin{equation}
 0 + \alpha
  = \alpha + 0
  = \alpha
  \label{zero}
\end{equation}

逆元と零元の性質($\alpha$の逆元は$-\alpha$)
\begin{equation}
 \alpha + (-\alpha)
  = (-\alpha) + \alpha
  =0
  \label{inv}
\end{equation}

\dotfill
乗法(掛け算)
\dotfill

乗法の単位元$1$の性質
\begin{equation}
 1\times \beta
  = \beta \times 1
  = \beta
  \label{one}
\end{equation}

\dotfill
分配法則
\dotfill

加法と乗法の性質
\begin{equation}
 a\times (b+c) = a\times b + a\times c
  \label{distri}
\end{equation}

\hrulefill

$(-1)\times (-1)$を示す為に
まず$(-1)\times 0 = 0$を示す。

\begin{align}
 (-1)\times 0
 =& (-1)\times ( 0 + 0 )\label{1st}\\
 =& (-1)\times 0 + (-1)\times 0\label{2nd}
\end{align}

式(\ref{1st})は零元の性質(\ref{zero})より$\alpha=0$とした式$0+0=0$から得られる。

式(\ref{2nd})は分配法則(\ref{distri})より$a=-1,\, b=c=0$とした式から得られる。

この式を次のように変形する。

\begin{align}
 (-1)\times 0 =& (-1)\times 0 + (-1)\times 0\\
 (-1)\times 0 + (- (-1)\times 0) =& (-1)\times 0 + (-1)\times 0  + (- (-1)\times 0)\label{3rd}\\
 0 =& (-1)\times 0\label{4th}
\end{align}

式(\ref{3rd})は両辺に$(-1)\times 0$の逆元$-(-1)\times 0$を加えている。

式(\ref{4th})は逆元と零元の性質(\ref{inv})より$\alpha = (-1)\times 0$とした式
$(-1)\times 0 + (-(-1)\times 0) = 0 $から得られる。

これにより$(-1)\times 0 =0$が示せた。

\begin{align}
 0 =& (-1)\times 0\\
 =& (-1)\times ((-1)+1)\label{5th}\\
 =& (-1)\times (-1) + (-1)\times 1\label{6th}\\
 =& (-1)\times (-1) + (-1)\label{7th}
\end{align}

式(\ref{5th})は逆元と零元の性質(\ref{inv})より$\alpha = 1$とした式$0=(-1)+1$から得られる。

式(\ref{6th})は分配法則(\ref{distri})より$a=-1,\, b=-1,\, c=1$とした式から得られる。

式(\ref{7th})は単位元の性質(\ref{one})より$\beta = -1$とした式$(-1)\times 1 = (-1)$から得られる。


\begin{align}
 0 =& (-1)\times (-1) + (-1)\\
 0 + 1 =& (-1)\times (-1) + (-1) +1\label{8th}\\
 0 + 1 =& (-1)\times (-1) + 0\label{9th}\\
 1 =& (-1)\times (-1)\label{10th}
\end{align}

式(\ref{8th})は両辺に$-1$の逆元$1$を加えている。

式(\ref{9th})は逆元と零元の性質(\ref{inv})より$\alpha = 1$とした式$(-1)+1 =0$から得られる。

式(\ref{9th})は零元の性質(\ref{zero})より
$\alpha = 1$とした式%$0 + 1 =1$
と
$\alpha = (-1)\times (-1)$とした式%$(-1)\times (-1) + 0 =(-1)\times (-1)$
から得られる。

以上により$(-1)\times (-1)=1$であることが示せた。






\hrulefill

\hrulefill


\begin{gather}
 (-1)\times 0 = (-1)\times (0+0) = (-1)\times 0 + (-1)\times 0\\
 0 = (-1)\times 0
\end{gather}

\begin{align}
 (-1)\times(-1)
 =& (-1)\times(-1) +0\\
 =& (-1)\times(-1) +(-1)+1\\
 =& (-1)\times(-1) +(-1)\times 1+1\\
 =& (-1)\times((-1) + 1)+1\\
 =& (-1)\times 0+1\\
 =& 0+1\\
 =& 1
\end{align}

\hrulefill


\end{document}

