\documentclass[12pt,b5paper]{ltjsarticle}

%\usepackage[margin=15truemm, top=5truemm, bottom=5truemm]{geometry}
%\usepackage[margin=10truemm,left=15truemm]{geometry}
\usepackage[margin=10truemm]{geometry}

\usepackage{amsmath,amssymb}
%\pagestyle{headings}
\pagestyle{empty}

%\usepackage{listings,url}
%\renewcommand{\theenumi}{(\arabic{enumi})}

%\usepackage{graphicx}

%\usepackage{tikz}
%\usetikzlibrary {arrows.meta}
%\usepackage{wrapfig}
%\usepackage{bm}

% ルビを振る
\usepackage{luatexja-ruby}	% required for `\ruby'

%% 核Ker 像Im Hom を定義
%\newcommand{\Img}{\mathop{\mathrm{Im}}\nolimits}
%\newcommand{\Ker}{\mathop{\mathrm{Ker}}\nolimits}
%\newcommand{\Hom}{\mathop{\mathrm{Hom}}\nolimits}

%\DeclareMathOperator{\Rot}{rot}
%\DeclareMathOperator{\Div}{div}
%\DeclareMathOperator{\Grad}{grad}
%\DeclareMathOperator{\arcsinh}{arcsinh}
%\DeclareMathOperator{\arccosh}{arccosh}
%\DeclareMathOperator{\arctanh}{arctanh}

\usepackage{url}

%\usepackage{listings,url}
%
%\lstset{
%%プログラム言語(複数の言語に対応,C,C++も可)
%  language = Python,
%%  language = Lisp,
%%  language = C,
%  %背景色と透過度
%  %backgroundcolor={\color[gray]{.90}},
%  %枠外に行った時の自動改行
%  breaklines = true,
%  %自動改行後のインデント量(デフォルトでは20[pt])
%  breakindent = 10pt,
%  %標準の書体
%%  basicstyle = \ttfamily\scriptsize,
%  basicstyle = \ttfamily,
%  %コメントの書体
%%  commentstyle = {\itshape \color[cmyk]{1,0.4,1,0}},
%  %関数名等の色の設定
%  classoffset = 0,
%  %キーワード(int, ifなど)の書体
%%  keywordstyle = {\bfseries \color[cmyk]{0,1,0,0}},
%  %表示する文字の書体
%  %stringstyle = {\ttfamily \color[rgb]{0,0,1}},
%  %枠 "t"は上に線を記載, "T"は上に二重線を記載
%  %他オプション:leftline,topline,bottomline,lines,single,shadowbox
%  frame = TBrl,
%  %frameまでの間隔(行番号とプログラムの間)
%  framesep = 5pt,
%  %行番号の位置
%  numbers = left,
%  %行番号の間隔
%  stepnumber = 1,
%  %行番号の書体
%%  numberstyle = \tiny,
%  %タブの大きさ
%  tabsize = 4,
%  %キャプションの場所("tb"ならば上下両方に記載)
%  captionpos = t
%}



\begin{document}

\hrulefill

参考文献

測度論 (1) ルベーグ測度と完全加法性

\url{http://racco.mikeneko.jp/Kougi/2018a/AMA/2018a_ama14.pdf}

\hrulefill

$X$を集合、$\mathcal{F}$を$X$の有限加法族とする。

外測度$m^{*}$について
$m^{*}$-可測集合全体を$\mathcal{M}_{m^{*}}$とする。

$\mathcal{F} \subset \mathcal{M}_{m^{*}}$
であり、
任意の$A \in \mathcal{F}$
に対し、
$m(A)=m^{*}(A)$が成り立つならば、
$(X,\mathcal{F})$上の有限加法的測度$m$は
完全加法的である。

\begin{equation}
 m\left( \bigsqcup_{n=1}^{\infty}A_{n} \right)
  = \sum_{n=1}^{\infty} m(A_{n})
\end{equation}


\dotfill

%$m^{*}$は$X$の外測度である。
%
%$\mathcal{F}$は
%$m^{*}$-可測全体の集合
%$\mathcal{M}_{m^{*}}$
%の部分集合である。
%$\mathcal{F} \subset \mathcal{M}_{m^{*}}$

$A_{k}\in\mathcal{F}
,\ A=\bigsqcup_{k=1}^{\infty}A_{k}$(各$k$について互いに素)
とする。
%$S \subset X$を部分集合とする。

%$\mathcal{F}$は
%有限加法族であるので、
%$\bigcup_{i=1}^{n}A_{i} \in\mathcal{F}$である。

%$A_{i}$が互いに素である集合とすると
%$m$は有限加法的測度であるので、
%有限加法性が成り立つ。
%\begin{equation}
% m\left( \bigsqcup_{i=1}^{n}A_{i} \right)
%  = \sum_{i=1}^{n}m(A_{i})
%\end{equation}
%
%$m^{*}$は外測度であるので、
%完全劣加法性から次が成り立つ。
%\begin{equation}
% m^{*}(A) = m^{*}\left( \bigcup_{n=1}^{\infty}A_{n} \right)
%  \leq \sum_{n=1}^{\infty}m^{*}(A_{n})
%\end{equation}
%
Carath\`{e}odory の意味で可測
から、
任意の集合
$S \subset X$
に対して
次が成り立つ。
\begin{equation}
 m^{*}(S) = m^{*}(S \cap A) + m^{*}(S \cap A^{c})
\label{eq_finite_additivity}
\end{equation}

%$S = (S \cap A) \cup (S \cap A^{c})$
%であるので
%$m$の有限加法性により
%次の式が成り立つ。
%\begin{equation}
% m(S) =  m(S \cap A) + m(S \cap A^{c})
%  
%\end{equation}

$A=\bigsqcup_{k=1}^{\infty}A_{k}$
より
$S \cap A = \bigsqcup_{k=1}^{\infty}(S \cap A_{k})$
なので、
外測度$m^{*}$の
完全劣加法性から次が成り立つ。
\begin{equation}
 m^{*}(S \cap A)
  = m^{*}\left( \bigsqcup_{k=1}^{\infty}(S \cap A_{k}) \right)
  \leq \sum_{k=1}^{\infty} m^{*}\left(S \cap A_{k} \right)
  \label{eq_subadditivity}
\end{equation}

$m=m^{*}$であるので、
\eqref{eq_finite_additivity}
と
\eqref{eq_subadditivity}
から次の不等式が成り立つ。
\begin{equation}
 m(S)
  %=  m(S \cap A) + m(S \cap A^{c})
  \leq
  \sum_{k=1}^{\infty} m\left(S \cap A_{k} \right)
   + m(S \cap A^{c})
\end{equation}


もし、
この不等号が
次のように等号であったとする。
\begin{equation}
 m(S)
  =
  \sum_{k=1}^{\infty} m\left(S \cap A_{k} \right)
   + m(S \cap A^{c})
\end{equation}

$S$は任意の集合なので
$S=A$とすれば
\begin{equation}
 m(A)
  =
  \sum_{k=1}^{\infty} m\left(A \cap A_{k} \right)
   + m(A \cap A^{c})
   =
  \sum_{k=1}^{\infty} m\left( A_{k} \right)
   + m(A \cap A^{c})
   = \sum_{k=1}^{\infty} m\left( A_{k} \right)
\end{equation}
である。
$A=\bigsqcup_{k=1}^{\infty}A_{k}$であるから
$m\left(\bigsqcup_{k=1}^{\infty}A_{k}\right)
=\sum_{k=1}^{\infty} m\left( A_{k} \right)$
が得られる。

そこで、
自然数$n$について次の不等式が成り立つことを
数学的帰納法を用いて示す。
\begin{equation}
 m^{*}(S)
  \geq
  \sum_{k=1}^{n} m^{*}\left(S \cap A_{k} \right)
   + m^{*}(S \cap A^{c})
\end{equation}

$n=1$の場合

$A_{1}\subset A$より
$A_{1}^{c} \supset A^{c}$となるので次が得られる。
\begin{equation}
 m^{*}(S)
 =
  m^{*}\left(S \cap A_{1} \right)+ m^{*}(S \cap A_{1}^{c})
   \geq
   m^{*}\left(S \cap A_{1} \right)+ m^{*}(S \cap A^{c})
\end{equation}

$n=l$の時成り立つと仮定し
$n=l+1$について考える。
\begin{equation}
 m^{*}(S)
  \geq
  \sum_{k=1}^{l} m^{*}\left(S \cap A_{k} \right)+ m^{*}(S \cap A^{c})
\end{equation}

この式の$S$は任意の集合であるので、
$S\cap A_{l+1}^{c}$で置き換える。
\begin{equation}
 m^{*}(S\cap A_{l+1}^{c})
  \geq
 \sum_{k=1}^{l} m^{*}\left(S\cap A_{l+1}^{c} \cap A_{k} \right)+ m^{*}(S\cap A_{l+1}^{c} \cap A^{c})
\end{equation}

$A_{l+1}$と$A_{k}$は互いに素であるので、
$A_{l+1}^{c} \cap A_{k} = A_{k}$である。
また、$A=\bigsqcup_{k=1}^{\infty} A_{k}$より
$A_{l+1}^{c} \cap A^{c} = A^{c}$である。
これにより次のような式となる。
\begin{equation}
 m^{*}(S\cap A_{l+1}^{c})
  \geq
 \sum_{k=1}^{l} m^{*}\left(S\cap A_{k} \right)+ m^{*}(S \cap A^{c})
\end{equation}

$A_{l+1}$は加速であるので、次の式が成り立つ。
\begin{equation}
 m^{*}(S) = m^{*}(S \cap A_{l+1}) + m^{*}(S \cap A_{l+1}^{c})
\end{equation}

これにより次の不等式が成り立つ。
\begin{align}
 m^{*}(S) &= m^{*}(S \cap A_{l+1}) + m^{*}(S \cap A_{l+1}^{c})\\
 & \geq
 m^{*}(S \cap A_{l+1})
 + \sum_{k=1}^{l} m^{*}\left(S\cap A_{k} \right)+ m^{*}(S \cap A^{c})\\
 & \geq
 \sum_{k=1}^{l+1} m^{*}\left(S\cap A_{k} \right)+ m^{*}(S \cap A^{c})
\end{align}

\hrulefill

\end{document}
