\documentclass[12pt,b5paper]{ltjsarticle}

%\usepackage[margin=15truemm, top=5truemm, bottom=5truemm]{geometry}
\usepackage[margin=10truemm]{geometry}

\usepackage{amsmath,amssymb}
%\pagestyle{headings}
\pagestyle{empty}

%\usepackage{listings,url}
%\renewcommand{\theenumi}{(\arabic{enumi})}

%\usepackage{graphicx}

%\usepackage{tikz}
%\usetikzlibrary {arrows.meta}
%\usepackage{wrapfig}	% required for `\wrapfigure' (yatex added)
%\usepackage{bm}	% required for `\bm' (yatex added)

% ルビを振る
%\usepackage{luatexja-ruby}	% required for `\ruby'

%% 核Ker 像Im Hom を定義
%\newcommand{\Img}{\mathop{\mathrm{Im}}\nolimits}
%\newcommand{\Ker}{\mathop{\mathrm{Ker}}\nolimits}
%\newcommand{\Hom}{\mathop{\mathrm{Hom}}\nolimits}

%\DeclareMathOperator{\Rot}{rot}
%\DeclareMathOperator{\Div}{div}
%\DeclareMathOperator{\Grad}{grad}
%\DeclareMathOperator{\arcsinh}{arcsinh}
%\DeclareMathOperator{\arccosh}{arccosh}
%\DeclareMathOperator{\arctanh}{arctanh}



\begin{document}


\begin{enumerate}
 \item 数列$\{a_n\}$の初項から第$n$項までの和を$S_n$とする。
       $S_n$は次を満たす。
       \begin{equation}
        S_1=1
         ,\qquad
        S_{n+1}-3S_n = n+1 \ (n\geq 1)
       \end{equation}
       \begin{enumerate}
        \item $S_n$を求めよ。
        \item $a_n$を求めよ。
       \end{enumerate}
 \item 数列$\{a_n\}$が次を満たす。
       \begin{equation}
        a_1=1,\qquad \sum_{k=1}^n ka_k = n^2a_n \ (n\geq 1)
       \end{equation}
       \begin{enumerate}
        \item $a_n$を$a_{n-1} \ (n\geq 2)$で表せ。
        \item $a_n$を求めよ。
       \end{enumerate}
\end{enumerate}

\hrulefill


\begin{enumerate}

 \item % 数列$\{a_n\}$の初項から第$n$項までの和を$S_n$とする。
       % $S_n$は次を満たす。
       % \begin{equation}
       %  S_1=1
       %   ,\qquad
       %  S_{n+1}-3S_n = n+1 \ (n\geq 1)
       % \end{equation}

       \begin{enumerate}
        \item % $S_n$を求めよ。

              $S_{n+1}-3S_n = n+1$より
              $S_{n+2}-3S_{n+1} = n+2$である。
              この2つの式の差を求める。
              \begin{gather}
               (S_{n+2}-3S_{n+1})-(S_{n+1}-3S_{n}) = (n+2)-(n+1)\\
               S_{n+2}-S_{n+1} = 3(S_{n+1} - S_{n}) + 1
              \end{gather}

              $b_n = S_{n+1}-S_{n}$と置くと
              $b_{n+1}=3b_{n}+1$が得られるので、
              $b_n$の一般項を求める。
              \begin{align}
               b_{n+1} =& 3b_{n}+1\\
               b_{n+1}+\frac{1}{2} =& 3 \left( b_{n}+\frac{1}{2} \right)
              \end{align}

              $b_n$の初項は$b_1=S_{2}-S_{1}=5-1=4$であるので、
              数列$\{ b_{n}+\frac{1}{2} \}$の初項は$\frac{9}{2}$である。
              一般項は次のようになる。
              \begin{equation}
               b_{n} + \frac{1}{2} = \frac{9}{2} \cdot 3^{n-1} = \frac{3^{n+1}}{2}
              \end{equation}

              $b_n = \frac{3^{n+1}}{2} - \frac{1}{2}$は
              $S_n$の階差数列である。
              この為、$S_n$の一般項は次のように求まる。
              \begin{align}
               S_n =& S_1 + \sum_{k=1}^{n-1}b_k\\
                =& 1 + \frac{9}{2}\sum_{k=1}^{n-1}3^{k-1} - \sum_{k=1}^{n-1}\frac{1}{2}\\
                =& 1 + \frac{9}{2}\cdot \frac{(3^{n-1}-1)}{3-1} - \frac{n-1}{2}\\
                =& \frac{1}{4} (4 + 3^{n+1} -9 -2(n-1))\\
                =& \frac{1}{4} (3^{n+1} -2n-3)
              \end{align}


              \dotfill

              \textbf{[ 別解 ]}

              $S_{n+1}-3S_n = n+1$を
              $S_{n+1} +a(n+1) +b = 3(S_n +an+b)$となるように
              $a,b$を求める。
              \begin{gather}
               S_{n+1} +a(n+1) +b = 3(S_n +an+b)\\
               %S_{n+1} +an+a +b = 3S_n +3an+3b\\
               S_{n+1} - 3S_n
               %= 3an+3b - an -a -b
               = 2an + (2b-a)
              \end{gather}

              これにより$a=\frac{1}{2},\ b=\frac{3}{4}$となる。

              $c_n = S_n +\frac{1}{2}n+\frac{3}{4}$とすると、
              $c_{n+1}=3c_n$である。
              $c_1=1+\frac{1}{2}+\frac{3}{4}=\frac{9}{4}$より
              $c_n$の一般項は次のようになる。
              \begin{equation}
               c_n = \frac{9}{4}\cdot 3^{n-1}
              \end{equation}

              $c_n = S_n +\frac{1}{2}n+\frac{3}{4}$であるので、
              $S_n$を求めると次のようになる。
              \begin{align}
               S_n =& \frac{9}{4}\cdot 3^{n-1} -\frac{1}{2}n -\frac{3}{4}\\
               =& \frac{1}{4}( 3^{n+1} - 2n-3 )
              \end{align}



              \dotfill

        \item % $a_n$を求めよ。

              $S_n$は初項から第$n$項までの和であるので、
              $a_n=S_n-S_{n-1}$である。

              \begin{align}
               a_n =& S_n - S_{n-1}\\
               =& \left( \frac{1}{4}( 3^{n+1} - 2n-3 ) \right)
               - \left( \frac{1}{4}( 3^{n} - 2(n-1)-3 ) \right)\\
               =& \frac{1}{4} \left( 2\cdot 3^n -2 \right)
               = \frac{1}{2} \left( 3^n -1 \right)
              \end{align}




              \dotfill

       \end{enumerate}

 \item % 数列$\{a_n\}$が次を満たす。
       % \begin{equation}
       %  a_1=1,\qquad \sum_{k=1}^n ka_k = n^2a_n \ (n\geq 1)
       % \end{equation}

       \begin{enumerate}
        \item % $a_n$を$a_{n-1} \ (n\geq 2)$で表せ。


              $\sum_{k=1}^n ka_k = n^2a_n$より
              $\sum_{k=1}^{n-1} ka_k = (n-1)^2a_{n-1} \ (n\geq 2)$
              である。

              これにより次の式が得られる。
              \begin{align}
               \sum_{k=1}^n ka_k =& \sum_{k=1}^{n-1} ka_k + na_n\\
               =& (n-1)^2a_{n-1} + na_n
              \end{align}

              条件の式より
              \begin{equation}
               n^2a_n = (n-1)^2a_{n-1} + na_n
              \end{equation}
              であるので、変形をして次の式が得られる。
              \begin{equation}
               a_n = \frac{n-1}{n}a_{n-1}
              \end{equation}

              \dotfill

        \item % $a_n$を求めよ。

              $a_n = \frac{n-1}{n}a_{n-1}$より
              $na_n=(n-1)a_{n-1}$である。
              これを繰り返すと次のような式が得られる。
              \begin{equation}
               na_n=(n-1)a_{n-1}
                =(n-2)a_{n-2}
                = \cdots
                =2a_2 =1a_1
              \end{equation}

              初項が$a_1=1$であるので、上記式は$1$である。
              $na_n=1$であるので
              \begin{equation}
               a_n=\frac{1}{n}
              \end{equation}
              である。
       \end{enumerate}
\end{enumerate}

\hrulefill


\end{document}

