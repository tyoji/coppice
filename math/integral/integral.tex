\documentclass[12pt,b5paper]{ltjsarticle}

%\usepackage[margin=15truemm, top=5truemm, bottom=5truemm]{geometry}
%\usepackage[margin=10truemm,left=15truemm]{geometry}
\usepackage[margin=10truemm]{geometry}

\usepackage{amsmath,amssymb}
%\pagestyle{headings}
\pagestyle{empty}

%\usepackage{listings,url}
%\renewcommand{\theenumi}{(\arabic{enumi})}

%\usepackage{graphicx}

%\usepackage{tikz}
%\usetikzlibrary {arrows.meta}
%\usepackage{wrapfig}
%\usepackage{bm}

% ルビを振る
%\usepackage{luatexja-ruby}

%% 核Ker 像Im Hom を定義
%\newcommand{\Img}{\mathop{\mathrm{Im}}\nolimits}
%\newcommand{\Ker}{\mathop{\mathrm{Ker}}\nolimits}
%\newcommand{\Hom}{\mathop{\mathrm{Hom}}\nolimits}

%\DeclareMathOperator{\Rot}{rot}
%\DeclareMathOperator{\Div}{div}
%\DeclareMathOperator{\Grad}{grad}
%\DeclareMathOperator{\arcsinh}{arcsinh}
%\DeclareMathOperator{\arccosh}{arccosh}
%\DeclareMathOperator{\arctanh}{arctanh}

%\usepackage{url}

%\usepackage{listings}
%
%\lstset{
%%プログラム言語(複数の言語に対応,C,C++も可)
%  language = Python,
%%  language = Lisp,
%%  language = C,
%  %背景色と透過度
%  %backgroundcolor={\color[gray]{.90}},
%  %枠外に行った時の自動改行
%  breaklines = true,
%  %自動改行後のインデント量(デフォルトでは20[pt])
%  breakindent = 10pt,
%  %標準の書体
%%  basicstyle = \ttfamily\scriptsize,
%  basicstyle = \ttfamily,
%  %コメントの書体
%%  commentstyle = {\itshape \color[cmyk]{1,0.4,1,0}},
%  %関数名等の色の設定
%  classoffset = 0,
%  %キーワード(int, ifなど)の書体
%%  keywordstyle = {\bfseries \color[cmyk]{0,1,0,0}},
%  %表示する文字の書体
%  %stringstyle = {\ttfamily \color[rgb]{0,0,1}},
%  %枠 "t"は上に線を記載, "T"は上に二重線を記載
%  %他オプション:leftline,topline,bottomline,lines,single,shadowbox
%  frame = TBrl,
%  %frameまでの間隔(行番号とプログラムの間)
%  framesep = 5pt,
%  %行番号の位置
%  numbers = left,
%  %行番号の間隔
%  stepnumber = 1,
%  %行番号の書体
%%  numberstyle = \tiny,
%  %タブの大きさ
%  tabsize = 4,
%  %キャプションの場所("tb"ならば上下両方に記載)
%  captionpos = t
%}

%\usepackage{cancel}
%\usepackage{bussproofs}
%\usepackage{proof}

\begin{document}

\begin{equation}
 \int_{0}^{1} x\sqrt{1-x} \; \mathrm{d}x
\end{equation}

\hrulefill


$t=1-x$と置く。
$t$を$x$で微分すると$\mathrm{d}t/\mathrm{d}x=-1$
であるので、
$\mathrm{d}t=-\mathrm{d}x$
が得られる。
$x$が0から1に変化すると
$t$は1から0に変化する。
そこで、変数を置換すると次の式が得られる。
\begin{equation}
 \int_{0}^{1} x\sqrt{1-x} \; \mathrm{d}x
  =
  \int_{1}^{0} (1-t)\sqrt{t} \; (-\mathrm{d}t)
\end{equation}


\begin{align}
 \int_{1}^{0} (1-t)\sqrt{t} \; (-\mathrm{d}t)
  &=
  \int_{0}^{1} (t^{\frac{1}{2}}-t^{\frac{3}{2}}) \; \mathrm{d}t
  =
 \left[ \frac{2}{3}t^{\frac{3}{2}} - \frac{2}{5}t^{\frac{5}{2}}\right]_{t=0}^{t=1}\\
 &=
 \left( \frac{2}{3}\times 1^{\frac{3}{2}} - \frac{2}{5}\times 1^{\frac{5}{2}} \right)
 - \left( \frac{2}{3}\times 0^{\frac{3}{2}} - \frac{2}{5}\times 0^{\frac{5}{2}} \right)\\
 &= \frac{4}{15}
\end{align}

\dotfill

\textbf{不定積分を先に行う場合}

$t=1-x$より$\mathrm{d}t=-\mathrm{d}x$となる。
\begin{align}
 \int x\sqrt{1-x} \; \mathrm{d}x
  &= \int (1-t)\sqrt{t} \; (-\mathrm{d}t)\\
  &= -\int (t^{\frac{1}{2}}-t^{\frac{3}{2}}) \; \mathrm{d}t\\
  &= -\frac{2}{3}t^{\frac{3}{2}} + \frac{2}{5}t^{\frac{5}{2}} + C\\
  &= -\frac{2}{3}(1-x)^{\frac{3}{2}} + \frac{2}{5}(1-x)^{\frac{5}{2}} + C
\end{align}

上の式より定積分を計算する。
\begin{align}
 \int_{0}^{1} x\sqrt{1-x} \; \mathrm{d}x
  &= \left[ -\frac{2}{3}(1-x)^{\frac{3}{2}} + \frac{2}{5}(1-x)^{\frac{5}{2}} \right]_{x=0}^{x=1}\\
 &= \frac{4}{15}
\end{align}


\hrulefill


\textbf{注意事項}

定積分と不定積分は異なる計算であるため、
等号で結ぶことは出来ない。
\begin{equation}
 \int_{0}^{1} x\sqrt{1-x} \; \mathrm{d}x
  \ne
  \int (1-t)\sqrt{t} \; \mathrm{d}t
\end{equation}



\newpage


\begin{equation}
 \lim_{n\to\infty} \frac{1}{n} \sum_{k=1}^{n} \left( \frac{k}{n}+\frac{1}{n} \right)
  \ne
  \int_{0}^{1}(x+0) \; \mathrm{d}x
\end{equation}


\hrulefill

\begin{equation}
 \frac{1}{n} \sum_{k=1}^{n} \left( \frac{k}{n}+\frac{1}{n} \right)
  =
  \frac{1}{n} \left( \sum_{k=1}^{n}\frac{k}{n} + \sum_{k=1}^{n}\frac{1}{n} \right)
  =
  \frac{1}{n} \left( \frac{n+1}{2} + 1 \right)
  =
  \frac{n+3}{2n}
\end{equation}

\begin{equation}
 \lim_{n\to\infty} \frac{1}{n} \sum_{k=1}^{n} \left( \frac{k}{n}+\frac{1}{n} \right)
  =
  \lim_{n\to\infty} \frac{n+3}{2n} =\frac{1}{2}
\end{equation}


\end{document}
