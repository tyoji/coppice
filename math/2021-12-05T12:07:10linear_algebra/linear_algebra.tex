\documentclass[10pt,b5paper]{ltjsarticle}

\usepackage[margin=15truemm, top=5truemm, bottom=5truemm]{geometry}

\usepackage{amsmath,amssymb}
\pagestyle{empty}

\usepackage{bm}

\usepackage{listings,url}


\begin{document}



\begin{enumerate}
 \item %\hrulefill
       行列$A$の固有値を求めよ。
       \begin{equation}
        A=\left(
           \begin{matrix}
            -5 & 0 & 0 & 3\\
            3 & -2 & 0 & -3\\
            0 & 0 & -2 & 0\\
            -6 & 0 & 0 & 4
           \end{matrix}
         \right)
       \end{equation}

       \dotfill

       固有値$\lambda$は固有方程式$\lvert A-\lambda E \rvert =0$を解くことで求まる。

       \dotfill

       \begin{align}
        \lvert A-\lambda E \rvert &= 0\\
        \begin{vmatrix}
            -5 -\lambda & 0 & 0 & 3\\
            3 & -2 -\lambda & 0 & -3\\
            0 & 0 & -2 -\lambda & 0\\
            -6 & 0 & 0 & 4 -\lambda
        \end{vmatrix}
        &=0\\
        (-5-\lambda)
        \begin{vmatrix}
            -2 -\lambda & 0 & -3\\
            0 & -2 -\lambda & 0\\
            0 & 0 & 4 -\lambda
        \end{vmatrix}
        -3
        \begin{vmatrix}
            3 & -2 -\lambda & 0\\
            0 & 0 & -2 -\lambda\\
            -6 & 0 & 0
        \end{vmatrix}
        &=0\\
        (-5-\lambda)(-2-\lambda)^2(4-\lambda)&=0
       \end{align}
       \underline{固有値は $-5, -2, 4$}

       \hrulefill
 \item 行列$A$が対角化可能か判定し、
       対角化可能であれば正則行列$P$を求め$A$を対角化せよ。
       \begin{equation}
        A=\left(
           \begin{matrix}
            -2 & 1 & 0\\
            0 & -2 & 1\\
            1 & -3 & 1
           \end{matrix}
         \right)
       \end{equation}

       \dotfill

       まず、固有値を求める。
       固有方程式$\lvert A-\lambda E \rvert =0$を計算すると
       $(\lambda+1)^3=0$で$\lambda = -1$となる。

       固有値$-1$の固有ベクトル$\bm{x}$を求める為に、
       $(A+E)\bm{x}=\bm{0}$を計算する。
       \begin{gather}
           \bm{x}=\left(\begin{matrix}x_1\\x_2\\x_3\end{matrix}\right)
            \quad
           \left(\begin{matrix}-1&1&0\\0&-1&1\\1&-3&2\end{matrix}\right)
           \left(\begin{matrix}x_1\\x_2\\x_3\end{matrix}\right)
           = \left(\begin{matrix}0\\0\\0\end{matrix}\right)\\
           \begin{cases}
            -x_1 + x_2 =0\\
            -x_2+x_3=0\\
            x_1-3x_2+2x_3=0
           \end{cases}
       \end{gather}
       重複度$3$だが、固有ベクトルは
       $\bm{x}=k\left(\begin{matrix}1\\1\\1\end{matrix}\right), k\ne0$
       と1次元になるので、行列$A$は対角化不可能である。

       \hrulefill
 \item 実対称行列$A$を直交行列によって対角化せよ。
       \begin{equation}
        A=\left(
           \begin{matrix}
            2 & 1 & 1\\
            1 & 2 & 1\\
            1 & 1 & 2
           \end{matrix}
          \right)
       \end{equation}

       \dotfill

       固有値$\lambda$を求めるために
       固有方程式$\lvert A-\lambda E \rvert =0$を計算する。
       \begin{align}
        \begin{vmatrix}
         2-\lambda & 1 & 1\\
         1 & 2-\lambda & 1\\
         1 & 1 & 2-\lambda
        \end{vmatrix}
        =0\\
        (2-\lambda)^3+2-3(2-\lambda)=0\\
        (\lambda-1)^2(\lambda-4)=0
       \end{align}
       固有値は$1, 4$である。

       固有ベクトル$\bm{x}={^t(x_1,x_2,x_3)}$を求める。
       
       固有値$4$の場合
       \begin{gather}
        \begin{pmatrix}
         2-4 & 1 & 1\\
         1 & 2-4 & 1\\
         1 & 1 & 2-4
        \end{pmatrix}
        \begin{pmatrix}
         x_1\\x_2\\x_3
        \end{pmatrix}
        =0\\
        \begin{cases}
         -2x_1+x_2+x_3=0\\
         x_1-2x_2+x_3=0\\
         x_1+x_2-2x_3=0\\
        \end{cases}\\
        \bm{x}=k_3\begin{pmatrix}1\\1\\1\end{pmatrix}
        \quad k_3\ne0
       \end{gather}

       固有値$1$の場合
       \begin{gather}
        \begin{pmatrix}
         2-1 & 1 & 1\\
         1 & 2-1 & 1\\
         1 & 1 & 2-1
        \end{pmatrix}
        \begin{pmatrix}
         x_1\\x_2\\x_3
        \end{pmatrix}
        =0\\
        x_1+x_2+x_3=0 \quad より \quad
        x_3=-x_1-x_2\\
        \bm{x}=k_1\begin{pmatrix}1\\0\\-1\end{pmatrix}
        +k_2\begin{pmatrix}0\\1\\-1\end{pmatrix}
        \quad (k_1,k_2)\ne(0,0)
       \end{gather}

       固有値$4$の固有空間は1次元なので、
       基底を正規化すると $\frac{1}{\sqrt{3}}{}^t(1, 1, 1)$

       固有値$1$の固有空間は2次元なので、
       シュミットの直交化法で正規直交基底を得る。
       $\frac{1}{\sqrt{2}}{}^t(1, 0, -1)$と$\frac{1}{\sqrt{6}}{}^t(-1, 2, -1)$

       これを用いて次のように直交行列$T$を作る。
       \begin{equation}
        T=
        \begin{pmatrix}
         \frac{1}{\sqrt{3}} & \frac{1}{\sqrt{2}} & -\frac{1}{\sqrt{6}}\\
         \frac{1}{\sqrt{3}} & 0 &\frac{2}{\sqrt{6}}\\
         \frac{1}{\sqrt{3}} & -\frac{1}{\sqrt{2}} & -\frac{1}{\sqrt{6}}
        \end{pmatrix}
       \end{equation}

       直交行列$T$を利用し、対称行列$A$を対角化すると次のようになる。
       \begin{equation}
        {}^tTAT =
         \begin{pmatrix}
          4 & 0 & 0\\
          0 & 1 & 0\\
          0 & 0 & 1
         \end{pmatrix}
       \end{equation}
       
       \hrulefill
 \item $n$次行列$A$の固有値を$\alpha$とする。
       $\alpha$に対する固有空間$W_{\alpha}$は
       $\mathbb{C}^n$の部分空間であることを示せ。

       \dotfill

       固有値を求める固有方程式$\lvert A - \lambda E\rvert =0$は
       変数$\lambda$の$n$次方程式である。
       複素数$\mathbb{C}$の範囲で考えれば$n$次方程式は
       $n$個の1次式の積に分解できる。(代数学の基本定理)

       固有値は重複度を含め$n$個になり、
       一つの固有値$\alpha$の重複度は$n$を超えない。
       この為、$\alpha$の固有空間$W_{\alpha}$の次元は
       $\alpha$の重複度を超えないので、
       $n$以下になり、$W_{\alpha} \subset \mathbb{C}^n$であることが分かる。
\end{enumerate}


\hrulefill

行列の計算確認 -sagemath-

\url{https://sagecell.sagemath.org/}

\begin{lstlisting}[language=Python,basicstyle={\small},frame=single,numbers=left]
A=Matrix([[2,1,1],[1,2,1],[1,1,2]])
B=Matrix(
    [[1/sqrt(3), 1/sqrt(2),-1/sqrt(6)],
     [1/sqrt(3),         0, 2/sqrt(6)],
     [1/sqrt(3),-1/sqrt(2),-1/sqrt(6)]]
)
C=B.transpose()
print("---")
print(C*B)
print("---")
print(C*A*B)
print("---")
\end{lstlisting}


\end{document}
