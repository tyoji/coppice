\documentclass[12pt,b5paper]{ltjsarticle}

%\usepackage[margin=15truemm, top=5truemm, bottom=5truemm]{geometry}
\usepackage[margin=15truemm]{geometry}

\usepackage{amsmath,amssymb}
%\pagestyle{headings}
\pagestyle{empty}

%\usepackage{listings,url}
%\renewcommand{\theenumi}{(\arabic{enumi})}

\usepackage{graphicx}

\usepackage{tikz}
\usetikzlibrary {arrows.meta}
\usepackage{wrapfig}	% required for `\wrapfigure' (yatex added)
\usepackage{bm}	% required for `\bm' (yatex added)
\usepackage{luatexja-ruby}	% required for `\ruby'
%% 像Im を定義
%\newcommand{\Img}{\mathop{\mathrm{Im}}\nolimits}


\begin{document}



\textbf{同値関係}

推移律、反射律、対称律を満たす関係を同値関係という。

$\sim$:同値関係
\begin{itemize}
 \item 反射律 $A\sim A$
 \item 対称率 $A\sim B \Rightarrow B\sim A$
 \item 推移率 $A\sim B , \, B\sim C \Rightarrow A\sim C$
\end{itemize}


\textbf{商集合}

集合$S$に同値関係$\sim$が与えられている場合、
商集合$S/\sim$が存在する。

商集合の元を同値類という。

商集合$S/\sim$は同値関係にある$S$の元を同じものとみなしてつくる集合である。



\hrulefill

\begin{enumerate}
 \item
      $n$:正の整数、$S=\{1,\dots,n\}$、$P(S)$:$S$の全ての部分集合

      $P(S)$上の同値関係$\sim$を次のように定義する。
      \begin{equation}
       A \sim B \stackrel{\mathrm{def}}{\Leftrightarrow}
        \# A = \# B
      \end{equation}
      この時、$\#P(S)/\sim$を求めよ。

      \dotfill

      同値関係$\sim$は部分集合の元の数が同じもので定義されている。
      元の個数により$n+1$種類に分けられる。
      $\#A = 0$なら$A=\emptyset$、
      $\#A = 1$なら$A=\{1\}, A=\{2\}, A=\{n\}$等、
      $\dots$
      $\#A = n$なら$A=S$である。


      これらがそれぞれ同値類を作るため
      $\#P(S)/\sim = n+1$である。

      \hrulefill

 \item
      $K$:体、$V$:$K$-線形空間、$W$:$V$の部分空間

      $v_1,\dots v_n\in V$に対していかが同値であることを示せ。
       \begin{enumerate}
        \item \label{vw}
             $[v_1],\dots,[v_n]\in V/W$は一次独立
        \item \label{w}
             任意の
             $a_1,\dots,a_n\in K$に対し
             \begin{equation}
              a_1v_1+\cdots+a_nv_n \in W
               \Rightarrow a_1=\cdots = a_n=0
             \end{equation}
       \end{enumerate}

      \dotfill

      \ref{vw} $\Rightarrow$ \ref{w}

      $[v_1],\dots,[v_n]\in V/W$は一次独立であるので、
      $a_i\in K \, (i=1,\dots n)$について
      \begin{equation}
       a_1[v_1]+\cdots + a_n[v_n]=0 \Rightarrow a_1=\cdots = a_n=0\label{def_lin}
      \end{equation}

      $a_1v_1+\cdots+a_nv_n \in W$であれば、
      $V/W$の同値類$[a_1v_1+\cdots+a_nv_n]$は$[0]$と等しい。
      つまり、
      \begin{equation}
       [a_1v_1+\cdots+a_nv_n] = a_1[v_1]+\cdots + a_n[v_n] =0
      \end{equation}
      であるので、
%      (\ref{def_lin})より
      $[v_i]$が一次独立であることから
      $a_1=\cdots =a_n=0$となる。

      \ref{vw} $\Leftarrow$ \ref{w}

      上の議論を逆にたどると結論が得られる。

      \begin{equation}
       a_1v_1+\cdots+a_nv_n \in W
        \Rightarrow a_1=\cdots = a_n=0
      \end{equation}

      $a_1v_1+\cdots+a_nv_n \in W$より$V/W$の同値類$[a_1v_1+\cdots+a_nv_n]$
      は$[a_1v_1+\cdots+a_nv_n]=[0]=0$である。
      \begin{equation}
       [a_1v_1+\cdots+a_nv_n] = a_1[v_1]+\cdots + a_n[v_n] =0
      \end{equation}
      であるが、
      $a_1v_1+\cdots+a_nv_n \in W$ならば$a_1=\cdots =a_n=0$であるので、
      $a_1[v_1]+\cdots + a_n[v_n] =0$ならば$a_1=\cdots =a_n=0$である。

      つまり、
      $[v_1],\dots,[v_n]$は一次独立である。




\end{enumerate}




\end{document}
