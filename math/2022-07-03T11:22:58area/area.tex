\documentclass[12pt,b5paper]{ltjsarticle}

%\usepackage[margin=15truemm, top=5truemm, bottom=5truemm]{geometry}
\usepackage[margin=15truemm]{geometry}

\usepackage{amsmath,amssymb}
%\pagestyle{headings}
\pagestyle{empty}

%\usepackage{listings,url}
%\renewcommand{\theenumi}{(\arabic{enumi})}

\usepackage{graphicx}

\usepackage{tikz}
\usetikzlibrary {arrows.meta}
\usepackage{wrapfig}	% required for `\wrapfigure' (yatex added)
\usepackage{bm}	% required for `\bm' (yatex added)

% ルビを振る
%\usepackage{luatexja-ruby}	% required for `\ruby'

%% 核Ker 像Im Hom を定義
%\newcommand{\Img}{\mathop{\mathrm{Im}}\nolimits}
%\newcommand{\Ker}{\mathop{\mathrm{Ker}}\nolimits}
%\newcommand{\Hom}{\mathop{\mathrm{Hom}}\nolimits}

\begin{document}

\hrulefill

$\mathbb{R}^2$上の定置関数を次のように定める。
\begin{equation}
 1 : \mathbb{R}^2 \to \{1\} , \quad (x,y) \mapsto 1
\end{equation}

有界集合$D\subset\mathbb{R}^2$上の関数$1$が積分可能である時、
$D$を面積確定であるという。
また、$\int_D1\mathrm{d}x\mathrm{d}y$を$D$の面積と言う。
$D$の面積を$\mathrm{Area}(D)$と書く。

\begin{enumerate}
 \item
      積分定理が使えるような2次元領域$D$は面積確定であることを示せ。
 \item
      $\mathbb{R}^2$上の3つの$C^1$-級ベクトル場
      $\bm{f}_1,\,\bm{f}_2,\,\bm{f}_3:\mathbb{R}^2\to\mathbb{R}^2$を
      次のように定める。
      \begin{equation}
       \bm{f}_1(x,y)=\frac{1}{2}\begin{pmatrix}-y\\x\end{pmatrix}
       , \
       \bm{f}_2(x,y)=\begin{pmatrix}0\\x\end{pmatrix}
       , \
       \bm{f}_3(x,y)=\begin{pmatrix}-y\\0\end{pmatrix}
      \end{equation}
      積分定理が使えるような2次元領域に対し、
      以下が成り立つことを示せ。
      \begin{equation}
       \mathrm{Area}(D)
        = \int_{\partial D}\bm{f}_1
        = \int_{\partial D}\bm{f}_2
        = \int_{\partial D}\bm{f}_3
      \end{equation}
\end{enumerate}

\dotfill

\begin{enumerate}
 \item
      領域$D$はある集合$I$に含まれていて無限に広がっておらず、
      縦線集合又は横線集合であるため、積分する順序が決まっている。

      $D$の境界$\partial D$は有限個曲線$C_i$を繋いだものである。

      これにより、$D$は境界に囲まれた領域であり、
      この領域は2変数の一方を固定すれば直線で表せる。
      この順序で積分することで面積を求めることが出来る。

 \item

      \begin{equation}
       \frac{1}{2}\begin{pmatrix}-y\\x\end{pmatrix}
       =
        \frac{1}{2}
       \begin{pmatrix}0\\x\end{pmatrix}
       +
        \frac{1}{2}
       \begin{pmatrix}-y\\0\end{pmatrix}
%      \end{equation}
      \text{ より }
%      \begin{equation}
       \bm{f}_1 = \frac{1}{2}\bm{f}_2 + \frac{1}{2}\bm{f}_3
      \end{equation}

      これにより次の式が成り立つ。
      \begin{equation}
       \int_{\partial D}\bm{f}_1
        =
        \int_{\partial D}\left(\frac{1}{2}\bm{f}_2 + \frac{1}{2}\bm{f}_3\right)
        = \frac{1}{2}\int_{\partial D}\bm{f}_2
        + \frac{1}{2}\int_{\partial D}\bm{f}_3
        \label{area_d}
      \end{equation}




      積分定理が式(\ref{Green's theorem})を満たす定理であるなら、
      $\bm{f}_2=\begin{pmatrix}0\\x\end{pmatrix}=\begin{pmatrix}P(x,y)\\Q(x,y)\end{pmatrix}$
      とすると次の式が得られる。
      \begin{equation}
       \iint_D \left( 1-0 \right)\mathrm{d}x\mathrm{d}y
        = \int_{\partial D}\left( 0\mathrm{d}x + x\mathrm{d}y \right)
        = \int_{\partial D}
           \begin{pmatrix}0\\x\end{pmatrix}\cdot\begin{pmatrix}\mathrm{d}x\\\mathrm{d}y\end{pmatrix}
      \end{equation}

      これにより$\mathrm{Area}(D)=\int_{\partial D}\bm{f}_2$が得られる。
      同様にして$\mathrm{Area}(D)=\int_{\partial D}\bm{f}_3$である。

      これを式(\ref{area_d})に当てはめると
      \begin{equation}
       \int_{\partial D}\bm{f}_1
        = \frac{1}{2}\int_{\partial D}\bm{f}_2
        + \frac{1}{2}\int_{\partial D}\bm{f}_3
        = \frac{1}{2}\int_{\partial D}\mathrm{Area}(D)
        + \frac{1}{2}\int_{\partial D}\mathrm{Area}(D)
        = \mathrm{Area}(D)
      \end{equation}
      が得られる。

      よって次の式が成り立つ。
      \begin{equation}
       \mathrm{Area}(D)
        = \int_{\partial D}\bm{f}_1
        = \int_{\partial D}\bm{f}_2
        = \int_{\partial D}\bm{f}_3
      \end{equation}

\end{enumerate}






\hrulefill

\textbf{グリーンの定理}

$D$を有界閉領域、
$P(x,y),\,Q(x,y)$を$C^1$-級関数の時、
次の式が成り立つ。
\begin{equation}
 \iint_D \left( \frac{\partial Q(x,y)}{\partial x}
          -\frac{\partial P(x,y)}{\partial y} \right)\mathrm{d}x\mathrm{d}y
 = \int_{\partial D}\left( P(x,y)\mathrm{d}x + Q(x,y)\mathrm{d}y \right)
 \label{Green's theorem}
\end{equation}


\end{document}
