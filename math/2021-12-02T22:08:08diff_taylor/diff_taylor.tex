\documentclass[12pt,b5paper]{ltjsarticle}

\usepackage[margin=15truemm, top=5truemm, bottom=0truemm]{geometry}

\usepackage{amsmath}
\usepackage{amssymb}
\pagestyle{empty}


\begin{document}



\begin{enumerate}
 \item %\hrulefill

       $\displaystyle
       f(z) := \frac{1}{1-2z}
       =\sum_{n=0}^{\infty}2^nz^n
       \quad (\lvert z \rvert < \frac{1}{2})$
       である。

       \hrulefill
       \begin{enumerate}\renewcommand{\theenumii}{\arabic{enumii}}
        \item %\hrulefill

             $\displaystyle
                \frac{1}{(1-2z)^2}
                = \sum_{n=0}^{\infty}a_nz^n$
              とマクローリン展開される係数$a_n$を求めよ。

              \dotfill

              $f(z)$の微分を考える。

              $f(z)=\frac{1}{1-2z}$の場合。
              \begin{equation}
               \frac{d}{dz}\left( \frac{1}{1-2z} \right)
                = -\frac{-2}{(1-2z)^2}
                = \frac{2}{(1-2z)^2}
              \end{equation}
              $f(z)=\sum_{n=0}^{\infty}2^nz^n$の場合、
              $\rvert z \lvert < \frac{1}{2}$において収束する為、
              項別に微分を行う。定数項は無くなる。
              \begin{equation}
               \frac{d}{dz}\left( \sum_{n=0}^{\infty}2^nz^n \right)
                = \sum_{n=1}^{\infty}2^nnz^{n-1}
              \end{equation}
              これらは$\rvert z \lvert < \frac{1}{2}$において等しいので
              \begin{align}
                \frac{2}{(1-2z)^2}
               =& \sum_{n=1}^{\infty}2^nnz^{n-1}\\
                \frac{1}{(1-2z)^2}
               =& \sum_{n=1}^{\infty}2^{n-1}nz^{n-1}
               = \sum_{n=0}^{\infty}2^{n}(n+1)z^{n}
              \end{align}
%
              以上より \underline{$a_n=2^n(n+1)$}
              
              \hrulefill
        \item %\hrulefill

              $\displaystyle
                \frac{1}{(1-2z)^3}
                = \sum_{n=0}^{\infty}b_nz^n$
              とマクローリン展開される係数$b_n$を求めよ。

              \dotfill

              先程の式
              $\frac{1}{(1-2z)^2}
              = \sum_{n=0}^{\infty}2^{n}(n+1)z^{n}$
              を微分する。
              \begin{align}
                \frac{d}{dz}\left(\frac{1}{(1-2z)^2}\right)
               =& -\frac{2\cdot(-2)}{(1-2z)^3}
               = \frac{2^2}{(1-2z)^3}\\
               %
               \frac{d}{dz}\left(\sum_{n=0}^{\infty}2^{n}(n+1)z^{n}\right)
               =& \sum_{n=1}^{\infty}2^{n}(n+1)nz^{n-1}\\
               =& \sum_{n=0}^{\infty}2^{n+1}(n+2)(n+1)z^{n}
              \end{align}
              $\frac{2^2}{(1-2z)^3} = \sum_{n=0}^{\infty}2^{n+1}(n+2)(n+1)z^{n}$より
              \underline{$b_n=2^{n-1}(n+2)(n+1)$}
       \end{enumerate}
       \hrulefill
 \item %\hrulefill

       $\displaystyle
       g(z) := \frac{z}{(1-z^3)^2}
       =\sum_{n=0}^{\infty}c_nz^n$
       とマクローリン展開される係数$c_n$
       を求めよ。

       \dotfill

       $\rvert z-\alpha \lvert <R$で正則な関数$t(z)$のテイラー展開
       \begin{equation}
        t(z)=\sum_{k=0}^{\infty}c_k(z-\alpha)^k, \quad c_k=\frac{t^{(k)}(\alpha)}{k!}
       \end{equation}
       $\alpha=0$の時をマクローリン展開と言う。
       \begin{equation}
        t(z)=\sum_{k=0}^{\infty}c_{k}z^k, \quad c_k=\frac{t^{(k)}(0)}{k!}
       \end{equation}

       \dotfill

       $\bar{g}(z)=(1-z^3)^{-1}$とする。
       $\rvert z \lvert <1$において
       $(1-z)^{-1}=\sum_{k=0}^{\infty}z^k$であるので、
       同様に考えると$\rvert z \lvert <1$において
       \begin{equation}
        \bar{g}(z)=\frac{1}{1-z^3} = \sum_{n=0}^{\infty}(z^3)^n
       \end{equation}
       である。
       $\bar{g}(z)=\frac{1}{1-z^3}$を微分する。
       \begin{equation}
        \frac{d}{dz}\bar{g}(z)=-\frac{-3z^2}{(1-z^3)^2}=3z\cdot\frac{z}{(1-z^3)^2} =3z\cdot g(z)
       \end{equation}
       $\bar{g}(z) = \sum_{n=0}^{\infty}z^{3n}$を微分する。
       \begin{align}
        \frac{d}{dz}\bar{g}(z) = \sum_{n=1}^{\infty}3nz^{3n-1}
       \end{align}
       $3z\cdot g(z)=\sum_{n=1}^{\infty}3nz^{3n-1}$より
       \begin{align}
        g(z) =& (3z)^{-1}\sum_{n=1}^{\infty}3nz^{3n-1}\\
        =& \sum_{n=1}^{\infty}nz^{3n-2}
        \quad
        = \sum_{n=0}^{\infty}(n+1)z^{3n+1}\\
        = & z + 2z^4+3z^7+4z^{10}+5z^{13}+6z^{16}+ \cdots
       \end{align}
       となるので、
        \begin{equation}
         c_n =
         \begin{cases}
          \frac{1}{3}(n+2) &\quad n\equiv 1 \mod 3\\
          0 &\quad n\not\equiv 1 \mod 3
         \end{cases}
        \end{equation}
\end{enumerate}

\end{document}
