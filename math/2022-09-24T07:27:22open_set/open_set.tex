\documentclass[12pt,b5paper]{ltjsarticle}

%\usepackage[margin=15truemm, top=5truemm, bottom=5truemm]{geometry}
\usepackage[margin=10truemm]{geometry}

\usepackage{amsmath,amssymb}
%\pagestyle{headings}
\pagestyle{empty}

%\usepackage{listings,url}
%\renewcommand{\theenumi}{(\arabic{enumi})}

%\usepackage{graphicx}

%\usepackage{tikz}
%\usetikzlibrary {arrows.meta}
%\usepackage{wrapfig}	% required for `\wrapfigure' (yatex added)
%\usepackage{bm}	% required for `\bm' (yatex added)

% ルビを振る
%\usepackage{luatexja-ruby}	% required for `\ruby'

%% 核Ker 像Im Hom を定義
%\newcommand{\Img}{\mathop{\mathrm{Im}}\nolimits}
%\newcommand{\Ker}{\mathop{\mathrm{Ker}}\nolimits}
%\newcommand{\Hom}{\mathop{\mathrm{Hom}}\nolimits}

%\DeclareMathOperator{\Rot}{rot}
%\DeclareMathOperator{\Div}{div}
%\DeclareMathOperator{\Grad}{grad}
%\DeclareMathOperator{\arcsinh}{arcsinh}
%\DeclareMathOperator{\arccosh}{arccosh}
%\DeclareMathOperator{\arctanh}{arctanh}



\begin{document}

\textbf{開集合}

集合$A$において、
任意の$a\in A$に対し
次を満たす $\varepsilon\in >0$ が存在する時
$A$を開集合という。

$d(a,b)<\varepsilon$となる全ての$b$が
$b\in A$である。
なお、$d(a,b)$は二点間の距離を表す。


つまり、$a\in A$の周りの点は必ず$A$に含まれる時に
$A$を開集合という。


\dotfill

区間
$(a,b) \subset \mathbb{R}$
について

$p_0 \in (a,b)$とするとき、
$\varepsilon$を次のように定める。
\begin{equation}
 \varepsilon = \min \left\{ \frac{p_0-a}{2}, \frac{b-p_0}{2} \right\}
\end{equation}

これにより 点$p_0$から距離$\varepsilon$未満の全ての点が
区間$(a,b)$に含まれる。

$p_0$は区間内のどの点であっても上記を満たす為、
$(a,b)$は開集合となる。


区間$(a,b]$は多くの点が開集合の定義を満たすが、
端の点$b\in (a,b]$はどれほど$\varepsilon$を小さくとっても
$b$より大きい点は区間$(a,b]$に含まれないので
開集合ではない。

\hrulefill

$n$次元開立方体$X\subset\mathbb{R}^n$は$\mathbb{R}^n$の開集合であることを示せ。
\begin{equation}
 X = \{ (x_1,x_2,x_3,\dots ,x_n) \mid -1<x_i<1 \ (i=1,2,3,\dots,n) \}
\end{equation}

\dotfill

点$p\in\mathbb{R}^n$を中心とした半径$r$の開球$B(p,r)$を次のように定義する。
\begin{equation}
 B(p,r) = \{ x \in\mathbb{R}^n \mid \lvert x-p \rvert < r \}
\end{equation}

任意の点$x\in X$について
開球$B(x,r)$が$X$に含まれるようになるには
$x\in X$に対して
開球の半径$r_x$
をうまく取る必要がある。

点$x\in X$から$X$の外部に最も近いのは
次の$2n$個の点のどれかになる。
\begin{equation}
 (\pm 1, 0,0,\dots , 0),\
 (0,\pm 1, 0,\dots , 0),\
 (0,0,\pm 1, \dots , 0),\
 \dots \ ,\
 (0,0,0,\dots , \pm 1)
\end{equation}

そこで、最も近い点との距離の半分を開球の半径$r_x$とすれば良い。

\begin{equation}
 2 r_x =
 \min \{
 | x_1 - 1 |,\ | x_1 + 1 |,\ | x_2 - 1 |,\ | x_2 + 1 |, \dots , | x_n - 1 |,\ | x_n + 1 |
 \}
\end{equation}

このように$r_x$を定めると
任意の点$x\in X$について
$B(x,r_x) \subset X$となる。

よって、$X$は$\mathbb{R}^n$の開集合である。

\end{document}

