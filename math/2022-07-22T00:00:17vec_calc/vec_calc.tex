\documentclass[12pt,b5paper]{ltjsarticle}

%\usepackage[margin=15truemm, top=5truemm, bottom=5truemm]{geometry}
\usepackage[margin=15truemm]{geometry}

\usepackage{amsmath,amssymb}
%\pagestyle{headings}
\pagestyle{empty}

%\usepackage{listings,url}
%\renewcommand{\theenumi}{(\arabic{enumi})}

\usepackage{graphicx}

\usepackage{tikz}
\usetikzlibrary {arrows.meta}
\usepackage{wrapfig}	% required for `\wrapfigure' (yatex added)
\usepackage{bm}	% required for `\bm' (yatex added)

% ルビを振る
\usepackage{luatexja-ruby}	% required for `\ruby'

%% 核Ker 像Im Hom を定義
%\newcommand{\Img}{\mathop{\mathrm{Im}}\nolimits}
%\newcommand{\Ker}{\mathop{\mathrm{Ker}}\nolimits}
%\newcommand{\Hom}{\mathop{\mathrm{Hom}}\nolimits}
\newcommand{\Rot}{\mathop{\mathrm{rot}}\nolimits}
\newcommand{\Div}{\mathop{\mathrm{div}}\nolimits}
\newcommand{\Grad}{\mathop{\mathrm{grad}}\nolimits}

\begin{document}

\hrulefill

\textbf{ナブラ $\nabla$}
\begin{equation}
 \nabla = \left( \frac{\partial}{\partial x}, \frac{\partial}{\partial y}, \frac{\partial}{\partial z} \right)
\end{equation}


ベクトル場$\bm{f}$
\begin{equation}
 \bm{f}: \mathbb{R}^3 \rightarrow \mathbb{R}^3,
  \quad
  (x,y,z) \mapsto ( f_1(x,y,z),f_2(x,y,z),f_3(x,y,z) )
\end{equation}

%\textbf{$D\bm{f}$}
%\begin{equation}
% D\bm{f}=\left( \frac{\partial \bm{f}}{\partial x}, \frac{\partial \bm{f}}{\partial y}, \frac{\partial \bm{f}}{\partial z} \right)
%  =
%  \begin{pmatrix}
%   \frac{\partial f_1}{\partial x} & \frac{\partial f_1}{\partial y} & \frac{\partial f_1}{\partial z}\\
%   \frac{\partial f_2}{\partial x} & \frac{\partial f_2}{\partial y} & \frac{\partial f_2}{\partial z}\\
%   \frac{\partial f_3}{\partial x} & \frac{\partial f_3}{\partial y} & \frac{\partial f_3}{\partial z}
%  \end{pmatrix}
%\end{equation}

\textbf{回転 $\Rot$}
\begin{equation}
 \Rot \bm{f} = \nabla\times\bm{f}
  = \left(
     \frac{\partial f_3}{\partial y} - \frac{\partial f_2}{\partial z},\
     \frac{\partial f_1}{\partial z} - \frac{\partial f_3}{\partial x},\
     \frac{\partial f_2}{\partial x} - \frac{\partial f_1}{\partial y}
    \right)
\end{equation}


\textbf{発散 $\Div$}
\begin{equation}
 \Div \bm{f} = \langle \nabla, \bm{f}\rangle
  = \frac{\partial f_1}{\partial x}
    + \frac{\partial f_2}{\partial y}
    + \frac{\partial f_3}{\partial z}
\end{equation}



%\textbf{内積、外積}
%\begin{align}
% \langle \bm{a}, \bm{a} \rangle =& \lvert\bm{a}\rvert^2 &
% \bm{a}\times\bm{a} =& 0\\
% \langle \bm{a}, \bm{b} \rangle =& \langle \bm{b}, \bm{a} \rangle &
%  \quad
% \bm{a}\times\bm{b} =& - \bm{b}\times\bm{a}\\
% \langle k\bm{a}, \bm{b} \rangle =& k \langle \bm{a}, \bm{b} \rangle &
%  k\bm{a} \times \bm{b} =& k (\bm{a} \times \bm{b})
% \\
% \langle \bm{a}, \bm{b}+\bm{c} \rangle =& \langle \bm{a}, \bm{b} \rangle + \langle \bm{a}, \bm{c} \rangle &
% \bm{a}\times (\bm{b}+\bm{c}) =& \bm{a}\times \bm{b} + \bm{a}\times \bm{c}
%\end{align}
%
%\textbf{三重積}
%\begin{align}
% \langle \bm{a}, \bm{b}\times\bm{c} \rangle
%  =& \langle \bm{b}, \bm{c}\times\bm{a} \rangle
%  = \langle \bm{c}, \bm{a}\times\bm{b} \rangle\\
% \bm{a}\times(\bm{b}\times\bm{c})
% =& \langle \bm{a}, \bm{c} \rangle\bm{b}
% - \langle \bm{a}, \bm{b} \rangle\bm{c}
%\end{align}
%
%\textbf{ヤコビの恒等式}
%\begin{equation}
% \bm{a}\times(\bm{b}\times\bm{c})
%  + \bm{b}\times(\bm{c}\times\bm{a})
%  + \bm{c}\times(\bm{a}\times\bm{b})
%  =0
%\end{equation}

\dotfill

ベクトルの回転や発散は2次元上で考える場合
3つ目の成分は0(定数)として考える。

\begin{align}
 \bm{f}(x,y) =& {}^{t}(f_1(x,y,0) \quad f_2(x,y,0) \quad 0)\\
 \Div \bm{f}(x,y) =& \frac{\partial}{\partial x}f_1(x,y) + \frac{\partial}{\partial y}f_2(x,y))\\
 \Rot \bm{f}(x,y)
  =& \left(
     \frac{\partial 0}{\partial y} - \frac{\partial f_2(x,y)}{\partial z},\
     \frac{\partial f_1(x,y)}{\partial z} - \frac{\partial 0}{\partial x},\
     \frac{\partial f_2(x,y)}{\partial x} - \frac{\partial f_1(x,y)}{\partial y}
    \right)\\
  =& \left(
     0,\
     0,\
     \frac{\partial f_2(x,y)}{\partial x} - \frac{\partial f_1(x,y)}{\partial y}
    \right)
\end{align}

\dotfill

\textbf{\ruby{Green}{グリーン}の定理}
\begin{equation}
 \int_{\partial D} \left( P\mathrm{d}x + Q\mathrm{d}y \right)
  =\int_{D} \left( \frac{\partial Q}{\partial x} - \frac{\partial P}{\partial y}\right)\mathrm{d}x\,\mathrm{d}y
\end{equation}

\textbf{\ruby{Gauss}{ガウス}の発散定理}
\begin{equation}
 \int_{\partial D} \langle \bm{A}, \bm{n} \rangle \mathrm{d}s
  =\int_{D} \Div\bm{A} \ \mathrm{d}S
\end{equation}

\textbf{\ruby{Stokes}{ストークス}の定理}
\begin{equation}
 \int_{\partial D} \langle \bm{A}, \mathrm{d}s \rangle
  =\int_{D} \langle \Rot\bm{A}, \mathrm{d}S \rangle
\end{equation}


\dotfill

\textbf{倍角の公式}
\begin{align}
 \sin2\theta =& 2\sin\theta \cos\theta\\
 \cos2\theta =& \cos^2\theta - \sin^2\theta
 = 2\cos^2\theta - 1
 = 1 - 2\sin^2\theta
\end{align}

\textbf{半角の公式}
\begin{align}
 \sin^2\theta =& \frac{1}{2}(1-\cos2\theta) & \cos^2\theta =& \frac{1}{2}(1+\cos2\theta)
\end{align}

\textbf{3倍角の公式}
\begin{align}
 \sin 3\theta =& 3\sin\theta - 4\sin^3\theta &
 \cos 3\theta =& 4\cos^3\theta - 3\cos\theta\\
 \sin^3\theta =& \frac{1}{4}(3\sin\theta - \sin 3\theta) &
 \cos^3\theta =& \frac{1}{4}(\cos 3\theta + 3\cos\theta)
\end{align}


\hrulefill

\begin{enumerate}
 \item
      ベクトル場$\bm{f}$と領域$D$を次のように定める。
      \begin{gather}
       \bm{f}(x,y) = {}^{t}(2x-3y^2 \quad 3x^2-4y^3)\\
       D = \{ (x,y)\in\mathbb{R}^2 \mid x^2+y^2\leq 1 \}
      \end{gather}

      この時、次を求めよ。
      \begin{equation}
       \int_D \Rot\bm{f} \mathrm{d}x\mathrm{d}y
      \end{equation}

\dotfill

      \begin{align}
       \int_D \Rot\bm{f} \mathrm{d}x\mathrm{d}y
       =& \int_D \left(\frac{\partial f_2}{\partial x} - \frac{\partial f_1}{\partial y}\right) \mathrm{d}x\mathrm{d}y
       = \int_{\partial D} ( f_1  \mathrm{d}x + f_2\mathrm{d}y )
      \end{align}

      $\partial D$は単位円の円周であるので、
      $x=\cos \theta,\ y=\sin \theta$とおいて積分を行う。

      \begin{gather}
       \mathrm{d}x = -\sin\theta \mathrm{d}\theta
        , \quad
       \mathrm{d}y = \cos\theta \mathrm{d}\theta\\
       f_1 = 2x-3y^2 = 2\cos\theta-3\sin^2\theta\\
       f_2 = 3x^2-4y^3 = 3\cos^2\theta -4\sin^3\theta
      \end{gather}

      \begin{align}
       & \int_{\partial D} ( f_1  \mathrm{d}x + f_2\mathrm{d}y )\\
       =& \int_{0}^{2\pi} ( -2\sin\theta\cos\theta+3\sin^3\theta + 3\cos^3\theta -4\sin^3\theta\cos\theta)\mathrm{d}\theta\\
       =& \int_{0}^{2\pi} \left( -2\sin2\theta+\frac{3}{4}(3\sin\theta - \sin 3\theta) + \frac{3}{4}(\cos 3\theta + 3\cos\theta) +\frac{1}{2}\sin4\theta \right)\mathrm{d}\theta\\
       =& \left[
         \cos2\theta + \frac{3}{4}\left(-3\cos\theta + \frac{1}{3}\cos 3\theta\right)
         + \frac{3}{4}\left(\frac{1}{3}\sin 3\theta + 3\sin\theta\right) -\frac{1}{8}\cos4\theta
       \right]_{\theta=0}^{\theta=2\pi}\\
       =& 0
      \end{align}
\hrulefill

 \item
      ベクトル場$\bm{f}$と領域$D$を次のように定める。
      \begin{gather}
       \bm{f}(x,y) = {}^{t}\left(\frac{x}{x^2+y^2} \quad  \frac{y}{x^2+y^2}\right)\\
       D = \{ (x,y)\in\mathbb{R}^2 \mid 1\leq  x^2+y^2\leq 4 \}
      \end{gather}

      この時、次を求めよ。
      \begin{equation}
       \int_D \Div\bm{f} \mathrm{d}x\mathrm{d}y
      \end{equation}

\dotfill

      $D$の境界は半径2の円周$S_2$と半径1の円周$S_1$であり、
      半径2の円は正の向き、半径1の円は負の向きとなる。
      この円周を極座標$x=r\cos\theta,\ y=r\sin\theta$で置いて積分を行う。
      半径より$r=2$または$r=1$である。
      \begin{equation}
       \bm{f}(r\cos\theta,r\sin\theta)
        = {}^{t}\left(\frac{r\cos\theta}{r^2} \quad  \frac{r\sin\theta}{r^2}\right)
        = {}^{t}\left(\frac{\cos\theta}{r} \quad  \frac{\sin\theta}{r}\right)
      \end{equation}

      これらを用いて次のように積分を行う。
      \begin{equation}
       \int_D \Div\bm{f} \mathrm{d}x\mathrm{d}y
       = \int_{\partial D} \langle \bm{f}, \bm{n} \rangle \mathrm{d}s
       = \int_{S_2} \langle \bm{f}, \bm{n} \rangle \mathrm{d}s
         + \int_{S_1} \langle \bm{f}, \bm{n} \rangle \mathrm{d}s\\
       =0
      \end{equation}

      半径2の円周($S_2$)上の積分は次のようになる。
      \begin{equation}
       \int_{S_2} \langle \bm{f}, \bm{n} \rangle \mathrm{d}s
        = \int_{0}^{2\pi} \left\langle \left(\frac{\cos\theta}{2}, \frac{\sin\theta}{2}\right),
        \left( -2\sin\theta, 2\cos\theta \right) \right\rangle \mathrm{d}\theta
        =0
      \end{equation}

      同様に、
      半径1の円周($S_1$)上の積分は次のようになる。
      \begin{equation}
       \int_{S_1} \langle \bm{f}, \bm{n} \rangle \mathrm{d}s
        = \int_{2\pi}^{0} \langle
            (\cos\theta, \sin\theta),(-\sin\theta, \cos\theta )
          \rangle \mathrm{d}\theta
        =0
      \end{equation}


\hrulefill

 \item
      ベクトル場$\bm{f},\bm{P}$と領域$D,S$を次のように定める。
      \begin{gather}
       \bm{f}(x,y,z) = {}^{t}(z^2+1 \quad xy \quad y^2+x)\\
       \bm{P}(u,v) = {}^{t}(u \quad v \quad 4-u^2-v^2)\\
       D = \{ (u,v)\in\mathbb{R}^2 \mid u^2+v^2\leq 4 \},\quad
       S = \bm{P}(D)
      \end{gather}

      この時、次を求めよ。
      \begin{equation}
       \int_S \langle \Rot\bm{f},\bm{n}\rangle \mathrm{d}A
      \end{equation}

\dotfill

      \begin{equation}
       \int_S \langle \Rot\bm{f},\bm{n}\rangle \mathrm{d}A
        =
        \int_{\partial S} \langle \bm{f},\mathrm{d}\bm{r}\rangle
      \end{equation}


      \begin{gather}
       D=\{ (r\cos\theta,r\sin\theta) \in\mathbb{R}^2 \mid 0\leq r\leq 2, 0\leq \theta < 2\pi \}\\
       S=\bm{P}(D) = \{ {}^{t}(r\cos\theta \quad r\sin\theta \quad 4-r^2 ) \in\mathbb{R}^3
         \mid  0\leq r\leq 2, 0\leq \theta < 2\pi \}
      \end{gather}
      つまり、$S$は$(0,0,4)$を頂点とした二次関数のグラフを回転させた図形である。
      境界$\partial S$は$z=0$の平面上の半径2の円周である。

      そこで、$x=2\cos\theta,y=2\sin\theta,z=0$とおく。
      \begin{equation}
       \mathrm{d}\bm{r}
        =
        \begin{pmatrix}
         \mathrm{d}x\\\mathrm{d}y\\0
        \end{pmatrix}
        =
        \begin{pmatrix}
         -2\sin\theta\mathrm{d}\theta\\2\cos\theta\mathrm{d}\theta\\0
        \end{pmatrix}
      \end{equation}


      \begin{align}
       \int_{\partial S} \langle \bm{f},\mathrm{d}\bm{r}\rangle
       =&
       \int_{0}^{2\pi} \left\langle
        \begin{pmatrix}
         z^2+1 \\ xy \\ y^2+x
        \end{pmatrix}
       ,
       \begin{pmatrix}
         \mathrm{d}x\\\mathrm{d}y\\0
       \end{pmatrix}
       \right\rangle\\
       =&
       \int_{0}^{2\pi} \left\langle
        \begin{pmatrix}
         1 \\ 4\cos\theta\sin\theta \\ 4\sin^2\theta+2\cos\theta
        \end{pmatrix}
       ,
       \begin{pmatrix}
         -2\sin\theta\mathrm{d}\theta\\2\cos\theta\mathrm{d}\theta\\0
       \end{pmatrix}
       \right\rangle\\
       =& \int_{0}^{2\pi} \left( -2\sin\theta + 8\sin\theta\cos^2\theta \right)\mathrm{d}\theta\\
       =& \int_{0}^{2\pi} \left( -2\sin\theta + 8\sin\theta - 8\sin^3\theta \right)\mathrm{d}\theta\\
       =& \int_{0}^{2\pi} 2\sin3\theta \mathrm{d}\theta\\
       =& \left[ -\frac{2}{3}\cos3\theta \right]_{0}^{2\pi} =0
      \end{align}




\hrulefill

\end{enumerate}

\hrulefill

\end{document}
