\documentclass[12pt,b5paper]{ltjsarticle}

%\usepackage[margin=15truemm, top=5truemm, bottom=5truemm]{geometry}
\usepackage[margin=15truemm]{geometry}

\usepackage{amsmath,amssymb}
%\pagestyle{headings}
\pagestyle{empty}

%\usepackage{listings,url}
%\renewcommand{\theenumi}{(\arabic{enumi})}

\usepackage{graphicx}

\usepackage{tikz}
\usetikzlibrary {arrows.meta}
\usepackage{wrapfig}	% required for `\wrapfigure' (yatex added)
\usepackage{bm}	% required for `\bm' (yatex added)

% ルビを振る
\usepackage{luatexja-ruby}	% required for `\ruby'

%% 核Ker 像Im Hom を定義
%\newcommand{\Img}{\mathop{\mathrm{Im}}\nolimits}
%\newcommand{\Ker}{\mathop{\mathrm{Ker}}\nolimits}
%\newcommand{\Hom}{\mathop{\mathrm{Hom}}\nolimits}
\newcommand{\Rot}{\mathop{\mathrm{rot}}\nolimits}
\newcommand{\Div}{\mathop{\mathrm{div}}\nolimits}

\begin{document}

\hrulefill

\begin{center}
 4元体$\mathbb{F}_4$の演算対応表

 \begin{tabular}{c||c|c|c|c||c||c||c|c|c|c}
  $+$ & 0 & 1 & $a$ & $b$ & & $\times$ & 0 & 1 & $a$ & $b$ \\
  \hline\hline
  0 & 0 & 1 & $a$ & $b$ & & 0 & 0 & 0 & 0 & 0 \\
  \hline
  1 & 1 & 0 & $b$ & $a$ & & 1 & 0 & 1 & $a$ & $b$ \\
  \hline
  $a$ & $a$ & $b$ & 0 & 1 & & $a$ & 0 & $a$ & $b$ & 1 \\
  \hline
  $b$ & $b$ & $a$ & 1 & 0 & & $b$ & 0 & $b$ & 1 & $a$ \\
 \end{tabular}
\end{center}

同様にして8元体$\mathbb{F}_8$の対応表を作れ。

\dotfill

有限体の標数は素数である。
標数とは乗法単位元$1$をその個数だけ足すと0になる数のことである。
4元体も8元体も標数は2であるので、$1+1=0$であり、同じ者同士の和が0になる。

2元体$\mathbb{F}_2$は$\mathbb{F}_2=\{0,1\}$であり、
その演算は整数の加法乗法と同じであり、$2=0$という規則を付け加えたものになる。

4元体は1変数多項式の加法乗法と同じであり、標数2($2=0$)と$x^2=x+1$を付け加えたものになる。
\begin{equation}
 \mathbb{F}_4= \{0,1,x,1+x\}
\end{equation}
この$\mathbb{F}_4$の演算が多項式の加法と乗法であり、
$2$が出てくると$0$に置き換え、$x^2$が出てくると$x+1$に置き換えることで体になる。
対応表では$a=x,\ b=1+x$としたものと同じである。

8元体$\mathbb{F}_8$は4元体$\mathbb{F}_4$の規則$x^2=x+1$を
$x^3=x+1$に変えたものになる。
\begin{equation}
 \mathbb{F}_8 = \{ 0,1,x,x+1,x^2,x^2+1,x^2+x,x^2+x+1 \}
\end{equation}
これに多項式としての加法と乗法を当てはめ、
$2$が出てくると$0$に置き換え、$x^3$が出てくると$x+1$に置き換える。

例えば、$x^2+1$と$x^2+x+1$をかけると次のようになる。
\begin{align}
 (x^2+1)\times(x^2+x+1)
 &= x^4+x^3+2x^2+x+1\\
 &= x(x+1) + (x+1) + x+1\\
 &= x^2+x + x+1 + x+1\\
 &= x^2+x
\end{align}

$\mathbb{F}_8$の元は
$a=x ,\ b=x+1 ,\ c=x^2 ,\ d=x^2+1 ,\ e=x^2+x ,\ f=x^2+x+1$と表記すると
次のようになる。

\begin{tabular}{c||c|c|c|c|c|c|c|c}
 $+$ & 0 & 1 & $a$ & $b$ & $c$ & $d$ & $e$ & $f$ \\
 \hline\hline
 0 & 0 & 1 & $a$ & $b$ & $c$ & $d$ & $e$ & $f$ \\
 \hline
 1 & 1 & 0 & $b$ & $a$ & $d$ & $c$ & $f$ & $e$ \\
 \hline
 $a$ & $a$ & $b$ & 0 & 1 & $e$ & $f$ & $c$ & $d$ \\
 \hline
 $b$ & $b$ & $a$ & 1 & 0 & $f$ & $e$ & $d$ & $c$ \\
 \hline
 $c$ & $c$ & $d$ & $e$ & $f$ & 0 & 1 & $a$ & $b$ \\
 \hline
 $d$ & $d$ & $c$ & $f$ & $e$ & 1 & 0 & $b$ & $a$ \\
 \hline
 $e$ & $e$ & $f$ & $c$ & $d$ & $a$ & $b$ & 0 & 1 \\
 \hline
 $f$ & $f$ & $e$ & $d$ & $c$ & $b$ & $a$ & 1 & 0
\end{tabular}
\quad
\begin{tabular}{c||c|c|c|c|c|c|c|c}
 $\times$
     & 0 &  1  & $a$ & $b$ & $c$ & $d$ & $e$ & $f$ \\
 \hline\hline
  0  & 0 &  0  &  0  &  0  &  0  &  0  &  0  &  0 \\
 \hline
  1  & 0 &  1  & $a$ & $b$ & $c$ & $d$ & $e$ & $f$ \\
 \hline
 $a$ & 0 & $a$ & $c$ & $e$ & $b$ &  1  & $f$ & $d$ \\
 \hline
 $b$ & 0 & $b$ & $e$ & $d$ & $f$ & $c$ &  1  & $a$ \\
 \hline
 $c$ & 0 & $c$ & $b$ & $f$ & $e$ & $a$ & $d$ & 1 \\
 \hline
 $d$ & 0 & $d$ &  1 & $c$ & $a$ & $f$ & $b$ & $e$ \\
 \hline
 $e$ & 0 & $e$ & $f$ &  1  & $d$ & $b$ & $a$ & $c$ \\
 \hline
 $f$ & 0 & $f$ & $d$ & $a$ & 1 & $e$ & $c$ & $b$
\end{tabular}

\hrulefill

\hrulefill

有限体は多項式環を既約多項式で割ったものと同型になる。

\begin{equation}
 \text{4元体 : } \mathbb{F}_2[X]/(x^2+x+1)
  \qquad
 \text{8元体 : } \mathbb{F}_2[X]/(x^3+x+1)
\end{equation}



\newpage

\hrulefill

$G,H$を群とする。
$G$の演算を$\bullet_G$とし、
$H$の演算を$\bullet_H$とする。
直積集合$G\times H$に演算$\bullet_{G\times H}$を次のように成分ごとの演算で定義する。
\begin{equation}
 (g_1,h_1) \bullet_{G\times H} (g_2,h_2) = (g_1 \bullet_G g_2, h_1 \bullet_H h_2)
\end{equation}

$\mathbb{Z}/3\mathbb{Z}\times \mathbb{Z}/3\mathbb{Z}$
と
$\mathbb{Z}/9\mathbb{Z}$
は同型か否かを判定せよ。


\dotfill

剰余類群は次のような集合である。
\begin{equation}
 \mathbb{Z}/3\mathbb{Z} =\{0,1,2\},\quad
 \mathbb{Z}/9\mathbb{Z} =\{0,1,2,3,4,5,6,7,8\}
\end{equation}

位数3の群は次のような演算が定義される。
\begin{center}
 \begin{tabular}{c||c|c|c}
  & $e$ & $a$ & $b$ \\
  \hline \hline
  $e$ & $e$ & $a$ & $b$ \\
  \hline
  $a$ & $a$ & $b$ & $e$ \\
  \hline
  $b$ & $b$ & $e$ & $a$
 \end{tabular}
\end{center}
つまり、$e=0,a=1,b=2$である加法群と同型なものしか存在しない為、
$\mathbb{Z}/3\mathbb{Z}$と同型である。

$\mathbb{Z}/9\mathbb{Z}$を剰余類群とすると、
整数を$9$で割った余りの集合となる。
この為、演算は整数の加法で行い、
$9$が現れると$0$に置き換えることで群となる。


$\mathbb{Z}/3\mathbb{Z}\times\mathbb{Z}/3\mathbb{Z}$
の元は$(n_1,n_2)$となるが、
$(0,0)$以外の元は3回足すと$(0,0)$になる。

\begin{gather}
 (0,2)+(0,2)+(0,2)=(0,6)=(0,0)\\
 (1,1)+(1,1)+(1,1)=(3,3)=(0,0)
\end{gather}

つまり、$(0,0)$は位数1であり、それ以外の元は位数3である。
しかし、$\mathbb{Z}/9\mathbb{Z}$の元$1$は位数9である。

よって、位数の異なるものに準同型写像が対応を取れないため
同型写像が存在しない。
これにより
$\mathbb{Z}/3\mathbb{Z}\times\mathbb{Z}/3\mathbb{Z}$
と
$\mathbb{Z}/9\mathbb{Z}$
は同型ではない。



\newpage

\hrulefill

送信者は2ビットのビット列$S$を送ろうとしている。
通信経路にて高々2ビットのノイズが入ることが想定される。

このビット列$S$に6ビットの訂正列を加えた8ビットを送ることで
誤り訂正が可能である。
この手順を構成し説明せよ。

\dotfill


高々2ビットのエラーが含まれる事を考え、
送信者が送るビット列のそれぞれのハミング距離は4以上ないといけない。

例えば、送るデータ列を次のようにする。
\begin{align}
 a =& (0,0, \ 0,0,0,0,0,0)\\
 b =& (0,1, \ 0,0,0,1,1,1)\\
 c =& (1,0, \ 1,1,1,0,0,0)\\
 d =& (1,1, \ 1,1,1,1,1,1)
\end{align}

これらのハミング距離$d_H$は次のようになる。
\begin{align}
 d_H(a,b)=& 4 &
 d_H(a,c)=& 4 &
 d_H(a,d)=& 8\\
 d_H(b,c)=& 8 &
 d_H(b,d)=& 4 &
 d_H(c,d)=& 4
\end{align}
ハミング距離は2つの文字列の異なる箇所の個数で定義する。

エラーが高々2ビットしか含まれないのであれば
受信したデータとハミング距離が最も小さいものが送られたと考えられる。

これにより送信するデータ列$a,b,c,d$からみて
ハミング距離が2以内の受信データは
そのデータが送られたものとして扱うことで誤り訂正が可能である。



\newpage

\hrulefill

以下のような誤り訂正・検出の手順を考える。
\begin{enumerate}
 \item
      送信者は100ビットの情報列を$10\times 10$の正方形に並べる
 \item\label{par_sum}
      正方形の各列各行についてそこに含まれる10個の数の和を求める。(2元体上)
 \item\label{total}
      すべての数の総和を求める。(2元体上)
 \item
      \ref{par_sum} 及び \ref{total} で得られた21個の数を並べ、
      訂正列として付加して送信する。
\end{enumerate}

この方式(垂直水平パリティ符号)を誤り訂正符号として用いた場合、
誤りが何ビットまでなら確実に訂正できるか。
また、誤り検出符号として用いた場合、
誤りが何ビットまでなら確実に検出できるか。

\dotfill


%この符号では3つの式でパリティを計算する。
%(行と列と総和)
\begin{itemize}
 \item
      1ビットの誤りがあると
      そのビットを含む行と列の和が合わなくなる。
      この行と列からエラーの含まれるビットが特定できるため、
      1ビットの誤りは訂正が可能である。

 \item
      2ビットの誤りがあると
      行の和が合わない場所が2行、列の和が合わない場所が2列
      発生する場合がある。
      この場合はエラーを含むビットを絞ることは出来るが、
      特定は出来ない。

 \item
      3ビットの誤りがあると
      他の情報列と1ビットの誤りと区別がつかない。

 \item
      4ビットの誤りがあると
      他の情報列と一致する場合がある。
\end{itemize}

以上により、
1ビットの誤りは訂正が可能である。
検出だけであれば3ビットの誤りまで検出可能である。









\end{document}
