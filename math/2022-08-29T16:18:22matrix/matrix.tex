\documentclass[12pt,b5paper]{ltjsarticle}

%\usepackage[margin=15truemm, top=5truemm, bottom=5truemm]{geometry}
\usepackage[margin=10truemm]{geometry}

\usepackage{amsmath,amssymb}
%\pagestyle{headings}
\pagestyle{empty}

%\usepackage{listings,url}
%\renewcommand{\theenumi}{(\arabic{enumi})}

%\usepackage{graphicx}

%\usepackage{tikz}
%\usetikzlibrary {arrows.meta}
%\usepackage{wrapfig}	% required for `\wrapfigure' (yatex added)
%\usepackage{bm}	% required for `\bm' (yatex added)

% ルビを振る
%\usepackage{luatexja-ruby}	% required for `\ruby'

%% 核Ker 像Im Hom を定義
%\newcommand{\Img}{\mathop{\mathrm{Im}}\nolimits}
%\newcommand{\Ker}{\mathop{\mathrm{Ker}}\nolimits}
%\newcommand{\Hom}{\mathop{\mathrm{Hom}}\nolimits}

%\DeclareMathOperator{\Rot}{rot}
%\DeclareMathOperator{\Div}{div}
%\DeclareMathOperator{\Grad}{grad}
%\DeclareMathOperator{\arcsinh}{arcsinh}
%\DeclareMathOperator{\arccosh}{arccosh}
%\DeclareMathOperator{\arctanh}{arctanh}



\begin{document}


固有値を使った計算

\hrulefill

\begin{equation}
 5x^2 + 4xy + 2y^2 =0
\end{equation}


\hrulefill

式を次のように行列で表す。

\begin{equation}
 5x^2 + 4xy + 2y^2 =
  \begin{pmatrix}
   x & y
  \end{pmatrix}
  \begin{pmatrix}
   5 & 2\\ 2 & 2
  \end{pmatrix}
  \begin{pmatrix}
   x \\ y
  \end{pmatrix}
  \label{1st}
\end{equation}

2次正方行列は対称行列であるので
直交行列$P\ (P^{-1}={}^{t}\!P)$を用いて対角化したものが次の式である。

\begin{equation}
 P
 \begin{pmatrix}
   5 & 2\\ 2 & 2
 \end{pmatrix}
 {}^{t}\!P
 =
 \begin{pmatrix}
   1 & 0\\ 0 & 6
 \end{pmatrix}
\end{equation}

式を変形する。

\begin{equation}
 \begin{pmatrix}
   5 & 2\\ 2 & 2
 \end{pmatrix}
 =
 {}^{t}\!P
 \begin{pmatrix}
   1 & 0\\ 0 & 6
 \end{pmatrix}
 P
\end{equation}

これを式(\ref{1st})に当てはめる。

\begin{align}
 5x^2 + 4xy + 2y^2 =&
  \begin{pmatrix}
   x & y
  \end{pmatrix}
  \begin{pmatrix}
   5 & 2\\ 2 & 2
  \end{pmatrix}
  \begin{pmatrix}
   x \\ y
  \end{pmatrix}\\
  =&
  \begin{pmatrix}
   x & y
  \end{pmatrix}
  {}^{t}\!P
  \begin{pmatrix}
   1 & 0\\ 0 & 6
  \end{pmatrix}
  P
  \begin{pmatrix}
   x \\ y
  \end{pmatrix}\\
  =&
  {}^{t}\!\left( P \begin{pmatrix} x \\ y \end{pmatrix} \right)
  \begin{pmatrix}
   1 & 0\\ 0 & 6
  \end{pmatrix}
  \left( P \begin{pmatrix} x \\ y \end{pmatrix} \right)
 \label{2nd}
\end{align}

$X,Y$を次のようにおく。
\begin{equation}
 \begin{pmatrix}X\\Y\end{pmatrix}
 =
  P \begin{pmatrix} x \\ y \end{pmatrix}
\end{equation}

これを式(\ref{2nd})に当てはめると次の式が得られる。

\begin{equation}
   {}^{t}\!\left( P \begin{pmatrix} x \\ y \end{pmatrix} \right)
  \begin{pmatrix}
   1 & 0\\ 0 & 6
  \end{pmatrix}
  \left( P \begin{pmatrix} x \\ y \end{pmatrix} \right)
  =
   {}^{t}\!\begin{pmatrix}X\\Y\end{pmatrix}
   \begin{pmatrix}
    1 & 0\\ 0 & 6
   \end{pmatrix}
   \begin{pmatrix}X\\Y\end{pmatrix}
   = X^2 + 6Y^2
\end{equation}

以上により
$5x^2 + 4xy + 2y^2 =0$は
座標変換
$\begin{pmatrix}X\\Y\end{pmatrix}= P\begin{pmatrix} x \\ y \end{pmatrix}$
によって
$X^2 + 6Y^2 =0$
となる。


\hrulefill



\end{document}

