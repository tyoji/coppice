\documentclass[12pt,b5paper]{ltjsarticle}

%\usepackage[margin=15truemm, top=5truemm, bottom=5truemm]{geometry}
%\usepackage[margin=10truemm,left=15truemm]{geometry}
\usepackage[margin=10truemm]{geometry}

\usepackage{amsmath,amssymb}
%\pagestyle{headings}
\pagestyle{empty}

%\usepackage{listings,url}
\renewcommand{\theenumi}{(\arabic{enumi})}

%\usepackage{graphicx}

%\usepackage{tikz}
%\usetikzlibrary {arrows.meta}
%\usepackage{wrapfig}
%\usepackage{bm}

% ルビを振る
%\usepackage{luatexja-ruby}	% required for `\ruby'

%% 核Ker 像Im Hom を定義
\newcommand{\Img}{\mathop{\mathrm{Im}}\nolimits}
\newcommand{\Ker}{\mathop{\mathrm{Ker}}\nolimits}
%\newcommand{\Hom}{\mathop{\mathrm{Hom}}\nolimits}

%\DeclareMathOperator{\Rot}{rot}
%\DeclareMathOperator{\Div}{div}
%\DeclareMathOperator{\Grad}{grad}
%\DeclareMathOperator{\arcsinh}{arcsinh}
%\DeclareMathOperator{\arccosh}{arccosh}
%\DeclareMathOperator{\arctanh}{arctanh}

\usepackage{url}

%\usepackage{listings}
%
%\lstset{
%%プログラム言語(複数の言語に対応,C,C++も可)
%  language = Python,
%%  language = Lisp,
%%  language = C,
%  %背景色と透過度
%  %backgroundcolor={\color[gray]{.90}},
%  %枠外に行った時の自動改行
%  breaklines = true,
%  %自動改行後のインデント量(デフォルトでは20[pt])
%  breakindent = 10pt,
%  %標準の書体
%%  basicstyle = \ttfamily\scriptsize,
%  basicstyle = \ttfamily,
%  %コメントの書体
%%  commentstyle = {\itshape \color[cmyk]{1,0.4,1,0}},
%  %関数名等の色の設定
%  classoffset = 0,
%  %キーワード(int, ifなど)の書体
%%  keywordstyle = {\bfseries \color[cmyk]{0,1,0,0}},
%  %表示する文字の書体
%  %stringstyle = {\ttfamily \color[rgb]{0,0,1}},
%  %枠 "t"は上に線を記載, "T"は上に二重線を記載
%  %他オプション:leftline,topline,bottomline,lines,single,shadowbox
%  frame = TBrl,
%  %frameまでの間隔(行番号とプログラムの間)
%  framesep = 5pt,
%  %行番号の位置
%  numbers = left,
%  %行番号の間隔
%  stepnumber = 1,
%  %行番号の書体
%%  numberstyle = \tiny,
%  %タブの大きさ
%  tabsize = 4,
%  %キャプションの場所("tb"ならば上下両方に記載)
%  captionpos = t
%}

%\usepackage{cancel}
%\usepackage{bussproofs}
%\usepackage{proof}

\begin{document}

\hrulefill

$V$を有限次ベクトル空間とする。
$f:V\to V$を線形写像とし、
$f\circ f =0$(零写像)とする。

\hrulefill

$\Img{f} \subset \Ker{f}$
である。

\dotfill

$\Img{f} \subset V$である。

$f\circ f =0$ より
${}^{\forall} \alpha \in \Img{f}$に対して$f(\alpha)=0$
である。

つまり、$\alpha \in \Ker{f}$である。

よって、
$\Img{f} \subset \Ker{f}$
である。

\hrulefill

$\dim{V} \geq 2 \dim{\Img{f}}$
である。

\dotfill

\textbf{ベクトル空間の次元定理}
\begin{equation}
 \dim{V}
  = \dim{\Img{f}} + \dim{\Ker{f}}
\end{equation}

上記証明より、
$\Img{f} \subset \Ker{f}$
であるので、
次元を比較すると次が得られる。
\begin{equation}
 \dim{\Img{f}} \leq  \dim{\Ker{f}}
\end{equation}

これを利用すると次の式が成り立つ。
\begin{equation}
  \dim{\Img{f}} + \dim{\Ker{f}}
  \geq \dim{\Img{f}} + \dim{\Img{f}}
\end{equation}

左辺は次元定理より$\dim{V}$となるので、
次が示せた。
\begin{equation}
 \dim{V} \geq 2 \dim{\Img{f}}
\end{equation}


\hrulefill

\end{document}
