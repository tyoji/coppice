\documentclass[12pt,b5paper]{ltjsarticle}

%\usepackage[margin=15truemm, top=5truemm, bottom=5truemm]{geometry}
%\usepackage[margin=10truemm,left=15truemm]{geometry}
\usepackage[margin=10truemm]{geometry}

\usepackage{amsmath,amssymb}
%\pagestyle{headings}
\pagestyle{empty}

%\usepackage{listings,url}
%\renewcommand{\theenumi}{(\arabic{enumi})}

%\usepackage{graphicx}

%\usepackage{tikz}
%\usetikzlibrary {arrows.meta}
%\usepackage{wrapfig}
%\usepackage{bm}

% ルビを振る
%\usepackage{luatexja-ruby}	% required for `\ruby'

%% 核Ker 像Im Hom を定義
%\newcommand{\Img}{\mathop{\mathrm{Im}}\nolimits}
%\newcommand{\Ker}{\mathop{\mathrm{Ker}}\nolimits}
%\newcommand{\Hom}{\mathop{\mathrm{Hom}}\nolimits}

%\DeclareMathOperator{\Rot}{rot}
%\DeclareMathOperator{\Div}{div}
%\DeclareMathOperator{\Grad}{grad}
%\DeclareMathOperator{\arcsinh}{arcsinh}
%\DeclareMathOperator{\arccosh}{arccosh}
%\DeclareMathOperator{\arctanh}{arctanh}



%\usepackage{listings,url}
%
%\lstset{
%%プログラム言語(複数の言語に対応,C,C++も可)
%  language = Python,
%%  language = Lisp,
%%  language = C,
%  %背景色と透過度
%  %backgroundcolor={\color[gray]{.90}},
%  %枠外に行った時の自動改行
%  breaklines = true,
%  %自動改行後のインデント量(デフォルトでは20[pt])
%  breakindent = 10pt,
%  %標準の書体
%%  basicstyle = \ttfamily\scriptsize,
%  basicstyle = \ttfamily,
%  %コメントの書体
%%  commentstyle = {\itshape \color[cmyk]{1,0.4,1,0}},
%  %関数名等の色の設定
%  classoffset = 0,
%  %キーワード(int, ifなど)の書体
%%  keywordstyle = {\bfseries \color[cmyk]{0,1,0,0}},
%  %表示する文字の書体
%  %stringstyle = {\ttfamily \color[rgb]{0,0,1}},
%  %枠 "t"は上に線を記載, "T"は上に二重線を記載
%  %他オプション:leftline,topline,bottomline,lines,single,shadowbox
%  frame = TBrl,
%  %frameまでの間隔(行番号とプログラムの間)
%  framesep = 5pt,
%  %行番号の位置
%  numbers = left,
%  %行番号の間隔
%  stepnumber = 1,
%  %行番号の書体
%%  numberstyle = \tiny,
%  %タブの大きさ
%  tabsize = 4,
%  %キャプションの場所("tb"ならば上下両方に記載)
%  captionpos = t
%}



\begin{document}

\hrulefill
\begin{enumerate}
 \item
      $S_3$を$3$次置換群とする。
      $f$を$S_3$から$\mathbb{Z}/19\mathbb{Z}$への
      準同型写像とする。
      
      このとき、任意の$\sigma\in S_3$に対して、
      $f(\sigma)=\bar{0}$であることを示せ。
      ただし、
      $\mathbb{Z}/19\mathbb{Z}$の元のことを
      $\bar{x}\ (x\in\mathbb{Z})$のように書く。

      \dotfill

      ${}^{\forall}\sigma\in S_{3}$に対して、
      $\sigma^{6}=e$ (単位元) である。

      $f$は
      加法群$\mathbb{Z}/19\mathbb{Z}$への
      準同型であるので、
      $f(e)=\bar{0}$である。
      $f(\sigma^6)=6f(\sigma)$であるので、
      $6f(\sigma)=\bar{0}$である。

      ${}^{\forall}\bar{m}\in\mathbb{Z}/19\mathbb{Z}$に対して、
      $\bar{m}$の位数は$19$の約数である。

      $f(\sigma)\in\mathbb{Z}/19\mathbb{Z}$
      であるので、位数は19の約数だが、
      $6f(\sigma)=\bar{0}$であるので、
      $f(\sigma)$の位数は6と19の公約数となる。

      よって、位数が1となることがわかるので、
      $f(\sigma)=\bar{0}$である。

      \hrulefill
 \item
      群$\mathbb{Z}/8\mathbb{Z}$を考える。
      $\mathbb{Z}/8\mathbb{Z}$の元のことを
      $\bar{x}\ (x\in\mathbb{Z})$のように書く。

      $\bar{a}\in\mathbb{Z}/8\mathbb{Z}$に対して、
      群の準同型$f_{\bar{a}}$を次のように定義する。
      \begin{equation}
       f_{\bar{a}} : \mathbb{Z}/8\mathbb{Z} \to \mathbb{Z}/8\mathbb{Z}
        ,\qquad
        \bar{x} \mapsto \bar{a}\cdot \bar{x} (=\overline{ax})
      \end{equation}

      このとき、
      $f_{\bar{a}}$が
      群$\mathbb{Z}/8\mathbb{Z}$の自己同型写像になるような
      $\bar{a}\in\mathbb{Z}/8\mathbb{Z}$は合計何個あるか
      答えよ。

      \P
      群$\mathbb{Z}/8\mathbb{Z}$の自己同型写像とは、
      $\mathbb{Z}/8\mathbb{Z}$から
      $\mathbb{Z}/8\mathbb{Z}$への
      群の同型写像のこと。

      \dotfill

      $\mathbb{Z}/8\mathbb{Z}=\{\bar{0},\bar{1},\dots,\bar{7}\}$
      は有限の加法群である。
      よって、$f_{\bar{a}}$が単射であれば全射となるので、
      全単射であることがいえる。

      よって、単射である性質
      $\mathrm{Ker}f_{\bar{a}} = \{\bar{0}\}$
      について調べる。

      $f_{\bar{a}}(\bar{0})=\bar{a}\bar{0}=\bar{0}$
      であるので、$\bar{x}\ne \bar{0}$に対して
      $f_{\bar{a}}(\bar{x})\ne \bar{0}$となればよい。

      $\bar{0}$は8の倍数であるので、
      $\bar{x}=\bar{1},\dots,\bar{7}$に対して
      $ax$が8の倍数にならないときの$a$を求めればよい。

      つまり、奇数の場合
      $f_{\bar{1}},f_{\bar{3}},f_{\bar{5}},f_{\bar{7}}$
      は自己同型写像である。

      $f_{\bar{0}}(\bar{1})=\bar{0},f_{\bar{2}}(\bar{4})=\bar{0},
      f_{\bar{4}}(\bar{2})=\bar{0},f_{\bar{6}}(\bar{4})=\bar{0}$
      であるので、これらは単射ではない。

      よって、自己同型写像となるのは4つ
      $f_{\bar{1}},f_{\bar{3}},f_{\bar{5}},f_{\bar{7}}$
      である。


\end{enumerate}

\hrulefill

\end{document}
