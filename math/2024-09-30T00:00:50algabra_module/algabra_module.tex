\documentclass[12pt,b5paper]{ltjsarticle}

%\usepackage[margin=15truemm, top=5truemm, bottom=5truemm]{geometry}
%\usepackage[margin=10truemm,left=15truemm]{geometry}
\usepackage[margin=10truemm]{geometry}

\usepackage{amsmath,amssymb}
%\pagestyle{headings}
\pagestyle{empty}

%\usepackage{listings,url}

% 記号変更
%\renewcommand{\theenumi}{(\arabic{enumi})}
\renewcommand{\labelenumi}{(\arabic{enumi})}


%\usepackage{graphicx}

\usepackage{tikz} % draw graphics : TikZ ist kein Zeichenprogramm
\usetikzlibrary{arrows.meta}

%\usepackage{wrapfig}
%\usepackage{bm} % ベクトルの矢印

%\usepackage{luatexja-ruby} % ルビを振る

%% 核Ker 像Im Hom を定義
\newcommand{\Ker}{\mathop{\mathrm{Ker}}\nolimits}
\newcommand{\Img}{\mathop{\mathrm{Img}}\nolimits}
%\newcommand{\Hom}{\mathop{\mathrm{Hom}}\nolimits}

%\DeclareMathOperator{\Rot}{rot}
%\DeclareMathOperator{\Div}{div}
%\DeclareMathOperator{\Grad}{grad}
%\DeclareMathOperator{\arcsinh}{arcsinh}
%\DeclareMathOperator{\arccosh}{arccosh}
%\DeclareMathOperator{\arctanh}{arctanh}

\usepackage{url} % URLの記述

%\usepackage{listings}
%
%\lstset{
%%プログラム言語(複数の言語に対応,C,C++も可)
%  language = Python,
%%  language = Lisp,
%%  language = C,
%  %背景色と透過度
%  %backgroundcolor={\color[gray]{.90}},
%  %枠外に行った時の自動改行
%  breaklines = true,
%  %自動改行後のインデント量(デフォルトでは20[pt])
%  breakindent = 10pt,
%  %標準の書体
%%  basicstyle = \ttfamily\scriptsize,
%  basicstyle = \ttfamily,
%  %コメントの書体
%%  commentstyle = {\itshape \color[cmyk]{1,0.4,1,0}},
%  %関数名等の色の設定
%  classoffset = 0,
%  %キーワード(int, ifなど)の書体
%%  keywordstyle = {\bfseries \color[cmyk]{0,1,0,0}},
%  %表示する文字の書体
%  %stringstyle = {\ttfamily \color[rgb]{0,0,1}},
%  %枠 "t"は上に線を記載, "T"は上に二重線を記載
%  %他オプション:leftline,topline,bottomline,lines,single,shadowbox
%  frame = TBrl,
%  %frameまでの間隔(行番号とプログラムの間)
%  framesep = 5pt,
%  %行番号の位置
%  numbers = left,
%  %行番号の間隔
%  stepnumber = 1,
%  %行番号の書体
%%  numberstyle = \tiny,
%  %タブの大きさ
%  tabsize = 4,
%  %キャプションの場所("tb"ならば上下両方に記載)
%  captionpos = t
%}

%\usepackage{cancel}
%\usepackage{bussproofs}
%\usepackage{proof}

\begin{document}


\hrulefill

$R$を可換環とする。
左$R$-加群$M$が与えられたとき、
$M \times R \to M$を$(x,a) \mapsto ax$で定めると、
$M$は右$R$-加群にもなることを示せ。
また、
$M$は両側$(R,R)$-加群にもなることを示せ。


\dotfill


\textbf{定義 加群}

加法群$M$に対して、環$R$との演算が次のように定義されているとする。
\begin{equation}\label{eq:def_module}
 R \times M \to M, \quad (r,m) \mapsto rm
\end{equation}
この演算が次の性質を満たすとき、
$M$は左$R$-加群であると言う。
\begin{enumerate}
 \item $1m=m \quad (1\in R, m\in M)$
 \item $a(bm)=(ab)m \quad (a,b\in R, m\in M)$
 \item $(a+b)m=am+bm \quad (a,b\in R, m\in M)$
 \item $a(m+n)=am+an \quad (a\in R, m,n\in M)$
\end{enumerate}

式\eqref{eq:def_module}の演算の左右を入れ替えることで
右$R$-加群が定義できる。


\dotfill

環$R$との演算の定義は
$(x,a) \mapsto ax$であるので、
加群の定義の性質を確認する。

右からの外部演算を$(x,a)$と書くこととする。
\begin{itemize}
 \item
      $1\in R,\; x\in M$について
      $(x,1) = 1x$であり、
      $1x$は左$R$-加群で定義されているので、
      $(x,1)=1x=x$である。
 \item
      $a,b \in R, \; x\in M$について
      $((x,b),a)=(bx,a)=abx$である。
      また、
      $ab$のと演算は
      $(x,ba)=bax$であるが、
      $R$は可換環であるので、
      $(x,ba)=bax=abx$である。
      つまり、
      $((x,b),a)=abx=(x,ab)$
      である。
 \item
      $a,b \in R, \; x\in M$について
      $(x, a+b)=(a+b)x$である。
      $(x,a)=ax,\; (x,b)=bx$より
      $(x, a+b)=(a+b)x=ax+bx=(x,a)+(x,b)$
      である。
 \item
      $a \in R, \; x,y\in M$について
      $(x+y, a)=a(x+y)$であり、
      $(x,a)=ax,\;(y,a)=ay$であるから
      $(x+y, a)=a(x+y)=ax+ay=(x,a)+(y,a)$である。
\end{itemize}

これにより右側からの演算が定義されるので右加群である。

$M$は
左$R$-加群であり
右$R$-加群でもある。
$a,b\in R,\;x\in M$として、
$a(x,b)$と$(ax,b)$について考える。
\begin{itemize}
 \item
      $a(x,b)=abx$
 \item
      $(ax,b)=bax$
\end{itemize}

$R$が可換環であるので、
$ba=ab$である。
その為、
$a(x,b)=abx=bax=(ax,b)$
となり、
両側加群となる。




\hrulefill

$M$を左$R$-加群とし、
$N \subset M$を左$R$-部分加群とする。
集合$M / N = \{ x+N \mid x\in M\}$に対して、
\begin{align}
 \label{eq:def_operation1}
 (x+N)+(y+N) &:= (x+y) +N && (x,y\in N)\\
 \label{eq:def_operation2}
 a(x+N) &:= ax + N && (a\in R, x\in N)
\end{align}
と定めると$M/N$は左$R$-加群になる。
これを左$R$-剰余加群(quotient module)と呼ぶ。

\begin{enumerate}
 \item $x + N = y + N \Leftrightarrow x-y \in N \; (x,y\in N)$を示せ。
 \item 上の2つの演算が well-defined であることを示せ。
\end{enumerate}

\dotfill

$x,y\in N$とする。

$x+N$は
$N$の元と$x$との和全体の集合である。

\begin{itemize}
 \item
      $x + N = y + N \Rightarrow x-y \in N$

      $x + N = y + N$は
      $n_{x}, n_{y}\in N$が存在し、
      $x+n_{x}=y+n_{y}$となることを意味する。
      両辺を移項することで、
      $x-y=n_{y}-n_{x}$が得られる。

      $N$は部分加群であるから
      $n_{y}-n_{x}\in N$であるので、
      $x-y\in N$である。

 \item
      $x + N = y + N \Leftarrow x-y \in N$

      $x-y \in N$より
      ある$n\in N$が存在し、
      $x-y=n$である。
      これにより、
      $x=y+n,\;y=x-n$が得られる。

      任意の$n_{x}\in N$に対して、
      $x+n_{x} = y+n+n_{x}$である。
      $n+n_{x} \in N$より
      $x+n_{x} \in y+N$である。
      つまり、
      $x+N \subset y+N$である。

      同様に
      任意の$n_{y}\in N$に対して、
      $y+n_{y} = x-n+n_{y}$である為、
      $y+n_{y} \in x+N$であり、
      $y+N \subset x+N$である。

      よって、
      $x+N = y+N$である。
\end{itemize}

よって、
$x + N = y + N \Longleftrightarrow x-y \in N$
である。


式\eqref{eq:def_operation1}、
式\eqref{eq:def_operation2}
で定義された演算が well-defined を示す。

具体的には
剰余類$x+N$の代表元の取り方に依存せず演算が成り立つことを確認する。


式\eqref{eq:def_operation1}
$(x+N)+(y+N) := (x+y) +N$

$x,\;y\in M$に対して、
$x^{\prime},\;y^{\prime} \in M$が存在し、
$x+N = x^{\prime} +N ,\; y+N = y^{\prime} +N$
とする。
次のように演算が定義されている。
\begin{gather}
 \label{eq:prof01}
  (x+N)+(y+N) = (x+y) +N\\
  \label{eq:prof02}
  (x^{\prime}+N)+(y^{\prime}+N) = (x^{\prime}+y^{\prime}) +N
\end{gather}
先程の証明より次が成り立つ。
\begin{gather}
 x+N = x^{\prime} +N \Longleftrightarrow x-x^{\prime} \in N\\
 y+N = y^{\prime} +N \Longleftrightarrow y-y^{\prime} \in N
\end{gather}

式\eqref{eq:prof01}と
式\eqref{eq:prof02}
の左辺は等しいため、
右辺が等しくなることを示せばよい。

$N$は部分加群であり、
$x-x^{\prime} \in N,\;y-y^{\prime} \in N$
より
$x-x^{\prime} +  y-y^{\prime} \in N$
である。
%つまり、$x+y -x^{\prime}-y^{\prime} \in N$である。

$x^{\prime},\;y^{\prime}$の
マイナス元が
$-x^{\prime},\;-y^{\prime}$
である為、
$x^{\prime}+y^{\prime}$の
マイナス元が
$-x^{\prime}-y^{\prime}$
である。
つまり、
$-(x^{\prime}+y^{\prime})=-x^{\prime}-y^{\prime}$
である。

この為、
$x-x^{\prime} +  y-y^{\prime} = (x+y) -(x^{\prime}+y^{\prime}) \in N$
であり、
$(x+y)+N =  (x^{\prime}+y^{\prime}) + N$
であることが言える。




\hrulefill

$M,N$を左$R$-加群とし、
$f:M\to N$を準同型とする。
このとき、
\begin{gather}
 \Ker{f} := \{ x\in M \mid f(x) = 0\}\\
 \Img{f} := \{ f(x) \in N \mid x\in M\}
\end{gather}
とおき、
それぞれを
$f$の核 (kernel)、
$f$の像 (image)
と呼ぶ。

\begin{enumerate}
 \item $\Ker{f}$は$M$の左$R$-部分加群であることを示せ。
 \item $\Img{f}$は$N$の左$R$-部分加群であることを示せ。
\end{enumerate}

\dotfill

\begin{enumerate}
 \item $\Ker{f}$は$M$の左$R$-部分加群であることを示す。

       $x,\;y\in\Ker{f}$とする。
       \begin{itemize}
        \item $f(x+y)=f(x)+f(y)=0+0=0$より$x + y\in\Ker{f}$である。
        \item $f(0)=f(0+0)=f(0)+f(0)$より$f(0)=0$となり、$0\in\Ker{f}$である。
        \item $f(0)=f(x+(-x))=f(x)+f(-x)$であり、
              $f(x)=0$より$-x\in\Ker{f}$である。
       \end{itemize}
       以上により$\Ker{f}$は加法群である。

       $\Ker{f} \subset M$より、$R$と$\Ker{f}$の演算は$M$の元となる。
       $r\in R,\; x\in\Ker{f}$とすると、
       $f(rx)=rf(x)=r0=0$より$rx\in\Ker{f}$である。

       他の性質は$M$は加群であり、
       $\Ker{f}$は$M$の部分加法群であることから従う。

       よって、$\Ker{f}$は$M$の左$R$-部分加群である。

 \item $\Img{f}$は$N$の左$R$-部分加群であることを示す。

       $x,\;y\in\Img{f}$とする。
       $x_{0},y_{0}\in M$が存在し、$x=f(x_{0}),y=f(y_{0})$である。
       \begin{itemize}
        \item $x+y=f(x_{0})+f(y_{0})=f(x_{0}+y_{0})$より$x + y\in\Img{f}$である。
        \item $f(0)=f(0+0)=f(0)+f(0)$より$f(0)=0$となり、$0\in\Img{f}$である。
        \item $0=f(0)=f(x_{0}+(-x_{0}))=f(x_{0})+f(-x_{0})=x+f(-x_{0})$であるため、
              $-x=f(-x_{0})$より$-x\in\Img{f}$である。
       \end{itemize}
       以上により$\Img{f}$は加法群である。

       $\Img{f} \subset N$より、$R$と$\Img{f}$の演算は$N$の元となる。
       $r\in R,\; x\in\Img{f}$とすると、
       $rx=rf(x_{0})=f(rx_{0})$であり、
       $rx_{0}\in M$より$rx\in\Img{f}$である。

       $\Img{f}$は$N$の部分加法群であり、
       $R$との演算が$N$の中で成り立っている為、
       その他の性質も従う。

       よって、$\Img{f}$は$N$の左$R$-部分加群である。
\end{enumerate}



\hrulefill

\textbf{準同型定理}

$M,N$を左$R$-加群とし、
$f:M\to N$を準同型とする。
このとき、
以下の写像は (well-defined であり)同型写像である。
\begin{equation}
 \bar{f} : M/ \Ker{f} \to \Img{f}, \quad x+\Ker{f} \mapsto f(x)
\end{equation}
特に、
$M/\Ker{f} \cong \Img{f}$
である。


この定理を示せ。

\dotfill

写像$f$は$f:M\to\Img{f}$とすると全射である。
$f=\bar{f}\circ g$となるように
$g:M\to M/\Ker{f}$とすると$g$も全射である。
\begin{equation}
 M \overset{g}{\longrightarrow} M / \Ker{f} \overset{\bar{f}}{\longrightarrow}  \Img{f}
\end{equation}
$f=\bar{f}\circ g$であるので、
$\bar{f}$も全射である。

$x,y\in \Img{f}$が$x=y$とする。
$x_{0},y_{0}\in M$が存在し、
$x=f(x_{0}),\;y=f(y_{0})$である。

$x_{0}-y_{0}\in M$であり、
$f(x_{0}-y_{0})=f(x_{0})+f(-y_{0})=x+(-y)=0$である。
つまり、
$x_{0}-y_{0}\in \Ker{f}$である。

$x_{0} +\Ker{f} = y_{0}+\Ker{f}$であることがわかる為、
$\bar{f}(x_{0}+\Ker{f})=f(x_{0})=x$、
$\bar{f}(y_{0}+\Ker{f})=f(y_{0})=y$
から、
$\bar{f}$は単射であることがわかる。

よって、
$\bar{f}$は全単射である。

その為、
$M/\Ker{f}$と$\Img{f}$は同型であり、
$M/\Ker{f} \cong \Img{f}$である。



\hrulefill

\begin{itemize}
 \item
      空でない集合$X$に対して、
      $S_{X}$を
      $X$から$X$への全単射全体の集合とする。
      写像の合成を演算として
      $S_{X}$は群をなす。

 \item
      群$G$と集合$X$に対して、
      写像$G \times X \to X, \; (g,x) \mapsto gx$
      が与えられていて、
      \begin{enumerate}
       \item $g(hx) = (gh)x \; (g,h \in G, \; x \in X)$
       \item $ex=x \; (x\in X, e は G の単位元)$
      \end{enumerate}
      が成り立つとき、
      $G$は$X$に左から作用するという。
\end{itemize}

群$G$が集合$X$に作用することと、
群の準同型$G\to S_{X}$が存在することが
同値であることを示せ。

\dotfill

$S_{X}$は$X\to X$の恒等写像を単位元とし、
$f\in S_{X}$に対し、逆写像$f^{-1}\in S_{X}$を逆元とする。


\begin{itemize}
 \item 群$G$が集合$X$に作用しているとする。

       $g\in G$に対し、$x\mapsto gx$と対応させる写像を$f_{g}$とする。
       このとき、
       $f_{g^{-1}}$は$f_{g}$の逆写像となり、
       $f_{g^{-1}} \circ f_{g} = f_{g} \circ f_{g^{-1}} = id_{X}$である。

       群$G$の元は全て逆元を持つから$f_{g}$は全単射となり、
       $f_{g}\in S_{X}$となる。

       つまり次のような準同型写像が存在する。
       \begin{equation}
        G \to S_{X} , \; g \mapsto f_{g}
       \end{equation}

 \item 群の準同型$G\to S_{X}$が存在するとする。

       \begin{equation}
        f: G \to S_{X}, \; g \mapsto f_{g}
       \end{equation}

       $f_{g}:X\to X$は全単射である。
       単位元$e\in G$に対し、$f_{e}$は恒等写像であり、
       $g^{-1}\in G$に対し、$f_{g^{-1}}$は$f_{g}$の逆写像である。

       \begin{equation}
        G\times X \to X , \; (g,x) \mapsto f_{g}(x)
       \end{equation}
       とする。

       $f_{e}$は恒等写像であるので、$f_{e}(x)=x$である。

       $g,h\in G$に対し、
       準同型$G\to S_{X}$より$f_{g}\circ f_{h}=f_{gh}$である。
       これにより、
       \begin{equation}
        f_{g}(f_{h}(x)) = (f_{g}\circ f_{h})(x)= f_{gh}(x)
       \end{equation}
       であるので、
       群$G$が集合$X$に作用している。

\end{itemize}

これらにより次の2つは同値であることがわかる。
\begin{itemize}
 \item 群$G$が集合$X$に作用する
 \item 群の準同型$G\to S_{X}$が存在する
\end{itemize}

\hrulefill









\end{document}
