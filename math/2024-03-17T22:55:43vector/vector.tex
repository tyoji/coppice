\documentclass[12pt,b5paper]{ltjsarticle}

%\usepackage[margin=15truemm, top=5truemm, bottom=5truemm]{geometry}
%\usepackage[margin=10truemm,left=15truemm]{geometry}
\usepackage[margin=10truemm]{geometry}

\usepackage{amsmath,amssymb}
%\pagestyle{headings}
\pagestyle{empty}

%\usepackage{listings,url}
\renewcommand{\theenumi}{(\arabic{enumi})}

%\usepackage{graphicx}

%\usepackage{tikz}
%\usetikzlibrary {arrows.meta}
%\usepackage{wrapfig}
%\usepackage{bm}

% ルビを振る
%\usepackage{luatexja-ruby}	% required for `\ruby'

%% 核Ker 像Im Hom を定義
%\newcommand{\Img}{\mathop{\mathrm{Im}}\nolimits}
%\newcommand{\Ker}{\mathop{\mathrm{Ker}}\nolimits}
%\newcommand{\Hom}{\mathop{\mathrm{Hom}}\nolimits}

%\DeclareMathOperator{\Rot}{rot}
%\DeclareMathOperator{\Div}{div}
%\DeclareMathOperator{\Grad}{grad}
%\DeclareMathOperator{\arcsinh}{arcsinh}
%\DeclareMathOperator{\arccosh}{arccosh}
%\DeclareMathOperator{\arctanh}{arctanh}

\usepackage{url}

%\usepackage{listings}
%
%\lstset{
%%プログラム言語(複数の言語に対応,C,C++も可)
%  language = Python,
%%  language = Lisp,
%%  language = C,
%  %背景色と透過度
%  %backgroundcolor={\color[gray]{.90}},
%  %枠外に行った時の自動改行
%  breaklines = true,
%  %自動改行後のインデント量(デフォルトでは20[pt])
%  breakindent = 10pt,
%  %標準の書体
%%  basicstyle = \ttfamily\scriptsize,
%  basicstyle = \ttfamily,
%  %コメントの書体
%%  commentstyle = {\itshape \color[cmyk]{1,0.4,1,0}},
%  %関数名等の色の設定
%  classoffset = 0,
%  %キーワード(int, ifなど)の書体
%%  keywordstyle = {\bfseries \color[cmyk]{0,1,0,0}},
%  %表示する文字の書体
%  %stringstyle = {\ttfamily \color[rgb]{0,0,1}},
%  %枠 "t"は上に線を記載, "T"は上に二重線を記載
%  %他オプション:leftline,topline,bottomline,lines,single,shadowbox
%  frame = TBrl,
%  %frameまでの間隔(行番号とプログラムの間)
%  framesep = 5pt,
%  %行番号の位置
%  numbers = left,
%  %行番号の間隔
%  stepnumber = 1,
%  %行番号の書体
%%  numberstyle = \tiny,
%  %タブの大きさ
%  tabsize = 4,
%  %キャプションの場所("tb"ならば上下両方に記載)
%  captionpos = t
%}

%\usepackage{cancel}
%\usepackage{bussproofs}
%\usepackage{proof}

\begin{document}

\hrulefill

三角形$ABC$について
$\lvert \overrightarrow{AB} \rvert =1,\;
\lvert \overrightarrow{AC} \rvert =2,\;
\lvert \overrightarrow{BC} \rvert =\sqrt{6}$
とし、
外接円の中心を$O$とする。
直線$AO$と
外接円との交点の内、
$A$以外の交点を$P$とする。

\begin{enumerate}
 \item
      $\overrightarrow{AB}$と
      $\overrightarrow{AC}$の
      内積$\overrightarrow{AB}\cdot\overrightarrow{AC}$
      を求めよ。
 \item
      $\overrightarrow{AP} = s\overrightarrow{AB} +t\overrightarrow{AC}$
      が
      成り立つような実数$s,t$を求めよ。
 \item
      直線$\overrightarrow{AP}$と
      直線$\overrightarrow{BC}$の
      交点を$D$とするとき、
      線分$AD$の長さ$\lvert AD \rvert$を求めよ。
\end{enumerate}

\hrulefill
\begin{enumerate}
 \item
      内積は次の式で求められる。
      \begin{equation}
       \overrightarrow{AB}\cdot\overrightarrow{AC}
        =
        \lvert \overrightarrow{AB} \rvert
        \lvert \overrightarrow{AC} \rvert
        \cos{\angle{BAC}}
      \end{equation}

      また、$\cos{\angle{BAC}}$は三角形$ABC$に対する余弦定理により
      次の式で求められる。
      \begin{equation}
       \cos{\angle{BAC}}
        =
        \frac{\lvert \overrightarrow{AB} \rvert^{2} + \lvert \overrightarrow{AC} \rvert^{2} - \lvert \overrightarrow{BC} \rvert^{2}}{2\lvert \overrightarrow{AB} \rvert\lvert \overrightarrow{AC} \rvert}
      \end{equation}

      これらを合わせると次が得られる。
      \begin{align}
       \overrightarrow{AB}\cdot\overrightarrow{AC}
        &=
        \frac{\lvert \overrightarrow{AB} \rvert^{2} + \lvert \overrightarrow{AC} \rvert^{2} - \lvert \overrightarrow{BC} \rvert^{2}}{2} \label{eq:inpro}\\
       &= \frac{1^{2}+2^{2}-\sqrt{6}^{2}}{2}
       = -\frac{1}{2}
      \end{align}

      \dotfill

      \textbf{別解}

      $\overrightarrow{BC} = \overrightarrow{AC} - \overrightarrow{AB}$
      であるので、次の式が得られる。
      \begin{equation}
       \lvert \overrightarrow{BC} \rvert^{2}
       = \lvert \overrightarrow{AC} - \overrightarrow{AB} \rvert^{2}
       = \lvert \overrightarrow{AC} \rvert^{2} -2\overrightarrow{AB}\cdot\overrightarrow{AC} + \lvert \overrightarrow{AB} \rvert^{2}
      \end{equation}
      これにより式\eqref{eq:inpro}が得られる。

      \hrulefill

 \item
      線分$AP$は外接円の直径である。
      その為、$\angle{ABP}=\angle{ACP}=\pi/2$であり、
      $\overrightarrow{AB} \cdot \overrightarrow{BP}
      = \overrightarrow{AC} \cdot \overrightarrow{CP} = 0$
      である。

      $\overrightarrow{AP} = s \overrightarrow{AB} + t \overrightarrow{AC}\;(s,t\in\mathbb{R})$
      より$\overrightarrow{BP},\overrightarrow{CP}$は
      次の様に$\overrightarrow{AB}$と$\overrightarrow{AC}$で表せる。
      \begin{align}
       \overrightarrow{BP}
        &= \overrightarrow{BA} + \overrightarrow{AP}
        = - \overrightarrow{AB} + s \overrightarrow{AB} + t \overrightarrow{AC}
        = (s-1) \overrightarrow{AB} + t \overrightarrow{AC}\\
       \overrightarrow{CP}
        &= \overrightarrow{CA} + \overrightarrow{AP}
        = - \overrightarrow{AC} + s \overrightarrow{AB} + t \overrightarrow{AC}
        = s \overrightarrow{AB} + (t-1) \overrightarrow{AC}
      \end{align}

      ここから内積を計算する。
      \begin{equation}
       \overrightarrow{AB} \cdot \overrightarrow{BP}
        = s-1 - \frac{t}{2}
        ,\qquad
       \overrightarrow{AC} \cdot \overrightarrow{CP}
        = -\frac{s}{2} +4(t-1)
      \end{equation}

      内積が0となるため、
      $s,t\in\mathbb{R}$は次の連立方程式を満たす。
      \begin{equation}
       \begin{cases}
        s-1 - \frac{t}{2} =0 \\
        -\frac{s}{2} +4(t-1) =0
       \end{cases}
      \end{equation}

      これを解くと
      $(s,t)=(\frac{8}{5},\;\frac{6}{5})$
      を得る。
      \begin{equation}
       \overrightarrow{AP} = \frac{8}{5} \overrightarrow{AB} + \frac{6}{5} \overrightarrow{AC}
      \end{equation}




%      \begin{align}
%       \lvert \overrightarrow{AP} \rvert^{2}
%       &= s^{2} \lvert \overrightarrow{AB} \rvert^{2}
%       +2st \overrightarrow{AB} \cdot \overrightarrow{AC}
%       + t^{2} \lvert \overrightarrow{AC} \rvert^{2}
%       = s^{2}-st+4t^{2}
%       \\
%       \overrightarrow{AB} \cdot \overrightarrow{AP}
%       &= s \overrightarrow{AB} \cdot \overrightarrow{AB}
%       + t \overrightarrow{AB} \cdot \overrightarrow{AC}
%       = s -\frac{t}{2}
%       \\
%       \overrightarrow{AC} \cdot \overrightarrow{AP}
%       &= s \overrightarrow{AC} \cdot \overrightarrow{AB}
%       + t \overrightarrow{AC} \cdot \overrightarrow{AC}
%       = -\frac{s}{2} +4t
%      \end{align}

      \hrulefill

 \item
      点$D$は辺$BC$上の点であるので、
      $0 < u < 1$となる実数を用いて次の様に表せる。
      \begin{equation}
       \overrightarrow{AD} = (1-u)\overrightarrow{AB} + u\overrightarrow{AC}
        \label{eq:ad1}
      \end{equation}


      また、点$D$は辺$AP$上の点であるので、
      $0 < v < 1$となる実数を用いて
      $\overrightarrow{AD} = v\overrightarrow{AP}$
      である。

      前問より$\overrightarrow{AD}$は次の様に表せる。
      \begin{equation}
       \overrightarrow{AD}
        = v\overrightarrow{AP}
        = \frac{8v}{5} \overrightarrow{AB} + \frac{6v}{5} \overrightarrow{AC}
        \label{eq:ad2}
      \end{equation}

      式\eqref{eq:ad1}、
      式\eqref{eq:ad2}
      より次の連立方程式が得られる。
      \begin{equation}
       \begin{cases}
        1-u &= \frac{8v}{5}\\
        u &= \frac{6v}{5}
       \end{cases}
      \end{equation}

      これを解くと
      $(u,v) = (\frac{3}{7}, \frac{5}{14})$
      である。

      $\overrightarrow{AD}$は次の様に表せる。
      \begin{equation}
       \overrightarrow{AD} = \frac{4}{7}\overrightarrow{AB} + \frac{3}{7}\overrightarrow{AC}
      \end{equation}

      $\lvert\overrightarrow{AD}\rvert^{2}$を計算する。
      \begin{align}
       \lvert \overrightarrow{AD} \rvert^{2}
        &= \left\lvert \frac{4}{7}\overrightarrow{AB}
       + \frac{3}{7}\overrightarrow{AC} \right\rvert^{2}\\
        &= \left(\frac{4}{7}\right)^{2} \lvert \overrightarrow{AB} \rvert^{2}
        + 2 \cdot \frac{4}{7} \cdot \frac{3}{7} \overrightarrow{AB} \cdot \overrightarrow{AC}
        + \left(\frac{3}{7}\right)^{2} \lvert \overrightarrow{AC} \rvert^{2}\\
        &= \left(\frac{4}{7}\right)^{2}
        - \frac{4}{7} \cdot \frac{3}{7}
        + \left(\frac{3}{7}\right)^{2} \cdot 2^{2}
        = \frac{40}{7^2}
      \end{align}

      これにより$\lvert\overrightarrow{AD}\rvert$が求まる。
      \begin{equation}
       \lvert\overrightarrow{AD}\rvert = \frac{2}{7}\sqrt{10}
      \end{equation}

\end{enumerate}
\hrulefill

\end{document}
