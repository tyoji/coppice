\documentclass[12pt,b5paper]{ltjsarticle}

%\usepackage[margin=15truemm, top=5truemm, bottom=5truemm]{geometry}
%\usepackage[margin=10truemm,left=15truemm]{geometry}
\usepackage[margin=10truemm]{geometry}

\usepackage{amsmath,amssymb}
%\pagestyle{headings}
\pagestyle{empty}

%\usepackage{listings,url}
%\renewcommand{\theenumi}{(\arabic{enumi})}

%\usepackage{graphicx}

%\usepackage{tikz}
%\usetikzlibrary {arrows.meta}
%\usepackage{wrapfig}
%\usepackage{bm}

% ルビを振る
\usepackage{luatexja-ruby}	% required for `\ruby'

%% 核Ker 像Im Hom を定義
%\newcommand{\Img}{\mathop{\mathrm{Im}}\nolimits}
%\newcommand{\Ker}{\mathop{\mathrm{Ker}}\nolimits}
%\newcommand{\Hom}{\mathop{\mathrm{Hom}}\nolimits}

%\DeclareMathOperator{\Rot}{rot}
%\DeclareMathOperator{\Div}{div}
%\DeclareMathOperator{\Grad}{grad}
%\DeclareMathOperator{\arcsinh}{arcsinh}
%\DeclareMathOperator{\arccosh}{arccosh}
%\DeclareMathOperator{\arctanh}{arctanh}

%\usepackage{url}

%\usepackage{listings}
%
%\lstset{
%%プログラム言語(複数の言語に対応,C,C++も可)
%  language = Python,
%%  language = Lisp,
%%  language = C,
%  %背景色と透過度
%  %backgroundcolor={\color[gray]{.90}},
%  %枠外に行った時の自動改行
%  breaklines = true,
%  %自動改行後のインデント量(デフォルトでは20[pt])
%  breakindent = 10pt,
%  %標準の書体
%%  basicstyle = \ttfamily\scriptsize,
%  basicstyle = \ttfamily,
%  %コメントの書体
%%  commentstyle = {\itshape \color[cmyk]{1,0.4,1,0}},
%  %関数名等の色の設定
%  classoffset = 0,
%  %キーワード(int, ifなど)の書体
%%  keywordstyle = {\bfseries \color[cmyk]{0,1,0,0}},
%  %表示する文字の書体
%  %stringstyle = {\ttfamily \color[rgb]{0,0,1}},
%  %枠 "t"は上に線を記載, "T"は上に二重線を記載
%  %他オプション:leftline,topline,bottomline,lines,single,shadowbox
%  frame = TBrl,
%  %frameまでの間隔(行番号とプログラムの間)
%  framesep = 5pt,
%  %行番号の位置
%  numbers = left,
%  %行番号の間隔
%  stepnumber = 1,
%  %行番号の書体
%%  numberstyle = \tiny,
%  %タブの大きさ
%  tabsize = 4,
%  %キャプションの場所("tb"ならば上下両方に記載)
%  captionpos = t
%}

%\usepackage{cancel}
%\usepackage{bussproofs}
%\usepackage{proof}

\begin{document}

%%%% %%%% %%%% %%%% %%%%
% 1
\hrulefill
\textbf{1}
\hrulefill

関数$f:[-\pi,\pi]\to\mathbb{C}$
を三角関数を用いて
$\displaystyle f(x)=\sum_{n\in\mathbb{Z}} \widehat{f}(n)e^{inx}$
とFourier級数展開を出来るか考える。

「$f$は可積分」、
「$\displaystyle \sum_{n\in\mathbb{Z}}\widehat{f}(n)$は絶対収束」
という仮定の下では
$\displaystyle \widehat{f}(n) = \frac{1}{2\pi}\int_{-\pi}^{\pi}f(x)e^{-inx}\mathrm{d}x$
と求めることが出来る。

これにより
$\displaystyle \sum_{n\in\mathbb{Z}}\widehat{f}(n)$
が収束するが、
級数は$f(x)$と一致するかは不明である。

関数$f$によっては、
そのFourier級数$\displaystyle \sum_{n\in\mathbb{Z}} \widehat{f}(n)e^{inx}$
が収束しない例もあり、
級数の収束するかと収束先が$f(x)$であるかを考える必要がある。

\hrulefill



%%%% %%%% %%%% %%%% %%%%
% 2
\hrulefill
\textbf{2}
\hrulefill

Poissonの定理
(${}^{\forall}f\in C_{per}[-\pi,\pi]$に対して、
$P_{r}f \overset{r \nearrow 1}{\longrightarrow} f$(一様収束))
を利用し、
$f\in C^{2}_{per}[-\pi,\pi]$ならば
フーリエ級数$f=\sum_{n\in\mathbb{Z}}\hat{f}(n)e^{inx}$に展開できる事が示せる。

これにより
$f\in C^{2}_{per}[-\pi,\pi]$は
Fourier展開が出来るための十分条件ではあるが、
$C^{2}_{per}[-\pi,\pi]$ではない関数でもFourier展開が可能なものもある。

\hrulefill


%%%% %%%% %%%% %%%% %%%%
% 3
\hrulefill
\textbf{3}
\hrulefill

$f\in C^{2}_{\mathrm{per}}[-\pi,\pi]$であれば、
Fourier展開
$\displaystyle f(x)=\sum_{n\in\mathbb{Z}} \widehat{f}(n)e^{inx}$
が出来るが、
$f\in C^{1}_{\mathrm{per}}[-\pi,\pi]$
の時も
このように展開ができる事が示せる。

まず、
$C_{\mathrm{per}}[-\pi,\pi]$に内積を定義し、
内積から距離を定義する。
%
この距離において
Besselの不等式
$\displaystyle \left\| \sum_{n=-M}^{N} \hat{f}(n)e_{n} \right\|^{2}_{2} \leq \|f\|^{2}_{2} \  (e_{n}(x) := e^{inx})$
が成立する。
この不等式を利用することで
$\displaystyle \sum_{n\in \mathbb{Z}} \hat{f}(n) < \infty$
ということが示せ、
$C^{1}$関数のFourier展開ができることがわかる。

$f\in C_{\mathrm{per}}[-\pi,\pi]$
の場合は
先程定義した距離においての収束を考えることで
Fourier展開ができる。

\hrulefill

\newpage

%%%% %%%% %%%% %%%% %%%%
% 4
\hrulefill
\textbf{4}
\hrulefill

$p<\infty$において
$L^{p}(\mu)$のコーシー列は
概収束する部分列を持つことから収束することがわかる。

$L^{\infty}(\mu)$のコーシー列は
$\mathrm{ess sup}$を考えることにより
収束することがわかる。
よって、$L^{p}(\mu)$は完備な距離空間である。

これにより、
$\displaystyle \sum_{n=-N}^{N}\hat{f}(n)e_{n}$は
距離$\| \cdot \|_{2}$により収束する。

%数列空間$\ell^{p}(\mathbb{N})$は完備である。

\hrulefill



%%%% %%%% %%%% %%%% %%%%
% 5
\hrulefill
\textbf{5}
\hrulefill

%
%$\{e_{n}\}_{n\in\mathbb{Z}}$の生成する$C_{\mathrm{per}}[-\pi,\pi]$の部分空間が
%ノルム$\|\cdot\|_{\infty}$において稠密である。
%
%${}^{\forall}f\in C[-1,1]$は多項式で一様近似できる。
%
%
%$p\in [1,\infty)$に対して
%$C_{c}^{\infty}(\mathbb{R}^{d})
% = \{ f: C^{\infty}級 かつ ある有界集合の外で0 \}$
% は
%$(L^{p}(\mathbb{R}^{d},\|\cdot\|_{p})$
%で稠密である。



$f \in L^{1}(\mathbb{R}^{d}), \; g\in L^{p}(\mathbb{R}^{d})$に対し、
畳み込み$f\ast g$ は $\|f\ast g\|_{p} \leq \|f\|_{1} \| g\|_{p}$を満たす。
また、
$f\in C_{c}^{\infty}(\mathbb{R}^{d}), \; g\in L^{p}(\mathbb{R}^{d})$に対し、
$f\ast g$は$C^{\infty}$級である。

これらを用いて
$p\in [1,\infty)$に対して
$C_{per}^{\infty}[-\pi,\pi]$
 は
$(L^{p}(-\pi,\pi),\|\cdot\|_{p})$
で稠密であることが示せる。
この稠密性より
$f\in L^{2}(-\pi,\pi)$の
Fourier級数の部分和$\sum_{n=-M}^{N}\widehat{f}(n)e_{n}$は
$M,N\to \infty$で$f$に$L^{2}$収束することが得られる。


\hrulefill



%%%% %%%% %%%% %%%% %%%%
% 6
\hrulefill
\textbf{6}
\hrulefill

$L^{2}(-\pi,\pi)$
は無限次元線形空間である。
ここにノルムを導入する。

ノルムから定義された位相を用いて
極限が定義できるので、
$\sum_{n=1}^{\infty}v_{n}s_{n}$
という式に意味をもたせられる。
%そこで、
任意の元$v$に対して
複素数列$(v_{n})_{n\in\mathbb{N}}$が唯一存在し
$\lim_{N\to\infty} \left\| v - \sum_{n=1}^{N}v_{n}s_{n} \right\| = 0$
を満たすとき、
$\{s_{n}\}_{n\in\mathbb{N}}$を
\ruby{Schauder}{シャウダー} 基底という。

有限次元の線形空間では
全てのノルムは同値であるが、
無限次元では成立しない為、
この極限はノルムに依存する。





\hrulefill

\end{document}
