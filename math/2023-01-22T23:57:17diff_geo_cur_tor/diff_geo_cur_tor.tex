\documentclass[12pt,b5paper]{ltjsarticle}

%\usepackage[margin=15truemm, top=5truemm, bottom=5truemm]{geometry}
%\usepackage[margin=10truemm,left=15truemm]{geometry}
\usepackage[margin=10truemm]{geometry}

\usepackage{amsmath,amssymb}
%\pagestyle{headings}
\pagestyle{empty}

%\usepackage{listings,url}
%\renewcommand{\theenumi}{(\arabic{enumi})}

%\usepackage{graphicx}

%\usepackage{tikz}
%\usetikzlibrary {arrows.meta}
%\usepackage{wrapfig}	% required for `\wrapfigure' (yatex added)
\usepackage{bm}	% required for `\bm' (yatex added)

% ルビを振る
%\usepackage{luatexja-ruby}	% required for `\ruby'

% カラー表示
\usepackage{color}

%% 核Ker 像Im Hom を定義
%\newcommand{\Img}{\mathop{\mathrm{Im}}\nolimits}
%\newcommand{\Ker}{\mathop{\mathrm{Ker}}\nolimits}
%\newcommand{\Hom}{\mathop{\mathrm{Hom}}\nolimits}

%\DeclareMathOperator{\Rot}{rot}
%\DeclareMathOperator{\Div}{div}
%\DeclareMathOperator{\Grad}{grad}
%\DeclareMathOperator{\arcsinh}{arcsinh}
%\DeclareMathOperator{\arccosh}{arccosh}
%\DeclareMathOperator{\arctanh}{arctanh}



%\usepackage{listings,url}
%
%\lstset{
%%プログラム言語(複数の言語に対応,C,C++も可)
%  language = Python,
%%  language = Lisp,
%%  language = C,
%  %背景色と透過度
%  %backgroundcolor={\color[gray]{.90}},
%  %枠外に行った時の自動改行
%  breaklines = true,
%  %自動改行後のインデント量(デフォルトでは20[pt])
%  breakindent = 10pt,
%  %標準の書体
%%  basicstyle = \ttfamily\scriptsize,
%  basicstyle = \ttfamily,
%  %コメントの書体
%%  commentstyle = {\itshape \color[cmyk]{1,0.4,1,0}},
%  %関数名等の色の設定
%  classoffset = 0,
%  %キーワード(int, ifなど)の書体
%%  keywordstyle = {\bfseries \color[cmyk]{0,1,0,0}},
%  %表示する文字の書体
%  %stringstyle = {\ttfamily \color[rgb]{0,0,1}},
%  %枠 "t"は上に線を記載, "T"は上に二重線を記載
%  %他オプション:leftline,topline,bottomline,lines,single,shadowbox
%  frame = TBrl,
%  %frameまでの間隔(行番号とプログラムの間)
%  framesep = 5pt,
%  %行番号の位置
%  numbers = left,
%  %行番号の間隔
%  stepnumber = 1,
%  %行番号の書体
%%  numberstyle = \tiny,
%  %タブの大きさ
%  tabsize = 4,
%  %キャプションの場所("tb"ならば上下両方に記載)
%  captionpos = t
%}



\begin{document}



\textbf{微分幾何}

\textbf{曲率と捩率}

\hrulefill

任意の媒介変数$t$による
空間曲線$\bm{x}(t)=(e^{t},e^{-t},\sqrt{2}t)$の
曲率と捩率を定義に従って求める。

%

まず、
$\displaystyle \frac{\mathrm{d}}{\mathrm{d}t}\bm{x}(t)=$\fbox{\textcolor{red}{\ref{pro01}}}
である。
よって、
$\displaystyle
 \left\lvert\frac{\mathrm{d}}{\mathrm{d}t}\bm{x}(t)\right\rvert
 =e^{t}+$\fbox{\textcolor{red}{\ref{pro02}}}
となるので、
$s$を弧長パラメーターとすると、
$\displaystyle \frac{\mathrm{d}s}{\mathrm{d}t}=e^{t}+$\fbox{\textcolor{red}{\ref{pro02}}}
である。
したがって、
$\displaystyle \bm{e}_{1}=\frac{\mathrm{d}}{\mathrm{d}s}\bm{x}(t)
=\left. \frac{\mathrm{d}\bm{x}(t)}{\mathrm{d}t} \middle/ \frac{\mathrm{d}s}{\mathrm{d}t} \right.
=$\fbox{\textcolor{red}{\ref{pro03}}}
となる。
次に、
$\displaystyle
 \frac{\mathrm{d}}{\mathrm{d}t}\bm{e}_{1}
 =\frac{(2,2,\fbox{\textcolor{red}{\ref{pro04}}})}{(e^{t}+e^{-t})^2}$
である。
これにより、
$\displaystyle \frac{\mathrm{d}}{\mathrm{d}s}\bm{e}_{1}
=\left. \frac{\mathrm{d}\bm{e}_{1}}{\mathrm{d}t} \middle/ \frac{\mathrm{d}s}{\mathrm{d}t} \right.=$\fbox{\textcolor{red}{\ref{pro05}}}
を得る。

%

よって、曲率
$\displaystyle \kappa = \left\lvert \frac{\mathrm{d}}{\mathrm{d}s}\bm{e}_{1} \right\rvert
 =\frac{\fbox{\textcolor{red}{\ref{pro06}}}}{(e^{t}+e^{-t})^2}$となる。
ここで、
$\displaystyle \bm{e}_{2}=\frac{ \frac{\mathrm{d}}{\mathrm{d}s}\bm{e}_{1} }{\kappa}$
、および
$\bm{e}_{3}=\bm{e}_{1}\times \bm{e}_{2}$
とおくと
$\displaystyle \bm{e}_{3}=\frac{(-e^{-t},\fbox{\textcolor{red}{\ref{pro07}}},\sqrt{2})}{e^{t}+e^{-t}}$
である。
さらに、
$\displaystyle
 \frac{\mathrm{d}}{\mathrm{d}t}\bm{e}_{3}
 =\frac{\fbox{\textcolor{red}{\ref{pro08}}}}{(e^{t}+e^{-t})^2}$
である。
$\displaystyle \frac{\mathrm{d}}{\mathrm{d}s}\bm{e}_{3}
= \left. \frac{\mathrm{d}\bm{e}_{3}}{\mathrm{d}t} \middle/ \frac{\mathrm{d}s}{\mathrm{d}t} \right.=-\tau \bm{e}_{2}$
なる$\tau$が捩率なので、
$\tau=\fbox{\textcolor{red}{\ref{pro09}}}$となる。



\hrulefill

\begin{enumerate}
 \item \label{pro01}
       $\frac{\mathrm{d}}{\mathrm{d}t}\bm{x}(t)=$\fbox{\textcolor{red}{\ref{pro01}}}
       \begin{equation}
        \frac{\mathrm{d}}{\mathrm{d}t}\bm{x}(t)=(e^{t},-e^{-t},\sqrt{2})
       \end{equation}
       \dotfill

 \item \label{pro02}
       $\left\lvert\frac{\mathrm{d}}{\mathrm{d}t}\bm{x}(t)\right\rvert
 =e^{t}+$\fbox{\textcolor{red}{\ref{pro02}}}
       \begin{equation}
        \left\lvert\frac{\mathrm{d}}{\mathrm{d}t}\bm{x}(t)\right\rvert
          =\sqrt{e^{2t}+e^{-2t}+2}
          =\sqrt{(e^{t}+e^{-t})^2}
          =e^{t}+e^{-t}
       \end{equation}
       \dotfill

 \item \label{pro03}
       $\bm{e}_{1}=\frac{\mathrm{d}}{\mathrm{d}s}\bm{x}(t)
       =\left. \frac{\mathrm{d}\bm{x}(t)}{\mathrm{d}t} \middle/ \frac{\mathrm{d}s}{\mathrm{d}t} \right.
       =$\fbox{\textcolor{red}{\ref{pro03}}}

       \begin{equation}
        \left. \frac{\mathrm{d}\bm{x}(t)}{\mathrm{d}t} \middle/ \frac{\mathrm{d}s}{\mathrm{d}t} \right.
         =\frac{1}{e^{t}+e^{-t}}(e^{t},-e^{-t},\sqrt{2})
       \end{equation}
       \dotfill

 \item \label{pro04}
       $\frac{\mathrm{d}}{\mathrm{d}t}\bm{e}_{1}
 =\frac{(2,2,\fbox{\textcolor{red}{\ref{pro04}}})}{(e^{t}+e^{-t})^2}$

       \begin{equation}
        \frac{\mathrm{d}}{\mathrm{d}t}\bm{e}_{1}
         =\frac{\mathrm{d}}{\mathrm{d}t}
         \left( \frac{1}{e^{t}+e^{-t}}(e^{t},-e^{-t},\sqrt{2}) \right)
         = \frac{1}{(e^{t}+e^{-t})^2}(2,2,\sqrt{2}(-e^{t}+e^{-t}))
       \end{equation}
       \dotfill

 \item \label{pro05}
       $\frac{\mathrm{d}}{\mathrm{d}s}\bm{e}_{1}
       =\left. \frac{\mathrm{d}\bm{e}_{1}}{\mathrm{d}t} \middle/ \frac{\mathrm{d}s}{\mathrm{d}t} \right.=$\fbox{\textcolor{red}{\ref{pro05}}}


       \begin{align}
        \left. \frac{\mathrm{d}\bm{e}_{1}}{\mathrm{d}t} \middle/ \frac{\mathrm{d}s}{\mathrm{d}t} \right.
         =& \frac{1}{e^{t}+e^{-t}}\frac{1}{(e^{t}+e^{-t})^2}(2,2,\sqrt{2}(-e^{t}+e^{-t}))\\
         =& \frac{1}{(e^{t}+e^{-t})^3}(2,2,\sqrt{2}(-e^{t}+e^{-t}))
       \end{align}
       \dotfill

 \item \label{pro06}
       $\kappa = \left\lvert \frac{\mathrm{d}}{\mathrm{d}s}\bm{e}_{1} \right\rvert
       =\frac{\fbox{\textcolor{red}{\ref{pro06}}}}{(e^{t}+e^{-t})^2}$

       \begin{equation}
        \left\lvert \frac{\mathrm{d}}{\mathrm{d}s}\bm{e}_{1} \right\rvert
         =\frac{1}{(e^{t}+e^{-t})^3}\sqrt{2^{2}+2^{2}+\sqrt{2}^{2}(-e^{t}+e^{-t})^{2}}
         =\frac{\sqrt{2}}{(e^{t}+e^{-t})^2}
       \end{equation}

       \dotfill

 \item \label{pro07}
       $\bm{e}_{3}=\frac{(-e^{-t},\fbox{\textcolor{red}{\ref{pro07}}},\sqrt{2})}{e^{t}+e^{-t}}$

       \begin{equation}
        \bm{e}_{2} = \frac{ \frac{\mathrm{d}}{\mathrm{d}s}\bm{e}_{1} }{\kappa}
         = \frac{1}{e^{t}+e^{-t}}(\sqrt{2},\sqrt{2},(-e^{t}+e^{-t}))
       \end{equation}

       \begin{align}
        & (e^{t},-e^{-t},\sqrt{2})
        \times (\sqrt{2},\sqrt{2},(-e^{t}+e^{-t}))\\
        =& \left(
          \begin{vmatrix}
           -e^{-t} & \sqrt{2}\\
           \sqrt{2} & -e^{t}+e^{-t}
          \end{vmatrix},\
          \begin{vmatrix}
           \sqrt{2} & e^{t}\\
           -e^{t}+e^{-t} &\sqrt{2}
          \end{vmatrix},\
          \begin{vmatrix}
           e^{t} & -e^{-t}\\
           \sqrt{2} & \sqrt{2}
          \end{vmatrix}
        \right)\\
        =& \left(
          -e^{-t}(e^{t}+e^{-t}),\
          e^{t}(e^{t}+e^{-t}),\
          \sqrt{2}(e^{t}+e^{-t})
        \right)\\
        =& (e^{t}+e^{-t}) \left(
          -e^{-t},e^{t},\sqrt{2}
        \right)
       \end{align}

       \begin{align}
        \bm{e}_{3} =& \bm{e}_{1}\times\bm{e}_{2}\\
         =& \frac{1}{e^{t}+e^{-t}}(e^{t},-e^{-t},\sqrt{2})
        \times \frac{1}{e^{t}+e^{-t}}(\sqrt{2},\sqrt{2},(-e^{t}+e^{-t}))\\
         & \frac{1}{e^{t}+e^{-t}}(-e^{-t},e^{t},\sqrt{2})
       \end{align}

       \dotfill

 \item \label{pro08}
       $\frac{\mathrm{d}}{\mathrm{d}t}\bm{e}_{3}
       =\frac{\fbox{\textcolor{red}{\ref{pro08}}}}{(e^{t}+e^{-t})^2}$

       \begin{align}
        \frac{\mathrm{d}}{\mathrm{d}t}\bm{e}_{3}
        =&
        \frac{1}{(e^{t}+e^{-t})^2}
        \left(
          1+e^{-2t}+1-e^{-2t},\
          e^{2t}+1-e^{2t}+1,\
          -\sqrt{2}(e^{t}-e^{-t})
        \right)\\
        =&
        \frac{1}{(e^{t}+e^{-t})^2}
        \left(
          2,\ 2,\ -\sqrt{2}(e^{t}-e^{-t})
        \right)
       \end{align}

       \dotfill

 \item \label{pro09}
       $\tau=\fbox{\textcolor{red}{\ref{pro09}}}$

       \begin{align}
        \tau =& -\left\lvert
                 \frac{\mathrm{d}\bm{e}_{3}}{\mathrm{d}t} \middle/ \frac{\mathrm{d}s}{\mathrm{d}t}
                \right\rvert\\
        =&  -\left\lvert
        \frac{1}{(e^{t}+e^{-t})^2}
        \left(
          2,\ 2,\ -\sqrt{2}(e^{t}-e^{-t})
        \right)
        \frac{1}{e^{t}+e^{-t}}
        \right\rvert\\
        =& -\frac{1}{(e^{t}+e^{-t})^3}(\sqrt{2(e^{t}+e^{-t})^2})\\
        =& -\frac{\sqrt{2}}{(e^{t}+e^{-t})^2}
       \end{align}



\end{enumerate}

\hrulefill

\end{document}
