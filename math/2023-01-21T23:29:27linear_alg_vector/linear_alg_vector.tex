\documentclass[12pt,b5paper]{ltjsarticle}

%\usepackage[margin=15truemm, top=5truemm, bottom=5truemm]{geometry}
%\usepackage[margin=10truemm,left=15truemm]{geometry}
\usepackage[margin=10truemm]{geometry}

\usepackage{amsmath,amssymb}
%\pagestyle{headings}
\pagestyle{empty}

%\usepackage{listings,url}
%\renewcommand{\theenumi}{(\arabic{enumi})}

%\usepackage{graphicx}

%\usepackage{tikz}
%\usetikzlibrary {arrows.meta}
%\usepackage{wrapfig}	% required for `\wrapfigure' (yatex added)
\usepackage{bm}

% ルビを振る
%\usepackage{luatexja-ruby}	% required for `\ruby'

%% 核Ker 像Im Hom を定義
%\newcommand{\Img}{\mathop{\mathrm{Im}}\nolimits}
%\newcommand{\Ker}{\mathop{\mathrm{Ker}}\nolimits}
%\newcommand{\Hom}{\mathop{\mathrm{Hom}}\nolimits}

%\DeclareMathOperator{\Rot}{rot}
%\DeclareMathOperator{\Div}{div}
%\DeclareMathOperator{\Grad}{grad}
%\DeclareMathOperator{\arcsinh}{arcsinh}
%\DeclareMathOperator{\arccosh}{arccosh}
%\DeclareMathOperator{\arctanh}{arctanh}



%\usepackage{listings,url}
%
%\lstset{
%%プログラム言語(複数の言語に対応,C,C++も可)
%  language = Python,
%%  language = Lisp,
%%  language = C,
%  %背景色と透過度
%  %backgroundcolor={\color[gray]{.90}},
%  %枠外に行った時の自動改行
%  breaklines = true,
%  %自動改行後のインデント量(デフォルトでは20[pt])
%  breakindent = 10pt,
%  %標準の書体
%%  basicstyle = \ttfamily\scriptsize,
%  basicstyle = \ttfamily,
%  %コメントの書体
%%  commentstyle = {\itshape \color[cmyk]{1,0.4,1,0}},
%  %関数名等の色の設定
%  classoffset = 0,
%  %キーワード(int, ifなど)の書体
%%  keywordstyle = {\bfseries \color[cmyk]{0,1,0,0}},
%  %表示する文字の書体
%  %stringstyle = {\ttfamily \color[rgb]{0,0,1}},
%  %枠 "t"は上に線を記載, "T"は上に二重線を記載
%  %他オプション:leftline,topline,bottomline,lines,single,shadowbox
%  frame = TBrl,
%  %frameまでの間隔(行番号とプログラムの間)
%  framesep = 5pt,
%  %行番号の位置
%  numbers = left,
%  %行番号の間隔
%  stepnumber = 1,
%  %行番号の書体
%%  numberstyle = \tiny,
%  %タブの大きさ
%  tabsize = 4,
%  %キャプションの場所("tb"ならば上下両方に記載)
%  captionpos = t
%}



\begin{document}

\hrulefill
\begin{equation}
  \bm{a}_1=
   \begin{pmatrix}
    1\\1\\1
   \end{pmatrix}
   ,\
  \bm{a}_2=
  \begin{pmatrix}
   1\\-1\\0
  \end{pmatrix}
  ,\
  \bm{a}_3=
  \begin{pmatrix}
   -2\\2\\3
  \end{pmatrix}
\end{equation}
3つのベクトル$\bm{a}_1,\bm{a}_2,\bm{a}_3\in\mathbb{R}^3$
と
スカラー$x,y\in\mathbb{R}$について
以下の問いに答えよ。
\begin{enumerate}
 \item
      内積$(\bm{a}_1,\bm{a}_2)$を求めよ。

      \dotfill

      \begin{equation}
       (\bm{a}_1,\bm{a}_2) =
        1\cdot 1 + 1\cdot (-1) + 1\cdot 0
        = 0
      \end{equation}

      \hrulefill

 \item
      $\bm{c}= \bm{a}_3 - x \bm{a}_1 - y \bm{a}_2$とおく。
      $\bm{c}$が$\bm{a}_1$と$\bm{a}_2$の両方と直交するとき、
      $x,y$と$\bm{c}$を求めよ。

      \dotfill

      次の内積を求める。
      \begin{equation}
        (\bm{a}_1,\bm{a}_1) = 3,
        (\bm{a}_2,\bm{a}_2) = 2,
        (\bm{a}_1,\bm{a}_2) = 0,
        (\bm{a}_1,\bm{a}_3) = 3,
        (\bm{a}_2,\bm{a}_3) = -4
      \end{equation}

      $\bm{c}$が$\bm{a}_1,\bm{a}_2$と直交するので、
      $(\bm{c},\bm{a}_1) = (\bm{c},\bm{a}_2) = 0$
      を満たす。
      \begin{align}
        (\bm{c},\bm{a}_1)
        =&
        (\bm{a}_3 - x \bm{a}_1 - y \bm{a}_2,\bm{a}_1)\\
        =&
        (\bm{a}_3,\bm{a}_1) -x(\bm{a}_1,\bm{a}_1) -y(\bm{a}_2,\bm{a}_1)
        =
        3-3x-0y =0\\
       x=&1\\
        (\bm{c},\bm{a}_1)
       =&
        (\bm{a}_3 - x \bm{a}_1 - y \bm{a}_2,\bm{a}_2)\\
        =&
        (\bm{a}_3,\bm{a}_2) -x(\bm{a}_1,\bm{a}_2) -y(\bm{a}_2,\bm{a}_2)
        =
        -4-0x-2y =0\\
       y=&-2
      \end{align}
      よって、答えは次のようになる。
      \begin{equation}
       x=1,\ y=-2,\
       \bm{c}=
        \bm{a}_3 - \bm{a}_1 +2 \bm{a}_2
        =\begin{pmatrix}
          -1 \\ -1 \\ 2
         \end{pmatrix}
      \end{equation}


      \hrulefill

 \item
      $\bm{a}_1,\bm{a}_2,\bm{c}$の大きさを
      それぞれ$\|\bm{a}_1\|,\|\bm{a}_2\|,\|\bm{c}\|$とし、
      $\displaystyle
      \bm{b}_1=\frac{\bm{a}_1}{\|\bm{a}_1\|},
      \bm{b}_2=\frac{\bm{a}_2}{\|\bm{a}_2\|},
      \bm{b}_3=\frac{\bm{c}}{\|\bm{c}\|}$
      とおく。
      このとき、$\bm{b}_1,\bm{b}_2,\bm{b}_3$をそれぞれ求めよ。

      \dotfill

      それぞれのベクトルの大きさは次の通り。
      $\|\bm{a}_1\| = \sqrt{3},\ \|\bm{a}_2\|= \sqrt{2},\ \|\bm{c}\| = \sqrt{6}$

      \begin{equation}
        \bm{b}_1=
         \frac{1}{\sqrt{3}}
         \begin{pmatrix}
          1\\1\\1
         \end{pmatrix}
        \ ,\bm{b}_2=
        \begin{pmatrix}
         1 \\ -1 \\ 0
        \end{pmatrix}
        \ ,\bm{b}_3=
        \frac{1}{\sqrt{6}}
        \begin{pmatrix}
         -1\\-1\\2
        \end{pmatrix}
      \end{equation}


      \hrulefill

\end{enumerate}

\hrulefill

\end{document}
