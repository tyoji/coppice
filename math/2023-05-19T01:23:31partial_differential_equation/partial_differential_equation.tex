\documentclass[12pt,b5paper]{ltjsarticle}

%\usepackage[margin=15truemm, top=5truemm, bottom=5truemm]{geometry}
%\usepackage[margin=10truemm,left=15truemm]{geometry}
\usepackage[margin=10truemm]{geometry}

\usepackage{amsmath,amssymb}
%\pagestyle{headings}
\pagestyle{empty}

%\usepackage{listings,url}
%\renewcommand{\theenumi}{(\arabic{enumi})}

%\usepackage{graphicx}

%\usepackage{tikz}
%\usetikzlibrary {arrows.meta}
%\usepackage{wrapfig}
%\usepackage{bm}

% ルビを振る
\usepackage{luatexja-ruby}	% required for `\ruby'

%% 核Ker 像Im Hom を定義
%\newcommand{\Img}{\mathop{\mathrm{Im}}\nolimits}
%\newcommand{\Ker}{\mathop{\mathrm{Ker}}\nolimits}
%\newcommand{\Hom}{\mathop{\mathrm{Hom}}\nolimits}

%\DeclareMathOperator{\Rot}{rot}
\DeclareMathOperator{\Div}{div}
%\DeclareMathOperator{\Grad}{grad}
%\DeclareMathOperator{\arcsinh}{arcsinh}
%\DeclareMathOperator{\arccosh}{arccosh}
%\DeclareMathOperator{\arctanh}{arctanh}



%\usepackage{listings,url}
%
%\lstset{
%%プログラム言語(複数の言語に対応,C,C++も可)
%  language = Python,
%%  language = Lisp,
%%  language = C,
%  %背景色と透過度
%  %backgroundcolor={\color[gray]{.90}},
%  %枠外に行った時の自動改行
%  breaklines = true,
%  %自動改行後のインデント量(デフォルトでは20[pt])
%  breakindent = 10pt,
%  %標準の書体
%%  basicstyle = \ttfamily\scriptsize,
%  basicstyle = \ttfamily,
%  %コメントの書体
%%  commentstyle = {\itshape \color[cmyk]{1,0.4,1,0}},
%  %関数名等の色の設定
%  classoffset = 0,
%  %キーワード(int, ifなど)の書体
%%  keywordstyle = {\bfseries \color[cmyk]{0,1,0,0}},
%  %表示する文字の書体
%  %stringstyle = {\ttfamily \color[rgb]{0,0,1}},
%  %枠 "t"は上に線を記載, "T"は上に二重線を記載
%  %他オプション:leftline,topline,bottomline,lines,single,shadowbox
%  frame = TBrl,
%  %frameまでの間隔(行番号とプログラムの間)
%  framesep = 5pt,
%  %行番号の位置
%  numbers = left,
%  %行番号の間隔
%  stepnumber = 1,
%  %行番号の書体
%%  numberstyle = \tiny,
%  %タブの大きさ
%  tabsize = 4,
%  %キャプションの場所("tb"ならば上下両方に記載)
%  captionpos = t
%}



\begin{document}


\hrulefill


\textbf{勾配 (gradient)}

関数$f$に対して、
$f$の勾配 (\ruby{gradient}{グラディエント})
$Df(=\nabla f)$
\begin{gather}
 f:\mathbb{R}^3 \to \mathbb{R},
  \quad
 (x_1,x_2,x_3)\mapsto y \qquad\
 Df = \nabla f = \left(
  \frac{\partial f}{\partial x_1},
  \frac{\partial f}{\partial x_2},
  \frac{\partial f}{\partial x_3}
  \right)
\end{gather}


\hrulefill


\textbf{偏微分方程式 ( partial differential equation )}

輸送方程式 ( transport equation )

\hrulefill

$u=u(x,t)\ (x\in\mathbb{R}^n,\ t\in\mathbb{R})$
$u: \mathbb{R}^n\times [ 0, \infty ) \to \mathbb{R}$

$Du=(u_{x_{1}},u_{x_{2}},\dots,u_{x_{n}})$

$b\in\mathbb{R}^n$




\hrulefill

\textbf{Report 1.1}

関数$u=u(x,t)\ (x\in\mathbb{R}^n,\ t\in\mathbb{R})$は
$u: \mathbb{R}^n\times [ 0, \infty ) \to \mathbb{R}$
とし、
$b\in\mathbb{R}^n$とする。
\begin{equation}
 u_{t} + b \cdot Du = 0
  \quad
  \text{in}\
 \mathbb{R}^{n} \times (0,\infty)
 \qquad
 z(s) = u(x+sb,t+s)
 \quad
 (s\in\mathbb{R})
 \label{eq001}
\end{equation}


この時、次の式が成り立つ。

\begin{equation}
 \dot{z}(s) = Du(x+sb, t+s)\cdot b + u_{t}(x+sb,t+s) =0
\end{equation}



\dotfill

$u:\mathbb{R}^{n}\times [0,\infty)\to\mathbb{R}$

合成関数の微分を用いて$z(s)$を$s$で微分する。
\begin{align}
 \frac{d}{d s}z(s)
  =& \frac{\partial z}{\partial x}\frac{d x}{d s}
   + \frac{\partial z}{\partial t}\frac{d t}{d s}\\
%  =& \frac{\partial u}{\partial x}(x+sb,t+s)\cdot\frac{d (x+sb)}{d s}
%   + \frac{\partial u}{\partial t}(x+sb,t+s)\cdot\frac{d (t+s)}{d s}\\
%  =& \frac{\partial u}{\partial x}(x+sb,t+s)\cdot\frac{d (x+sb)}{d s}
%   + u_{t}
\end{align}

第2項$\frac{\partial z}{\partial t}\frac{d t}{d s}$は次のように計算できる。
\begin{equation}
 \frac{\partial z}{\partial t}\frac{d t}{d s}
  = \frac{\partial u}{\partial t}(x+sb,t+s)
  = u_{t}(x+sb,t+s)
\end{equation}

第1項$\frac{\partial z}{\partial x}\frac{d x}{d s}$は
多変数関数の微分であるので、
$x=(x_1,\dots,x_n)\in\mathbb{R}^n$より
次のようになる。
\begin{align}
 \frac{\partial z}{\partial x}\frac{d x}{d s}
  =& \sum_{i=1}^{n}\frac{\partial z}{\partial x_i}\frac{d x_i}{d s}
  = \sum_{i=1}^{n}\frac{\partial u}{\partial x_i}(x_i+sb_i)\cdot\frac{d (x_i+sb_i)}{d s}\\
  =& \sum_{i=1}^{n} u_{x_i}(x_{i}+sb_{i})\cdot b_{i}
  = Du(x+sb)\cdot b
\end{align}

よって、
$z(s)$を$s$で微分すると
$Du(x+sb, t+s)\cdot b + u_{t}(x+sb,t+s)$が得られる。

式\eqref{eq001}より
$\mathbb{R}^{n} \times (0,\infty)$上で、
$u_{t} + b \cdot Du = 0$
であるので、
$t+s>0$において
$Du(x+sb, t+s)\cdot b + u_{t}(x+sb,t+s)=0$となる。


\hrulefill

\textbf{Report 1.2}

$b\in\mathbb{R}^{n}$
,\quad
$g:\mathbb{R}^{n}\to\mathbb{R}$



\begin{equation}
 \begin{cases}
  u_{t} + b\cdot Du = 0 & \text{in} \ \mathbb{R}^{n}\times (0,\infty)\\
  u=g & \text{on} \ \mathbb{R}^{n}\times \{ t=0 \}
 \end{cases}
 \label{eq02b}
\end{equation}

\begin{equation}
 u(x,t) = g(x-tb) \quad (x\in\mathbb{R}^{n},\ t\geq 0)
 \label{eq02a}
\end{equation}

\eqref{eq02a}で定義される$u(x,t)$は
\eqref{eq02b}を満たすことを示せ。

\dotfill

$u(x,t) = g(x-tb)$より
$t=0$の時は
$u(x,0) = g(x)$である。

$t>0$において、
$u_{t}$を計算する。
これは$t$で偏微分を行うので、
$g(x-tb)$を偏微分する。
\begin{equation}
 u_{t}
  = \frac{\partial}{\partial t}g(x-tb)
  = \frac{\partial g}{\partial x_{1}}(x-tb)\times (-b_{1})+\cdots+\frac{\partial g}{\partial x_{n}}(x-tb)\times (-b_{n})
\end{equation}

同様に$b\cdot Du$も計算する。
\begin{align}
 b\cdot Du
 =& (b_{1},\dots,b_{n})\cdot(u_{x_{1}},\dots,u_{x_{n}})\\
 =& (b_{1},\dots,b_{n})\cdot\left(\frac{\partial g}{\partial x_{1}}(x-tb),\dots,\frac{\partial g}{\partial x_{n}}(x-tb) \right)\\
 =& b_{1}\frac{\partial g}{\partial x_{1}}(x-tb)+\cdots b_{n}\frac{\partial g}{\partial x_{n}}(x-tb)
\end{align}

よって、$u_{t}+b\cdot Du =0$となる。

\hrulefill

\newpage

\textbf{Report 1.3}

$V$は$U\subset\mathbb{R}^n$の
任意のなめらかな部分領域とし、
$\Div{F}$が連続であるとする。
この時、
$\displaystyle \int_{V}\Div{F}dx=0$
であれば
$U$上で
$\Div{F}=0$
となることを示せ。

\dotfill

$V$は任意であるので、
$U$上の任意の点$P$と
$P$を中心とした半径$\varepsilon$の
球$B$上の積分
$\int_{B}\Div{F}dx=0$を考える。

$B$の体積$\lvert B\rvert$と
$(\Div{F})(P)$を用いて
$\int_{B}\Div{F}dx$は
$\lvert B \rvert (\Div{F})(P)$
に近似できる。

$\int_{B}\Div{F}dx=0$より
$\lim_{\varepsilon\to 0}\lvert B \rvert (\Div{F})(P) =0$
であるが、
両辺を$\varepsilon^n$で割ることにより
$(\Div{F})(P) =0$
が得られる。

${}^{\forall}P\in U$であったので、
$U$上で$\Div{F}=0$となる。


\hrulefill

\textbf{Report 1.4}

$x\in\mathbb{R}^n,\ x\ne 0$
の時、
\begin{equation}
 \Phi(x) =
  \begin{cases}
   -\frac{1}{2\pi}\log{\lvert x \rvert} & (n=2)\\
   \frac{1}{n(n-2)\alpha(n)}\frac{1}{\lvert x \rvert^{n-2}} & (n\geq 3)
  \end{cases}
\end{equation}
とする。
この時、ある定数$C>0$を用いて次の不等式が成り立つ。
\begin{equation}
 \lvert D\Phi(x)\rvert \leq \frac{C}{\lvert x \rvert^{n-1}}
  ,\quad
 \lvert D^{2}\Phi(x)\rvert \leq \frac{C}{\lvert x \rvert^{n}}
\end{equation}


\dotfill

$x\in\mathbb{R}^n$より
$x=(x_{1},\dots,x_{n})$とする。

$\displaystyle D\Phi(x)=
\left(\frac{\partial \Phi(x)}{\partial x_{1}},\dots,\frac{\partial \Phi(x)}{\partial x_{n}}\right)$
より、各成分を計算する。

$n=2$の場合
\begin{gather}
 \frac{\partial \Phi(x)}{\partial x_{i}}
  = -\frac{1}{2\pi} \frac{2x_{i}}{x_{1}^{2}+x_{2}^{2}}
  = -\frac{1}{\pi} \frac{x_{i}}{\lvert x \rvert^2} %\\
\end{gather}
これより、
\begin{equation}
 \lvert D\Phi(x)\rvert
  = \sqrt{\left( -\frac{1}{\pi} \frac{x_{1}}{\lvert x \rvert^2} \right)^2
  + \left( -\frac{1}{\pi} \frac{x_{2}}{\lvert x \rvert^2} \right)^2}
  = \frac{1}{\pi\lvert x \rvert}
\end{equation}

$n\geq 3$の場合
\begin{equation}
 \frac{\partial \Phi(x)}{\partial x_{i}}
  =\frac{1}{n(n-2)\alpha(n)} \frac{\partial}{\partial x_{i}} \frac{1}{\lvert x \rvert^{n-2}}
  = \frac{1}{n(n-2)\alpha(n)} \frac{-(n-2)x_{i}}{\lvert x \rvert^{n}}
\end{equation}
これにより、
\begin{equation}
 \lvert D\Phi(x)\rvert
  = \sqrt{\sum_{i=1}^{n} \left( \frac{-x_{i}}{n\alpha(n)\lvert x \rvert^{n}} \right)^2}
  = \frac{1}{n\alpha(n)\lvert x \rvert^{n-1}}
\end{equation}

よって、
$n\geq2$に対して、次を満たす定数$C>0$が存在する。
\begin{equation}
 \lvert D\Phi(x)\rvert
  \leq \frac{C}{\lvert x \rvert^{n-1}}
\end{equation}

\dotfill

$\displaystyle D^2 \Phi(x)=
\left(\frac{\partial^2 \Phi(x)}{\partial x_{1}^2},\dots,\frac{\partial^2 \Phi(x)}{\partial x_{n}^2}\right)$
より各項を計算する。

$n=2$の場合
\begin{gather}
 \frac{\partial^2 \Phi(x)}{\partial x_{i}^2}
  = \frac{\partial}{\partial x_{i}}
    \left( -\frac{1}{\pi} \frac{x_{i}}{\lvert x \rvert^2} \right)
  = -\frac{1}{\pi} \frac{x_{1}^2+x_{2}^{2} -2x_{i}^2}{(x_{1}^{2}+x_{2}^{2})^2}
  = -\frac{1}{\pi} \frac{\lvert x \rvert^{2} -2x_{i}^2}{\lvert x \rvert^4}
\end{gather}

\begin{align}
 \lvert D^2\Phi(x)\rvert^2
  =&
  \left(
         -\frac{1}{\pi} \frac{-x_{1}^2+x_{2}^{2}}{\lvert x \rvert^4}
  \right)^2 + \left(
         -\frac{1}{\pi} \frac{x_{1}^2-x_{2}^{2}}{\lvert x \rvert^4}
  \right)^2\\
  =&
  \frac{1}{\pi^2}
  \left(
    \frac{x_{1}^4-2x_{1}^{2}x_{2}^{2}+x_{2}^{4}}{\lvert x \rvert^8}
    +
    \frac{x_{1}^4-2x_{1}^{2}x_{2}^{2}+x_{2}^{4}}{\lvert x \rvert^8}
  \right)\\
 =&
 \frac{2}{\pi^2}
 \frac{\lvert x \rvert^4 -4x_{1}^{2}x_{2}^{2}}{\lvert x \rvert^8}\\
 \leq&
  \frac{2}{\pi^2}
  \frac{\lvert x \rvert^4}{\lvert x \rvert^8}
 =\frac{2}{\pi^2}\frac{1}{\lvert x \rvert^4}
\end{align}

\begin{equation}
 \lvert D \Phi(x)\rvert \leq \frac{\sqrt{2}}{\pi}\frac{1}{\lvert x \rvert^2}
\end{equation}

$n\geq 3$の場合
\begin{align}
 \frac{\partial^2 \Phi(x)}{\partial x_{i}^2}
  =& \frac{\partial}{\partial x_{i}}
  \left( \frac{-1}{n\alpha(n)} \frac{x_{i}}{\lvert x \rvert^{n}} \right)
  =\frac{-1}{n\alpha(n)} \frac{\partial}{\partial x_{i}} \frac{x_{i}}{\lvert x \rvert^{n}}\\
  =& \frac{-1}{n\alpha(n)} \frac{\lvert x \rvert^{n}-x_{i}\cdot \frac{n}{2}\lvert x \rvert^{n-2}\cdot 2x_i}{\lvert x \rvert^{2n}}
  = \frac{-1}{n\alpha(n)} \frac{\lvert x \rvert^{n}-nx_{i}^{2}\lvert x \rvert^{n-2}}{\lvert x \rvert^{2n}}
\end{align}

\begin{align}
 \lvert D^2\Phi(x)\rvert^2
  =& \sum_{i=1}^{n} \left( \frac{-1}{n\alpha(n)} \frac{\lvert x \rvert^{n} - nx_{i}^{2} \lvert x \rvert^{n-2}}{\lvert x \rvert^{2n}} \right)^2 \\
   =& \frac{1}{n^2(\alpha(n))^2} \sum_{i=1}^{n}  \frac{\lvert x \rvert^{2n} - 2\lvert x \rvert^{n} n x_{i}^{2} \lvert x \rvert^{n-2}  + n^2x_{i}^{4} \lvert x \rvert^{2n-4}}{\lvert x \rvert^{4n}} \\
   =& \frac{1}{n^2(\alpha(n))^2}   \frac{n \lvert x \rvert^{2n} - 2\lvert x \rvert^{n} n \lvert x \rvert^{n-2}\sum_{i=1}^{n}x_{i}^{2}  + n^2 \lvert x \rvert^{2n-4} \sum_{i=1}^{n}x_{i}^{4}}{\lvert x \rvert^{4n}} \\
  =& \frac{1}{n^2(\alpha(n))^2}   \frac{n \lvert x \rvert^{2n} - 2n \lvert x \rvert^{2n} + n^2 \lvert x \rvert^{2n-4} \sum_{i=1}^{n}x_{i}^{4}}{\lvert x \rvert^{4n}} \\
 \leq& \frac{1}{n^2(\alpha(n))^2}   \frac{n \lvert x \rvert^{2n} - 2n \lvert x \rvert^{2n} + n^2 \lvert x \rvert^{2n-4} \left(\sum_{i=1}^{n}x_{i}^{2}\right)^2}{\lvert x \rvert^{4n}} \\
 =& \frac{n^2-n}{n^2(\alpha(n))^2}   \frac{ \lvert x \rvert^{2n} }{\lvert x \rvert^{4n}} \\
  =& \frac{n^2-n}{n^2(\alpha(n))^2}   \frac{ 1 }{\lvert x \rvert^{2n}}
\end{align}


\begin{equation}
 \lvert D^2\Phi(x)\rvert
  =
  \frac{\sqrt{n^2-n}}{n\alpha(n)} \frac{ 1 }{\lvert x \rvert^{n}}
\end{equation}

よって、
$n\geq2$に対して、次を満たす定数$C>0$が存在する。
\begin{equation}
 \lvert D^2\Phi(x)\rvert
  \leq \frac{C}{\lvert x \rvert^{n}}
\end{equation}


\hrulefill

\end{document}
