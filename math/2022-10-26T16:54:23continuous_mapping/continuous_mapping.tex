\documentclass[12pt,b5paper]{ltjsarticle}

%\usepackage[margin=15truemm, top=5truemm, bottom=5truemm]{geometry}
\usepackage[margin=10truemm]{geometry}

\usepackage{amsmath,amssymb}
%\pagestyle{headings}
\pagestyle{empty}

%\usepackage{listings,url}
%\renewcommand{\theenumi}{(\arabic{enumi})}

%\usepackage{graphicx}

%\usepackage{tikz}
%\usetikzlibrary {arrows.meta}
%\usepackage{wrapfig}	% required for `\wrapfigure' (yatex added)
%\usepackage{bm}	% required for `\bm' (yatex added)

% ルビを振る
%\usepackage{luatexja-ruby}	% required for `\ruby'

%% 核Ker 像Im Hom を定義
%\newcommand{\Img}{\mathop{\mathrm{Im}}\nolimits}
%\newcommand{\Ker}{\mathop{\mathrm{Ker}}\nolimits}
%\newcommand{\Hom}{\mathop{\mathrm{Hom}}\nolimits}

%\DeclareMathOperator{\Rot}{rot}
%\DeclareMathOperator{\Div}{div}
%\DeclareMathOperator{\Grad}{grad}
%\DeclareMathOperator{\arcsinh}{arcsinh}
%\DeclareMathOperator{\arccosh}{arccosh}
%\DeclareMathOperator{\arctanh}{arctanh}



%\usepackage{listings,url}
%
%\lstset{
%%プログラム言語(複数の言語に対応,C,C++も可)
%%  language = Python,
%  language = Lisp,
%  %背景色と透過度
%  %backgroundcolor={\color[gray]{.90}},
%  %枠外に行った時の自動改行
%  breaklines = true,
%  %自動改行後のインデント量(デフォルトでは20[pt])
%  breakindent = 10pt,
%  %標準の書体
%%  basicstyle = \ttfamily\scriptsize,
%  basicstyle = \ttfamily,
%  %コメントの書体
%%  commentstyle = {\itshape \color[cmyk]{1,0.4,1,0}},
%  %関数名等の色の設定
%  classoffset = 0,
%  %キーワード(int, ifなど)の書体
%%  keywordstyle = {\bfseries \color[cmyk]{0,1,0,0}},
%  %表示する文字の書体
%  %stringstyle = {\ttfamily \color[rgb]{0,0,1}},
%  %枠 "t"は上に線を記載, "T"は上に二重線を記載
%  %他オプション:leftline,topline,bottomline,lines,single,shadowbox
%  frame = TBrl,
%  %frameまでの間隔(行番号とプログラムの間)
%  framesep = 5pt,
%  %行番号の位置
%  numbers = left,
%  %行番号の間隔
%  stepnumber = 1,
%  %行番号の書体
%%  numberstyle = \tiny,
%  %タブの大きさ
%  tabsize = 4,
%  %キャプションの場所("tb"ならば上下両方に記載)
%  captionpos = t
%}



\begin{document}



\hrulefill
\textbf{定義}
\hrulefill

\textbf{位相空間}

集合$X$に部分集合族$\mathcal{O}$が
開集合の公理をみたすとき、
その集合と集合族の組$(X,\mathcal{O})$を
位相空間という。

\textbf{開集合の公理}

$X$:集合、$\mathcal{O}$:部分集合族
\begin{enumerate}
 \item $X\in\mathcal{O},\ \emptyset\in\mathcal{O}$
 \item $A_i\in\mathcal{O}$に対して$\bigcap_{i=1}^{n}A_i\in\mathcal{O}$
 \item $A_k\in\mathcal{O}$に対して$\bigcup_{k\in\Lambda}A_k\in\mathcal{O}$
\end{enumerate}

\textbf{距離空間}

集合$X$に距離関数$d:X\times X \to \mathbb{R}$が定義される時
集合と関数の組$(X,d)$を距離空間という。

\textbf{距離関数}

関数$d:X\times X \to \mathbb{R}$が次を満たす時
距離関数という。
\begin{enumerate}
 \item $d(x,y)\geq 0,\ d(x,y)=0 \Leftrightarrow x=y$
 \item $d(x,y)=d(y,x)$
 \item $d(x,z)\leq d(x,y)+d(y,z)$
\end{enumerate}

\textbf{距離位相}

集合$X$に対し距離関数を用いて
開集合族$\mathcal{O}$が定められる時
$(X,\mathcal{O})$を距離位相空間という。

$\mathcal{O}=\{ O\subset X \mid
 {}^{\forall}x\in O, {}^{\exists}\varepsilon>0
 \ s.t.\ U_{\varepsilon}(x)\subset O
 \}$
 ($U_{\varepsilon}(x)$は点$x$に於ける$\varepsilon$-近傍である)



\textbf{離散位相}

全ての部分集合が開集合である時、
離散位相という。

\textbf{密着位相}

全体集合と空集合のみが開集合となる時、
密着位相という。


\dotfill

位相空間は
何を開集合と定めるかによって
異なる性質を持つ。




\hrulefill
\textbf{問題}
\hrulefill

\begin{enumerate}
 \item
      有限集合上の距離位相空間は
      離散空間であることを示せ。

\dotfill

      $X$を有限集合とし、$d$を距離関数とする。

      $x\in X$とする。
      $x$と異なる$X$の点との距離の内、
      最小のものを$m$とする。
      \begin{equation}
       m = d(x,\{x\}^{c})
      \end{equation}

      $x$の近傍$U_{m}(x)$は$x$のみの集合となる。
      $U_{m}(x)=\{x\}$より
      集合$\{x\}$は$U_{m}(x) \subset \{x\}$を満たす為、
      開集合である。

      任意の点$x\in X$に対して1点集合$\{x\}$が開集合となる。
      この為、複数の点が含まれた集合$\{x_1,\dots,x_n\}$も
      各点における近傍が含まれるため開集合となる。

      つまり任意の部分集合が開集合となるので
      離散位相となる。

\hrulefill
 \item
      距離空間において$x\in A$が$A$の内点であるための
      必要十分条件は
      $d(x,A^{c})>0$であることを示せ。

      \dotfill
      $x\in A \text{が内点}\ \Rightarrow \ d(x,A^{c})>0$
      \dotfill

      $x\in A$ が内点であれば$x$の近傍$U$が存在し
      $x\in U \subset A$である。
%      $x\in A$より$x\not\in A^{c}$であるので、
      ${}^\forall{}y\in U$に対して
      $d(x,A^{c})>d(x,y)$である為、
      $d(x,A^{c})>0$となる。

      \dotfill
      $x\in A \text{が内点}\ \Leftarrow \ d(x,A^{c})>0$
      \dotfill

      $d(x,A^{c})>0$より、
      $2\varepsilon=d(x,A^{c})$とすると、
      $x$の近傍$U_{\varepsilon}(x)$と
      $A^{c}$の共通部分は空集合である。
      $U_{\varepsilon}(x)\cap A^{c} = \emptyset$

      よって、$U_{\varepsilon}(x)\subset A$であるので、
      $x\in A$は内点である。

\dotfill


      \begin{equation}
       x\in A \quad \Leftrightarrow \quad d(x,A^{c})>0
      \end{equation}
      このように思えるので、$d(x,A^{c})$の定義を確認する必要があり。

\hrulefill
 \item
      位相空間$X$が離散位相空間であるための
      必要十分条件は
      全ての1点集合が開集合となることであることを示せ。

      \dotfill
      $X \text{が離散位相} \ \Rightarrow \ \text{1点集合が開集合}$
      \dotfill

      $X$が離散位相空間である時、全ての部分集合は開集合となる。
      1点からなる集合$\{x\}$も部分集合であるので
      これらも開集合となる。

      \dotfill
      $X \text{が離散位相} \ \Leftarrow \ \text{1点集合が開集合}$
      \dotfill

      ${}^{\forall}x\in X$において$\{x\}$が開集合であるとする。

      $x\ne y$のとき、$\{x\}\cap\{y\}=\emptyset$より
      $\emptyset$は開集合となる。

      %$\{x\}\cup\{y\}=\{x,y\}$
      $X$の全ての部分集合も1点集合の和集合として表せる。
      任意個の和集合も開集合となるので
      任意の部分集合も開集合である。

      よって、
      $X$は離散位相空間となる。

\hrulefill

\end{enumerate}

\hrulefill

\end{document}
