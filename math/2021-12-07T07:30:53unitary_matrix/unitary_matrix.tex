\documentclass[10pt,b5paper]{ltjsarticle}

%\usepackage[margin=15truemm, top=5truemm, bottom=5truemm]{geometry}
\usepackage[margin=15truemm]{geometry}

\usepackage{amsmath,amssymb}
\pagestyle{empty}

\usepackage{bm}
\usepackage{listings,url}

\begin{document}




直交行列$A$を対角化するユニタリ行列を求めよ。
\begin{equation}
 A=\frac{1}{2}
    \begin{pmatrix}
     1 & -1 & -1 & -1\\
     1 & 1 & 1 & -1\\
     1 & -1 & 1 & 1\\
     1 & 1 & -1 & 1
    \end{pmatrix}
\end{equation}

\dotfill

次の手順で求めた。
\begin{enumerate}
 \item 固有値を求める
 \item 固有ベクトルを求める
 \item 固有空間の次元を確認し、対角化可能か判断する
 \item それぞれの固有空間の正規直交基底をとる(シュミットの直交化法)
 \item 基底を並べて行列(ユニタリ行列)をつくる
\end{enumerate}

ここでの内積は複素数における内積(複素内積、エルミート内積)である。
つまり、$x_i,y_i\in\mathbb{C}$に対し、
$\bm{x}=(x_1,\dots,x_n)$と$\bm{y}=(y_1,\dots,y_n)$の内積を
$\sum_{i=1}^{\infty}x_i\overline{y_i}$とする。
($\overline{y_i}$は$y_i$の共役複素数)

ここでは$\bm{x}, \bm{y}$の内積を$(\bm{x}\cdot\bm{y})$と書いた。



\dotfill

固有方程式$\lvert A-\lambda E \rvert =0$を計算し
固有値$\lambda$を求める。
\begin{align}
 \lvert A-\lambda E \rvert =0\\
 (\lambda^2-\lambda+1)^2 =0\\
 \lambda=\frac{1}{2}\pm \frac{\sqrt{3}}{2}i
\end{align}
固有値は
$\frac{1}{2}+ \frac{\sqrt{3}}{2}i=\exp{(\frac{\pi}{3}i)}$
と
$\frac{1}{2}- \frac{\sqrt{3}}{2}i=\exp{(-\frac{\pi}{3}i)}$
で、
それぞれ重複度は2である。



固有値$\exp{(\frac{\pi}{3}i)}$の固有ベクトル$\bm{x}$を求める。
\begin{gather}
 \left( A-\exp{(\frac{\pi}{3}i)} E \right)\bm{x} =0\\
 \bm{x}= k_1 \bm{v_1} + k_2 \bm{v_2}\\
 \bm{v_1}=
 \begin{pmatrix}
  1\\0\\\exp{(-\frac{2\pi}{3}i)}\\\exp{(-\frac{\pi}{3}i)}
 \end{pmatrix}
 =
 \begin{pmatrix}
  1\\0\\-\frac{1}{2}\\\frac{1}{2}
 \end{pmatrix}
 +i
 \begin{pmatrix}
  0\\0\\-\frac{\sqrt{3}}{2}\\-\frac{\sqrt{3}}{2}
 \end{pmatrix}
 , \quad
 \bm{v_2}=
 \begin{pmatrix}
  0\\1\\\exp{(\frac{2\pi}{3}i)}\\\exp{(-\frac{2\pi}{3}i)}
 \end{pmatrix}
 =
 \begin{pmatrix}
  0\\1\\-\frac{1}{2}\\-\frac{1}{2}
 \end{pmatrix}
 +i
 \begin{pmatrix}
  0\\0\\\frac{\sqrt{3}}{2}\\-\frac{\sqrt{3}}{2}
 \end{pmatrix}
\end{gather}




同様に固有値$\exp{(-\frac{\pi}{3}i)}$の固有ベクトル$\bm{x}$を求める。
\begin{gather}
 \left( A-\exp{(-\frac{\pi}{3}i)} E \right)\bm{x} =0\\
 \bm{x}= k_3 \bm{v_3} + k_4 \bm{v_4}\\
 \bm{v_3}=
 \begin{pmatrix}
  1\\0\\\exp{(\frac{2\pi}{3}i)}\\\exp{(\frac{\pi}{3}i)}
 \end{pmatrix}
 =
 \begin{pmatrix}
  1\\0\\-\frac{1}{2}\\\frac{1}{2}
 \end{pmatrix}
 +i
 \begin{pmatrix}
  0\\0\\\frac{\sqrt{3}}{2}\\\frac{\sqrt{3}}{2}
 \end{pmatrix}
 , \quad
 \bm{v_4}=
 \begin{pmatrix}
  0\\1\\\exp{(-\frac{2\pi}{3}i)}\\\exp{(\frac{2\pi}{3}i)}
 \end{pmatrix}
 =
 \begin{pmatrix}
  0\\1\\-\frac{1}{2}\\-\frac{1}{2}
 \end{pmatrix}
 +i
 \begin{pmatrix}
  0\\0\\-\frac{\sqrt{3}}{2}\\\frac{\sqrt{3}}{2}
 \end{pmatrix}
\end{gather}


$P=(\bm{v_1}, \bm{v_2}, \bm{v_3}, \bm{v_4})$と置くと
$P^{-1}AP$で対角化可能である。

2つの固有空間は直交しているが、$\bm{v_1}, \bm{v_2}$の組と$\bm{v_3}, \bm{v_4}$の組は
直交していない。
それぞれの組にシュミットの直交化法を用いて基底を取り直す。
内積は複素数におけるエルミート内積であり、
内積$(\bm{v_1}\cdot \bm{v_2})$は
$\bm{v_1}$と$\overline{\bm{v_2}}$(成分が共役複素数)の成分ごとの積で求められる。

$\bm{v_1}$の大きさは$\sqrt{(\bm{v_1}\cdot \bm{v_1})}=\sqrt{3}$であるので、
基底$\bm{e_1}$は$\bm{e_1}=\frac{1}{\sqrt{3}}\bm{v_1}$である。
$\bm{e_1}$を利用し$\bm{v_2}$を正規直交化する。
\begin{gather}
 (\bm{v_2}\cdot \bm{e_1})=-i\\
 \bm{v_2^\prime} = \bm{v_2} - (\bm{v_2}\cdot \bm{e_1})\bm{e_1}\\
 \sqrt{(\bm{v_2^\prime} \cdot \bm{v_2^\prime})}=\sqrt{2}\\
 \bm{e_2}=\frac{1}{\sqrt{2}}\bm{v_2^\prime}
\end{gather}

正規直交基底として次のような$\bm{e_1}, \bm{e_2}$を得る。
\begin{equation}
  \bm{e_1}=
   \begin{pmatrix}
    \frac{1}{\sqrt{3}}\\
    0\\
    \frac{1}{\sqrt{3}}\exp{(\frac{-2\pi}{3}i)}\\
    \frac{1}{\sqrt{3}}\exp{(\frac{-\pi}{3}i)}
   \end{pmatrix}
   , \quad
   \bm{e_2}=
   \begin{pmatrix}
    \frac{1}{\sqrt{6}}i\\
    \frac{1}{\sqrt{2}}\\
    \frac{1}{\sqrt{6}}i\\
    -\frac{1}{\sqrt{6}}
   \end{pmatrix}
\end{equation}


同様に$\bm{v_3}, \bm{v_4}$を正規直交基底$\bm{e_3}, \bm{e_4}$を求める。
$\sqrt{(\bm{v_3}\cdot \bm{v_3})}=\sqrt{3}$より
$\bm{e_3}=\frac{1}{\sqrt{3}}\bm{v_3}$とおく。
\begin{gather}
 (\bm{v_4}\cdot \bm{e_3})=i\\
 \bm{v_4^\prime} = \bm{v_4} - (\bm{v_4}\cdot \bm{e_3})\bm{e_3}\\
 \sqrt{(\bm{v_4^\prime} \cdot \bm{v_4^\prime})}=\sqrt{2}\\
 \bm{e_4}=\frac{1}{\sqrt{2}}\bm{v_4^\prime}
\end{gather}
これよりt次のような$\bm{e_3}, \bm{e_4}$を得る。
\begin{equation}
  \bm{e_3}=
   \begin{pmatrix}
    \frac{1}{\sqrt{3}}\\
    0\\
    \frac{1}{\sqrt{3}}\exp{(\frac{2\pi}{3}i)}\\
    \frac{1}{\sqrt{3}}\exp{(\frac{\pi}{3}i)}
   \end{pmatrix}
   , \quad
  \bm{e_4}=
   \begin{pmatrix}
    -\frac{1}{\sqrt{6}}i\\
    \frac{1}{\sqrt{2}}\\
    -\frac{1}{\sqrt{6}}i\\
    \frac{1}{\sqrt{6}}
   \end{pmatrix}
\end{equation}

$e_1,\dots,e_4$を並べた行列$U$がユニタリ行列となる。

\begin{equation}
 U=(\bm{e_1}, \bm{e_2}, \bm{e_3}, \bm{e_4}) =
  \begin{pmatrix}
   \frac{1}{\sqrt{3}} & \frac{1}{\sqrt{6}}i & \frac{1}{\sqrt{3}} & -\frac{1}{\sqrt{6}}i\\
                    0 &   \frac{1}{\sqrt{2}} &                  0 &  \frac{1}{\sqrt{2}}\\
   \frac{1}{\sqrt{3}}\exp{(\frac{-2\pi}{3}i)} & \frac{1}{\sqrt{6}}i
           & \frac{1}{\sqrt{3}}\exp{(\frac{2\pi}{3}i)} & -\frac{1}{\sqrt{6}}i\\
    \frac{1}{\sqrt{3}}\exp{(\frac{-\pi}{3}i)} & -\frac{1}{\sqrt{6}}
           &  \frac{1}{\sqrt{3}}\exp{(\frac{\pi}{3}i)} & \frac{1}{\sqrt{6}}
  \end{pmatrix}
\end{equation}


\begin{gather}
 U^{*}U=UU^{*}=E\\
 U^{*}AU=
 \begin{pmatrix}
  \exp{(\frac{\pi}{3}i)} & 0 & 0 & 0\\
  0 & \exp{(\frac{\pi}{3}i)} & 0 & 0\\
  0 & 0 & \exp{(-\frac{\pi}{3}i)} & 0\\
  0 & 0 & 0 & \exp{(-\frac{\pi}{3}i)}
 \end{pmatrix}
\end{gather}


\hrulefill

行列の計算確認は次のサイトを利用し計算した。

\textbf{Matrix calculator}
\quad
\url{https://matrixcalc.org/ja/}

\textbf{SageMathCell}
\quad
\url{https://sagecell.sagemath.org/}

以下のコードはSageMathで実行した。

\begin{lstlisting}[language=Python,basicstyle={\small},frame=single,numbers=left]
A=Matrix([[1,-1,-1,-1],[1,1,1,-1],[1,-1,1,1],[1,1,-1,1]])*1/2
print("A=")
print(A)
print("A*At=")
print(A*A.transpose())
print("Aの行列式:", A.det())
print("固有多項式:", A.fcp())
print("固有値:", A.eigenvalues())
print("固有値と固有ベクトル:", A.eigenvectors_right())

print("---固有空間の基底---")
#ベクトル
v1=vector([1,0,exp(-2*pi*I/3),exp(-pi*I/3)])
v2=vector([0,1,exp(2*pi*I/3),exp(-2*pi*I/3)])
v3=vector([1,0,exp(2*pi*I/3),exp(pi*I/3)])
v4=vector([0,1,exp(-2*pi*I/3),exp(2*pi*I/3)])
print("v1=", v1)
print("v2=", v2)
print("v3=", v3)
print("v4=", v4)
#内積
print("基底同士の内積-複素内積-エルミート内積")
print("v1,v1 :", expand(v1.inner_product(v1.conjugate())))
print("v1,v2 :", expand(v1.inner_product(v2.conjugate())))
print("v1,v3 :", expand(v1.inner_product(v3.conjugate())))
print("v1,v4 :", expand(v1.inner_product(v4.conjugate())))
print("--")
print("v2,v1 :", expand(v2.inner_product(v1.conjugate())))
print("v2,v2 :", expand(v2.inner_product(v2.conjugate())))
print("v2,v3 :", expand(v2.inner_product(v3.conjugate())))
print("v2,v4 :", expand(v2.inner_product(v4.conjugate())))
print("--")
print("v3,v1 :", expand(v3.inner_product(v1.conjugate())))
print("v3,v2 :", expand(v3.inner_product(v2.conjugate())))
print("v3,v3 :", expand(v3.inner_product(v3.conjugate())))
print("v3,v4 :", expand(v3.inner_product(v4.conjugate())))
print("--")
print("v4,v1 :", expand(v4.inner_product(v1.conjugate())))
print("v4,v2 :", expand(v4.inner_product(v2.conjugate())))
print("v4,v3 :", expand(v4.inner_product(v3.conjugate())))
print("v4,v4 :", expand(v4.inner_product(v4.conjugate())))
print("--")

#対角化
P=Matrix([v1,v2,v3,v4]).transpose()
print("==対角化 P^{-1}AP ==")
print(expand(P.inverse()*A*P))
print("====")

# 以下、直交基底の計算
e1=v1/sqrt(3)
v2a=v2-v2.inner_product(e1.conjugate())*e1
print("v2とe1の内積", expand(v2.inner_product(e1.conjugate())))
print("v2'の大きさ", expand(v2a.inner_product(v2a.conjugate())))
e2=v2a/sqrt(2)
print("e1 = ", e1)
print("e2 = ", e2)

e3=v3/sqrt(3)
v4a=v4-v4.inner_product(e3.conjugate())*e3
print("v4とe3の内積", expand(v4.inner_product(e3.conjugate())))
print("v4'の大きさ", expand(v4a.inner_product(v4a.conjugate())))
e4=v4a/sqrt(2)
print("e3 = ", e3)
print("e4 = ", e4)

U=Matrix([e1,e2,e3,e4]).transpose()
print("-ユニタリ行列U-")
print(expand(U))
print("--ユニタリ行列の確認 UU^*--")
print(expand(U*U.transpose().conjugate()))
print("--ユニタリ行列の確認 U^*U--")
print(expand(U.transpose().conjugate()*U))
print("--ユニタリ行列Uを使った対角化--")
print(expand(U.transpose().conjugate()*A*U))
\end{lstlisting}


\end{document}
