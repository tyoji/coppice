\documentclass[12pt,b5paper]{ltjsarticle}

%\usepackage[margin=15truemm, top=5truemm, bottom=5truemm]{geometry}
\usepackage[margin=15truemm]{geometry}

\usepackage{amsmath,amssymb}
%\pagestyle{headings}
\pagestyle{empty}

%\usepackage{listings,url}
%\renewcommand{\theenumi}{(\arabic{enumi})}

\usepackage{graphicx}

\usepackage{tikz}
\usetikzlibrary {arrows.meta}
\usepackage{wrapfig}	% required for `\wrapfigure' (yatex added)
\usepackage{bm}	% required for `\bm' (yatex added)

% ルビを振る
%\usepackage{luatexja-ruby}	% required for `\ruby'

%% 核Ker 像Im Hom を定義
%\newcommand{\Img}{\mathop{\mathrm{Im}}\nolimits}
%\newcommand{\Ker}{\mathop{\mathrm{Ker}}\nolimits}
%\newcommand{\Hom}{\mathop{\mathrm{Hom}}\nolimits}
%% 回転rot 発散div 勾配grad を定義
%\newcommand{\Rot}{\mathop{\mathrm{rot}}\nolimits}
%\newcommand{\Div}{\mathop{\mathrm{div}}\nolimits}
%\newcommand{\Grad}{\mathop{\mathrm{grad}}\nolimits}
%% 階数rank を定義
\newcommand{\Rnk}{\mathop{\mathrm{rank}}\nolimits}


\begin{document}

\hrulefill

\textbf{Jordan標準形}

ジョルダン標準形はジョルダン細胞の順序の差を除いて
一つだけ存在する。

この為、ジョルダン標準形に変形する正則行列が求まれば
必ずジョルダン標準形が決まる。


\dotfill

\textbf{Jordan標準形の求め方}

$n$次正方行列$A$のJordan標準形の求め方

\begin{enumerate}
 \item
      固有値$\lambda$を求める。
      対角化可能であれば対角行列を求めて終了。

 \item
      \begin{enumerate}
       \item
            固有ベクトル$\bm{x}_{\lambda}$を用いて
            方程式$(A-\lambda E)\bm{x} = \bm{x}_{\lambda}$
            の解$\bm{x}$を求める。

       \item
            もし、解$\bm{x}$が求まれば、
            新しく方程式$(A-\lambda E)\bm{x}_1 = \bm{x}$を作り
            $\bm{x}_1$を求める。

       \item
            同じことを繰り返しベクトルの列$\bm{x}_1,\bm{x}_2,\dots$を求める。

       \item
            このようにすべての固有ベクトル$\bm{x}_{\lambda}$に対して
            方程式を繰り返し解く。

            つまり、$\bm{x}_{\lambda}=(A-\lambda E)^k \bm{p}$と考え、
            次のようなベクトル列を求めることになる。
            \begin{equation}
            (A-\lambda E)^{k-1} \bm{p},(A-\lambda E)^{k-2} \bm{p},\dots,
            (A-\lambda E)^2 \bm{p},(A-\lambda E) \bm{p},\bm{p}
            \end{equation}

      \end{enumerate}

 \item
      求まったベクトルを全部並べて行列$P$を作ると
      $P^{-1}AP$がジョルダン標準形となる。

\end{enumerate}

\hrulefill

次の行列のJordan標準形を求めよ。
\begin{align}
 A=&
  \begin{pmatrix}
   5 & 4 & 2 & 1\\
   0 & 1 & -1 & -1\\
   -1 & -1 & 3 & 0\\
   1 & 1 & -1 & 2
  \end{pmatrix}
  &
 B=&
  \begin{pmatrix}
   4 & -4 & -11 & 11\\
   7 & -16 & -48 & 46\\
   -6 & 16 & 43 & -38\\
   -3 & 9 & 23 & -19
  \end{pmatrix}
 \\
 C=&
  \begin{pmatrix}
   3 & 1 & 0 & 0\\
   -1 & 1 & 0 & 0\\
   0 & 2 & 2 & 1\\
   2 & 2 & 0 & 2
  \end{pmatrix}
  &
 D=&
  \begin{pmatrix}
   5 & 4 & 2 & 1 & 1\\
   0 & 1 & -1 & -1 & 1\\
   -1 & -1 & 3 & 0 & 1\\
   1 & 1 & -1 & 2 & 1\\
   0 & 0 & 2 & 2 & 2
  \end{pmatrix}
\end{align}

%%%%%%%%%% A %%%%%%%%%%
\dotfill
\textbf{$A$のJordan標準形}
\dotfill

$A$の固有値$\lambda$を求める。
\begin{equation}
 \det(A-\lambda E)=(\lambda-1)(\lambda-2)(\lambda-4)^2=0
  \qquad \lambda=1,2,4
\end{equation}

それぞれの固有ベクトル$\bm{x}_{\lambda}$は次の通り。
\begin{equation}
 \lambda =1 ,\ \bm{x}_{\lambda_1}=\begin{pmatrix}-1\\1\\0\\0\end{pmatrix},
 \qquad
 \lambda =2 ,\ \bm{x}_{\lambda_2}=\begin{pmatrix}1\\-1\\0\\1\end{pmatrix},
 \qquad
 \lambda =4 ,\ \bm{x}_{\lambda_4}=\begin{pmatrix}1\\0\\-1\\1\end{pmatrix}
\end{equation}
固有ベクトルが4つ無いので対角化は出来ない。

方程式$(A-\lambda E)\bm{x}=\bm{x}_{\lambda}$の解$\bm{x}$を求める。
\begin{align}
 \Rnk (A-E) =& 3, & \Rnk ((A-E) \ \ \bm{x}_{\lambda_1}) =& 4 \\
 \Rnk (A-2E) =& 3, & \Rnk ((A-2E) \ \ \bm{x}_{\lambda_2}) =& 4
\end{align}
上記の様に階数が求まる。
これにより$\bm{x}_{\lambda_1},\bm{x}_{\lambda_2}$の方程式に解はない。
\begin{align}
 (A-4E)\bm{x} =& \bm{x}_{\lambda_4}\\
 \begin{pmatrix}
  1 & 4 & 2 & 1\\
  0 & -3 & -1 & -1\\
  -1 & -1 & -1 & 0\\
  1 & 1 & -1 & -2
 \end{pmatrix}
 \bm{x}
 =&
 \begin{pmatrix} 1 \\ 0 \\ -1 \\ 1 \end{pmatrix}
 &
 \bm{x}=
 k\begin{pmatrix} 1 \\ 0 \\ -1 \\ 1 \end{pmatrix}
 +\begin{pmatrix} 1 \\ 0 \\ 0 \\ 0 \end{pmatrix}
\end{align}

固有ベクトルとこの$k=0$の時の解を並べて行列$P$を作る。
これによりジョルダン標準形$P^{-1}AP$が求まる。
\begin{equation}
 P=
 \begin{pmatrix}
  -1 & 1 & 1 & 1\\
  1 & -1 & 0 & 0\\
  0 & 0 & -1 & 0\\
  0 & 1 & 1 & 0
 \end{pmatrix}
 ,\quad
 P^{-1}AP=
 \begin{pmatrix}
  1 & 0 & 0 & 0\\
  0 & 2 & 0 & 0\\
  0 & 0 & 4 & 1\\
  0 & 0 & 0 & 4
 \end{pmatrix}
 =J_1(1) \oplus J_1(2) \oplus J_2(4)
\end{equation}


%%%%%%%%%% B %%%%%%%%%%
\dotfill
\textbf{$B$のJordan標準形}
\dotfill

$B$の固有値$\lambda$を求める。
\begin{equation}
 \det(B-\lambda E)=(\lambda-3)^4=0
  \qquad \lambda=3
\end{equation}

固有ベクトル$\bm{x}_{\lambda_3}$は次の通り。
\begin{equation}
 \lambda =3 ,\ \bm{x}_{\lambda_3}=\begin{pmatrix}-1\\-3\\2\\1\end{pmatrix}
\end{equation}
%固有ベクトルが4つ無いので対角化は出来ない。

固有ベクトル$\bm{x}_{\lambda_3}$を用いて
方程式$(B-3E)\bm{x}=\bm{x}_{\lambda_3}$を解く。
\begin{align}
 \begin{pmatrix}
  1 & -4 & -11 & 11\\
  7 & -19 & -48 & 46\\
  -6 & 16 & 40 & -38\\
  -3 & 9 & 23 & -22
 \end{pmatrix}
 \bm{x}
 =&
 \begin{pmatrix} -1 \\ -3 \\ 2 \\ 1 \end{pmatrix}
 &
 \bm{x}=&
 k_1\begin{pmatrix} -1 \\ -3 \\ 2 \\ 1 \end{pmatrix}
 +\begin{pmatrix} -3 \\ -6 \\ 2 \\ 0 \end{pmatrix}\\
 \begin{pmatrix}
  1 & -4 & -11 & 11\\
  7 & -19 & -48 & 46\\
  -6 & 16 & 40 & -38\\
  -3 & 9 & 23 & -22
 \end{pmatrix}
 \bm{x}
 =&
 \begin{pmatrix} -3 \\ -6 \\ 2 \\ 0 \end{pmatrix}
 &
 \bm{x}=&
 k_2\begin{pmatrix} -1 \\ -3 \\ 2 \\ 1 \end{pmatrix}
 +\begin{pmatrix} -19 \\ -37 \\ 12 \\ 0 \end{pmatrix}\\
 \begin{pmatrix}
  1 & -4 & -11 & 11\\
  7 & -19 & -48 & 46\\
  -6 & 16 & 40 & -38\\
  -3 & 9 & 23 & -22
 \end{pmatrix}
 \bm{x}
 =&
 \begin{pmatrix} -19 \\ -37 \\ 12 \\ 0 \end{pmatrix}
 &
 \bm{x}=&
 k_3\begin{pmatrix} -1 \\ -3 \\ 2 \\ 1 \end{pmatrix}
 +\begin{pmatrix} -118 \\ -231 \\ 75 \\ 0 \end{pmatrix}
\end{align}

固有ベクトル$\bm{x}_{\lambda_3}$と求めた3つのベクトルを並べ、
行列$P$を作る。
この行列$P$により$B$のジョルダン標準形$P^{-1}BP$は次のように求まる。
\begin{equation}
 P=
 \begin{pmatrix}
  -1 & -3 & -19 & -118\\
  -3 & -6 & -37 & -231\\
  2 & 2 & 12 & 75\\
  1 & 0 & 0 & 0
 \end{pmatrix}
 ,\quad
 P^{-1}BP=
 \begin{pmatrix}
  3 & 1 & 0 & 0\\
  0 & 3 & 1 & 0\\
  0 & 0 & 3 & 1\\
  0 & 0 & 0 & 3
 \end{pmatrix}
 =J_4(3)
\end{equation}


%%%%%%%%%% C %%%%%%%%%%
\dotfill
\textbf{$C$のJordan標準形}
\dotfill

$C$の固有値$\lambda$を求める。
\begin{equation}
 \det(C-\lambda E)=(\lambda-2)^4=0
  \qquad \lambda=2
\end{equation}

固有ベクトル$\bm{x}_{\lambda_2}$は次の通り。
\begin{equation}
 \lambda =2 ,\
 \bm{x}_{\lambda_2}=
 \begin{pmatrix}0\\0\\1\\0\end{pmatrix},\
 \begin{pmatrix}1\\-1\\0\\2\end{pmatrix}
\end{equation}

固有ベクトル$\bm{x}_{\lambda_2}$を用いて
方程式$(C-2E)\bm{x}=\bm{x}_{\lambda_2}$を解く。
\begin{align}
 \begin{pmatrix}
  1 & 1 & 0 & 0\\
  -1 & -1 & 0 & 0\\
  0 & 2 & 0 & 1\\
  2 & 2 & 0 & 0
 \end{pmatrix}
 \bm{x}
 =&
 \begin{pmatrix} 0 \\ 0 \\ 1 \\ 0 \end{pmatrix}
 &
 \bm{x}=&
 k_1\begin{pmatrix} 0 \\ 0 \\ 1 \\ 0 \end{pmatrix}
 +\begin{pmatrix} 0 \\ 0 \\ 0 \\ 1 \end{pmatrix}\\
 \begin{pmatrix}
  1 & 1 & 0 & 0\\
  -1 & -1 & 0 & 0\\
  0 & 2 & 0 & 1\\
  2 & 2 & 0 & 0
 \end{pmatrix}
 \bm{x}
 =&
 \begin{pmatrix} 1 \\ -1 \\ 0 \\ 2 \end{pmatrix}
 &
 \bm{x}=&
 k_2\begin{pmatrix} 0 \\ 0 \\ 1 \\ 0 \end{pmatrix}
 +\begin{pmatrix} 0 \\ 1 \\ 0 \\ -2 \end{pmatrix}
\end{align}

2つの固有ベクトル$\bm{x}_{\lambda_3}$に対応したベクトルを並べ、
行列$P$を作る。
この行列$P$により$C$のジョルダン標準形$P^{-1}CP$は次のように求まる。
\begin{equation}
 P=
 \begin{pmatrix}
  0 & 0 & 1 & 0\\
  0 & 0 & -1 & 1\\
  1 & 0 & 0 & 0\\
  0 & 1 & 2 & -2
 \end{pmatrix}
 ,\quad
 P^{-1}CP=
 \begin{pmatrix}
  2 & 1 & 0 & 0\\
  0 & 2 & 0 & 0\\
  0 & 0 & 2 & 1\\
  0 & 0 & 0 & 2
 \end{pmatrix}
 =J_2(2) \oplus J_2(2)
\end{equation}



%%%%%%%%%% D %%%%%%%%%%
\dotfill
\textbf{$D$のJordan標準形}
\dotfill

$D$の固有値$\lambda$を求める。
\begin{equation}
 \det(D-\lambda E)=-\lambda(\lambda-1)(\lambda-4)^3=0
  \qquad \lambda=0,1,4
\end{equation}

固有ベクトル$\bm{x}_{\lambda}$は次の通り。
\begin{equation}
 [\lambda =0] ,
 \bm{x}_{\lambda_0}=
 \begin{pmatrix}27\\-32\\-7\\-9\\16\end{pmatrix},\quad
%
 [\lambda =1] ,
 \bm{x}_{\lambda_1}=
 \begin{pmatrix}-1\\1\\0\\0\\0\end{pmatrix},\quad
%
 [\lambda =4] ,
 \bm{x}_{\lambda_4}=
 \begin{pmatrix}1\\0\\-1\\1\\0\end{pmatrix}
\end{equation}

方程式$(D-\lambda E)\bm{x}=\bm{x}_{\lambda}$の解を求める。
階数を調べることにより$\bm{x}_{\lambda_0},\bm{x}_{\lambda_1}$
の方程式には解はない。
\begin{align}
 \Rnk(D-0E)=&4, & \quad \Rnk((D-0E) \ \ \bm{x}_{\lambda_0})=&5\\
 \Rnk(D-1E)=&4, & \quad \Rnk((D-1E) \ \ \bm{x}_{\lambda_1})=&5
\end{align}

固有ベクトル$\bm{x}_{\lambda_4}$を用いて
方程式$(D-4E)\bm{x}=\bm{x}_{\lambda_4}$を解く。
\begin{align}
 \begin{pmatrix}
  1 & 4 & 2 & 1 & 1 \\
  0 & -3 & -1 & -1 & 1 \\
  -1 & -1 & -1 & 0 & 1 \\
  1 & 1 & -1 & -2 & 1 \\
  0 & 0 & 2 & 2 & -2
 \end{pmatrix}
 \bm{x}
 =&
 \begin{pmatrix} 1 \\ 0 \\ -1 \\ 1 \\ 0 \end{pmatrix}
 &
 \bm{x}=&
 k_1\begin{pmatrix} 1 \\ 0 \\ -1 \\ 1 \\ 0 \end{pmatrix}
 +\begin{pmatrix} 1 \\ 0 \\ 0 \\ 0 \\ 0 \end{pmatrix}\\
 \begin{pmatrix}
  1 & 4 & 2 & 1 & 1 \\
  0 & -3 & -1 & -1 & 1 \\
  -1 & -1 & -1 & 0 & 1 \\
  1 & 1 & -1 & -2 & 1 \\
  0 & 0 & 2 & 2 & -2
 \end{pmatrix}
 \bm{x}
 =&
 \begin{pmatrix} 1 \\ 0 \\ 0 \\ 0 \\ 0 \end{pmatrix}
 &
 \bm{x}=&
 k_2\begin{pmatrix} 1 \\ 0 \\ -1 \\ 1 \\ 0 \end{pmatrix}
 +\frac{1}{3}\begin{pmatrix} 0 \\ 0 \\ 1 \\ 0 \\ 1 \end{pmatrix}
\end{align}

固有ベクトル$\bm{x}_{\lambda_0},\bm{x}_{\lambda_1},\bm{x}_{\lambda_4}$と
$\bm{x}_{\lambda_4}$を用いた方程式から求めたベクトルを並べ、
行列$P$を作る。
この行列$P$により$C$のジョルダン標準形$P^{-1}DP$は次のように求まる。
\begin{equation}
 P=
 \begin{pmatrix}
  27 & -1 & 1 & 1 & 0\\
  -32 & 1 & 0 & 0 & 0\\
  -7 & 0 & -1 & 0 & 1/3\\
  -9 & 0 & 1 & 0 & 0\\
  16 & 0 & 0 & 0 & 1/3
 \end{pmatrix}
 ,\quad
 P^{-1}DP=
 \begin{pmatrix}
  0 & 0 & 0 & 0 & 0 \\
  0 & 1 & 0 & 0 & 0 \\
  0 & 0 & 4 & 1 & 0\\
  0 & 0 & 0 & 4 & 1\\
  0 & 0 & 0 & 0 & 4
 \end{pmatrix}
\end{equation}

これにより
ジョルダン標準形は$P^{-1}DP = J_1(0) \oplus J_1(1) \oplus J_3(4)$
となる。

\hrulefill

\end{document}
