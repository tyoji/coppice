\documentclass[12pt,b5paper]{ltjsarticle}

%\usepackage[margin=15truemm, top=5truemm, bottom=5truemm]{geometry}
%\usepackage[margin=10truemm,left=15truemm]{geometry}
\usepackage[margin=10truemm]{geometry}

\usepackage{amsmath,amssymb}
%\pagestyle{headings}
\pagestyle{empty}

%\usepackage{listings,url}
%\renewcommand{\theenumi}{(\arabic{enumi})}

%\usepackage{graphicx}

%\usepackage{tikz}
%\usetikzlibrary {arrows.meta}
%\usepackage{wrapfig}
%\usepackage{bm}

% ルビを振る
%\usepackage{luatexja-ruby}	% required for `\ruby'

%% 核Ker 像Im Hom を定義
%\newcommand{\Img}{\mathop{\mathrm{Im}}\nolimits}
%\newcommand{\Ker}{\mathop{\mathrm{Ker}}\nolimits}
%\newcommand{\Hom}{\mathop{\mathrm{Hom}}\nolimits}

%\DeclareMathOperator{\Rot}{rot}
%\DeclareMathOperator{\Div}{div}
%\DeclareMathOperator{\Grad}{grad}
%\DeclareMathOperator{\arcsinh}{arcsinh}
%\DeclareMathOperator{\arccosh}{arccosh}
%\DeclareMathOperator{\arctanh}{arctanh}



%\usepackage{listings,url}
%
%\lstset{
%%プログラム言語(複数の言語に対応,C,C++も可)
%  language = Python,
%%  language = Lisp,
%%  language = C,
%  %背景色と透過度
%  %backgroundcolor={\color[gray]{.90}},
%  %枠外に行った時の自動改行
%  breaklines = true,
%  %自動改行後のインデント量(デフォルトでは20[pt])
%  breakindent = 10pt,
%  %標準の書体
%%  basicstyle = \ttfamily\scriptsize,
%  basicstyle = \ttfamily,
%  %コメントの書体
%%  commentstyle = {\itshape \color[cmyk]{1,0.4,1,0}},
%  %関数名等の色の設定
%  classoffset = 0,
%  %キーワード(int, ifなど)の書体
%%  keywordstyle = {\bfseries \color[cmyk]{0,1,0,0}},
%  %表示する文字の書体
%  %stringstyle = {\ttfamily \color[rgb]{0,0,1}},
%  %枠 "t"は上に線を記載, "T"は上に二重線を記載
%  %他オプション:leftline,topline,bottomline,lines,single,shadowbox
%  frame = TBrl,
%  %frameまでの間隔(行番号とプログラムの間)
%  framesep = 5pt,
%  %行番号の位置
%  numbers = left,
%  %行番号の間隔
%  stepnumber = 1,
%  %行番号の書体
%%  numberstyle = \tiny,
%  %タブの大きさ
%  tabsize = 4,
%  %キャプションの場所("tb"ならば上下両方に記載)
%  captionpos = t
%}



\begin{document}


\hrulefill

\textbf{$\sigma$-加法族}

集合$X$の集合族$\Sigma$が
「$\sigma$-加法族である」
とは次を満たすときをいう。
\begin{enumerate}
 \item $X \in \Sigma$
 \item $A \in \Sigma \Rightarrow A^{c}\in\Sigma$
 \item $A_{i}\in\Sigma \ (i\in\mathbb{N})
       \Rightarrow \bigcup_{i=1}^{\infty}A_{i}\in\Sigma$
\end{enumerate}

\textbf{生成される$\sigma$-加法族}

$X$の部分集合族$\mathcal{A}$について、
$\mathcal{A}$を含む最小の$\sigma$-加法族を
$\sigma_{X}(\mathcal{A})$と表す。
\begin{equation}
 \sigma_{X}(\mathcal{A})=
  \bigcap_{\substack{\mathcal{M}:\sigma\text{-加法族} \\ \mathcal{A}\subset \mathcal{M}}}\mathcal{M}
\end{equation}

\textbf{ボレル$\sigma$-加法族}

$(X,\mathcal{O})$を位相空間とする。
$\sigma_{X}(\mathcal{O})$を
$X$上のボレル$\sigma$-加法族といい、
$\mathcal{B}(X)$と表す。

\hrulefill

\begin{enumerate}
 \item
      $X=\{ 1,2,3,4,5\},\mathcal{A}=\{ \{1\}, \{3,4\}\}$
      とする時、
      $\sigma_{X}(\mathcal{A})$を具体的に書け。

      \dotfill

      \begin{align}
       \{1\}^{c} =& \{2,3,4,5\} &
       \{3,4\}^{c} =& \{1,2,5\} &
       \{1\} \cap \{3,4\} =& \emptyset\\
       \{1\} \cup \{3,4\} =& \{1,3,4\} &
       \{1\}^{c} \cup \{3,4\}^{c} =& X
      \end{align}

      よって、$\sigma$-加法族$\sigma_{X}(\mathcal{A})$
      は次のようになる。
      \begin{equation}
       \sigma_{X}(\mathcal{A})
        =\{
        \emptyset,
          \{1\}, \{3,4\},
          \{1,2,5\}, \{1,3,4\},
          \{2,3,4,5\}, X
        \}
      \end{equation}

      \hrulefill
 \item
      $\mathbb{R}$には通常の位相を入れるものとする。
      この位相を$\mathcal{O}$とする。
      \begin{enumerate}
       \item
            $\mathbb{R}$の開集合$U(\ne \emptyset)$
            を任意にとる。
            有理数$x\in U$に対し、
            $I_{x}=\bigcup_{\substack{I:\text{開区間}\\x\in I \subset U}}I$
            と定義する時、
            $U=\bigcup_{x\in U\cap \mathbb{Q}}I_{x}$が成り立つことを示せ。

            \dotfill

            $U \supset \bigcup_{x\in U\cap \mathbb{Q}}I_{x}$
            を示す。

            ${}^{\forall}\alpha\in\bigcup_{x\in U\cap \mathbb{Q}}I_{x}$とする。
            このとき、$\alpha\in I_{x}$となる$I_{x}$が存在する。

            $I_{x}=\bigcup_{\substack{I:\text{開区間}\\x\in I \subset U}}I$
            より、$I_{x} \subset U$であるため、
            $\alpha\in U$である。



            $U \subset \bigcup_{x\in U\cap \mathbb{Q}}I_{x}$
            を示す。

            ${}^{\forall}\beta\in U$とする。
            $\beta$が有理数と無理数の場合を考える。
            
            $\beta$が有理数であれば、
            $\beta\in I_{\beta} \subset \bigcup_{x\in U\cap \mathbb{Q}}I_{x}$
            である。

            $\beta$が無理数とする。
            $U$は開集合であるので、
            $\varepsilon$-近傍
            $U_{\varepsilon}(\beta)=(\beta-\varepsilon, \beta+\varepsilon)$
            で
            $U_{\varepsilon}(\beta) \subset U$となるものが存在する。
            この$\varepsilon$-近傍に含まれる有理数$x_{\beta}$を
            一つ取ってくると、
            $U_{\varepsilon}(\beta) \subset I_{x_{\beta}}$
            である。
            よって、
            $U_{\varepsilon}(\beta) \subset \bigcup_{x\in U\cap \mathbb{Q}}I_{x}$
            であるので、
            $\beta \in\bigcup_{x\in U\cap \mathbb{Q}}I_{x}$
            となる。
            


            \hrulefill
       \item
            $\mathcal{A}=\{ (-\infty,a] \mid a\in\mathbb{R} \}$
            とする時、
            $\sigma_{\mathbb{R}}(\mathcal{A}) = \mathcal{B}(\mathbb{R})$
            が成り立つことを示せ。

            \dotfill

            $\sigma_{\mathbb{R}}(\mathcal{A}) \supset \mathcal{B}(\mathbb{R})$
            と
            $\sigma_{\mathbb{R}}(\mathcal{A}) \subset \mathcal{B}(\mathbb{R})$
            を示す。

            \fbox{
            $\sigma_{\mathbb{R}}(\mathcal{A}) \supset \mathcal{B}(\mathbb{R})$
            }

            ${}^{\forall}O\in\mathcal{O}$
            とする。
            $O=(a,b) \ (a,b\in\mathbb{R})$である。
            
            $A,B_{n}\in\mathcal{A}$を
            $A=(-\infty,a],\ B_{n}=(-\infty,b-1/n]$
            とする。
            これにより、$O$を次のようにかける。
            \begin{equation}
             O=A^{c} \cap \bigcup_{n=1}^{\infty}B_{n}
              \in\sigma_{\mathbb{R}}(\mathcal{A})
            \end{equation}
            よって、
            $\mathcal{O} \subset \sigma_{\mathbb{R}}(\mathcal{A})$であり、
            $\mathcal{B}(\mathbb{R}) \subset \sigma_{\mathbb{R}}(\mathcal{A})$
            である。

            \fbox{
            $\sigma_{\mathbb{R}}(\mathcal{A}) \subset \mathcal{B}(\mathbb{R})$
            }

            ${}^{\forall}A\in\mathcal{A}$とする。
            この時、$A=(-\infty,a] \ (a\in\mathbb{R})$である。
            ${}^{\forall}n\in\mathbb{N}$に対し、
            $B_{n}=(-\infty,a+1/n)$
            とおけば、
            $B_{n}\in\mathcal{O}$であり、
            $B_{n}\in \sigma_{\mathbb{R}}(\mathcal{O})=\mathcal{B}(\mathbb{R})$
            である。
            これより、
            $A=\bigcap_{n=1}^{\infty}B_{n}\in\mathcal{B}(\mathbb{R})$である。
            よって、
            $\mathcal{A}\subset \mathcal{B}(\mathbb{R})$より、
            $\sigma_{\mathbb{R}}(\mathcal{A}) \subset \mathcal{B}(\mathbb{R})$
            となる。


            以上により、
            $\sigma_{\mathbb{R}}(\mathcal{A}) = \mathcal{B}(\mathbb{R})$
            である。

      \end{enumerate}

\end{enumerate}

\hrulefill

\end{document}
