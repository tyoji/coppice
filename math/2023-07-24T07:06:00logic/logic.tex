\documentclass[12pt,b5paper]{ltjsarticle}

%\usepackage[margin=15truemm, top=5truemm, bottom=5truemm]{geometry}
%\usepackage[margin=10truemm,left=15truemm]{geometry}
\usepackage[margin=10truemm]{geometry}

\usepackage{amsmath,amssymb}
%\pagestyle{headings}
\pagestyle{empty}

%\usepackage{listings,url}
%\renewcommand{\theenumi}{(\arabic{enumi})}

%\usepackage{graphicx}

%\usepackage{tikz}
%\usetikzlibrary {arrows.meta}
%\usepackage{wrapfig}
%\usepackage{bm}

% ルビを振る
\usepackage{luatexja-ruby}	% required for `\ruby'

%% 核Ker 像Im Hom を定義
%\newcommand{\Img}{\mathop{\mathrm{Im}}\nolimits}
%\newcommand{\Ker}{\mathop{\mathrm{Ker}}\nolimits}
%\newcommand{\Hom}{\mathop{\mathrm{Hom}}\nolimits}

%\DeclareMathOperator{\Rot}{rot}
%\DeclareMathOperator{\Div}{div}
%\DeclareMathOperator{\Grad}{grad}
%\DeclareMathOperator{\arcsinh}{arcsinh}
%\DeclareMathOperator{\arccosh}{arccosh}
%\DeclareMathOperator{\arctanh}{arctanh}

\usepackage{url}

%\usepackage{listings}
%
%\lstset{
%%プログラム言語(複数の言語に対応,C,C++も可)
%  language = Python,
%%  language = Lisp,
%%  language = C,
%  %背景色と透過度
%  %backgroundcolor={\color[gray]{.90}},
%  %枠外に行った時の自動改行
%  breaklines = true,
%  %自動改行後のインデント量(デフォルトでは20[pt])
%  breakindent = 10pt,
%  %標準の書体
%%  basicstyle = \ttfamily\scriptsize,
%  basicstyle = \ttfamily,
%  %コメントの書体
%%  commentstyle = {\itshape \color[cmyk]{1,0.4,1,0}},
%  %関数名等の色の設定
%  classoffset = 0,
%  %キーワード(int, ifなど)の書体
%%  keywordstyle = {\bfseries \color[cmyk]{0,1,0,0}},
%  %表示する文字の書体
%  %stringstyle = {\ttfamily \color[rgb]{0,0,1}},
%  %枠 "t"は上に線を記載, "T"は上に二重線を記載
%  %他オプション:leftline,topline,bottomline,lines,single,shadowbox
%  frame = TBrl,
%  %frameまでの間隔(行番号とプログラムの間)
%  framesep = 5pt,
%  %行番号の位置
%  numbers = left,
%  %行番号の間隔
%  stepnumber = 1,
%  %行番号の書体
%%  numberstyle = \tiny,
%  %タブの大きさ
%  tabsize = 4,
%  %キャプションの場所("tb"ならば上下両方に記載)
%  captionpos = t
%}

\usepackage{cancel}
\usepackage{bussproofs}
\usepackage{proof}

\begin{document}

\hrulefill

参考文献

ゲンツェンの自然演繹法 --- Mukai Kuniaki 慶応大学

\url{http://web.sfc.keio.ac.jp/~mukai/modular/gentzen-NK.pdf}

\hrulefill

自然演繹
--Natural deduction

\hrulefill
\textbf{NK}
\hrulefill

以下にある、9つの推論規則
を有限回組み合わせて得られる
「論理式たちを並べた樹形図」のことを
 NK の証明図という。

\begin{enumerate}
 \item
      仮定の宣言
      \begin{center}
       仮定 $n: A$
      \end{center}
      \dotfill
 \item
      $\land$導入規則
      \begin{prooftree}
        \AxiomC{$A$}
        \AxiomC{$B$}
        \RightLabel{($I \land$)}
        \BinaryInfC{$A\land B$}
      \end{prooftree}
      \dotfill
 \item
      $\land$除去規則
      \begin{prooftree}
        \AxiomC{$A \land B$}
        \RightLabel{($E \land$)}
        \UnaryInfC{$A$}

       \DisplayProof
       ,\qquad
        \AxiomC{$A \land B$}
        \RightLabel{($E \land$)}
        \UnaryInfC{$B$}
      \end{prooftree}
      \dotfill
 \item
      $\to$導入規則
      \begin{prooftree}
       \AxiomC{仮定$n: A$}
       \def\extraVskip{-1.5pt}
       \noLine
       \UnaryInfC{$\vdots$}
       \def\extraVskip{2pt}
       \noLine
       \UnaryInfC{$B$}
       \RightLabel{($I \to$, 仮定$n$)}
       \UnaryInfC{$A \to B$}
      \end{prooftree}
      \dotfill
 \item
      $\to$除去規則
      \begin{prooftree}
        \AxiomC{$A$}
        \AxiomC{$A \to B$}
        \RightLabel{($E \to$)}
        \BinaryInfC{$B$}
      \end{prooftree}
      \dotfill
 \item
      $\neg$導入規則
      \begin{prooftree}
       \AxiomC{仮定$\cancel{n: A}$}
       \def\extraVskip{-1.5pt}
       \noLine
       \UnaryInfC{$\vdots$}
       \def\extraVskip{2pt}
       \noLine
       \UnaryInfC{$\bot$}
       \RightLabel{($I \neg$, 仮定$n$)}
       \UnaryInfC{$\neg A$}
      \end{prooftree}
      \dotfill
 \item
      $\neg$除去規則
      \begin{prooftree}
       \AxiomC{仮定$\cancel{n: \neg A}$}
       \def\extraVskip{-1.5pt}
       \noLine
       \UnaryInfC{$\vdots$}
       \def\extraVskip{2pt}
       \noLine
       \UnaryInfC{$\bot$}
       \RightLabel{($E \neg$, 仮定$n$)}
       \UnaryInfC{$A$}
      \end{prooftree}
      \dotfill
 \item
      $\lor$導入規則
      \begin{prooftree}
       \AxiomC{$A$}
       \RightLabel{($I \lor$)}
       \UnaryInfC{$A \lor B$}
       %
       \DisplayProof
       ,\qquad
       \AxiomC{$B$}
       \RightLabel{($I \lor$)}
       \UnaryInfC{$A \lor B$}
      \end{prooftree}
      \dotfill
 \item
      $\lor$除去規則
      \begin{prooftree}
       \AxiomC{$A \lor B$}
       %
       \AxiomC{仮定$\cancel{n:A}$}
       \def\extraVskip{-1.5pt}
       \noLine
       \UnaryInfC{$\vdots$}
       \def\extraVskip{2pt}
       \noLine
       \UnaryInfC{$C$}
       %
       \AxiomC{仮定$\cancel{m:B}$}
       \def\extraVskip{-1.5pt}
       \noLine
       \UnaryInfC{$\vdots$}
       \def\extraVskip{2pt}
       \noLine
       \UnaryInfC{$C$}
       %
       \RightLabel{($E \lor$, 仮定$n,m$)}
       \TrinaryInfC{$C$}
      \end{prooftree}


\end{enumerate}


\hrulefill
\textbf{HK}
\hrulefill

以下のどれかの形をした論理式をすべて集めた集合を
HK の (論理)公理
と呼ぶ
\begin{enumerate}
 \item $A \to (B \to A)$
 \item $(A \to (B \to C )) \to ((A \to B) \to (A \to C))$
 \item $(\neg B \to \neg A) \to (A \to B)$
 \item
      \begin{enumerate}
       \item $(A \lor B) \to (\neg A \to B)$
       \item $(\neg A \to B) \to (A \lor B)$
      \end{enumerate}
 \item
      \begin{enumerate}
       \item $(A \land B) \to \neg (A \to \neg B)$
       \item $\neg (A \to \neg B) \to (A \land B)$
      \end{enumerate}
\end{enumerate}

「$A$と$A \to B$から $B$を導く」という規則を
モーダス・ポーネンス
という

上記5つの公理と
1つの推論規則(モーダス・ポーネンス)
のみを用いて
HK の証明図は作る

\hrulefill

\begin{enumerate}
  \item
       \textbf{命題論理}

      $A$を命題論理の論理式とする。
      $A\to A$の
      HK
      における証明図を書け。
      また、$(\neg A) \vee A$
      の
      NK
      における証明図を書け。

      \dotfill

       $\vdash_{\mathrm{HK}} A \to A$
       の証明図

       \begin{enumerate}
        \renewcommand{\labelenumii}{[\arabic{enumii}]}
        \item
             HK の 公理1 より
             (公理1の$B$を$A\to A$に置き換えた)

             $D_{0} = A \to ( (A\to A) \to A)$

        \item
             HK の 公理2 より
             (公理2の$B$を$A\to A$に、$C$を$A$に置き換えた)

             $D_{1} = (A \to ((A\to A) \to A)) \to ((A \to (A\to A)) \to (A \to A))$

        \item
             $D_{0}$と$D_{1}$にモーダスポーネンスを適用

             $D_{2} = (A \to (A\to A)) \to (A \to A)$

        \item
             HK の 公理1 より

             $D_{3} = A \to (A\to A)$

        \item
             $D_{3}$と$D_{2}$にモーダスポーネンスを適用

             $D_{4} = A \to A$
       \end{enumerate}


%
%%       \begin{prooftree}
%%        %ここに証明図を書く
%%        \AxiomC{$(\Phi \land \Psi)$}
%%        \UnaryInfC{$\Phi$}
%%       \end{prooftree}
%
%       \begin{prooftree}
%        %ここに証明図を書く
%        \AxiomC{$\Phi$}
%        \AxiomC{$\Psi$}
%        \BinaryInfC{$\Phi \land \Psi$}
%       \end{prooftree}
%
%
%       \begin{prooftree}
%        %ここに証明図を書く
%        \AxiomC{$\Phi$}
%        \AxiomC{$\Psi$}
%        \BinaryInfC{$\Phi \land \Psi$}
%        \AxiomC{$X \to \Upsilon$}
%        \BinaryInfC{$(\Phi \land \Psi) \land (X \to \Upsilon)$}
%       \end{prooftree}
%
%       \begin{prooftree}
%        %ここに証明図を書く
%        \AxiomC{$\Gamma$}
%        \AxiomC{$[\Phi]_1$}
%        \BinaryInfC{$\Psi$}
%        \RightLabel{{\scriptsize 1}}
%        \UnaryInfC{$\Phi \to \Psi$}
%       \end{prooftree}
%
%
%
%       \begin{prooftree}
%        \AxiomC{A}
%        \AxiomC{B}
%        \AxiomC{C}
%        \BinaryInfC{D}
%        \RightLabel{\scriptsize(1)}
%        \BinaryInfC{E}
%        \DisplayProof
%        \qquad
%        \AxiomC{A}
%       \end{prooftree}
%
%%       \begin{prooftree}
%%\AX and \AXC abbreviate \Axiom and \AxiomC
%%\UI and \UIC abbreviate \UnaryInf and \UnaryInfC
%%\BI and \BIC abbreviate \BinaryInf and \BinaryInfC
%%\TI and \TIC abbreviate \TrinaryInf and \TrinaryInfC
%%\DP abbreviates \DisplayProof
%%       \end{prooftree}
%
%
%       \infer[(L\to)]{A, A \to B \vdash B}{A \vdash A \& B \vdash B}


\dotfill

       $\vdash_{\mathrm{NK}} (\neg A) \lor A$
       の証明図

%       \begin{prooftree}
%        \AxiomC{\neg A}
%        \AxiomC{A}
%%        \RightLabel{\scriptsize(1)}
%        \BinaryInfC{\top}
%       \end{prooftree}
%
%
%       \begin{prooftree}
%       \AxiomC{仮定$\cancel{1:\neg (A \lor \neg A)}$}
%       \def\extraVskip{-1.5pt}
%       \noLine
%       \UnaryInfC{$\vdots$}
%       \def\extraVskip{2pt}
%       \noLine
%       \UnaryInfC{$\bot$}
%       \RightLabel{($E \neg$, 仮定$1$)}
%       \UnaryInfC{$A \lor \neg A$}
%       \end{prooftree}
%
%       \begin{prooftree}
%        \AxiomC{仮定$\cancel{2:\neg A}$}
%        \def\extraVskip{-1.5pt}
%        \noLine
%        \UnaryInfC{$\vdots$}
%        \def\extraVskip{2pt}
%        \noLine
%        \UnaryInfC{$\bot$}
%        \RightLabel{($E \neg$, 仮定$2$)}
%        \UnaryInfC{$A$}
%        \RightLabel{($I \lor$)}
%        \UnaryInfC{$A \lor \neg A$}
%       \end{prooftree}
%
%       \begin{prooftree}
%       \AxiomC{仮定$\cancel{3:\neg (A \lor \neg A)}$}
%       \def\extraVskip{-1.5pt}
%       \noLine
%       \UnaryInfC{$\vdots$}
%       \def\extraVskip{2pt}
%       \noLine
%       \UnaryInfC{$\bot$}
%       \RightLabel{($E \neg$, 仮定$3$)}
%       \UnaryInfC{$A \lor \neg A$}
%       \end{prooftree}



       \begin{prooftree}
        \AxiomC{\cancel{仮定$1:\neg ( (\neg A) \lor A)$}}
         %
        \AxiomC{\cancel{仮定$2: A$}}
        \RightLabel{($I \lor$)}
        \UnaryInfC{$(\neg A) \lor A$}
%
        \RightLabel{($I \land$)}
        \BinaryInfC{$\bot$}
        \RightLabel{($I \neg$, 仮定$2$)}
        \UnaryInfC{$\neg A$}
%
        \RightLabel{($I \lor$)}
        \UnaryInfC{$\neg A \lor A$}
%
        \AxiomC{\cancel{仮定$3:\neg ( (\neg A) \lor A)$}}
        \RightLabel{($I \land$)}
        \BinaryInfC{$\bot$}
        \RightLabel{($E \neg$, 仮定$1,3$)}
        \UnaryInfC{$(\neg A) \lor A$}
       \end{prooftree}





      \hrulefill

  \item
       \textbf{述語論理}


       \begin{enumerate}
        \item
             $L_{1}=\{c,f\}, L_{2}=\{P\}$
             とする。
             ただし、
             $c$は定数記号、
             $f$は2変数関数記号、
             $P$は1変数述語記号である。
             
             $L_{3}=L_{1}\cup L_{2}$とする。
             自由変数を1つ、
             束縛変数を1つもつ
             $L_{3}$論理式で、
             $i< 3$に対しては
             $L_{i}$論理式とはならない
             論理式の例を挙げよ。

             また、
             $L_{2}$文だが、
             $L_{1}$文ではないような文の例を挙げよ。

             更に、
             $L_{1}$文にも
             $L_{2}$文にもなっているような分の例
             を与えよ。

             \dotfill


             $\forall v_{0}P( f(v_{0},v_{1}) )$
             は
             $f\in L_{3}, \; P\in L_{3}$であるので
             $L_{3}$論理式である。
             変数は$v_{0},v_{1}$とあり、
             $\forall v_{0}$とあるので
             束縛変数が1つと自由変数が1つである。
             $P \not\in L_{1}$であるので$L_{1}$論理式ではなく、
             $f\not\in L_{2}$であるので$L_{2}$論理式でもない。

             $\exists v_{0}P(v_{0})$は
             束縛変数が一つのみであるので、
             $L_{2}$文であるが、$P \not\in L_{1}$であるので
             $L_{1}$文ではない。

             $\forall v_{0} \exists v_{1} (v_{0} = v_{1})$は
             $L_{1}$文でも$L_{2}$文でもある。


             \hrulefill

        \item
             $L=\{f,P\}$とする。
             ただし、
             $f$は2変数関数記号、
             $P$は2変数述語記号である。

             $L$構造$M$と$N$で次の性質を満たすような例を挙げよ。
             \begin{equation}
              M \vDash \forall x \, \exists y \, \left(
                P\left( f\left(x,y\right), y \right)
              \right)
              , \qquad
              N \vDash \neg \left(
                \forall x \, \exists y \, P\left( f\left(x,y\right), y \right)
              \right)
             \end{equation}

             \dotfill

             $M = (\mathbb{Z}, \, 0, \, -, \, <)$とする。
             この時、
             任意の$x$に対して
             $x-y < y$となる$y$が存在する。
             具体的には$y>\frac{x}{2}$を満たす$y$を取ってくればよい。

             $N \vDash \neg \left(
                \forall x \, \exists y \, P\left( f\left(x,y\right), y \right)
              \right)$
             は次のように
             $N \vDash
             \exists x \, \forall y \,
             \neg P\left( f\left(x,y\right), y \right)$
             と考えられる。
             \begin{equation}
               \neg \left(
                \forall x \, \exists y \, P\left( f\left(x,y\right), y \right)
              \right)
               \quad \Leftrightarrow \quad
               \exists x \, \forall y \, \neg P\left( f\left(x,y\right), y \right)
             \end{equation}


             $N=(\mathbb{Z}, \, 0, \, \times, \, <)$とすると、
             ある$x$が存在して
             任意の$y$に対して$xy<y$とならない。
             具体的には$x=1$であれば、任意の$y$に対して$y<y$を満たさない。




             \hrulefill

       \end{enumerate}







%
%      \dotfill
%
%      $\varphi(x)$を任意の$L$論理式とする。
%      $\vdash_{\mathrm{NK}} \forall x \varphi(x) \leftrightarrow \neg \exists x \neg \varphi(x)$
%       を示す
%      NK
%      の証明図をかけ。
%      また、
%       $\vdash_{\mathrm{NK}} \exists x \varphi(x) \leftrightarrow \neg \forall x \neg \varphi(x)$
%      を示せ。
%
%
%      \dotfill
%
%
%       $\forall x \varphi(x) \leftrightarrow \neg \exists x \neg \varphi(x)$
%       とは
%       $\forall x \varphi(x) \rightarrow \neg \exists x \neg \varphi(x)$
%       と
%       $\neg \exists x \neg \varphi(x) \rightarrow \forall x \varphi(x)$
%       の2つを省略して書いたものである。
%
%
%       $\forall x \varphi(x) \rightarrow \neg \exists x \neg \varphi(x)$
%       \begin{prooftree}
%        \AxiomC{$\forall x \varphi(x)$}
%        \RightLabel{($E \forall$)}
%        \UnaryInfC{$\varphi(x)$}
%       \end{prooftree}
%
%
%
%       $\neg \exists x \neg \varphi(x) \rightarrow \forall x \varphi(x)$
%       \begin{prooftree}
%        \AxiomC{$\neg \exists x \neg \varphi(x)$}
%        \RightLabel{($E \exists$)}
%        \UnaryInfC{$\varphi(x)$}
%       \end{prooftree}
%
%      \hrulefill

\end{enumerate}

\hrulefill

\end{document}
