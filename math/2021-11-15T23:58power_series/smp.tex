\documentclass[12pt,b5paper]{ltjsarticle}
%\documentclass{jsarticle}

%\usepackage{amsmath}

\pagestyle{empty}

\begin{document}

\begin{equation}
 1-x^3+x^6-x^9+ \cdots
 =  \frac{1}{1+x^3}
\end{equation}

上の式は次の式から求まる。

\begin{equation}
 \sum_{i=0}^n (-x^3)^i =
 1+(-x^3)+(-x^3)^2+\cdots + (-x^3)^n
\end{equation}

\begin{equation}
 (-x^3)\sum_{i=0}^n (-x^3)^i =
 (-x^3)(1+(-x^3)+(-x^3)^2+\cdots + (-x^3)^n)
\end{equation}

上記の2つの式の差を考えると
\begin{equation}
 (1-(-x^3))\sum_{i=0}^n (-x^3)^i =
 1 - (-x^3)^{n+1}
\end{equation}

$1+x^3 \ne 0$の時、
式を変形し極限を取ると
\begin{equation}
 \lim_{n\rightarrow\infty}\sum_{i=0}^n (-x^3)^i =
 \lim_{n\rightarrow\infty}\frac{1 - (-x^3)^{n+1}}{1+x^3}
\end{equation}

%\begin{equation}
% \lim_{n\rightarrow\infty}\sum_{i=0}^n (-x^3)^i =
% \frac{1 + \lim_{n\rightarrow\infty}(-1)^n(x^3)^{n+1}}{1+x^3}
%\end{equation}

$\lim_{n\rightarrow\infty}(-x^3)^{n+1}$が
$|x|<1$のとき収束するので、
$|x|<1$のとき次の式が成り立つ。
\begin{equation}
 1-x^3+x^6-x^9+ \cdots
 = \frac{1}{1+x^3}
\end{equation}

この式の定積分を$|x|<1$の範囲で行うので、
\begin{equation}
 \int_0^1(1-x^3+x^6-x^9+ \cdots)dx
  = \int_0^1\frac{1}{1+x^3}dx
\end{equation}
となる。

\end{document}
