\documentclass[12pt,b5paper]{ltjsarticle}

%\usepackage[margin=15truemm, top=5truemm, bottom=5truemm]{geometry}
\usepackage[margin=15truemm]{geometry}

\usepackage{amsmath,amssymb}
%\pagestyle{headings}
\pagestyle{empty}

%\usepackage{listings,url}
%\renewcommand{\theenumi}{(\arabic{enumi})}

\usepackage{graphicx}

\usepackage{tikz}
\usetikzlibrary {arrows.meta}
\usepackage{wrapfig}    % required for `\wrapfigure' (yatex added)
\usepackage{bm} % required for `\bm' (yatex added)

% ルビを振る
%\usepackage{luatexja-ruby}      % required for `\ruby'

%% 核Ker 像Im Hom を定義
%\newcommand{\Img}{\mathop{\mathrm{Im}}\nolimits}
%\newcommand{\Ker}{\mathop{\mathrm{Ker}}\nolimits}
%\newcommand{\Hom}{\mathop{\mathrm{Hom}}\nolimits}
\newcommand{\Rot}{\mathop{\mathrm{rot}}\nolimits}
\newcommand{\Div}{\mathop{\mathrm{div}}\nolimits}


\begin{document}

\hrulefill

次の行列のJordan標準形を求めよ。
\begin{equation}
 A=
 \begin{pmatrix}
  2 & 0 & 1 & 0\\
  2 & 0 & 1 & 1\\
  0 & 1 & 0 & 0\\
  1 & 0 & 0 & 0
 \end{pmatrix}
 ,\quad
 B=
 \begin{pmatrix}
  2 & 0 & 1 & 0 & 0\\
  0 & 2 & 0 & 1 & 0\\
  0 & 1 & 0 & 0 & 0\\
  1 & 0 & 0 & 0 & 0\\
  2 & 0 & 1 & 0 & 1
 \end{pmatrix}
\end{equation}

\dotfill

$A$の固有方程式
\begin{equation}
 \det(A-\lambda E)= \lambda^4-2\lambda^3-\lambda^2-1=0
\end{equation}

固有値は
実数だと
 $-0.844$ぐらいと$2.471$ぐらい
 で、
 虚数が
 $0.187 \pm 0.667i$ぐらい。


\dotfill

$B$の固有方程式
\begin{equation}
 \det(B-\lambda E)=-(\lambda-1)^3(\lambda^2-2\lambda-1)=0
\end{equation}
ここから固有値は$\lambda = 1, 1\pm\sqrt{2}$の3つであることがわかる。

それぞれの固有値に対応する固有ベクトル
\begin{align}
 \lambda =& 1,&
 \bm{x}_\lambda =& \begin{pmatrix} 0\\0\\0\\0\\1 \end{pmatrix}\\
 \lambda =& 1+\sqrt{2},&
 \bm{x}_\lambda =& %\begin{pmatrix} 2-\sqrt{2} \\ 2-\sqrt{2} \\ -4+3\sqrt{2} \\ -4+3\sqrt{2} \\ 1 \end{pmatrix}\\
 \begin{pmatrix} 2 \\ 2 \\ -2+2\sqrt{2} \\ -2+2\sqrt{2} \\ 2+\sqrt{2} \end{pmatrix}\\
 \lambda =& 1-\sqrt{2},&
 \bm{x}_\lambda =& %\begin{pmatrix} 2+\sqrt{2} \\ 2+\sqrt{2} \\ -4-3\sqrt{2} \\ -4-3\sqrt{2} \\ 1 \end{pmatrix}
 \begin{pmatrix} 2 \\ 2 \\ -2-2\sqrt{2} \\ -2-2\sqrt{2} \\ 2-\sqrt{2} \end{pmatrix}
\end{align}

$\lambda =1$の時、%$(B-\lambda E)\bm{x}_\lambda=0$であるが、
$(B-\lambda E)\bm{x}_1=\bm{x}_\lambda$となるベクトル$\bm{x}_1$を求める。

\begin{equation}
 \bm{x}_1
       = k_1\begin{pmatrix} 0\\0\\0\\0\\1 \end{pmatrix}
       + \begin{pmatrix} 1\\-1\\-1\\1\\0 \end{pmatrix}
\end{equation}

$\bm{x}_1$の$k_1=0$として
同様に$(B-E)\bm{x}_2=\bm{x}_1$となるベクトル$\bm{x}_2$を求める。

\begin{equation}
 \bm{x}_2
       = k_2\begin{pmatrix} 0\\0\\0\\0\\1 \end{pmatrix}
       + \begin{pmatrix} -1\\1\\2\\-2\\0 \end{pmatrix}
\end{equation}

これらのベクトルと列に並べた行列$P$を作る。
\begin{equation}
 P=
  \begin{pmatrix}
   0 & 1 & -1 & 2 & 2 \\
   0 & -1 & 1 & 2 & 2 \\
   0 & -1 & 2 & -2+2\sqrt{2} & -2-2\sqrt{2} \\
   0 & 1 & -2 & -2+2\sqrt{2} & -2-2\sqrt{2} \\
   1 & 0 & 0 & 2+\sqrt{2} & 2-\sqrt{2}
  \end{pmatrix}
\end{equation}

これを用いて$B$のジョルダン標準形は次のようになる。
\begin{equation}
 P^{-1}BP=
  \begin{pmatrix}
   1 & 1 & 0 & 0 & 0 \\
   0 & 1 & 1 & 0 & 0 \\
   0 & 0 & 1 & 0 & 0 \\
   0 & 0 & 0 & 1+\sqrt{2} & 0 \\
   0 & 0 & 0 & 0 & 1-\sqrt{2}
  \end{pmatrix}
  =J_3(1) \oplus J_1(1+\sqrt{2}) \oplus J_1(1-\sqrt{2})
\end{equation}


\end{document}
