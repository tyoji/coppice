\documentclass[12pt,b5paper]{ltjsarticle}

%\usepackage[margin=15truemm, top=5truemm, bottom=5truemm]{geometry}
\usepackage[margin=10truemm]{geometry}

\usepackage{amsmath,amssymb}
%\pagestyle{headings}
\pagestyle{empty}

%\usepackage{listings,url}
%\renewcommand{\theenumi}{(\arabic{enumi})}

%\usepackage{graphicx}

%\usepackage{tikz}
%\usetikzlibrary {arrows.meta}
%\usepackage{wrapfig}	% required for `\wrapfigure' (yatex added)
%\usepackage{bm}	% required for `\bm' (yatex added)

% ルビを振る
%\usepackage{luatexja-ruby}	% required for `\ruby'

%% 核Ker 像Im Hom を定義
%\newcommand{\Img}{\mathop{\mathrm{Im}}\nolimits}
%\newcommand{\Ker}{\mathop{\mathrm{Ker}}\nolimits}
%\newcommand{\Hom}{\mathop{\mathrm{Hom}}\nolimits}

%\DeclareMathOperator{\Rot}{rot}
%\DeclareMathOperator{\Div}{div}
%\DeclareMathOperator{\Grad}{grad}
%\DeclareMathOperator{\arcsinh}{arcsinh}
%\DeclareMathOperator{\arccosh}{arccosh}
%\DeclareMathOperator{\arctanh}{arctanh}



\begin{document}

\hrulefill
\textbf{Laplace 変換}
\hrulefill

%\textbf{性質}

%\begin{align}
% \mathcal{L}[t^n] =& \frac{n!}{s^{n+1}} &
% \mathcal{L}[\sin \omega t] =& \frac{\omega}{s^{2}+\omega^2} &
% \mathcal{L}[\cos \omega t] =& \frac{s}{s^{2}+\omega^2}
%\end{align}

\begin{equation}
 \mathcal{L}[t^n] = \frac{n!}{s^{n+1}}, \quad
 \mathcal{L}[e^{at}] = \frac{1}{s-a}, \quad
 \mathcal{L}[\sin \omega t] = \frac{\omega}{s^2+\omega^2}, \quad
 \mathcal{L}[\cos \omega t] = \frac{s}{s^2+\omega^2}
\end{equation}

\begin{equation}
 \mathcal{L}[af(t)+bg(t)] = a\mathcal{L}[f(t)] + b\mathcal{L}[g(t)]
\end{equation}


$a>0$、$F(s)=\mathcal{L}[f(t)]$とする。
\begin{equation}
 \mathcal{L}[f(at)] = \frac{1}{a}F\left(\frac{s}{a}\right)
  \qquad
 \mathcal{L}[e^{at}f(t)] = F(s-a)
\end{equation}


\hrulefill
\textbf{問題}
\hrulefill

次の関数$F(s)$から逆ラプラス変換で$f(t)$を求めよ。
\begin{enumerate}
 \item
      \begin{equation}
       F(s) = \frac{3}{s^3+3s^2+3s+1}
      \end{equation}

 \item
      \begin{equation}
       F(s) = \frac{s+1}{s(s+2)^2}
      \end{equation}

 \item
      \begin{equation}
       F(s) = \frac{2s+5}{s^2+4s+8}
      \end{equation}

\end{enumerate}



\dotfill

\begin{enumerate}
 \item

      \begin{equation}
       F(s) = \frac{3}{s^3+3s^2+3s+1}
        = \frac{3}{(s+1)^3}
      \end{equation}

      $t^2$と$e^{-t}$のラプラス変換が
      \begin{equation}
       \mathcal{L}[t^2] = \frac{2!}{s^{2+1}} = \frac{2}{s^{3}}
       ,\qquad
       \mathcal{L}[e^{-t}] = \frac{1}{s+1}
      \end{equation}
      であるので、
      $t^{2}e^{-t}$のラプラス変換は次のようになる。
      \begin{equation}
       \mathcal{L}[t^{2}e^{-t}] = \frac{2}{(s+1)^{3}}
      \end{equation}

      つまり、次のようなラプラス変換があることが分かる。
      \begin{equation}
       \mathcal{L}\left[ \frac{3}{2}t^{2}e^{-t} \right] = \frac{3}{(s+1)^{3}}
      \end{equation}

      この為、
      逆ラプラス変換は次のようになる。
      \begin{equation}
       \mathcal{L}^{-1}\left[ \frac{3}{s^3+3s^2+3s+1} \right]
        = \frac{3}{2}t^{2}e^{-t}
      \end{equation}

      \dotfill

 \item

      \begin{equation}
       F(s) = \frac{s+1}{s(s+2)^2}
        = \frac{1}{4}\left( \frac{1}{s} + \frac{2}{(s+2)^2} + \frac{-1}{s+2} \right)
      \end{equation}

      $1$、$t$、$e^{-2t}$ のラプラス変換は次のようになる。
      \begin{equation}
       \mathcal{L}[1] = \frac{0!}{s^{0+1}} = \frac{1}{s}
        ,\quad
       \mathcal{L}[t] = \frac{1!}{s^{1+1}} = \frac{1}{s^{2}}
        ,\quad
       \mathcal{L}[e^{-2t}] = \frac{1}{s-(-2)} = \frac{1}{s+2}
      \end{equation}

      これを組み合わせて$te^{-2t}$のラプラス変換を考える。
      \begin{equation}
       \mathcal{L}[te^{-2t}] = \frac{1}{(s+2)^2}
      \end{equation}

      これらにより次のようなラプラス変換があることが分かる。
      \begin{equation}
       \mathcal{L}\left[ 1 + 2te^{-2t} - e^{-2t} \right]
        = \frac{1}{s} + \frac{2}{(s+2)^2} - \frac{1}{s+2}
      \end{equation}

      よって、求めるべき逆変換は次のような式となる。
      \begin{equation}
       \mathcal{L}^{-1}\left[ \frac{s+1}{s(s+2)^2} \right]
        = \frac{1}{4}( 1 + 2te^{-2t} - e^{-2t} )
      \end{equation}

      \dotfill

 \item

      \begin{align}
       F(s) =& \frac{2s+5}{s^2+4s+8}
         = \frac{2(s+2)+1}{(s+2)^2+2^2}\\
         =& \frac{1}{2}\cdot\frac{2}{(s+2)^2+2^2} + 2\cdot\frac{s+2}{(s+2)^2+2^2}
      \end{align}

      $\sin \omega t$、
      $\cos \omega t$、
      $e^{-2t}$のラプラス変換を考える。
      \begin{equation}
       \mathcal{L}[\sin \omega t] = \frac{\omega}{s^2+\omega^2}, \quad
       \mathcal{L}[\cos \omega t] = \frac{s}{s^2+\omega^2}, \quad
       \mathcal{L}[e^{-2t}] = \frac{1}{s+2}
      \end{equation}

      これを合わせると次のラプラス変換が得られる。
      \begin{equation}
       \mathcal{L}[e^{-2t}\sin \omega t] = \frac{\omega}{(s+2)^2+\omega^2}, \quad
       \mathcal{L}[e^{-2t}\cos \omega t] = \frac{s+2}{(s+2)^2+\omega^2}
      \end{equation}

      以上により次のラプラス変換が得られる。
      \begin{equation}
       \mathcal{L}\left[ \frac{1}{2}e^{-2t}\sin 2t + 2e^{-2t}\cos 2t \right]
                   = \frac{1}{2}\cdot\frac{2}{(s+2)^2+2^2} + 2\cdot\frac{s+2}{(s+2)^2+2^2}
      \end{equation}

      よって、求めるべき逆変換は次のような式となる。
      \begin{equation}
        \mathcal{L}^{-1}\left[ \frac{2s+5}{s^2+4s+8} \right]
         =\frac{1}{2}e^{-2t} \left( \sin 2t + 4\cos 2t \right)
      \end{equation}


\end{enumerate}


\end{document}

