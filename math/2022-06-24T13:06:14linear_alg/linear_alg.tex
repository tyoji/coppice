\documentclass[12pt,b5paper]{ltjsarticle}

%\usepackage[margin=15truemm, top=5truemm, bottom=5truemm]{geometry}
\usepackage[margin=15truemm]{geometry}

\usepackage{amsmath,amssymb}
%\pagestyle{headings}
\pagestyle{empty}

%\usepackage{listings,url}
%\renewcommand{\theenumi}{(\arabic{enumi})}

\usepackage{graphicx}

\usepackage{tikz}
\usetikzlibrary {arrows.meta}
\usepackage{wrapfig}	% required for `\wrapfigure' (yatex added)
\usepackage{bm}	% required for `\bm' (yatex added)
\usepackage{luatexja-ruby}	% required for `\ruby'

%% 核Ker 像Im Hom を定義
\newcommand{\Img}{\mathop{\mathrm{Im}}\nolimits}
\newcommand{\Ker}{\mathop{\mathrm{Ker}}\nolimits}
\newcommand{\Hom}{\mathop{\mathrm{Hom}}\nolimits}

\begin{document}

\hrulefill

$K$ \quad 体

$V$ \quad $K$-線形空間

$f:V\to V$ \quad $K$-線形写像

\hrulefill

$f\circ f=f$ならば、
$V=\Ker f\bigoplus \Img f$
となることを示せ。

\dotfill

準同型定理より
\begin{equation}
 V / \Ker f \cong \Img f
\end{equation}
これより$\dim_K V = \dim_K \Ker f + \dim_K \Img f$

$f\circ f = f$より
$f(f(v))=f(v)$である。
この為、写像$f$を$\Img f$に制限した写像は恒等写像である。
\begin{equation}
 f\mid_{\mathrm{Im}f}: \Img f \to \Img f , \quad f(v) \mapsto f(v)
\end{equation}
この為、$\Ker f \cap \Img f = \{0\}$であることが分かる。

つまり、$V=\Ker f\bigoplus \Img f$である。


\hrulefill

任意の集合$I$に対し、
\begin{equation}
 \prod_{i\in I}K \cong \Hom_{K}(\bigoplus_{i\in I}K, K)
\end{equation}
を示せ。

\dotfill

$\prod_{i\in I}K$
と
$\Hom_{K}(\bigoplus_{i\in I}K, K)$
は線形空間である。

$h\in \Hom_{K}(\bigoplus_{i\in I}K, K)$
は線形写像であり、
$\bigoplus_{i\in I}K$の元との内積を取る
$\prod_{i\in I}K$の元に対応する。
$\bigoplus_{i\in I}K$の成分は有限個を除いて全て$0$であるので、
$\prod_{i\in I}K$の元との内積は有限和となり、$K$の元となる。

この為、次の線形写像$f$は全射となる。
\begin{equation}
 f:\prod_{i\in I}K \to \Hom_{K}(\bigoplus_{i\in I}K, K)
\end{equation}

$\Hom_{K}(\bigoplus_{i\in I}K, K)$の零元は
$\bigoplus_{i\in I}K$上の零写像であるので、
$\Ker f$は$\prod_{i\in I}K$の零元のみとなる。
この為、$f$は単射でもある。

よって、
\begin{equation}
 \prod_{i\in I}K \cong \Hom_{K}(\bigoplus_{i\in I}K, K)
\end{equation}
である。

\hrulefill

整数$m$を$m\geq 0$とする。
\begin{enumerate}
 \item $\Ker f^m \subset \Ker f^{m+1}$ (ただし、$f^0$は恒等写像)
 \item $\Ker f^m = \Ker f^{m+1}$ならば、
       ${}^{\forall}p\geq 0$に対して$\Ker f^m = \Ker f^{m+p}$
 \item $n=\dim_K V$のとき、$\Ker f^n = \Ker f^{n+1}$
\end{enumerate}

\dotfill

\begin{enumerate}
 \item
      $\Ker f^0=\{0\}$であるので、$\Ker f^0 \subset \Ker f^1$である。

      ${}^{\forall}x\in\Ker f^m$の時、$f^m(x)=0$である。
      $f^{m+1}(x)=f(f^m(x))=f(0)=0$となるので、$x\in\Ker f^{m+1}$。

      よって、$\Ker f^m \subset \Ker f^{m+1}$である。

 \item
      $\Ker f^m = \Ker f^{m+1}$であるので、準同型定理より
      \begin{equation}
       \Img f^m \cong V/\Ker f^m = V/\Ker f^{m+1} \cong \Img f^{m+1}
      \end{equation}
      となる。

      線形写像$f:V\to V$を$\Img f^{m}$に制限すると
      \begin{equation}
       f |_{\mathrm{Im} f^m} : \Img f^m \to \Img f^{m+1}
      \end{equation}
      は同型写像である。
      同様に
      $f$を$\Img f^{m+1}$に制限すると
      \begin{equation}
       \Img f^{m+1} \cong f(\Img f^{m+1})=\Img f^{m+2}
      \end{equation}
      となり、
      \begin{equation}
       V/\Ker f^{m+1} \cong
       \Img f^{m+1} \cong \Img f^{m+2}
       \cong V/\Ker f^{m+2}
      \end{equation}
      となる。

      $\Ker f^{m+1} \subset \Ker f^{m+2}$であるので、
      $\Ker f^{m+1} = \Ker f^{m+2}$である。

      同様の議論を繰り返すことにより
      \begin{equation}
       \Ker f^m = \Ker f^{m+p}
      \end{equation}
      が得られる。


 \item
      $\Ker$の列
      \begin{equation}
       \Ker f^0 \subset \cdots \subset \Ker f^m \subset \Ker f^{m+1} \subset \cdots
      \end{equation}
      は$\Ker f^n=\Ker f^{n+1}$となる$n$があれば、それ以降全て等しくなる。
      \begin{equation}
       \Ker f^0 \subset \cdots \subset \Ker f^n = \Ker f^{m+1} = \cdots
      \end{equation}

      $\Ker f^m$は$V$の部分空間であるので
      $\dim_K \Ker f^m \leq \dim_K V$である。

      $\Ker f^m \ne \Ker f^{m+1}$であれば
      $\dim_K \Ker f^m < \dim_K \Ker f^{m+1}$である。

      これにより$n=\dim_K V$以降は全て等しくなる。
      \begin{equation}
       \Ker f^n = \Ker f^{n+1} = \Ker f^{n+2} = \cdots
      \end{equation}



\end{enumerate}

\hrulefill

\begin{enumerate}
 \item
      複素2次正方行列$A$について、
      $A\ne 0$かつ$A^2=0$をみたすものは
      $\begin{pmatrix} 0 & 1 \\ 0 & 0 \end{pmatrix}$と相似であることを示せ。
 \item
      複素3次正方行列$A$について、
      $A^2\ne 0$かつ$A^3=0$をみたすものは
      $\begin{pmatrix} 0 & 1 & 0 \\ 0 & 0 & 1 \\ 0 & 0 & 0 \end{pmatrix}$と相似であることを示せ。
\end{enumerate}

\dotfill

\begin{enumerate}
 \item
      あるベクトル$\bm{p}\in\mathbb{C}^2$が存在し$A\bm{p}\ne \bm{0}$とする。

      2つのベクトル$\bm{p},\,A\bm{p}$は次のようにして一次独立である事がわかる。

      次の式に$A$をかける。
      \begin{equation}
       a_0\bm{p} + a_1A\bm{p} =\bm{0} \quad (a_0,\,a_1\in\mathbb{C})
      \end{equation}
      これにより$a_0A\bm{p}=\bm{0}$となり、$a_0=0$が分かる。
      これにより$a_1=0$となるので、
      $\bm{p},\,A\bm{p}$は一次独立である。

      そこでこのベクトルを並べて行列$P$を作る。
      \begin{equation}
       P = \begin{pmatrix} A\bm{p} & \bm{p} \end{pmatrix}
      \end{equation}
      列ベクトルが独立なので$P$は正則行列である。

      \begin{align}
       AP =& A\begin{pmatrix} A\bm{p} & \bm{p} \end{pmatrix}
       = \begin{pmatrix} \bm{0} & A\bm{p} \end{pmatrix}\\
       =& \begin{pmatrix} A\bm{p} & \bm{p} \end{pmatrix}
       \begin{pmatrix} 0 & 1 \\ 0 & 0 \end{pmatrix}
       = P \begin{pmatrix} 0 & 1 \\ 0 & 0 \end{pmatrix}
      \end{align}

      $P^{-1}$をかけることにより
      $P^{-1}AP=\begin{pmatrix} 0 & 1 \\ 0 & 0 \end{pmatrix}$
      となる。

      これにより$A$は
      $\begin{pmatrix} 0 & 1 \\ 0 & 0 \end{pmatrix}$
      と相似である。


 \item
      $A^2\bm{p}\ne 0$となるベクトル$\bm{p}\in\mathbb{C}^3$を取ってくる。

      3つのベクトル$\bm{p},\,A\bm{p},\,A^2\bm{p}$について
      \begin{equation}
       a_0\bm{p} + a_1A\bm{p} + a_2A^2\bm{p} = \bm{0}
      \end{equation}
      を考える。
      $A^2$をかけると$a_0A^2\bm{p}=\bm{0}$となり$a_0=0$が得られる。
      $A$をかけて$a_0=0$を当てはめると$a_1A^2\bm{p}=\bm{0}$となり$a_1=0$が得られる。
      $a_0=a_1=0$より$a_2=0$となり、
      ベクトル$\bm{p},\,A\bm{p},\,A^2\bm{p}$は一次独立であることが分かる。

      このベクトルを用いて正則行列$P$を次のように定める。
      \begin{equation}
       P = \begin{pmatrix} A^2\bm{p} & A\bm{p} & \bm{p} \end{pmatrix}
      \end{equation}

      \begin{align}
       AP =& A \begin{pmatrix} A^2\bm{p} & A\bm{p} & \bm{p} \end{pmatrix}
       = \begin{pmatrix} \bm{0} & A^2\bm{p} & A\bm{p} \end{pmatrix}\\
       =& \begin{pmatrix} A^2\bm{p} & A\bm{p} & \bm{p} \end{pmatrix}
       \begin{pmatrix} 0 & 1 & 0 \\ 0 & 0 & 1 \\ 0 & 0 & 0 \end{pmatrix}
       = P \begin{pmatrix} 0 & 1 & 0 \\ 0 & 0 & 1 \\ 0 & 0 & 0 \end{pmatrix}
      \end{align}

      これにより
      $P^{-1}AP=\begin{pmatrix} 0 & 1 & 0 \\ 0 & 0 & 1 \\ 0 & 0 & 0 \end{pmatrix}$
      であるので
      $A$は$\begin{pmatrix} 0 & 1 & 0 \\ 0 & 0 & 1 \\ 0 & 0 & 0 \end{pmatrix}$に相似である。


\end{enumerate}

\hrulefill

\end{document}
