\documentclass[12pt,b5paper]{ltjsarticle}

%\usepackage[margin=15truemm, top=5truemm, bottom=5truemm]{geometry}
%\usepackage[margin=10truemm,left=15truemm]{geometry}
\usepackage[margin=10truemm]{geometry}

\usepackage{amsmath,amssymb}
%\pagestyle{headings}
\pagestyle{empty}

%\usepackage{listings,url}
%\renewcommand{\theenumi}{(\arabic{enumi})}

%\usepackage{graphicx}

%\usepackage{tikz}
%\usetikzlibrary {arrows.meta}
%\usepackage{wrapfig}	% required for `\wrapfigure' (yatex added)
%\usepackage{bm}	% required for `\bm' (yatex added)

% ルビを振る
%\usepackage{luatexja-ruby}	% required for `\ruby'

%% 核Ker 像Im Hom を定義
%\newcommand{\Img}{\mathop{\mathrm{Im}}\nolimits}
%\newcommand{\Ker}{\mathop{\mathrm{Ker}}\nolimits}
%\newcommand{\Hom}{\mathop{\mathrm{Hom}}\nolimits}

%\DeclareMathOperator{\Rot}{rot}
%\DeclareMathOperator{\Div}{div}
%\DeclareMathOperator{\Grad}{grad}
%\DeclareMathOperator{\arcsinh}{arcsinh}
%\DeclareMathOperator{\arccosh}{arccosh}
%\DeclareMathOperator{\arctanh}{arctanh}



%\usepackage{listings,url}
%
%\lstset{
%%プログラム言語(複数の言語に対応,C,C++も可)
%  language = Python,
%%  language = Lisp,
%%  language = C,
%  %背景色と透過度
%  %backgroundcolor={\color[gray]{.90}},
%  %枠外に行った時の自動改行
%  breaklines = true,
%  %自動改行後のインデント量(デフォルトでは20[pt])
%  breakindent = 10pt,
%  %標準の書体
%%  basicstyle = \ttfamily\scriptsize,
%  basicstyle = \ttfamily,
%  %コメントの書体
%%  commentstyle = {\itshape \color[cmyk]{1,0.4,1,0}},
%  %関数名等の色の設定
%  classoffset = 0,
%  %キーワード(int, ifなど)の書体
%%  keywordstyle = {\bfseries \color[cmyk]{0,1,0,0}},
%  %表示する文字の書体
%  %stringstyle = {\ttfamily \color[rgb]{0,0,1}},
%  %枠 "t"は上に線を記載, "T"は上に二重線を記載
%  %他オプション:leftline,topline,bottomline,lines,single,shadowbox
%  frame = TBrl,
%  %frameまでの間隔(行番号とプログラムの間)
%  framesep = 5pt,
%  %行番号の位置
%  numbers = left,
%  %行番号の間隔
%  stepnumber = 1,
%  %行番号の書体
%%  numberstyle = \tiny,
%  %タブの大きさ
%  tabsize = 4,
%  %キャプションの場所("tb"ならば上下両方に記載)
%  captionpos = t
%}



\begin{document}

\hrulefill

\textbf{微分幾何}

%測地線


\hrulefill

\begin{enumerate}
 \item
      曲面片$S$に線分$C$が含まれているとする。
      このとき、
      $C$は$S$の測地線であることを示せ。

      \dotfill

      線分$C$の長さを$L$とし、
      弧長パラメータで$\gamma(s)$と表現できるとする。

      曲面$S$はパラメータ$u,v$により$p(u,v)$とかけるのであれば、
      $\gamma (s) = p(u(s),v(s))$である。
%      \begin{equation}
%       \frac{\mathrm{d}}{\mathrm{d}s}\gamma(s)
%        =\frac{\mathrm{d}p(u,v)}{\mathrm{d}u}\frac{\mathrm{d}u(s)}{\mathrm{d}s}
%        +\frac{\mathrm{d}p(u,v)}{\mathrm{d}v}\frac{\mathrm{d}v(s)}{\mathrm{d}s}
%        =
%      \end{equation}
      これを$s$で微分する。
      \begin{align}
       \gamma^{\prime}(s)
        =& p_{u}(u,v)u^{\prime}(s) + p_{v}(u,v)v^{\prime}(s)\\
        =& \left(p_{u}(u,v), p_{v}(u,v)\right)\cdot \left(u^{\prime}(s), v^{\prime}(s)\right)
      \end{align}

      $(p_u,p_v)$は曲面$p(u,v)$の接ベクトルであるので、
      $\gamma^{\prime}(s)$も曲面上の接ベクトルである。

      $\gamma(s)$は弧長パラメータであるので、
      $\gamma^{\prime}(s)\cdot\gamma^{\prime}(s)=1$である。
      $\gamma^{\prime}(s)\cdot\gamma^{\prime}(s)=\lvert\gamma^{\prime}(s) \rvert^2$
      より
      両辺を$s$で微分すると次のようになる。
      \begin{equation}
       (\lvert\ \gamma^{\prime}(s) \rvert^2)^{\prime}
        = 2 \gamma^{\prime}(s)\cdot \gamma^{\prime\prime}(s) =0
      \end{equation}

      これにより$\gamma^{\prime}(s)$と$\gamma^{\prime\prime}(s)$
      は直交するが、
      $\gamma^{\prime}(s)$は接ベクトルであるので、
      $\gamma^{\prime\prime}(s)$は法ベクトルである。
      よって、
      $\gamma(s)$が表す線分$C$は測地線である。

      \hrulefill

 \item
      曲面片$S$は平面$P$に関する折り返しで対称であるとする。
      $S$の$P$による切り口$S\cap P$が曲線片であるとき、
      $S\cap P$は$S$の測地線であることを示せ。

      \dotfill

      \hrulefill

 \item
      $C$を有限な長さを持つ、曲率が$0$になる点がない曲線片とする。
      $C$の弧長パラメータ表示を$p:I=[0,L]\to S$とし、
      $b(s)$を点$p(s)$における従法線ベクトルとする。

      $x(s,t)=p(s)+tb(s)$とおく。

      \begin{enumerate}
       \item
            $\varepsilon>0$を十分小さくとると
            $x$は$I\times (-\varepsilon,\varepsilon)$上の
            正則曲面であることを示せ。

            \dotfill

            $x(s,t)$を$s,t$でそれぞれ偏微分する。
            \begin{equation}
             x_{s}(s,t)=p^{\prime}(s)+tb^{\prime}(s)
              ,\
             x_{t}(s,t)=b(s)
            \end{equation}

            $x$が正則曲面であるためには$x_{s}(s,t),x_{t}(s,t)$が
            一次独立であればよい。

%            曲線上では曲率が$0$にならないので、
%            $p^{\prime\prime}(s)\ne 0$である。

%            つまり、$p^{\prime}(s)$は
%            $s$の値により変化するベクトルであり、
%            $p^{\prime}(s)\ne 0$である。
            $t=0$の時、
            $x_{s}(s,0)=p^{\prime}(s),\ x_{t}(s,t)=b(s)$
            であり、この2つのベクトルは直交する。
            ここに小さな値$\varepsilon$を加え
            $x_{s}(s,\varepsilon),\ x_{t}(s,t)$が一次従属でないようにすることはできる。
            つまり、
            $I\times (-\varepsilon,\varepsilon)$上で正則となるような
            $\varepsilon$はとってこれる。

            \hrulefill

       \item
            $C$はこの曲面の測地線であることを示せ。

            \dotfill

            $p^{\prime\prime}(s)$は$p^{\prime}(s)$
            と直交するベクトルであるが、
            $b(s)$とも直交する。
            つまり、
            曲面$I\times (-\varepsilon,\varepsilon)$上で、
            $p^{\prime\prime}(s)$は法線ベクトルである。

            よって、
            $C$は測地線である。

            \hrulefill

      \end{enumerate}
\end{enumerate}


\hrulefill

\end{document}
