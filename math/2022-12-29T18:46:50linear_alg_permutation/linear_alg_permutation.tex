\documentclass[12pt,b5paper]{ltjsarticle}

%\usepackage[margin=15truemm, top=5truemm, bottom=5truemm]{geometry}
%\usepackage[margin=10truemm,left=15truemm]{geometry}
\usepackage[margin=10truemm]{geometry}

\usepackage{amsmath,amssymb}
%\pagestyle{headings}
\pagestyle{empty}

%\usepackage{listings,url}
%\renewcommand{\theenumi}{(\arabic{enumi})}

%\usepackage{graphicx}

%\usepackage{tikz}
%\usetikzlibrary {arrows.meta}
%\usepackage{wrapfig}	% required for `\wrapfigure' (yatex added)
%\usepackage{bm}	% required for `\bm' (yatex added)

% ルビを振る
%\usepackage{luatexja-ruby}	% required for `\ruby'

%% 核Ker 像Im Hom を定義
%\newcommand{\Img}{\mathop{\mathrm{Im}}\nolimits}
%\newcommand{\Ker}{\mathop{\mathrm{Ker}}\nolimits}
%\newcommand{\Hom}{\mathop{\mathrm{Hom}}\nolimits}

%\DeclareMathOperator{\Rot}{rot}
%\DeclareMathOperator{\Div}{div}
%\DeclareMathOperator{\Grad}{grad}
%\DeclareMathOperator{\arcsinh}{arcsinh}
%\DeclareMathOperator{\arccosh}{arccosh}
%\DeclareMathOperator{\arctanh}{arctanh}



%\usepackage{listings,url}
%
%\lstset{
%%プログラム言語(複数の言語に対応,C,C++も可)
%  language = Python,
%%  language = Lisp,
%%  language = C,
%  %背景色と透過度
%  %backgroundcolor={\color[gray]{.90}},
%  %枠外に行った時の自動改行
%  breaklines = true,
%  %自動改行後のインデント量(デフォルトでは20[pt])
%  breakindent = 10pt,
%  %標準の書体
%%  basicstyle = \ttfamily\scriptsize,
%  basicstyle = \ttfamily,
%  %コメントの書体
%%  commentstyle = {\itshape \color[cmyk]{1,0.4,1,0}},
%  %関数名等の色の設定
%  classoffset = 0,
%  %キーワード(int, ifなど)の書体
%%  keywordstyle = {\bfseries \color[cmyk]{0,1,0,0}},
%  %表示する文字の書体
%  %stringstyle = {\ttfamily \color[rgb]{0,0,1}},
%  %枠 "t"は上に線を記載, "T"は上に二重線を記載
%  %他オプション:leftline,topline,bottomline,lines,single,shadowbox
%  frame = TBrl,
%  %frameまでの間隔(行番号とプログラムの間)
%  framesep = 5pt,
%  %行番号の位置
%  numbers = left,
%  %行番号の間隔
%  stepnumber = 1,
%  %行番号の書体
%%  numberstyle = \tiny,
%  %タブの大きさ
%  tabsize = 4,
%  %キャプションの場所("tb"ならば上下両方に記載)
%  captionpos = t
%}



\begin{document}

\hrulefill

\textbf{線形代数 置換 互換}

$n$次の置換$\sigma$に対し、
以下を満たす自然数の組$(i,j)$の
個数を$f(\sigma)$
とする。
\begin{equation}
 1\leq i < j \leq n, \sigma(i) >\sigma(j)
\end{equation}
このとき、
$\sigma$
が奇置換
$\Leftrightarrow f(\sigma)$
が奇数を示せ。

\dotfill

$\sigma$が奇置換とする。

%奇置換であるので、
%$\tau_i$を互換として、
%$\sigma=\tau_m\tau_{m-1}\tau_{m-2}\cdots\tau_{1}$
%($m$は奇数)
%と書くことが出来る。

$i$と$j$を入れ替える互換を$(i,j)$とすると、
$\sigma$は奇数個の互換の積で表せる。

%$\tau_k$をある互換とすると、
%\begin{equation}
% \prod_{i<j}(\tau_k(j)-\tau_k(i))
%  =- \prod_{i<j}(j-i)
%\end{equation}
%である。
%
%
%\begin{equation}
% \prod_{i<j}(\sigma(j)-\sigma(i))<0
%\end{equation}
%である。
%
%$\sigma(j)-\sigma(i)$が負になるのが奇数個存在する。

%つまり、符号$\mathrm{sgn}(\sigma)=-1$である。

つまり、
次を満たす$i,j$の組が奇数個存在する。
\begin{equation}
 1\leq i < j \leq n, \sigma(i) >\sigma(j)
\end{equation}

よって、
$f(\sigma)$が奇数となることがわかる。

\dotfill

逆に
$f(\sigma)$が奇数であるとする。

置換$\sigma$は
\begin{equation}
 1\leq i < j \leq n, \sigma(i) >\sigma(j)
\end{equation}
を満たすものが奇数個あるということである。

これを満たす$i,j$で互換$(i,j)$を作れば
$\sigma$はこの$(i,j)$の積で表せる。

%これは次の式が成り立つことを意味する。
%\begin{equation}
% \prod_{i<j}(\sigma(j)-\sigma(i))<0
%\end{equation}

つまり、奇数個の互換の積で表せるため、
奇置換ということがわかる。



\hrulefill

\end{document}
