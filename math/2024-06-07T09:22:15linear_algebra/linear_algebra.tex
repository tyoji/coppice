\documentclass[12pt,b5paper]{ltjsarticle}

%\usepackage[margin=15truemm, top=5truemm, bottom=5truemm]{geometry}
%\usepackage[margin=10truemm,left=15truemm]{geometry}
\usepackage[margin=10truemm]{geometry}

\usepackage{amsmath,amssymb}

%% 定理環境
%\usepackage{amsthm}
%\newtheorem{theo}{Theorem}
%
%%\pagestyle{headings}
%\pagestyle{empty}

%\usepackage{listings,url}
%\renewcommand{\theenumi}{(\arabic{enumi})}

%\usepackage{graphicx}

%\usepackage{tikz}
%\usetikzlibrary {arrows.meta}
%\usepackage{wrapfig}
%\usepackage{bm}

% ルビを振る
%\usepackage{luatexja-ruby}	% required for `\ruby'

%% 核Ker 像Im Hom を定義
%\newcommand{\Ker}{\mathop{\mathrm{Ker}}\nolimits}
%\newcommand{\Img}{\mathop{\mathrm{Im}}\nolimits}
%\newcommand{\Ran}{\mathop{\mathrm{Ran}}\nolimits}
%\newcommand{\Hom}{\mathop{\mathrm{Hom}}\nolimits}

%\DeclareMathOperator{\Rot}{rot}
%\DeclareMathOperator{\Div}{div}
%\DeclareMathOperator{\Grad}{grad}
%\DeclareMathOperator{\arcsinh}{arcsinh}
%\DeclareMathOperator{\arccosh}{arccosh}
%\DeclareMathOperator{\arctanh}{arctanh}

\usepackage{url}

%\usepackage{listings}
%
%\lstset{
%%プログラム言語(複数の言語に対応,C,C++も可)
%  language = Python,
%%  language = Lisp,
%%  language = C,
%  %背景色と透過度
%  %backgroundcolor={\color[gray]{.90}},
%  %枠外に行った時の自動改行
%  breaklines = true,
%  %自動改行後のインデント量(デフォルトでは20[pt])
%  breakindent = 10pt,
%  %標準の書体
%%  basicstyle = \ttfamily\scriptsize,
%  basicstyle = \ttfamily,
%  %コメントの書体
%%  commentstyle = {\itshape \color[cmyk]{1,0.4,1,0}},
%  %関数名等の色の設定
%  classoffset = 0,
%  %キーワード(int, ifなど)の書体
%%  keywordstyle = {\bfseries \color[cmyk]{0,1,0,0}},
%  %表示する文字の書体
%  %stringstyle = {\ttfamily \color[rgb]{0,0,1}},
%  %枠 "t"は上に線を記載, "T"は上に二重線を記載
%  %他オプション:leftline,topline,bottomline,lines,single,shadowbox
%  frame = TBrl,
%  %frameまでの間隔(行番号とプログラムの間)
%  framesep = 5pt,
%  %行番号の位置
%  numbers = left,
%  %行番号の間隔
%  stepnumber = 1,
%  %行番号の書体
%%  numberstyle = \tiny,
%  %タブの大きさ
%  tabsize = 4,
%  %キャプションの場所("tb"ならば上下両方に記載)
%  captionpos = t
%}

%\usepackage{cancel}
%\usepackage{bussproofs}
%\usepackage{proof}

\begin{document}

%$V$をベクトル空間とし
次のように$W$を定めると$W$は$V^{4}$の部分空間である。
\begin{equation}
 W=
  \left\{
   \begin{pmatrix} x\\y\\z\\w \end{pmatrix} \in V^{4}
   \; \middle| \; 2y-z+w=0
  \right\}
\end{equation}

この時、
$W$の基底を求めよ。

\hrulefill

$W$の元は$2y-z+w=0$を満たす。
ベクトルは成分ごとの和で加法を定めているため、
次のように$x,y,z$ごとで分けた3つの元の和で表せる。
\begin{equation}
 \begin{pmatrix} x\\y\\z\\w \end{pmatrix}
 = \begin{pmatrix} x\\y\\z\\-2y+z \end{pmatrix}
 = \begin{pmatrix} x\\0\\0\\0 \end{pmatrix}
 + \begin{pmatrix} 0\\y\\0\\-2y \end{pmatrix}
 + \begin{pmatrix} 0\\0\\z\\z \end{pmatrix}
\end{equation}

$x,y,z\in V$は$V^{4}$のスカラーであるので、
上記式は次のようになる。
\begin{equation}
 \begin{pmatrix} x\\0\\0\\0 \end{pmatrix}
 + \begin{pmatrix} 0\\y\\0\\-2y \end{pmatrix}
 + \begin{pmatrix} 0\\0\\z\\z \end{pmatrix}
 =
  x\begin{pmatrix} 1\\0\\0\\0 \end{pmatrix}
 + y\begin{pmatrix} 0\\1\\0\\-2 \end{pmatrix}
 + z\begin{pmatrix} 0\\0\\1\\1 \end{pmatrix}
\end{equation}

よって、$W$の基底は次の3つである。
\begin{equation}
 \begin{pmatrix} 1\\0\\0\\0 \end{pmatrix},\quad
 \begin{pmatrix} 0\\1\\0\\-2 \end{pmatrix},\quad
 \begin{pmatrix} 0\\0\\1\\1 \end{pmatrix}
\end{equation}


\end{document}
