\documentclass[12pt,b5paper]{ltjsarticle}

%\usepackage[margin=15truemm, top=5truemm, bottom=5truemm]{geometry}
%\usepackage[margin=10truemm,left=15truemm]{geometry}
\usepackage[margin=10truemm]{geometry}

\usepackage{amsmath,amssymb}
%\pagestyle{headings}
\pagestyle{empty}

%\usepackage{listings,url}
%\renewcommand{\theenumi}{(\arabic{enumi})}

%\usepackage{graphicx}

%\usepackage{tikz}
%\usetikzlibrary {arrows.meta}
%\usepackage{wrapfig}
%\usepackage{bm}

% ルビを振る
\usepackage{luatexja-ruby}	% required for `\ruby'

%% 核Ker 像Im Hom を定義
%\newcommand{\Img}{\mathop{\mathrm{Im}}\nolimits}
%\newcommand{\Ker}{\mathop{\mathrm{Ker}}\nolimits}
%\newcommand{\Hom}{\mathop{\mathrm{Hom}}\nolimits}

%\DeclareMathOperator{\Rot}{rot}
%\DeclareMathOperator{\Div}{div}
%\DeclareMathOperator{\Grad}{grad}
%\DeclareMathOperator{\arcsinh}{arcsinh}
%\DeclareMathOperator{\arccosh}{arccosh}
%\DeclareMathOperator{\arctanh}{arctanh}

\usepackage{url}

%\usepackage{listings}
%
%\lstset{
%%プログラム言語(複数の言語に対応,C,C++も可)
%  language = Python,
%%  language = Lisp,
%%  language = C,
%  %背景色と透過度
%  %backgroundcolor={\color[gray]{.90}},
%  %枠外に行った時の自動改行
%  breaklines = true,
%  %自動改行後のインデント量(デフォルトでは20[pt])
%  breakindent = 10pt,
%  %標準の書体
%%  basicstyle = \ttfamily\scriptsize,
%  basicstyle = \ttfamily,
%  %コメントの書体
%%  commentstyle = {\itshape \color[cmyk]{1,0.4,1,0}},
%  %関数名等の色の設定
%  classoffset = 0,
%  %キーワード(int, ifなど)の書体
%%  keywordstyle = {\bfseries \color[cmyk]{0,1,0,0}},
%  %表示する文字の書体
%  %stringstyle = {\ttfamily \color[rgb]{0,0,1}},
%  %枠 "t"は上に線を記載, "T"は上に二重線を記載
%  %他オプション:leftline,topline,bottomline,lines,single,shadowbox
%  frame = TBrl,
%  %frameまでの間隔(行番号とプログラムの間)
%  framesep = 5pt,
%  %行番号の位置
%  numbers = left,
%  %行番号の間隔
%  stepnumber = 1,
%  %行番号の書体
%%  numberstyle = \tiny,
%  %タブの大きさ
%  tabsize = 4,
%  %キャプションの場所("tb"ならば上下両方に記載)
%  captionpos = t
%}

%\usepackage{cancel}
%\usepackage{bussproofs}
%\usepackage{proof}



\begin{document}

\hrulefill

無限級数入門

\url{https://www.juen.ac.jp/math/nakagawa/Series.pdf}

\dotfill

ライプニッツの級数のやさしい証明

\url{http://www.core.kochi-tech.ac.jp/m_inoue/work/pdf/math_tale/05.pdf}

\hrulefill


\textbf{\ruby{Leibniz}{ライプニッツ}級数}
\begin{equation}
 \sum_{n=0}^{\infty} \frac{(-1)^{n-1}}{2n-1} = \frac{\pi}{4}
\end{equation}

\dotfill

\textbf{等比級数の和}
\begin{equation}
 \sum_{k=0}^{n}ar^{k} =
  \begin{cases}
   \frac{a(1-r^{n+1})}{1-r} & (r\ne 1)\\
   a(n+1) & (r=1)
  \end{cases}
\end{equation}

$r\ne 1$の場合の
無限等比級数の和
\begin{equation}
 \sum_{k=0}^{\infty}ar^{k}
  = \lim_{n\to \infty}\sum_{k=0}^{n}ar^{k}
  = \lim_{n\to \infty}
   \frac{a(1-r^{n+1})}{1-r}
\end{equation}

$\lvert r \rvert <1$の時、
$\lim_{n\to\infty}r^{n+1} = 0$である為、
\begin{equation}\label{eq:geo_sq}
 \sum_{k=0}^{\infty}ar^{k}
  = \frac{a}{1-r}
\end{equation}

\hrulefill

無限等比級数
$\displaystyle \sum_{n=0}^{\infty} (-1)^{n}x^{2n}$
は
\begin{equation}
 \sum_{n=0}^{\infty} (-1)^{n}x^{2n} = \sum_{n=0}^{\infty} \left(-x^{2}\right)^{n}
\end{equation}
 であるので、
 $\lvert -x^{2} \rvert <1$において
 公式\eqref{eq:geo_sq}が使え、
 次のようになる。
 \begin{equation}
  \sum_{n=0}^{\infty} (-1)^{n}x^{2n} = \sum_{n=0}^{\infty} (-x^{2})^{n}
   = \frac{1}{1-(-x^{2})}
   = \frac{1}{1+x^{2}}
 \end{equation}

 \dotfill

 
 \begin{equation}\label{eq:lim_leib}
  \sum_{n=0}^{\infty} (-1)^{n}x^{2n} = \frac{1}{1+x^{2}}
 \end{equation}

 式\eqref{eq:lim_leib}の両辺を$0$から$x$において積分する。

 \begin{equation}\label{eq:int}
  \int_{0}^{x} \left( \sum_{n=0}^{\infty} (-1)^{n}x^{2n} \right) \mathrm{d}x
   = \int_{0}^{x} \left( \frac{1}{1+x^{2}} \right) \mathrm{d}x
 \end{equation}

$\arctan{x}$の導関数は
\begin{equation}
 \frac{\mathrm{d}}{\mathrm{d}x}\arctan{x} = \frac{1}{1+x^2}
\end{equation}
であるので、
次のように定積分で表せる。
\begin{equation}
 \arctan{x} = \int_{0}^{x} \frac{1}{1+x^2}\mathrm{d}x
\end{equation}

 式\eqref{eq:int}の右辺は$\arctan{x}$である。


 左辺は式\eqref{eq:lim_leib}が
 $\lvert x \rvert <1$の範囲において
 一様収束する為、
 積分と和を入れ替えることができる。(項別積分)
 \begin{equation}
  \int_{0}^{x} \left( \sum_{n=0}^{\infty} (-1)^{n}x^{2n} \right) \mathrm{d}x
   =
   \sum_{n=0}^{\infty} \left( \int_{0}^{x} (-1)^{n}x^{2n} \mathrm{d}x \right)
 \end{equation}


 そこで、一つの項の積分を行う。
 \begin{equation}
  \int_{0}^{x} (-1)^{n}x^{2n} \mathrm{d}x
   = \left[ \frac{(-1)^{n}}{2n+1} x^{2n+1} \right]_{0}^{x}
   = \frac{(-1)^{n}}{2n+1} x^{2n+1}
 \end{equation}

 式\eqref{eq:int}の左辺は次のようになる。
 \begin{equation}
  \int_{0}^{1} \left( \sum_{n=0}^{\infty} (-1)^{n}x^{2n} \right) \mathrm{d}x
   = \sum_{n=0}^{\infty} \frac{(-1)^{n}}{2n+1} x^{2n+1}
 \end{equation}

式\eqref{eq:int}は$\lvert x \rvert <1$において次のようになる。
(グレゴリー級数)
 \begin{equation}
  \sum_{n=0}^{\infty} \frac{(-1)^{n}}{2n+1} x^{2n+1}
   = \arctan{x}
 \end{equation}


 そこで、$x\to 1-0$と極限を取ると次のライプニッツの公式が得られる。
 \begin{equation}
  \sum_{n=0}^{\infty} \frac{(-1)^{n}}{2n+1}
   = \frac{\pi}{4}
 \end{equation}



\hrulefill

\end{document}
