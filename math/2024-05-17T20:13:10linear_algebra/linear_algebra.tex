\documentclass[12pt,b5paper]{ltjsarticle}

%\usepackage[margin=15truemm, top=5truemm, bottom=5truemm]{geometry}
%\usepackage[margin=10truemm,left=15truemm]{geometry}
\usepackage[margin=10truemm]{geometry}

\usepackage{amsmath,amssymb}
%\pagestyle{headings}
\pagestyle{empty}

%\usepackage{listings,url}
%\renewcommand{\theenumi}{(\arabic{enumi})}

%\usepackage{graphicx}

%\usepackage{tikz}
%\usetikzlibrary {arrows.meta}
%\usetikzlibrary{intersections,calc,arrows.meta}

%\usepackage{wrapfig}
\usepackage{bm}

% ルビを振る
%\usepackage{luatexja-ruby}	% required for `\ruby'

%% 核Ker 像Im Hom を定義
%\newcommand{\Img}{\mathop{\mathrm{Im}}\nolimits}
%\newcommand{\Ker}{\mathop{\mathrm{Ker}}\nolimits}
%\newcommand{\Hom}{\mathop{\mathrm{Hom}}\nolimits}

%\DeclareMathOperator{\Rot}{rot}
%\DeclareMathOperator{\Div}{div}
%\DeclareMathOperator{\Grad}{grad}
%\DeclareMathOperator{\arcsinh}{arcsinh}
%\DeclareMathOperator{\arccosh}{arccosh}
%\DeclareMathOperator{\arctanh}{arctanh}



%\usepackage{listings,url}
%
%\lstset{
%%プログラム言語(複数の言語に対応,C,C++も可)
%  language = Python,
%%  language = Lisp,
%%  language = C,
%  %背景色と透過度
%  %backgroundcolor={\color[gray]{.90}},
%  %枠外に行った時の自動改行
%  breaklines = true,
%  %自動改行後のインデント量(デフォルトでは20[pt])
%  breakindent = 10pt,
%  %標準の書体
%%  basicstyle = \ttfamily\scriptsize,
%  basicstyle = \ttfamily,
%  %コメントの書体
%%  commentstyle = {\itshape \color[cmyk]{1,0.4,1,0}},
%  %関数名等の色の設定
%  classoffset = 0,
%  %キーワード(int, ifなど)の書体
%%  keywordstyle = {\bfseries \color[cmyk]{0,1,0,0}},
%  %表示する文字の書体
%  %stringstyle = {\ttfamily \color[rgb]{0,0,1}},
%  %枠 "t"は上に線を記載, "T"は上に二重線を記載
%  %他オプション:leftline,topline,bottomline,lines,single,shadowbox
%  frame = TBrl,
%  %frameまでの間隔(行番号とプログラムの間)
%  framesep = 5pt,
%  %行番号の位置
%  numbers = left,
%  %行番号の間隔
%  stepnumber = 1,
%  %行番号の書体
%%  numberstyle = \tiny,
%  %タブの大きさ
%  tabsize = 4,
%  %キャプションの場所("tb"ならば上下両方に記載)
%  captionpos = t
%}



\begin{document}

ここではベクトル空間は実数体$\mathbb{R}$上の空間とする。

ベクトル空間$V$の元$\bm{v}_{1},\bm{v}_{2}\in V$に対して、
$\{\bm{v}_{1},\bm{v}_{2}\}$で生成される部分空間を
$\langle \bm{v}_{1},\;\bm{v}_{2} \rangle$と書く。
$\langle \bm{v}_{1},\;\bm{v}_{2} \rangle$は次のように定義される。
\begin{equation}
 \langle \bm{v}_{1},\;\bm{v}_{2} \rangle
  =
  \{ c_{1}\bm{v}_{1}+c_{2}\bm{v}_{2} \in V \mid c_{1},c_{2}\in\mathbb{R} \}
\end{equation}

\hrulefill

$\bm{a}=\begin{pmatrix} 1 \\ -1 \\ 0 \end{pmatrix},\;
\bm{b}=\begin{pmatrix} 1 \\ 0 \\ 2 \end{pmatrix},\;
\bm{c}=\begin{pmatrix} 0 \\ 1 \\ 2 \end{pmatrix}$
の時、次を示せ。
\begin{enumerate}
 \item $\langle \bm{a},\;\bm{b},\;\bm{c} \rangle = \langle \bm{a},\;\bm{b} \rangle$
 \item $\langle \bm{a},\;\bm{b},\;\bm{c} \rangle = \langle \bm{a},\;\bm{c} \rangle$
 \item $\langle \bm{a},\;\bm{b},\;\bm{c} \rangle = \langle \bm{b},\;\bm{c} \rangle$
\end{enumerate}
\dotfill

ベクトル$\bm{a},\bm{b},\bm{c}$は
$\bm{c}=\bm{b}-\bm{a}$を満たす。

\begin{enumerate}
 \item
      ${}^{\forall}\bm{v}\in\langle \bm{a},\;\bm{b},\;\bm{c} \rangle$とする。

      ベクトル$\bm{v}$はスカラー$s_{1},s_{2},s_{3}\in\mathbb{R}$を用いて
      $\bm{v}=s_{1}\bm{a}+s_{2}\bm{b}+s_{3}\bm{c}$と書ける。

      $\bm{c}=\bm{b}-\bm{a}$次の式が得られる。
      \begin{equation}
       \bm{v}
        =s_{1}\bm{a}+s_{2}\bm{b}+s_{3}\bm{c}
        =s_{1}\bm{a}+s_{2}\bm{b}+s_{3}(\bm{b}-\bm{a})
        =(s_{1}-s_{3})\bm{a}+(s_{2}+s_{3})\bm{b}
      \end{equation}

      これにより$\bm{v}\in \langle \bm{a},\;\bm{b} \rangle$である。
      つまり、$\langle \bm{a},\;\bm{b},\;\bm{c} \rangle \subset \langle \bm{a},\;\bm{b} \rangle$である。

      逆に
      ${}^{\forall}\bm{w}\in\langle \bm{a},\bm{b} \rangle$とする。

      ベクトル$\bm{w}$はスカラー$t_{1},t_{2}\in\mathbb{R}$を用いて
      $\bm{w}=t_{1}\bm{a}+t_{2}\bm{b}$と書ける。

      $0\in\mathbb{R}$を使って、
      $\bm{w}=t_{1}\bm{a}+t_{2}\bm{b}+0\bm{c}$と書けるので、
      $\bm{w} \in \langle \bm{a},\;\bm{b},\;\bm{c} \rangle$である。
      つまり、
      $\langle \bm{a},\;\bm{b} \rangle \subset \langle \bm{a},\;\bm{b},\;\bm{c} \rangle$である。

      これらにより
      $\langle \bm{a},\;\bm{b},\;\bm{c} \rangle = \langle \bm{a},\;\bm{b} \rangle$である。

 \item
      上記と同様に示せる。

      省略すると$\bm{b}=\bm{a}+\bm{c}$を用いて次のようになる。
      \begin{equation}
       \langle \bm{a},\;\bm{b},\;\bm{c} \rangle
        = \langle \bm{a},\; \bm{a}+\bm{c} ,\;\bm{c} \rangle
        = \langle \bm{a},\;\bm{c} \rangle
      \end{equation}
 \item
      $\bm{a}=\bm{b}-\bm{c}$を用いて次のようになる。
      \begin{equation}
       \langle \bm{a},\bm{b},\bm{c} \rangle
        = \langle \bm{b}-\bm{c},\;\bm{b},\;\bm{c} \rangle
        = \langle \bm{b},\;\bm{c} \rangle
      \end{equation}
\end{enumerate}

\hrulefill

$V$をベクトル空間とし、
$V$のベクトル$\bm{a},\bm{b},\bm{c},\bm{d}$が
$\bm{a}-\bm{b}+2\bm{c}-3\bm{d}=\bm{0}$を満たす時、
$\langle \bm{a},\;\bm{b},\;\bm{c},\;\bm{d} \rangle = \langle \bm{a},\;\bm{b},\;\bm{c} \rangle$を満たせ。

\dotfill

$\bm{a}-\bm{b}+2\bm{c}-3\bm{d}=\bm{0}$より、
$\bm{d}$は次のように表せる。
\begin{equation}
\bm{d}=\frac{1}{3}\bm{a}-\frac{1}{3}\bm{b}+\frac{2}{3}\bm{c}
\end{equation}

これを用いて次が得られる。
\begin{equation}
 \langle \bm{a},\;\bm{b},\;\bm{c},\;\bm{d} \rangle
  = \left\langle \bm{a},\;\bm{b},\;\bm{c},\;\frac{1}{3}\bm{a}-\frac{1}{3}\bm{b}+\frac{2}{3}\bm{c} \right\rangle
  = \langle \bm{a},\;\bm{b},\;\bm{c} \rangle
\end{equation}


\end{document}
