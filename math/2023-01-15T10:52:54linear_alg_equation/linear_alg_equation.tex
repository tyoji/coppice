\documentclass[12pt,b5paper]{ltjsarticle}

%\usepackage[margin=15truemm, top=5truemm, bottom=5truemm]{geometry}
%\usepackage[margin=10truemm,left=15truemm]{geometry}
\usepackage[margin=10truemm]{geometry}

\usepackage{amsmath,amssymb}
%\pagestyle{headings}
\pagestyle{empty}

%\usepackage{listings,url}
%\renewcommand{\theenumi}{(\arabic{enumi})}

\usepackage{graphicx}

%\usepackage{tikz}
%\usetikzlibrary {arrows.meta}
%\usepackage{wrapfig}
%\usepackage{bm}

% ルビを振る
\usepackage{luatexja-ruby}

%% 核Ker 像Im Hom を定義
%\newcommand{\Img}{\mathop{\mathrm{Im}}\nolimits}
%\newcommand{\Ker}{\mathop{\mathrm{Ker}}\nolimits}
%\newcommand{\Hom}{\mathop{\mathrm{Hom}}\nolimits}

%\DeclareMathOperator{\Rot}{rot}
%\DeclareMathOperator{\Div}{div}
%\DeclareMathOperator{\Grad}{grad}
%\DeclareMathOperator{\arcsinh}{arcsinh}
%\DeclareMathOperator{\arccosh}{arccosh}
%\DeclareMathOperator{\arctanh}{arctanh}



%\usepackage{listings,url}
%
%\lstset{
%%プログラム言語(複数の言語に対応,C,C++も可)
%  language = Python,
%%  language = Lisp,
%%  language = C,
%  %背景色と透過度
%  %backgroundcolor={\color[gray]{.90}},
%  %枠外に行った時の自動改行
%  breaklines = true,
%  %自動改行後のインデント量(デフォルトでは20[pt])
%  breakindent = 10pt,
%  %標準の書体
%%  basicstyle = \ttfamily\scriptsize,
%  basicstyle = \ttfamily,
%  %コメントの書体
%%  commentstyle = {\itshape \color[cmyk]{1,0.4,1,0}},
%  %関数名等の色の設定
%  classoffset = 0,
%  %キーワード(int, ifなど)の書体
%%  keywordstyle = {\bfseries \color[cmyk]{0,1,0,0}},
%  %表示する文字の書体
%  %stringstyle = {\ttfamily \color[rgb]{0,0,1}},
%  %枠 "t"は上に線を記載, "T"は上に二重線を記載
%  %他オプション:leftline,topline,bottomline,lines,single,shadowbox
%  frame = TBrl,
%  %frameまでの間隔(行番号とプログラムの間)
%  framesep = 5pt,
%  %行番号の位置
%  numbers = left,
%  %行番号の間隔
%  stepnumber = 1,
%  %行番号の書体
%%  numberstyle = \tiny,
%  %タブの大きさ
%  tabsize = 4,
%  %キャプションの場所("tb"ならば上下両方に記載)
%  captionpos = t
%}



\begin{document}

\hrulefill

\begin{equation}
 A
  \begin{pmatrix}
   x\\y\\z
  \end{pmatrix}
  =
  \begin{pmatrix}
   0\\0\\0
  \end{pmatrix}
  ,\qquad
 A=
  \begin{pmatrix}
   1 & 0 & -1\\
   1 & a & 0\\
   -2 & -1 & a
  \end{pmatrix}
\end{equation}

上記連立方程式が
自明でない解をもつときの$a$の値を求め、
その時の解を求めよ。

\dotfill


\begin{equation}
 A
  \begin{pmatrix}
   x\\y\\z
  \end{pmatrix}
  =
  \begin{pmatrix}
   x-z \\x + ay \\ -2x-y+az
  \end{pmatrix}
  =
  \begin{pmatrix}
   0\\0\\0
  \end{pmatrix}
  \quad
  \Leftrightarrow
  \quad
  \begin{cases}
   x-z=0 \\ x+ay=0 \\ -2x-y+az=0
  \end{cases}
\end{equation}

最初の式から$x=z$を満たすことがわかるので、
$z$を$x$に置き換えると
$x+ay=0,(a-2)x-y=0$となる。
ここから$y$を代入し消去すると
\begin{equation}
 x+a(a-2)x=0
  \ \Leftrightarrow\
  x(a^2-2a+1)=0
  \ \Leftrightarrow\
  x(a-1)^2=0
\end{equation}

$x(a-1)^2=0$より$x=0$または$a=1$である。

$x=0$の時、$y=0,z=0$となる為、自明な解である。

そこで、$a=1$の場合を考える。
この時の行列$A$の階数は2である。
\begin{equation}
 \mathrm{rank} A
  = \mathrm{rank}
  \begin{pmatrix}
   1 & 0 & -1\\
   1 & 1 & 0\\
   -2 & -1 & 1
  \end{pmatrix}
  = \mathrm{rank}
 \begin{pmatrix}
   1 & 0 & -1\\
   1 & 1 & 0\\
   0 & 0 & 0
 \end{pmatrix}
  = 2
\end{equation}

その為、$A$は退化し、方程式の解は1次元の解空間となる。

$x=z,y=(a-2)x$であるので、方程式の解は次のようになる。

\begin{equation}
 \begin{pmatrix}
  x \\ y \\ z
 \end{pmatrix}
 =
 \begin{pmatrix}
  x \\ (a-2)x \\ x
 \end{pmatrix}
 =
 \begin{pmatrix}
  x \\ -x \\ x
 \end{pmatrix}
 =
 x
 \begin{pmatrix}
  1 \\ -1 \\ 1
 \end{pmatrix}
\end{equation}

$x$は任意の数でいいので
定数$c$を用いると次のようになる。
\begin{equation}
 \begin{pmatrix}
  x \\ y \\ z
 \end{pmatrix}
 =
 c
 \begin{pmatrix}
  1 \\ -1 \\ 1
 \end{pmatrix}
\end{equation}


\hrulefill

\end{document}
