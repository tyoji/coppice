\documentclass[12pt,b5paper]{ltjsarticle}

%\usepackage[margin=15truemm, top=5truemm, bottom=5truemm]{geometry}
%\usepackage[margin=10truemm,left=15truemm]{geometry}
\usepackage[margin=10truemm]{geometry}

\usepackage{amsmath,amssymb}
%\pagestyle{headings}
\pagestyle{empty}

%\usepackage{listings,url}
%\renewcommand{\theenumi}{(\arabic{enumi})}

\usepackage{graphicx}

%\usepackage{tikz}
%\usetikzlibrary {arrows.meta}
%\usepackage{wrapfig}
%\usepackage{bm}

% ルビを振る
%\usepackage{luatexja-ruby}	% required for `\ruby'

%% 核Ker 像Im Hom を定義
%\newcommand{\Img}{\mathop{\mathrm{Im}}\nolimits}
%\newcommand{\Ker}{\mathop{\mathrm{Ker}}\nolimits}
%\newcommand{\Hom}{\mathop{\mathrm{Hom}}\nolimits}

%\DeclareMathOperator{\Rot}{rot}
%\DeclareMathOperator{\Div}{div}
%\DeclareMathOperator{\Grad}{grad}
%\DeclareMathOperator{\arcsinh}{arcsinh}
%\DeclareMathOperator{\arccosh}{arccosh}
%\DeclareMathOperator{\arctanh}{arctanh}



%\usepackage{listings,url}
%
%\lstset{
%%プログラム言語(複数の言語に対応,C,C++も可)
%  language = Python,
%%  language = Lisp,
%%  language = C,
%  %背景色と透過度
%  %backgroundcolor={\color[gray]{.90}},
%  %枠外に行った時の自動改行
%  breaklines = true,
%  %自動改行後のインデント量(デフォルトでは20[pt])
%  breakindent = 10pt,
%  %標準の書体
%%  basicstyle = \ttfamily\scriptsize,
%  basicstyle = \ttfamily,
%  %コメントの書体
%%  commentstyle = {\itshape \color[cmyk]{1,0.4,1,0}},
%  %関数名等の色の設定
%  classoffset = 0,
%  %キーワード(int, ifなど)の書体
%%  keywordstyle = {\bfseries \color[cmyk]{0,1,0,0}},
%  %表示する文字の書体
%  %stringstyle = {\ttfamily \color[rgb]{0,0,1}},
%  %枠 "t"は上に線を記載, "T"は上に二重線を記載
%  %他オプション:leftline,topline,bottomline,lines,single,shadowbox
%  frame = TBrl,
%  %frameまでの間隔(行番号とプログラムの間)
%  framesep = 5pt,
%  %行番号の位置
%  numbers = left,
%  %行番号の間隔
%  stepnumber = 1,
%  %行番号の書体
%%  numberstyle = \tiny,
%  %タブの大きさ
%  tabsize = 4,
%  %キャプションの場所("tb"ならば上下両方に記載)
%  captionpos = t
%}



\begin{document}

\hrulefill

群$G$の元$x$の位数
とは、
$x$のみで生成された部分群の
位数$\lvert \langle x \rangle \rvert$のことを指す。

\hrulefill

$G$を有限群、
$H$をその部分群とする。

\begin{enumerate}
 \item
      $H$ の位数 $\lvert H \rvert$は
      $G$ の位数 $\lvert G \rvert$の約数である。

      \dotfill

      $g\in G$に対して
      剰余類$gH=\{gh \mid h\in H\}$を考える。

      このとき、写像$f$を次のように定義する。
      \begin{equation}
       f:H\to gH ,\quad h \mapsto gh
      \end{equation}

      このとき、${}^{\forall} gh \in gH$に対して
      $h\in H$が存在するので、
      $f$は全射である。

      また、$f(h_{1})=f(h_{2})$とすれば
      $gh_{1}=gh_{2}$であり、左から$g^{-1}$をかければ
      $h_{1}=h_{2}$となり、
      $f$は単射である。

      $f$は全単射であり、$H,\;gH$は有限集合であるので、
      $\lvert H \rvert = \lvert gH \rvert$である。

      $H \cap gH \ne \emptyset$
      とする。
      つまり、$h_{1},h_{2} \in H$
      が存在し、
      $h_{1}=gh_{2}$となる。
      右から$h_{2}^{-1}$を書けることで、
      $h_{1}h_{2}^{-1}=g$となり、
      $g\in H$である。
      $H$は部分群であるから
      $H = gH$である。
      つまり、次が言える。
      \begin{equation}
       H \cap gH \ne \emptyset
        \Rightarrow H= gH
      \end{equation}

      $\alpha,\beta\in G$について
      $\alpha H \cap \beta H \ne \emptyset$
      とする。
      このとき、
      $x \in \alpha H \cap \beta H$
      が存在する。
      つまり、
      $x=\alpha h_{1} = \beta h_{2}$
      となる$h_{1},h_{2}\in H$が存在する。
      右から$h_{1}^{-1}$をかけると
      $\alpha = \beta h_{2} h_{1}^{-1} \in \beta H$
      である。
      これにより
      $\beta^{-1}\alpha \in H$であるため
      $H = \beta^{-1}\alpha H$となり、
      $\beta H = \alpha H$となる。

      $\alpha,\beta\in G$について
      $\alpha H \cap \beta H = \emptyset$
      または$\alpha H = \beta H$である。

      $H$は部分群であるから
      単位元$e\in H$を含むので、
      次の式が成り立つ。
      \begin{equation}
       G = \bigcup_{g\in G} gH
      \end{equation}

      ${}^{\forall}g,g^{\prime}\in G$に対して
      $gH=g^{\prime}H$ または $gH \cap g^{\prime}H = \emptyset$
      であるので、部分群$H$の位数は$G$の位数を割り切ることが出来る。


      \hrulefill

 \item
      $G$の元の位数は
      $G$ の位数 $\lvert G \rvert$の約数である。

      \dotfill

      $g\in G$が生成する部分群$\langle g \rangle$は
      $G$の部分群であるので、
      $g$の位数は$G$の位数の約数である。

      \hrulefill

\end{enumerate}


\hrulefill


\end{document}
