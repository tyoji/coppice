\documentclass[12pt,b5paper]{ltjsarticle}

%\usepackage[margin=15truemm, top=5truemm, bottom=5truemm]{geometry}
\usepackage[margin=15truemm]{geometry}

\usepackage{amsmath,amssymb}
%\pagestyle{headings}
\pagestyle{empty}

%\usepackage{listings,url}
%\renewcommand{\theenumi}{(\arabic{enumi})}

\usepackage{graphicx}

\usepackage{tikz}
\usetikzlibrary {arrows.meta}
\usepackage{wrapfig}	% required for `\wrapfigure' (yatex added)
\usepackage{bm}	% required for `\bm' (yatex added)

% ルビを振る
%\usepackage{luatexja-ruby}	% required for `\ruby'

%% 核Ker 像Im Hom を定義
%\newcommand{\Img}{\mathop{\mathrm{Im}}\nolimits}
%\newcommand{\Ker}{\mathop{\mathrm{Ker}}\nolimits}
%\newcommand{\Hom}{\mathop{\mathrm{Hom}}\nolimits}
\newcommand{\Rot}{\mathop{\mathrm{rot}}\nolimits}
\newcommand{\Div}{\mathop{\mathrm{div}}\nolimits}

\begin{document}

\hrulefill

$D$を$\mathbb{R}^2$の部分集合(開集合?)とする。

$\bm{f}$を$C^1$-級ベクトル場とする。
\begin{equation}
 \bm{f}: D \to \mathbb{R}^2 , \quad (x,y) \mapsto \begin{pmatrix}f_1(x,y)\\f_2(x,y)\end{pmatrix}
\end{equation}

\begin{itemize}
 \item
%      $\bm{f}$の回転を$\Rot \bm{f}$を書き、次のように定義する。
      \begin{equation}
       (\Rot\bm{f})(x,y)
        = \frac{\partial f_2}{\partial x}(x,y) - \frac{\partial f_1}{\partial y}(x,y)
      \end{equation}
 \item
%      $\bm{f}$の発散を$\Div\bm{f}$を書き、次のように定義する。
      \begin{equation}
       (\Div\bm{f})(x,y)
        =\frac{\partial f_1}{\partial x}(x,y) + \frac{\partial f_2}{\partial y}(x,y)
      \end{equation}
 \item
      \begin{equation}
       (*\bm{f})(x,y)
        = \begin{pmatrix}-f_2(x,y)\\f_1(x,y)\end{pmatrix}
      \end{equation}
\end{itemize}

\hrulefill

$\Rot\bm{f}=-\Div(*\bm{f})$を示せ。

\dotfill

\begin{align}
 \Div(*\bm{f}) =& \Div\begin{pmatrix}-f_2(x,y)\\f_1(x,y)\end{pmatrix}\\
 =& -\frac{\partial f_2}{\partial x}(x,y) + \frac{\partial f_1}{\partial y}(x,y)
 = -\Rot\bm{f}
\end{align}
よって、$\Rot\bm{f}=-\Div(*\bm{f})$となる。

\hrulefill

$\bm{g}=*\bm{f}$とおく。

$\bm{g}$に対してGreenの定理を用いると
\begin{equation}
 \int_{D}\Rot\bm{g}\;\mathrm{d}x\mathrm{d}y
  =\int_{\partial D}\bm{g}
\end{equation}
であり、
$\Rot\bm{g} = \Div\bm{f}$となることを示せ。

\dotfill

\begin{equation}
 \bm{g}=\begin{pmatrix}-f_2(x,y)\\f_1(x,y)\end{pmatrix}
 ,\quad
  \Rot\bm{g}=\Rot\begin{pmatrix}-f_2(x,y)\\f_1(x,y)\end{pmatrix}
  =\frac{\partial f_1}{\partial x}(x,y) + \frac{\partial f_2}{\partial y}(x,y)
\end{equation}

グリーンの定理より右辺は
\begin{equation}
 \int_{\partial D}\bm{g}
  = \int_{D}\left(\frac{\partial f_1}{\partial x}(x,y) - \frac{\partial (-f_2)}{\partial y}(x,y)\right)\mathrm{d}x\mathrm{d}y
\end{equation}
であり、左辺は
\begin{equation}
 \int_{D}\Rot\bm{g}\;\mathrm{d}x\mathrm{d}y
  = \int_{D}\left(\frac{\partial f_1}{\partial x}(x,y) + \frac{\partial f_2}{\partial y}(x,y)\right)\mathrm{d}x\mathrm{d}y
\end{equation}
である為、
 $\int_{D}\Rot\bm{g}\;\mathrm{d}x\mathrm{d}y=\int_{\partial D}\bm{g}$
である。

また、
\begin{equation}
 \Div\bm{f}=\frac{\partial f_1}{\partial x}(x,y) + \frac{\partial f_2}{\partial y}(x,y)
\end{equation}
であるので、
$\Rot\bm{g}=\Div\bm{f}$
である。


\hrulefill

\end{document}
