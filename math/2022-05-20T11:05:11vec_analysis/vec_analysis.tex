\documentclass[12pt,b5paper]{ltjsarticle}

%\usepackage[margin=15truemm, top=5truemm, bottom=5truemm]{geometry}
\usepackage[margin=15truemm]{geometry}

\usepackage{amsmath,amssymb}
%\pagestyle{headings}
\pagestyle{empty}

%\usepackage{listings,url}
%\renewcommand{\theenumi}{(\arabic{enumi})}

\usepackage{graphicx}

\usepackage{tikz}
\usetikzlibrary {arrows.meta}
\usepackage{bm}	% required for `\bm' (yatex added)
\usepackage{luatexja-ruby}      % required for `\ruby'
%% 像Im を定義
%\newcommand{\Img}{\mathop{\mathrm{Im}}\nolimits}

\begin{document}

\hrulefill

\begin{enumerate}
 \item 
       \begin{enumerate}
        \item
             次の中から開集合を選択せよ。
             \begin{align}
               A &= \{\bm{x}\in\mathbb{R}^2 \mid 1\leq \lvert x \rvert <2  \}\\
               B &= \{\bm{x}=(x_1, x_2)\in\mathbb{R}^2 \mid x_1>0, \; x_2>0 \}\\
               C &= \{\bm{x}\in\mathbb{R}^3 \mid \lvert \bm{x} \rvert<1 \} \cup \{(1,0,0)\}\\
               D &= \bigcap_{k=1}^{\infty} D_k \qquad \left(
                   D_k= \left\{\bm{x}\in\mathbb{R}^2 \mid
                   \lvert \bm{x} \rvert < \frac{1}{k} \right\}\right)\\
               E &= \bigcup_{k=1}^{\infty} E_k \qquad \left(
               E_k= \left\{\bm{x}\in\mathbb{R}^2 \mid
               \lvert \bm{x} \rvert < 1- \frac{1}{k} \right\}\right)
             \end{align}
        \item
             $D= \{(x,\;y)\in\mathbb{R}^2 \mid x>0,\; y>0\}$とし、
             $f: D\rightarrow\mathbb{R}$を$f(x,\;y)=x^y$と定義する。
             $(a,\;b)\in D$について次の式を求めよ。
             \begin{equation}
              \frac{\partial f}{\partial x}(a,b), \qquad \frac{\partial f}{\partial y}(a,b)
             \end{equation}
        \item
             写像$\bm{f}$を次のように定める。
             \begin{equation}
              \bm{f}:\mathbb{R}^2\rightarrow\mathbb{R}^3
               \qquad \bm{f}(x_1,\;x_2)
                 = \begin{pmatrix}x_1^2\\-x_1^2+x_2\\-4x_1+x_2^2\end{pmatrix}
             \end{equation}
             また、$\bm{n}={}^{t}\!(1\ 2\ 1)$とする。

             この時、次の条件を満たす$(a,\;b)\in\mathbb{R}^2$を求めよ。
             \begin{equation}
              \frac{\partial \bm{f}}{\partial x_i}(a,b)
               \ と\ \bm{n} \ は直交 \quad (i=1,2)
             \end{equation}
       \end{enumerate}
 \item
      $D_1,\;D_2\subset \mathbb{R}^n$を開集合とする。
      この時、$D_1\cap D_2$と$D_1\cup D_2$は$\mathbb{R}^n$の開集合となることを示せ。
 \item
      $D\subset \mathbb{R}^2$を開集合とし、
      $f:D\rightarrow \mathbb{R}$を$C^2$級関数とする。
      この時、次の式を満たすことを示せ。
      \begin{equation}
       \frac{\partial^2 f}{\partial x\partial y}
        =\frac{\partial^2 f}{\partial y\partial x}
      \end{equation}
\end{enumerate}

\hrulefill

\textbf{開集合}

$\mathbb{R}^n$の部分集合$U$が次を満たす時、開集合という。
($d(x,y)$は距離関数)
\begin{equation}
 {}^{\forall} x\in U, \ {}^{\exists}\varepsilon >0\
  s.t.\ y\in\mathbb{R}^n,\; d(x,\;y)<\varepsilon \Rightarrow y\in U\label{103656_18May22}
\end{equation}

開集合の公理

集合$X$の部分集合族$\mathcal{O}$が次の3つを満たす時、
$\mathcal{O}$の要素を開集合という。
\begin{enumerate}
 \item $\phi\in\mathcal{O}, \ X\in\mathcal{O}$
 \item $\displaystyle U_i\in\mathcal{O}\ (i=1,\dots,n) \Rightarrow \bigcap_{i=1}^{n}U_i\in\mathcal{O}$
 \item $\displaystyle U_{\lambda}\in\mathcal{O}\ (\lambda\in\Lambda) \Rightarrow \bigcup_{\lambda\in\Lambda}U_{\lambda}\in\mathcal{O}$
\end{enumerate}


式(\ref{103656_18May22})により定義した開集合は開集合の公理を満たす。

\dotfill

%$r\in\mathbb{R}, r\ne 0$

\begin{align}
 \frac{\mathrm{d}}{\mathrm{d}x}x^r = rx^{r-1} \qquad (r\ne 0)\\
 \frac{\mathrm{d}}{\mathrm{d}x}r^x = r^x \log r \qquad (r> 0)
\end{align}

\dotfill

$f(x,y)$の偏導関数の定義

\begin{align}
 f_x(x,y) = \frac{\partial}{\partial x}f(x,y)
&=\lim_{h\rightarrow 0} \frac{f(x+h,y)-f(x,y)}{h}\\
 f_y(x,y) = \frac{\partial}{\partial y}f(x,y)
&=\lim_{h\rightarrow 0} \frac{f(x,y+h)-f(x,y)}{h}
\end{align}

\dotfill

平均値の定理(\ruby{Lagrange}{ラグランジュ})

関数$f(x)$を閉区間$[a,b]$で連続、開区間$(a,b)$で微分可能とする。

この時、次を満たす$c\in (a,b)$が存在する。
\begin{equation}
 \frac{f(b)-f(a)}{b-a} = f^{\prime}(c)
\end{equation}

また、区間の幅を$h=b-a$とおき、
$c=a+\theta h \ (0<\theta<1)$
とすると次の式を得る。
\begin{equation}
 f(b)-f(a) = f^{\prime}(a+\theta h)h
\end{equation}

\hrulefill

次の中から開集合を選択せよ。
\begin{align}
 A &= \{\bm{x}\in\mathbb{R}^2 \mid 1\leq \lvert x \rvert <2  \}\\
 B &= \{\bm{x}=(x_1, x_2)\in\mathbb{R}^2 \mid x_1>0, \; x_2>0 \}\\
 C &= \{\bm{x}\in\mathbb{R}^3 \mid \lvert \bm{x} \rvert<1 \} \cup \{(1,0,0)\}\\
 D &= \bigcap_{k=1}^{\infty} D_k \qquad \left(
   D_k= \left\{\bm{x}\in\mathbb{R}^2 \mid
   \lvert \bm{x} \rvert < \frac{1}{k} \right\}\right)\\
 E &= \bigcup_{k=1}^{\infty} E_k \qquad \left(
   E_k= \left\{\bm{x}\in\mathbb{R}^2 \mid
   \lvert \bm{x} \rvert < 1- \frac{1}{k} \right\}\right)
\end{align}

\begin{itemize}
 \item $A$について

       $(1,0)\in A$であるが、
       ${}^{\forall}\varepsilon>0$とすると$(1-\varepsilon,0)\not\in A$である。

       $A$は$(1,0)$の$\varepsilon$-近傍を含まないので開集合ではない。

 \item $B$について

       $(x_1,x_2)\in B$に対し、$\varepsilon=\min\{x_1/2,\;x_2/2\}$とする。
       $(x_1,x_2)$から距離$\varepsilon$の範囲にある全ての点は$B$に含まれる。

       $B$は任意の点$(x_1,x_2)$の$\varepsilon$-近傍を含むので開集合である。

 \item $C$について

       $(1,0,0)\in C$であるが、
       ${}^{\forall}\varepsilon>0$とすると$(1+\varepsilon,0,0)\not\in C$である。

       $C$は$(1,0)$の$\varepsilon$-近傍を含まないので開集合ではない。

 \item $D$について

       各$D_k$は開集合であり、$D_1 \supset D_2 \supset D_3 \supset \cdots$である。

       $(0,0)\in D$であるが、
       $\varepsilon>0$とすると
       $0< 1/l < \varepsilon$となる$l\in\mathbb{N}$が存在する。
       つまり、$(\varepsilon,0)\not\in D_l$である。

       $D$は$(0,0)$の近傍を含まないため$D$は開集合でない。

 \item $E$について

       各$E_k$は開集合であり、
       $\phi=E_1\subset E_2\subset \cdots \subset \left\{\bm{x}\in\mathbb{R}^2 \mid
   \lvert \bm{x} \rvert < 1 \right\}$
       である。

       $\bm{p}\in E$に対して、$\bm{p}\in E_l$が存在する。
       $\bm{p}$の$\varepsilon$-近傍は$E_l$に含まれるから$E$にも含まれる。
       よって、$E$は開集合である。
\end{itemize}

以上より、開集合は$B$と$E$の2つである。

\dotfill


$D= \{(x,\;y)\in\mathbb{R}^2 \mid x>0,\; y>0\}$とし、
$f: D\rightarrow\mathbb{R}$を$f(x,\;y)=x^y$と定義する。
$(a,\;b)\in D$について次の式を求めよ。
\begin{equation}
 \frac{\partial f}{\partial x}(a,b), \qquad \frac{\partial f}{\partial y}(a,b)
\end{equation}


\begin{equation}
 \frac{\partial f}{\partial x}(x,y) = yx^{y-1},
  \qquad
  \frac{\partial f}{\partial y}(x,y) = x^y \log x
\end{equation}
これにより求める式は次のようになる。
\begin{equation}
 \frac{\partial f}{\partial x}(a,b) = ba^{b-1},
  \qquad
  \frac{\partial f}{\partial y}(a,b) = a^b \log a
\end{equation}

\dotfill

写像$\bm{f}$を次のように定める。
\begin{equation}
 \bm{f}:\mathbb{R}^2\rightarrow\mathbb{R}^3
  \qquad \bm{f}(x_1,\;x_2)
    = \begin{pmatrix}x_1^2\\-x_1^2+x_2\\-4x_1+x_2^2\end{pmatrix}
\end{equation}
また、$\bm{n}={}^{t}\!(1\ 2\ 1)$とする。

この時、次の条件を満たす$(a,\;b)\in\mathbb{R}^2$を求めよ。
\begin{equation}
 \frac{\partial \bm{f}}{\partial x_i}(a,b)
  \ と\ \bm{n} \ は直交 \quad (i=1,2)
\end{equation}

$\bm{f}$を$x_1,\;x_2$で偏微分する。
\begin{equation}
 \frac{\partial \bm{f}}{\partial x_1}(a,b)
    = \begin{pmatrix}2a\\-2a\\-4\end{pmatrix},
 \quad
 \frac{\partial \bm{f}}{\partial x_2}(a,b)
    = \begin{pmatrix}0\\1\\2b\end{pmatrix},
\end{equation}

この2つの式は$\bm{n}$と直行する為、
それぞれの内積が0になる。
\begin{equation}
 2a-4a-4 = 0,
 \quad
 0+ 2+2b=0
\end{equation}

この結果、$(a,b)=(-2,-1)$である。

\dotfill

$D_1,\;D_2\subset \mathbb{R}^n$を開集合とする。
この時、$D_1\cap D_2$と$D_1\cup D_2$は$\mathbb{R}^n$の開集合となることを示せ。

${}^{\forall}\bm{p}\in D_1\cap D_2$とする。

開集合の定義はある実数$\varepsilon >0$が存在し、
点$\bm{p}$からの距離が$\varepsilon$に満たない範囲の全ての点が
$\bm{p}$と同じ集合に含まれていることである。

$\bm{p}\in D_1\cap D_2$より
$\bm{p}\in D_1$かつ$\bm{p}\in D_2$である。

$\bm{p}\in D_1$である為、$\varepsilon_1>0$が存在し開集合の定義を満たす。
同様に$\bm{p}\in D_2$から$\varepsilon_2>0$が存在する。
そこで$\varepsilon = \min\{\varepsilon_1, \varepsilon_2\}$とすると、
$D_1,\;D_2$の両方の開集合の定義を満たす。

この$\varepsilon$-近傍内の点は$D_1 \cap D_2$に含まれる為、
$D_1 \cap D_2$が開集合であることが分かる。

$\bm{p}\in D_1 \cup D_2$とする。
もし、$\bm{p}\in D_1$であれば、
$\bm{p}$の$\varepsilon$-近傍は$D_1$に含まれる為、$D_1 \cup D_2$にも含まれる。
$D_2$も同様に考えることで
$D_1 \cup D_2$が開集合であることが分かる。


\dotfill


\newpage



$D\subset \mathbb{R}^2$を開集合とし、
$f:D\rightarrow \mathbb{R}$を$C^2$級関数とする。
この時、次の式を満たすことを示せ。
\begin{equation}
 \frac{\partial^2 f}{\partial x\partial y}
  =\frac{\partial^2 f}{\partial y\partial x}
\end{equation}


導関数の定義に従い式を書き換える。
\begin{align}
 & \frac{\partial^2 f}{\partial x\partial y}(x,y)\\
 &= \frac{\partial}{\partial x}\left(\frac{\partial f}{\partial y} (x,y)\right)\\
 &= \frac{\partial}{\partial x}\left(\lim_{h_y\rightarrow 0}\frac{f(x,y+h_y)-f(x,y)}{h_y}\right)\\
 &=\lim_{h_x\rightarrow 0} \frac{1}{h_x}\left(
 \lim_{h_y\rightarrow 0}\frac{f(x+h_x,y+h_y)-f(x+h_x,y)}{h_y}
 - \lim_{h_y\rightarrow 0}\frac{f(x,y+h_y)-f(x,y)}{h_y}
 \right)\\
 &=\lim_{h_x\rightarrow 0} \lim_{h_y\rightarrow 0}
 \frac{1}{h_xh_y}\left( f(x+h_x,y+h_y)-f(x+h_x,y) - f(x,y+h_y)+f(x,y) \right)\\
 &= \lim_{h_x\rightarrow 0} \lim_{h_y\rightarrow 0} \frac{\Delta}{h_xh_y}
\end{align}
ただし、$\Delta$は次のように置く。
\begin{equation}
 \Delta =f(x+h_x,y+h_y)-f(x+h_x,y) - f(x,y+h_y)+f(x,y)\label{Del0}
\end{equation}
同様にして次も得られる。
\begin{equation}
 \frac{\partial^2 f}{\partial y\partial x}(x,y)
 =\lim_{h_y\rightarrow 0} \lim_{h_x\rightarrow 0}
 \frac{\Delta}{h_xh_y}
\end{equation}

ここで、
\begin{equation}
 g_1(X) = f(X,y+h_y)-f(X,y)
\end{equation}
とおくと、
\begin{align}
 \Delta =& g_1(x+h_x) - g_1(x)\\
 =& g_1^{\prime}(x+\theta_1 h_x)h_x & (0<\theta_1<1) \cdots \text{平均値の定理}
\end{align}
である。

\begin{equation}
 g_1^{\prime}(X)
  = \frac{\mathrm{d}}{\mathrm{d}X}g_1(X)
  = \frac{\partial}{\partial X}f(X,y+h_y)-\frac{\partial}{\partial X}f(X,y)
  = f_x(X,y+h_y)-f_x(X,y)
\end{equation}
より
\begin{equation}
 \Delta = ( f_x(x+\theta_1 h_x, y+h_y) - f_x(x+\theta_1 h_x, y) )h_x
\end{equation}
である。
\begin{equation}
 g_2(Y) = f_x(x+\theta_1 h_x,Y)
\end{equation}
とおくと
\begin{align}
 \Delta =& ( g_2(y+h_y) - g_2(y) )h_x\\
 =& ( g_2^{\prime}(y+\theta_2h_y)h_y )h_x & (0<\theta_2<1) \cdots \text{平均値の定理}
\end{align}

\begin{equation}
 g_2^{\prime}(Y) = \frac{\mathrm{d}}{\mathrm{d}Y}g_2(Y)
  = \frac{\partial}{\partial Y}f_x(x+\theta_1h_x,Y)
  = f_{xy}(x+\theta_1h_x, Y)
\end{equation}
より$\Delta$は次のように表せる。
\begin{equation}
 \Delta = f_{xy}(x+\theta_1 h_x,\; y+\theta_2h_y)h_xh_y \qquad (0<\theta_i<1, \ i=1,2)
  \label{002624_20May22}
\end{equation}
この式は$\Delta$の式(\ref{Del0})を$x$について式を変形した後、
$y$について変形を行っている。

逆に
式(\ref{Del0})を$y$について式を変形した後、
$x$について変形を行うと次の式が得られる。
\begin{equation}
 \Delta = f_{yx}(x+\theta_3 h_x,\; y+\theta_4h_y)h_xh_y \qquad (0<\theta_i<1, \ i=3,4)\label{002641_20May22}
\end{equation}

式(\ref{002624_20May22})と式(\ref{002641_20May22})を使うと次の2つの式が得られる。
\begin{align}
 \lim_{(h_x,h_y)\rightarrow(0,0)}\frac{\Delta}{h_xh_y}
 = \lim_{(h_x,h_y)\rightarrow(0,0)}f_{xy}(x+\theta_1 h_x,\; y+\theta_2h_y)
 = f_{xy}(x,\; y)\\
 \lim_{(h_x,h_y)\rightarrow(0,0)}\frac{\Delta}{h_xh_y}
 = \lim_{(h_x,h_y)\rightarrow(0,0)}f_{yx}(x+\theta_3 h_x,\; y+\theta_4h_y)
 = f_{yx}(x,\; y)
\end{align}

左辺が等しい為、$f_{xy}(x,y)=f_{yx}(x,y)$であることが示せる。
\end{document}
