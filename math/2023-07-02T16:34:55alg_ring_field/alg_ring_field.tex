\documentclass[12pt,b5paper]{ltjsarticle}

%\usepackage[margin=15truemm, top=5truemm, bottom=5truemm]{geometry}
%\usepackage[margin=10truemm,left=15truemm]{geometry}
\usepackage[margin=10truemm]{geometry}

\usepackage{amsmath,amssymb}
%\pagestyle{headings}
\pagestyle{empty}

%\usepackage{listings,url}
%\renewcommand{\theenumi}{(\arabic{enumi})}

%\usepackage{graphicx}

%\usepackage{tikz}
%\usetikzlibrary {arrows.meta}
%\usepackage{wrapfig}
%\usepackage{bm}

% ルビを振る
\usepackage{luatexja-ruby}	% required for `\ruby'

%% 核Ker 像Im Hom を定義
%\newcommand{\Img}{\mathop{\mathrm{Im}}\nolimits}
%\newcommand{\Ker}{\mathop{\mathrm{Ker}}\nolimits}
%\newcommand{\Hom}{\mathop{\mathrm{Hom}}\nolimits}

%\DeclareMathOperator{\Rot}{rot}
%\DeclareMathOperator{\Div}{div}
%\DeclareMathOperator{\Grad}{grad}
%\DeclareMathOperator{\arcsinh}{arcsinh}
%\DeclareMathOperator{\arccosh}{arccosh}
%\DeclareMathOperator{\arctanh}{arctanh}

\usepackage{url}

\usepackage{listings,url}

\lstset{
%プログラム言語(複数の言語に対応,C,C++も可)
  language = Python,
%  language = Lisp,
%  language = C,
  %背景色と透過度
  %backgroundcolor={\color[gray]{.90}},
  %枠外に行った時の自動改行
  breaklines = true,
  %自動改行後のインデント量(デフォルトでは20[pt])
  breakindent = 10pt,
  %標準の書体
%  basicstyle = \ttfamily\scriptsize,
  basicstyle = \ttfamily,
  %コメントの書体
%  commentstyle = {\itshape \color[cmyk]{1,0.4,1,0}},
  %関数名等の色の設定
  classoffset = 0,
  %キーワード(int, ifなど)の書体
%  keywordstyle = {\bfseries \color[cmyk]{0,1,0,0}},
  %表示する文字の書体
  %stringstyle = {\ttfamily \color[rgb]{0,0,1}},
  %枠 "t"は上に線を記載, "T"は上に二重線を記載
  %他オプション:leftline,topline,bottomline,lines,single,shadowbox
  frame = TBrl,
  %frameまでの間隔(行番号とプログラムの間)
  framesep = 5pt,
  %行番号の位置
  numbers = left,
  %行番号の間隔
  stepnumber = 1,
  %行番号の書体
%  numberstyle = \tiny,
  %タブの大きさ
  tabsize = 4,
  %キャプションの場所("tb"ならば上下両方に記載)
  captionpos = t
}



\begin{document}

\hrulefill

\begin{enumerate}
 \item
      虚数単位$i=\sqrt{-1}$とし、
      $\alpha,\beta,\gamma$を次のように定める。
      \begin{equation}
       \alpha = \cos\left(\frac{2\pi}{7}\right)
                + i\sin\left(\frac{2\pi}{7}\right)
        ,\quad
        \beta = \sqrt[11]{2} \left( = 2^{\frac{1}{11}}\right)
        ,\quad
        \gamma = \cos\left(\frac{2\pi}{7}\right)
      \end{equation}

      この時、次の値を求めよ。
      \begin{enumerate}
       \item
            $\left[ \: \mathbb{Q}(\alpha,\beta):\mathbb{Q} \: \right]$

            \dotfill

            $\beta^{11}=2$であるので、
            $\beta$は多項式$X^{11}-2 \in \mathbb{Q}[X]$の解である。
            $X^11-2$は$\mathbb{Q}[X]$上既約であるので、
            体の拡大次数は
            $\left[ \: \mathbb{Q}(\beta):\mathbb{Q} \: \right] = 11$
            である。

            $\alpha = \exp{\left( \frac{2\pi}{7}i \right)}$より
            $\alpha^{7} = \exp{\left( 2\pi i\right)} = 1$であるので、
            $\alpha$は多項式$X^7-1\in\mathbb{Q}[X]$の解である。
            $X^7-1$は$\mathbb{Q}[X]$上可約であり、
            次の2つの多項式の積に分けられる。
            \begin{equation}
             X^7-1 = (X-1) (X^6 + X^5 + X^4 + X^3 + X^2 + X + 1 )
            \end{equation}
            $\alpha$は上記$6$次式の解となるので、
            $\left[ \: \mathbb{Q}(\alpha):\mathbb{Q} \: \right] = 6$
            である。

            $\left[ \: \mathbb{Q}(\alpha,\beta):\mathbb{Q} \: \right]
            =
            \left[ \: \mathbb{Q}(\alpha,\beta):\mathbb{Q}(\beta) \: \right]
            \left[ \: \mathbb{Q}(\beta):\mathbb{Q} \: \right]$
            であるので、
            $\left[ \: \mathbb{Q}(\alpha,\beta):\mathbb{Q}(\beta) \: \right]$
            がわかればよい。

            $\mathbb{Q}(\beta)$は$\mathbb{Q}$係数のベクトル空間で
            基底を$\beta^{0},\cdots,\beta^{10}$とする事ができる。
            $\alpha$は虚数であるので、$\mathbb{Q}(\beta)$の元ではない。
            よって、
            $\left[ \: \mathbb{Q}(\alpha,\beta):\mathbb{Q}(\beta) \: \right]
            =\left[ \: \mathbb{Q}(\alpha):\mathbb{Q} \: \right]$
            である。

            これにより
            $\left[ \: \mathbb{Q}(\alpha,\beta):\mathbb{Q} \: \right]
            = 6\times 11 = 66$
            である。

            \hrulefill
       \item
            $\left[ \: \mathbb{Q}(\alpha,\gamma):\mathbb{Q}(\gamma) \: \right]$

            \dotfill

            $\gamma = \cos\left(\frac{2\pi}{7}\right)$より
            $1-\gamma^2 = \sin\left(\frac{2\pi}{7}\right)$である。
            つまり、
            $\sin\left(\frac{2\pi}{7}\right) \in \mathbb{Q}(\gamma)$である。

            $\alpha = \gamma + i(1-\gamma^2)$であるので、
            $\alpha \in \mathbb{Q}(\gamma,i)$である。

            $i$は多項式$X^2+1 \in \mathbb{Q}(\gamma)[X]$の解であり、
            この多項式は既約である。

            よって、
            $\left[ \: \mathbb{Q}(\gamma,i):\mathbb{Q}(\gamma) \: \right]=2$
            である。

            $\alpha \in \mathbb{Q}(\gamma,i)$より
            $\mathbb{Q}(\alpha,\gamma) \subset \mathbb{Q}(\gamma,i)$であり、
            $\mathbb{Q}(\alpha,\gamma) \ne \mathbb{Q}(\gamma)$である。
            よって、
            $\left[ \: \mathbb{Q}(\alpha,\gamma):\mathbb{Q}(\gamma) \: \right]=2$
            である。


            \hrulefill

      \end{enumerate}

      \hrulefill
 \item
      $\alpha = \sqrt[4]{2} - \sqrt{2} -1$とする。

      この時、次の式を満たす$a_{i}\in\mathbb{Q} \: (i=0,\dots,3)$
      を求めよ。
      \begin{equation}
       \frac{1}{\alpha} = a_{0} + a_{1}\alpha  + a_{2}\alpha^{2} + a_{3}\alpha^{3}
      \end{equation}
      \dotfill

      $\mathbb{Q}(\sqrt[4]{2})$
      を$\mathbb{Q}$上のベクトル空間と考えると
      基底に$1,\sqrt[4]{2},\sqrt[4]{2}^{2},\sqrt[4]{2}^{3}$がとれる。
      よって、
      $\alpha \in \mathbb{Q}(\sqrt[4]{2})$
      である。

      $\alpha$は$\sqrt[4]{2}$を含んでいるので、
      $\alpha \not\in \mathbb{Q}(\sqrt{2})$である。

      よって、次のような関係がある。
      \begin{equation}
       \mathbb{Q}(\sqrt{2})
        \subsetneq
        \mathbb{Q}(\alpha)
        \subset
        \mathbb{Q}(\sqrt[4]{2})
      \end{equation}

      拡大次数はそれぞれ次のようになる。
      \begin{equation}
       \left[ \: \mathbb{Q}(\sqrt{2}):\mathbb{Q} \: \right]=2
        ,\quad
        \left[ \: \mathbb{Q}(\sqrt[4]{2}):\mathbb{Q} \: \right]=4
        ,\quad
        \left[ \: \mathbb{Q}(\sqrt[4]{2}):\mathbb{Q}(\sqrt{2}) \: \right]=2
      \end{equation}

      $\mathbb{Q}(\alpha) \ne \mathbb{Q}(\sqrt{2})$であることから
      $\left[ \: \mathbb{Q}(\sqrt[4]{2}):\mathbb{Q}(\alpha) \: \right]=1$
      であり、
      $\left[ \: \mathbb{Q}(\alpha):\mathbb{Q} \: \right]=4$
      である。

      $\mathbb{Q}(\alpha)$は$\mathbb{Q}$上のベクトル空間で
      基底を$1,\alpha,\alpha^{2},\alpha^{3}$と取ってこれる。
      そこで、
      $\alpha$の逆元$\alpha^{-1}$を基底を用いて表す。

      $\alpha = \sqrt[4]{2} - \sqrt{2} -1 = \sqrt[4]{2} - (\sqrt{2} +1)$
      であるので、
      $\left( \sqrt[4]{2} + (\sqrt{2} +1) \right)$をかけて根号を消す。
      \begin{align}
       \alpha \left( \sqrt[4]{2} + (\sqrt{2} +1) \right)
       &= \left( \sqrt[4]{2} - (\sqrt{2} +1) \right)\left( \sqrt[4]{2} + (\sqrt{2} +1) \right)\\
       &= \sqrt[4]{2}^{2} - (\sqrt{2} +1)^{2}
%       = \sqrt{2} - (3 + 2\sqrt{2})
       = - 3 - \sqrt{2}
      \end{align}
%
      これに$(- 3 + \sqrt{2})$をかける。
      \begin{align}
       \alpha \left( \sqrt[4]{2} + (\sqrt{2} +1) \right)(- 3 + \sqrt{2})
       &= (- 3 - \sqrt{2})(- 3 + \sqrt{2})
       = 7
      \end{align}

      よって、次の式が得られる。
      \begin{equation}
       \alpha \left( \sqrt[4]{2} + (\sqrt{2} +1) \right)(- 3 + \sqrt{2}) \cdot \frac{1}{7}
        =1
      \end{equation}


      これにより$\alpha$の逆元は
      $\left( \sqrt[4]{2} + (\sqrt{2} +1) \right)(- 3 + \sqrt{2}) \cdot \frac{1}{7}$
      ということである。
      \begin{equation}
       \alpha^{-1}=
       \left( \sqrt[4]{2} + (\sqrt{2} +1) \right)(- 3 + \sqrt{2}) \cdot \frac{1}{7}
%        = \left(-3\sqrt[4]{2} -3\sqrt{2} -3 + \sqrt[4]{2}^{3} +2+ \sqrt[4]{2}^{2} \right) \cdot \frac{1}{7}
        = \frac{1}{7}\left( -1 -3 \sqrt[4]{2} - 2 \sqrt[4]{2}^{2} + 2 \sqrt[4]{2}^{3}  \right)
      \end{equation}



      $\alpha^{2},\alpha^{3}$は次のように求められる。
      \begin{align}
       \alpha^{2}
        &= (\sqrt[4]{2} - \sqrt{2} -1)^{2}
%        &= \sqrt[4]{2}^{2} +2 +1 -2 \sqrt[4]{2}^{3} +2 \sqrt[4]{2}^{2} -2 \sqrt[4]{2}\\
        = 3 -2 \sqrt[4]{2} + 3 \sqrt[4]{2}^{2} - 2 \sqrt[4]{2}^{3}\\
       \alpha^{3}
       &=  (\sqrt[4]{2} - \sqrt{2} -1)^{3}
       =  -13 +9 \sqrt[4]{2} - 8 \sqrt[4]{2}^{2} + 7 \sqrt[4]{2}^{3}
      \end{align}

      
      $\mathbb{Q}(\sqrt[4]{2})=\mathbb{Q}(\alpha)$
      であるので、
      基底を
      $1,\sqrt[4]{2},\sqrt[4]{2}^{2},\sqrt[4]{2}^{3}$
      から
      $1,\alpha,\alpha^{2},\alpha^{3}$へ変換する。
      基底は次の行列で変換できるので
      逆行列をかけ、$\sqrt[4]{2}^{k}$から$\alpha^{k}$への変換行列を求める。
      \begin{align}
       \begin{pmatrix}
        1 \\ \alpha \\ \alpha^{2} \\ \alpha^{3}
       \end{pmatrix}
       =
       \begin{pmatrix}
        1 & 0 & 0 & 0 \\
        -1 & 1 & -1 & 0 \\
        3 & -2 & 3 & -2 \\
        -13 & 9 & -8 & 7
       \end{pmatrix}
       \begin{pmatrix}
        1 \\ \sqrt[4]{2} \\ \sqrt[4]{2}^{2} \\ \sqrt[4]{2}^{3}
       \end{pmatrix}
       \\
       \begin{pmatrix}
        1 \\ \sqrt[4]{2} \\ \sqrt[4]{2}^{2} \\ \sqrt[4]{2}^{3}
       \end{pmatrix}
       =
       \begin{pmatrix}
        1 & 0 & 0 & 0 \\
        -1 & 1 & -1 & 0 \\
        3 & -2 & 3 & -2 \\
        -13 & 9 & -8 & 7
       \end{pmatrix}
       ^{-1}
       \begin{pmatrix}
        1 \\ \alpha \\ \alpha^{2} \\ \alpha^{3}
       \end{pmatrix}
      \end{align}
      
      これにより
      $\alpha^{-1}$
      を基底変換して表現する。
      \begin{align}
       \alpha^{-1}
       &=
       \frac{1}{7}\left( -1 -3 \sqrt[4]{2} - 2 \sqrt[4]{2}^{2} + 2 \sqrt[4]{2}^{3}  \right)\\
       &=
       \frac{1}{7}
       \begin{pmatrix}
        -1 & -3 & -2 & 2
       \end{pmatrix}
       \begin{pmatrix}
        1 \\ \sqrt[4]{2} \\ \sqrt[4]{2}^{2} \\ \sqrt[4]{2}^{3}
       \end{pmatrix}\\
       &=
       \frac{1}{7}
       \begin{pmatrix}
        -1 & -3 & -2 & 2
       \end{pmatrix}
       \begin{pmatrix}
        1 & 0 & 0 & 0 \\
        -1 & 1 & -1 & 0 \\
        3 & -2 & 3 & -2 \\
        -13 & 9 & -8 & 7
       \end{pmatrix}
       ^{-1}
       \begin{pmatrix}
        1 \\ \alpha \\ \alpha^{2} \\ \alpha^{3}
       \end{pmatrix}
      \end{align}


      ここで係数部分を計算する。
      \begin{align}
       & \frac{1}{7}
       \begin{pmatrix}
        -1 & -3 & -2 & 2
       \end{pmatrix}
       \begin{pmatrix}
        1 & 0 & 0 & 0 \\
        -1 & 1 & -1 & 0 \\
        3 & -2 & 3 & -2 \\
        -13 & 9 & -8 & 7
       \end{pmatrix}
       ^{-1}\\
       =&
       \frac{1}{7}
       \begin{pmatrix}
        -1 & -3 & -2 & 2
       \end{pmatrix}
       \left(
       \frac{1}{9}
       \begin{pmatrix}
        9 & 0 & 0 & 0 \\
        10 & 5 & 7 & 2 \\
        1 & -4 & 7 & 2 \\
        5 & -11 & -1 & 1
       \end{pmatrix}
       \right)\\
       =&
       -\frac{1}{63}
       \begin{pmatrix}
        31 & 29 & 37 & 8
       \end{pmatrix}
      \end{align}

      よって、座標変換した$\alpha^{-1}$は
      次のようにかける。
      \begin{equation}
       \alpha^{-1}
        =
       -\frac{1}{63}
       \begin{pmatrix}
        31 & 29 & 37 & 8
       \end{pmatrix}
       \begin{pmatrix}
        1 \\ \alpha \\ \alpha^{2} \\ \alpha^{3}
       \end{pmatrix}
       =
       -\frac{31}{63}
       -\frac{29}{63}\alpha
       -\frac{37}{63}\alpha^{2}
       -\frac{8}{63}\alpha^{3}
      \end{equation}

      これにより
      係数
      $a_{i}\in\mathbb{Q} \: (i=0,\dots,3)$
      は次のようになる。

      \begin{equation}
       \begin{pmatrix}
        a_{0} & a_{1} & a_{2} & a_{3}
       \end{pmatrix}
       =
       \begin{pmatrix}
        -\frac{31}{63} & -\frac{29}{63} & -\frac{37}{63} & -\frac{8}{63}
       \end{pmatrix}
      \end{equation}



      \hrulefill

\end{enumerate}

\hrulefill

2問目の計算用 コード

\url{https://sagecell.sagemath.org/}
\begin{lstlisting}
x=2^(1/4)
print("--- alpha ---")
def f(x): return x - x^2 -1
print(f(x))
print("--- alpha^2 ---")
print(expand(f(x)^2))
print("--- alpha^3 ---")
print(expand(f(x)^3))

print("=== first ===")
print(expand(
    f(x)*(x +x^2 +1)
))
print("=== 2nd ===")
print(expand(
    f(x)*(x +x^2 +1)*(-3+x^2)
))

print("### alpha inverse ###")
print(expand(
    (x +x^2 +1)*(-3+x^2)
))

print("coefficient alpha^{-1}")
B=matrix([-1,-3,-2,2])
print(B)

print("$$$ transform basis matrix $$$")
A=matrix([[1,0,0,0],[-1,1,-1,0],[3,-2,3,-2],[-13,9,-8,7]])
print(A)

print("A.inverse()")
print(A.inverse())

print("B*A.inverse()")
print(B*A.inverse())
\end{lstlisting}


\end{document}
