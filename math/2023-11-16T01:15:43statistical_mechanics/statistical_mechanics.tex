\documentclass[12pt,a4paper]{ltjsarticle}

%\usepackage[margin=15truemm, top=5truemm, bottom=5truemm]{geometry}
%\usepackage[margin=10truemm,left=15truemm]{geometry}
\usepackage[margin=10truemm]{geometry}

\usepackage{amsmath,amssymb}
%\pagestyle{headings}
\pagestyle{empty}

%\usepackage{listings,url}
%\renewcommand{\theenumi}{(\arabic{enumi})}

%\usepackage{graphicx}

%\usepackage{tikz}
%\usetikzlibrary {arrows.meta}
%\usepackage{wrapfig}
%\usepackage{bm}

% ルビを振る
%\usepackage{luatexja-ruby}	% required for `\ruby'

%% 核Ker 像Im Hom を定義
%\newcommand{\Img}{\mathop{\mathrm{Im}}\nolimits}
%\newcommand{\Ker}{\mathop{\mathrm{Ker}}\nolimits}
%\newcommand{\Hom}{\mathop{\mathrm{Hom}}\nolimits}

%\DeclareMathOperator{\Rot}{rot}
%\DeclareMathOperator{\Div}{div}
%\DeclareMathOperator{\Grad}{grad}
%\DeclareMathOperator{\arcsinh}{arcsinh}
%\DeclareMathOperator{\arccosh}{arccosh}
%\DeclareMathOperator{\arctanh}{arctanh}

%\usepackage{url}

%\usepackage{listings}
%
%\lstset{
%%プログラム言語(複数の言語に対応,C,C++も可)
%  language = Python,
%%  language = Lisp,
%%  language = C,
%  %背景色と透過度
%  %backgroundcolor={\color[gray]{.90}},
%  %枠外に行った時の自動改行
%  breaklines = true,
%  %自動改行後のインデント量(デフォルトでは20[pt])
%  breakindent = 10pt,
%  %標準の書体
%%  basicstyle = \ttfamily\scriptsize,
%  basicstyle = \ttfamily,
%  %コメントの書体
%%  commentstyle = {\itshape \color[cmyk]{1,0.4,1,0}},
%  %関数名等の色の設定
%  classoffset = 0,
%  %キーワード(int, ifなど)の書体
%%  keywordstyle = {\bfseries \color[cmyk]{0,1,0,0}},
%  %表示する文字の書体
%  %stringstyle = {\ttfamily \color[rgb]{0,0,1}},
%  %枠 "t"は上に線を記載, "T"は上に二重線を記載
%  %他オプション:leftline,topline,bottomline,lines,single,shadowbox
%  frame = TBrl,
%  %frameまでの間隔(行番号とプログラムの間)
%  framesep = 5pt,
%  %行番号の位置
%  numbers = left,
%  %行番号の間隔
%  stepnumber = 1,
%  %行番号の書体
%%  numberstyle = \tiny,
%  %タブの大きさ
%  tabsize = 4,
%  %キャプションの場所("tb"ならば上下両方に記載)
%  captionpos = t
%}

%\usepackage{cancel}
%\usepackage{bussproofs}
%\usepackage{proof}

\begin{document}

\hrulefill

ベクトル空間$V$上の線形変換全体の集合を$\mathrm{End}(V)$と表す。
$\mathrm{End}(V) = \mathrm{Hom}(V,V)$である。
\begin{equation}
 %{}^{\forall}
  f \in \mathrm{End}(V)
  \; \overset{\mathrm{def}}{\Longleftrightarrow}\;
  f:V\to V
\end{equation}

\dotfill

遷移行列
\begin{equation}
 T^{(N)}(u) := \mathrm{tr}_{V^{(0)}}\left( R^{0N}(u)R^{0N-1}(u) \cdots R^{02}(u)R^{01}(u) \right)
  \in \mathrm{End} \left( \bigotimes_{i=1}^{N} V^{(i)} \right)
\end{equation}

\begin{align}
 T^{(N)}(u) &= \left( T^{(N)}(u)^{j_{1}\dots j_{N}}_{j_{1}^{\prime}\dots j_{N}^{\prime}} \right)^{j_{1}\dots j_{N}}_{j_{1}^{\prime}\dots j_{N}^{\prime}}
  \\
  T^{(N)}(u)^{j_{1}\dots j_{N}}_{j_{1}^{\prime}\dots j_{N}^{\prime}}
  &:=
  \sum_{i_{0},\dots ,i_{N-1}}
  R^{i_{0}j_{N}}_{i_{N-1}j^{\prime}_{N}}(u)
   R^{i_{N-1}j_{N-1}}_{i_{N-2}j^{\prime}_{N-1}}(u)
   \dots
   R^{i_{2}j_{2}}_{i_{1}j^{\prime}_{2}}(u)
   R^{i_{1}j_{1}}_{i_{0}j^{\prime}_{1}}(u)
\end{align}


\dotfill

\textbf{YBE Yang-Baxter 方程式}

$R(u)\in \mathrm{End}(V\otimes V)$とする。

$R(u) e_{i}\otimes e_{j} = \sum_{i^{\prime}j^{\prime}} e_{i_{i^{\prime}}}\otimes e_{j^{\prime}} R^{i^{\prime}j^{\prime}}_{ij}(u)$として
$R(u)=\left( R^{i^{\prime}j^{\prime}}_{ij}(u) \right)_{\substack{i,j=0,\dots,N-1 \\ i^{\prime},j^{\prime}=0,\dots,N-1}}$

YBE は $\mathrm{End}(V\otimes V\otimes V)$上の
\begin{equation}
 R^{01}(u) R^{02}(u+v) R^{12}(v) = R^{12}(v) R^{02}(u+v) R^{01}(u) \in \mathrm{End}(\overset{0}{V} \otimes \overset{1}{V} \otimes \overset{2}{V})
\end{equation}

$R^{01}$は
$\overset{0}{V} \otimes \overset{1}{V} \otimes \overset{2}{V}$の
$\overset{0}{V} \otimes \overset{1}{V}$上に
\begin{equation}
 R^{01}(u) e_{i} \otimes e_{j} \otimes e_{k}
  = \sum_{i,j} e_{i^{\prime}} \otimes e_{j^{\prime}} \otimes e_{k}
  R^{i^{\prime}j^{\prime}}_{ij}(u)
\end{equation}

\dotfill

$R(u)=u \mathrm{Id} + P$ とする。
YBE ($R^{01}(u) R^{02}(u+v) R^{12}(v) = R^{12}(v) R^{02}(u+v) R^{01}(u)$) は次のように表される。
\begin{equation}
 \begin{split}
  (u \mathrm{Id}^{01} + P^{01})& ((u+v) \mathrm{Id}^{02} + P^{02}) (v \mathrm{Id}^{12} + P^{12})\\
  &=   (v \mathrm{Id}^{12} + P^{12}) ((u+v) \mathrm{Id}^{02} + P^{02}) (u \mathrm{Id}^{01} + P^{01})
 \end{split}
\end{equation}

\begin{gather}
 R^{01}(u) = u \cdot \mathrm{Id}^{01} + P^{01}\\
 R^{02}(u+v) = (u+v) \cdot \mathrm{Id}^{02} + P^{02}\\
 R^{12}(v) = v \cdot \mathrm{Id}^{12} + P^{12}
\end{gather}


\dotfill

$\dim{V}=2$の時、
Yang-Baxter 方程式 の McGuire-Yang の解

\begin{align}
 R(u)
  &= u \cdot \mathrm{Id}_{\mathbb{C}^{2} \otimes \mathbb{C}^{2}} + P
  = \left( R_{ab,cd}(u) \right)_{ab,cd}
  = \left( R^{ab}_{cd}(u) \right)^{ab}_{cd}\\
  &=
  \begin{pmatrix}
   R^{00}_{00}(u) & R^{00}_{01}(u) & \cdots & R^{00}_{11}(u)\\
   R^{01}_{00}(u) & R^{01}_{01}(u) & \cdots & R^{01}_{11}(u)\\
   \vdots & \vdots & \ddots & \vdots\\
   R^{11}_{00}(u) & R^{11}_{01}(u) & \cdots & R^{11}_{11}(u)\\
  \end{pmatrix}
 =
  \begin{pmatrix}
   u+1 & 0 & 0 & 0\\
   0 & u & 1 & 0\\
   0 & 1 & u & 0\\
   0 & 0 & 0 & u+1\\
  \end{pmatrix}
\end{align}

\begin{align}
 R^{00}_{00}(u) &= R^{11}_{11}(u) = u+1\\
 R^{01}_{01}(u) &= R^{10}_{10}(u) = u\\
 R^{01}_{10}(u) &= R^{10}_{01}(u) = 1\\
 R^{ab}_{cd}(u) &= 0 \; (a+b \ne c+d)
\end{align}

\hrulefill

$V=\mathbb{C}^{2}= \langle e_{1},e_{2}\rangle$

以下を行列表示せよ。
\begin{enumerate}
 \item
      $T^{(1)}(u) \in \mathrm{End} \left( V^{(1)} \right)$

      \dotfill

      \begin{align}
%       T^{(1)}(u) &= \mathrm{tr}_{V^{(0)}}\left( R^{01}(u) \right) \in \mathrm{End}(V^{(1)})\\
       %
       T^{(1)}(u) &= \left( T^{(1)}(u)^{j_{1}}_{j_{1}^{\prime}} \right)^{j_{1}}_{j_{1}^{\prime}}\\
       %
       T^{(1)}(u)^{j_{1}}_{j_{1}^{\prime}}
       &= \sum_{i_{0}} R^{i_{0}j_{1}}_{i_{0}j_{1}^{\prime}}(u)\\
       T^{(1)}(u)^{1}_{1}
       &= R^{{1}{1}}_{{1}{1}}(u) + R^{{2}{1}}_{{2}{1}}(u) = u+1 + u =2u+1 \\
       T^{(1)}(u)^{1}_{2}
       &= R^{{1}{1}}_{{1}{2}}(u) + R^{{2}{1}}_{{2}{2}}(u) = 0+ 0 =0\\
       T^{(1)}(u)^{2}_{1}
       &= R^{{1}{2}}_{{1}{1}}(u) + R^{{2}{2}}_{{2}{1}}(u) = 0+ 0 =0\\
       T^{(1)}(u)^{2}_{2}
       &= R^{{1}{2}}_{{1}{2}}(u) + R^{{2}{2}}_{{2}{2}}(u) = u + u+1 =2u+1
       %
%       (u \cdot \mathrm{Id}_{V} + P)(e_{1})
%       &= u \cdot \mathrm{Id}_{V}(e_{1}) + P(e_{1})
%       =u e_{1} + e_{1} = (u+1)e_{1}\\
%       %
%       (u \cdot \mathrm{Id}_{V} + P)(e_{2})
%       &= u \cdot \mathrm{Id}_{V}(e_{2}) + P(e_{2})
%       =u e_{2} + e_{2} = (u+1)e_{2}
      \end{align}

%      $A \in \mathrm{End} \left( V^{(1)} \right)$

      \begin{equation}
       T^{(1)}(u)
        =
        \begin{pmatrix}
         2u+1 & 0\\
         0 & 2u+1
        \end{pmatrix}
      \end{equation}


      \hrulefill

 \item
      $T^{(2)}(u) \in \mathrm{End} \left( V^{(1)}\otimes V^{(2)} \right)$

      \dotfill

       \begin{equation}
        T^{(2)}(u) = \left( T^{(2)}(u)^{j_{1}j_{2}}_{j_{1}^{\prime}j_{2}^{\prime}} \right)^{j_{1}j_{2}}_{j_{1}^{\prime}j_{2}^{\prime}}
         ,\quad
       %
        T^{(2)}(u)^{j_{1}j_{2}}_{j_{1}^{\prime}j_{2}^{\prime}}
        = \sum_{i_{0},i_{1}} R^{i_{0}j_{2}}_{i_{1}j_{2}^{\prime}}(u) R^{i_{1}j_{1}}_{i_{0}j_{1}^{\prime}}(u)
       %
       \end{equation}



      \begin{align}
       T^{(2)}(u)^{11}_{11}
       &= \sum_{i_{0}i_{1}} R^{i_{0}1}_{i_{1}1}(u) R^{i_{1}1}_{i_{0}1}(u)\\
       &= R^{11}_{11}(u) R^{11}_{11}(u) + R^{21}_{11}(u) R^{11}_{21}(u) + R^{11}_{21}(u) R^{21}_{11}(u)  + R^{21}_{21}(u) R^{21}_{21}(u) \nonumber \\
       &= (u+1)^{2} + 0 + 0 + u^{2} = 2u^{2}+ 2u +1\\
       %
       T^{(2)}(u)^{12}_{11}
       &= \sum_{i_{0}i_{1}} R^{i_{0}2}_{i_{1}1}(u) R^{i_{1}1}_{i_{0}1}(u)\\
       &= R^{12}_{11}(u) R^{11}_{11}(u) + R^{22}_{11}(u) R^{11}_{21}(u) + R^{12}_{21}(u) R^{21}_{11}(u)  + R^{22}_{21}(u) R^{21}_{21}(u) \nonumber \\
       &= 0 + 0 + 0 + 0 = 0\\
       %
       T^{(2)}(u)^{21}_{11}
       &= T^{(2)}(u)^{11}_{12} = T^{(2)}(u)^{11}_{21} = 0
      \end{align}
       %
      \begin{align}
       T^{(2)}(u)^{22}_{11}
       &= \sum_{i_{0}i_{1}} R^{i_{0}2}_{i_{1}1}(u) R^{i_{1}2}_{i_{0}1}(u)\\
       &= R^{12}_{11}(u) R^{12}_{11}(u) + R^{22}_{11}(u) R^{12}_{21}(u) + R^{12}_{21}(u) R^{22}_{11}(u)  + R^{22}_{21}(u) R^{22}_{21}(u) \nonumber \\
       &= 0 + 0 + 0 + 0 = 0\\
       %
       T^{(2)}(u)^{11}_{22} &= 0\\
       T^{(2)}(u)^{12}_{12}
       &= \sum_{i_{0}i_{1}} R^{i_{0}2}_{i_{1}2}(u) R^{i_{1}1}_{i_{0}1}(u)\\
       &= R^{12}_{12}(u) R^{11}_{11}(u) + R^{22}_{12}(u) R^{11}_{21}(u) + R^{12}_{22}(u) R^{21}_{11}(u)  + R^{22}_{22}(u) R^{21}_{21}(u) \nonumber \\
       &= u(u+1) + 0 + 0 + (u+1)u = 2u^{2}+2u\\
       %
       T^{(2)}(u)^{21}_{21} &=2u^{2}+2u\\
       %
       T^{(2)}(u)^{12}_{21}
       &= \sum_{i_{0}i_{1}} R^{i_{0}2}_{i_{1}1}(u) R^{i_{1}1}_{i_{0}2}(u)\\
       &= R^{12}_{11}(u) R^{11}_{12}(u) + R^{22}_{11}(u) R^{11}_{22}(u) + R^{12}_{21}(u) R^{21}_{12}(u)  + R^{22}_{21}(u) R^{21}_{22}(u) \nonumber \\
       &= 0 + 0+ 1\cdot 1+ 0 = 1\\
       %
       T^{(2)}(u)^{21}_{12} &= 1
      \end{align}
       %
      \begin{align}
       T^{(2)}(u)^{21}_{22}
       &= T^{(2)}(u)^{12}_{22} =  T^{(2)}(u)^{22}_{21} =  T^{(2)}(u)^{22}_{12} =0\\
       %
       T^{(2)}(u)^{22}_{22}
       &= \sum_{i_{0}i_{1}} R^{i_{0}2}_{i_{1}2}(u) R^{i_{1}2}_{i_{0}2}(u)\\
       &= R^{12}_{12}(u) R^{12}_{12}(u) + R^{22}_{12}(u) R^{12}_{22}(u) + R^{12}_{22}(u) R^{22}_{12}(u)  + R^{22}_{22}(u) R^{22}_{22}(u) \nonumber \\
       &= u^{2} + 0 + 0 + (u+1)^{2} = 2u^{2}+ 2u +1
      \end{align}


      \begin{equation}
       T^{(2)}(u) =
        \begin{pmatrix}
         2u^{2}+ 2u +1 & 0 & 0 & 0\\
         0 & 2u^{2}+2u & 1 & 0\\
         0 & 1 & 2u^{2}+2u & 0\\
         0 & 0 & 0 & 2u^{2}+ 2u +1
        \end{pmatrix}
      \end{equation}


      \hrulefill

 \item
      $T^{(3)}(u) \in \mathrm{End} \left( V^{(1)}\otimes V^{(2)}\otimes V^{(3)} \right)$

      \dotfill


       \begin{equation}
        T^{(3)}(u) = \left( T^{(3)}(u)^{j_{1}j_{2}j_{3}}_{j_{1}^{\prime}j_{2}^{\prime}j_{3}^{\prime}} \right)^{j_{1}j_{2}j_{3}}_{j_{1}^{\prime}j_{2}^{\prime}j_{3}^{\prime}}
       \end{equation}

      \begin{align}
        T^{(3)}(u)^{j_{1}j_{2}j_{3}}_{j_{1}^{\prime}j_{2}^{\prime}j_{3}^{\prime}}
        &= \sum_{i_{0},i_{1},i_{2}}
       R^{i_{0}j_{3}}_{i_{2}j_{3}^{\prime}}(u)
       R^{i_{2}j_{2}}_{i_{1}j_{2}^{\prime}}(u)
       R^{i_{1}j_{1}}_{i_{0}j_{1}^{\prime}}(u)\\
       &=
       R^{{1}j_{3}}_{{1}j_{3}^{\prime}}(u)
       R^{{1}j_{2}}_{{1}j_{2}^{\prime}}(u)
       R^{{1}j_{1}}_{{1}j_{1}^{\prime}}(u)
       +
       R^{{1}j_{3}}_{{2}j_{3}^{\prime}}(u)
       R^{{2}j_{2}}_{{1}j_{2}^{\prime}}(u)
       R^{{1}j_{1}}_{{1}j_{1}^{\prime}}(u)\\
       &+
       R^{{1}j_{3}}_{{1}j_{3}^{\prime}}(u)
       R^{{1}j_{2}}_{{2}j_{2}^{\prime}}(u)
       R^{{2}j_{1}}_{{1}j_{1}^{\prime}}(u)
       +
       R^{{1}j_{3}}_{{2}j_{3}^{\prime}}(u)
       R^{{2}j_{2}}_{{2}j_{2}^{\prime}}(u)
       R^{{2}j_{1}}_{{1}j_{1}^{\prime}}(u)\\
       &+
       R^{{2}j_{3}}_{{1}j_{3}^{\prime}}(u)
       R^{{1}j_{2}}_{{1}j_{2}^{\prime}}(u)
       R^{{1}j_{1}}_{{2}j_{1}^{\prime}}(u)
       +
       R^{{2}j_{3}}_{{2}j_{3}^{\prime}}(u)
       R^{{2}j_{2}}_{{1}j_{2}^{\prime}}(u)
       R^{{1}j_{1}}_{{2}j_{1}^{\prime}}(u)\\
       &+
       R^{{2}j_{3}}_{{1}j_{3}^{\prime}}(u)
       R^{{1}j_{2}}_{{2}j_{2}^{\prime}}(u)
       R^{{2}j_{1}}_{{2}j_{1}^{\prime}}(u)
       +
       R^{{2}j_{3}}_{{2}j_{3}^{\prime}}(u)
       R^{{2}j_{2}}_{{2}j_{2}^{\prime}}(u)
       R^{{2}j_{1}}_{{2}j_{1}^{\prime}}(u)
      \end{align}



      \begin{align}
       T^{(3)}(u)^{{1}{1}{1}}_{{1}{1}{1}}
       &= T^{(3)}(u)^{{2}{2}{2}}_{{2}{2}{2}} = (u+1)^{3} + u^{3}\\
       %
       T^{(3)}(u)^{{1}{1}{2}}_{{1}{1}{2}}
       &= T^{(3)}(u)^{{1}{2}{1}}_{{1}{2}{1}}
       = T^{(3)}(u)^{{2}{1}{1}}_{{2}{1}{1}}
       = u(u+1)(2u+1)\\
       %
       T^{(3)}(u)^{{1}{2}{2}}_{{1}{2}{2}}
       &= T^{(3)}(u)^{{2}{1}{2}}_{{2}{1}{2}}
       = T^{(3)}(u)^{{2}{2}{1}}_{{2}{2}{1}}
       = u(u+1)(2u+1)\\
       %
       T^{(3)}(u)^{{1}{1}{2}}_{{1}{2}{1}}
       &= T^{(3)}(u)^{{1}{2}{1}}_{{2}{1}{1}}
       = T^{(3)}(u)^{{1}{2}{2}}_{{2}{2}{1}}
       = T^{(3)}(u)^{{2}{1}{1}}_{{1}{1}{2}}\\
       &= T^{(3)}(u)^{{2}{1}{2}}_{{1}{2}{2}}
       = T^{(3)}(u)^{{2}{2}{1}}_{{2}{1}{2}}
       = u+1\\
       %
       T^{(3)}(u)^{{2}{1}{2}}_{{2}{2}{1}}
       &= T^{(3)}(u)^{{1}{2}{2}}_{{2}{1}{2}}
       = T^{(3)}(u)^{{1}{1}{2}}_{{2}{1}{1}}
       = T^{(3)}(u)^{{2}{2}{1}}_{{1}{2}{2}}\\
       &= T^{(3)}(u)^{{2}{1}{1}}_{{1}{2}{1}}
       = T^{(3)}(u)^{{1}{2}{1}}_{{1}{1}{2}}
       = u
      \end{align}




      \begin{align}
       T^{(3)}(u) &=
        \begin{pmatrix}
         T^{111}_{111} & T^{111}_{112} & T^{111}_{121} & T^{111}_{122} & T^{111}_{211} & T^{111}_{212} & T^{111}_{221} & T^{111}_{222}\\
         T^{112}_{111} & T^{112}_{112} &  &  &  &  &  & \vdots\\
         \vdots & & \ddots & & & & & \vdots \\
         T^{222}_{111} & \cdots &  &  & & & & T^{222}_{222}
        \end{pmatrix}\\
       &=
        \begin{pmatrix}
         T^{111}_{111} & 0 & 0 & 0 & 0 & 0 & 0 & 0\\
         0 & T^{112}_{112} & T^{112}_{121} & 0 & T^{112}_{211} & 0 & 0 & 0 \\
         0 & T^{121}_{112} & T^{121}_{121} & 0 & T^{121}_{211} & 0 & 0 & 0 \\
         0 & 0 & 0 & T^{122}_{122} & 0 & T^{122}_{212} & T^{122}_{221} & 0 \\
         0 & T^{211}_{112} & T^{211}_{121} & 0 & T^{211}_{211} & 0 & 0 & 0 \\
         0 & 0 & 0 & T^{212}_{122} & 0 & T^{212}_{212} & T^{212}_{221} & 0 \\
         0 & 0 & 0 & T^{221}_{122} & 0 & T^{221}_{212} & T^{221}_{221} & 0 \\
         0 & 0 & 0 & 0 & 0 & 0 & 0 & T^{222}_{222}
        \end{pmatrix}\\
       &=
        \begin{pmatrix}
         (u+1)^{3}+u^{3} & 0 & 0 & 0 & 0 & 0 & 0 & 0\\
         0 & \alpha & u+1 & 0 & u & 0 & 0 & 0 \\
         0 & u & \alpha & 0 & u+1 & 0 & 0 & 0 \\
         0 & 0 & 0 & \alpha & 0 & u & u+1 & 0 \\
         0 & u+1 & u & 0 & \alpha & 0 & 0 & 0 \\
         0 & 0 & 0 & u+1 & 0 & \alpha & u & 0 \\
         0 & 0 & 0 & u & 0 & u+1 & \alpha & 0 \\
         0 & 0 & 0 & 0 & 0 & 0 & 0 &  (u+1)^{3}+u^{3}
        \end{pmatrix}
      \end{align}

      スペースの関係で$\alpha = u(u+1)(2u+1)$と置いている。



      \hrulefill

\end{enumerate}

\hrulefill

\end{document}
